%%Version 5.0, including the source files,  is published under a
%%Creative Commons Attribution-NonCommercial-ShareAlike 4.0 International licence
%%LaTeX file is very standard, and should work on any modern system.
%%File encoding is ASCII
%%Line endings are Unix-style (LF alone; \n)
%%If your text editor doesn't preserve the line endings, this will totally mess up the files
%%as many lines end in %.
%%Use defnb.tex to change fonts or margins.
%%The main file GT.tex was written in Scientific Workplace 5.50.
%%If you own SWP 5.5, you can use it to edit GT.tex;
%%others should ignore the %TCIDATA lines in preamble.

\documentclass[a4paper,11pt,final,openany]{memoir}
\usepackage{array}
\usepackage{amsxtra}
\usepackage[final]{graphicx}
\usepackage[amsmath,hyperref,thmmarks]{ntheorem}
\usepackage{natbib}
\usepackage{verbatim}
\usepackage{amsmath}
\usepackage{amsfonts}
\usepackage{amssymb}
\usepackage[notref,notcite]{showkeys}
\usepackage[colorlinks=true,bookmarksnumbered=true,pdfpagemode=None,final]%
{hyperref}%
\setcounter{MaxMatrixCols}{30}
%TCIDATA{OutputFilter=latex2.dll}
%TCIDATA{Version=5.50.0.2960}
%TCIDATA{CSTFile=memoir.cst}
%TCIDATA{Created=December 20, 2004}
%TCIDATA{LastRevised=Wednesday, November 18, 2020 19:33:34}
%TCIDATA{<META NAME="ViewPercent" CONTENT="135">}
%TCIDATA{<META NAME="GraphicsSave" CONTENT="32">}
%TCIDATA{<META NAME="SaveForMode" CONTENT="1">}
%TCIDATA{BibliographyScheme=Manual}
%TCIDATA{Language=British English}
\makeindex
\theoremnumbering{arabic}
\theoremheaderfont{\scshape}
\RequirePackage{latexsym}
\theorembodyfont{\slshape}
\theoremseparator{}
\newtheorem{X}{X}[chapter]
\newtheorem{corollary}[X]{Corollary}
\newtheorem{E}[X]{}
\newtheorem{lemma}[X]{Lemma}
\newtheorem{proposition}[X]{Proposition}
\newtheorem{theorem}[X]{Theorem}
\theorembodyfont{\upshape}
\newtheorem{definition}[X]{Definition}
\newtheorem{example}[X]{Example}
\newtheorem{question}[X]{Question}
\newtheorem{remark}[X]{Remark}
\newtheorem{plain}[X]{}
\newtheorem{summary}[X]{Summary}
\theorembodyfont{\small}
\newtheorem*{nt}{Notes}
\newtheorem{aside}[X]{Aside}
\theorembodyfont{\normalsize}
\newtheorem{Y}{Y}[chapter]
\newtheorem{exercise}[Y]{}
\theoremstyle{nonumberplain}
\theorembodyfont{\normalsize}
\theoremsymbol{\ensuremath{_\Box}}
\RequirePackage{amssymb}
\newtheorem{proof}{Proof.}
\newtheorem{sproof}{Sketch of proof.}
\newtheorem{pf}{Proof}
\newtheorem{proof1}{First Proof}
\newtheorem{proof2}{Second Proof}
\setsubsecheadstyle{\large\scshape\raggedright}
%%%Miscellaneous. Some of these only work in the Memoir class.

%%so def doesn't open lines. Use \overset{\df}{=}
\newcommand{\df}{\smash{\lower.12em\hbox{\textup{\tiny def}}}}

%%Turn off blackboxes in draft.
\overfullrule=0pt

%%This gets math in running heads to come out correctly.
\usepackage[overload]{textcase}

%%blueblack --- for links. Is best --- 0.5 not visible.
\usepackage{xcolor}
\definecolor{bblue}{rgb}{0.0, 0.0, 0.7}

%%fix headheight warning in Memoir
\setlength{\headheight}{14pt}

%%To stop footnotes in theorems coming out slanted
\renewcommand{\foottextfont}{\upshape\footnotesize}

%%For natbib: cite gives Author Year
\let\cite=\citealt

%%So memoir can print bibs
\renewcommand{\sc}{\scshape}

%%Turn emph into italic+bold
\newcommand{\eb}[1]{{\itshape\bfseries#1}}
\renewcommand{\emph}{\eb}

%%improves spacing
\usepackage{microtype}

%%%FONTS

%% This should work for everyone.
\usepackage{mathptmx}

%% To use the STIX fonts, put
%\usepackage{stix2}
%%in GT.tex after \document class and uncomment a line after tikzcd.
%% With the stix fonts, my system gives a fatal error unless I change
%% \chapter{Representations of Finite Groups} to ... finite groups.

%% I use the mtpro2 fonts, but they cost money.
%\usepackage{mathptmx}
%\usepackage[scaled=0.92]{helvet}
%\usepackage[subscriptcorrection,slantedGreek,nofontinfo]{mtpro2}

%% To use the free version of the mtpro2 fonts, replace the last line with
%\usepackage[lite,subscriptcorrection,slantedGreek,nofontinfo]{mtpro2}

\usepackage[bitstream-charter]{mathdesign}

%%%Diagrams
\usepackage{tikz}
\usepackage{tikz-cd}
\usetikzlibrary{matrix,arrows,positioning,decorations.pathmorphing}
\usetikzlibrary{decorations.markings,shapes.geometric}
\tikzset{commutative diagrams/column sep/Huge/.initial=24ex}

%%Stix gives weird arrow heads in tikzcd; the next line restores the status quo.
%\tikzcdset{arrow style=math font}


%%%Define the chapterstyle daleif2
\makechapterstyle{daleif2}{
\renewcommand\chapnamefont{\normalfont\Large\scshape\raggedleft}
\renewcommand\chaptitlefont{\normalfont\Huge\bfseries\sffamily\raggedleft}
\renewcommand\chapternamenum{}
%\renewcommand\printchapternum{%
%\makebox[0pt][l]{\hspace{0.4em}%
%\resizebox{!}{4ex}{\chapnamefont\bfseries\sffamily\thechapter}}}
\renewcommand\afterchapternum{\par\hspace{1.5cm}\hrule\vskip\midchapskip}
}

%%%Counters and numbering
\maxsecnumdepth{chapter}
\maxtocdepth{section}
\renewcommand{\theequation}{\arabic{equation}}
\setcounter{page}{-1}

%%Change the numbering of exercises to section-number.
\renewcommand{\theY}{\arabic{chapter}-\arabic{Y}}
\renewcommand{\theX}{\arabic{chapter}.\arabic{X}}

%%Get equations numbered consecutively throughout.
\usepackage{chngcntr}
\counterwithout{equation}{chapter}


%%%Lists

%Get rid of black bullets, which I find ugly
\renewcommand{\labelitemii}{$\circ$\hspace{0.07in}}
\renewcommand{\labelitemi}{$\diamond$\hspace{0.07in}}

%%This requires memoir
\tightlists

%Change enumerate to letters/small roman
\renewcommand{\theenumi}{{\alph{enumi}}}
\renewcommand{\labelenumi}{\upshape{(\theenumi)}}
\renewcommand{\theenumii}{\roman{enumii}}
\renewcommand{\labelenumii}{\upshape{\theenumii)}}

%%%These definitions are to get the file running smoothly in Scientific Workplace
%%%They fix problem with SWP misshandling some commands....
\newcommand\bquote{\begin{quote}}
\newcommand\equote{\end{quote}}
\newcommand\bcomment{\begin{comment}}
\newcommand\ecomment{\end{comment}}
\newcommand\bsmall{\begin{small}}
\newcommand\esmall{\end{small}}
\newcommand\bhangparas{\begin{hangparas}{2em}{1}} %%SWP screws this up completely
\newcommand\ehangparas{\end{hangparas}}
\newcommand\tableoc{\tableofcontents}
%To get \textstyle and \displaystyle painlessly
\newcommand{\tstyle}{\textstyle}
\newcommand{\dstyle}{\displaystyle}

%%%Setup for hyperref
\hypersetup{linkcolor=bblue,anchorcolor=bblue,citecolor=bblue,filecolor=bblue,urlcolor=bblue}
\hypersetup{pdftitle={Group Theory},pdfauthor={J.S. Milne},pdfkeywords={groups}}

%%%Page setup. Choose one.

%%This gives a text size of 5.5in x 9.0in on a4paper
%\usepackage[textwidth=5.5in,textheight=9.0in,centering,a4paper]{geometry}

%% Phone screen size
\usepackage[textwidth=3.32in,textheight=5.7in,pdftex,centering,papersize={3.5in,6.7in}]{geometry}

%%This gives a text size of 5.5in x 9.0in with margins of 0.5in (better for viewing on screen)
%\usepackage[papersize={6.5in,10.0in},margin=0.5in,pdftex]{geometry}

%%This loads various definitions (shortcuts).

%%Commands SWP won't recognize.
\newcommand{\bcomment}{\begin{comment}}\newcommand\ecomment{\end{comment}}
\newcommand{\bfootnotesize}{\begin{footnotesize}}\newcommand\efootnotesize{\end{footnotesize}}
\newcommand{\bquote}{\begin{quote}}\newcommand\equote{\end{quote}}
\newcommand{\bsmall}{\begin{small}}\newcommand\esmall{\end{small}}
\newcommand{\btable}{\begin{table}}\newcommand{\etable}{\end{table}}
\newcommand{\dstyle}{\displaystyle}
\newcommand{\edocument}{\end{document}}
\newcommand{\fsize}{\footnotesize}
\newcommand{\mdskip}{\medskip}
\newcommand{\tableoc}{\tableofcontents}%* removes "contents" from TOC
\newcommand{\textnf}{\textnormal}
\newcommand{\tstyle}{\textstyle}

%hyphenation
\hyphenation{Grot-hen-dieck Kron-ecker}

%Blackboard bold one. since alas there's no "\Bbb{1}";
\def\1{{1\mkern-7mu1}}

%Operators, which are set in mathrm with a little space after unless ( follows.
%\let\del=\nabla
%\newcommand\Cor}{Cor}}  Already defined.
%\DeclareMathOperator{\div}{div}%%This is already defined in TeX??
%\newcommand{\Proj}{\mathsf{Proj}}%%Problem here with Proj below.
\DeclareMathOperator{\ad}{ad}
\DeclareMathOperator{\Ad}{Ad}
\DeclareMathOperator{\art}{art}
%\DeclareMathOperator{\ar}{ar}%%Conflicts with xy
\DeclareMathOperator{\Aut}{Aut}
\DeclareMathOperator{\Bd}{Bd}
\DeclareMathOperator{\Br}{Br}
\DeclareMathOperator{\Card}{Card}
%\DeclareMathOperator{\ch}{\mathrm{char}}%%\char creates chaos -- used internally. Avoid this; use \mathrm{char} instead.
\DeclareMathOperator{\Cl}{Cl}
\DeclareMathOperator{\codim}{codim}
\DeclareMathOperator{\Coker}{Coker}
\DeclareMathOperator{\Corr}{Corr}
\DeclareMathOperator{\disc}{disc}
\DeclareMathOperator{\Div}{Div}
\DeclareMathOperator{\divv}{div}
\DeclareMathOperator{\dlim}{\varinjlim}%%\nolimits}%%Causes problems with ArXiVE
\DeclareMathOperator{\Dlim}{\underrightarrow{\mathrm{Lim}}}%2-category inverse limit
\DeclareMathOperator{\End}{End}
\DeclareMathOperator{\ev}{ev}
\DeclareMathOperator{\Ext}{Ext}
\DeclareMathOperator{\Filt}{Filt}
\DeclareMathOperator{\Frob}{Frob}
\DeclareMathOperator{\Gal}{Gal}
\DeclareMathOperator{\GL}{GL}
\DeclareMathOperator{\Gr}{Gr} %%what does this mean???
\DeclareMathOperator{\GSp}{GSp}
\DeclareMathOperator{\GU}{GU}
\DeclareMathOperator{\Hg}{Hg}
\DeclareMathOperator{\Hom}{Hom}
\DeclareMathOperator{\id}{id}
\DeclareMathOperator{\Id}{Id}
\DeclareMathOperator{\im}{Im}  %Image \Im is already used by TeX.
\DeclareMathOperator{\Ilim}{\underleftarrow{\mathrm{Lim}}}%2-category inverse limit
\DeclareMathOperator{\Ind}{Ind}
\DeclareMathOperator{\Inf}{Inf}
\DeclareMathOperator{\inn}{inn}
\DeclareMathOperator{\inv}{inv}
\DeclareMathOperator{\Inv}{Inv}
\DeclareMathOperator{\Isom}{Isom}
\DeclareMathOperator{\Ker}{Ker}
\DeclareMathOperator{\Lie}{Lie}
\DeclareMathOperator{\Map}{Map}
\DeclareMathOperator{\meas}{meas}
\DeclareMathOperator{\Mor}{Mor}
\DeclareMathOperator{\mt}{mt}
\DeclareMathOperator{\MT}{MT}
\DeclareMathOperator{\Nm}{Nm}
\DeclareMathOperator{\ob}{ob}
\DeclareMathOperator{\ord}{ord}
\DeclareMathOperator{\Out}{Out}
\DeclareMathOperator{\PGL}{PGL}
\DeclareMathOperator{\Pic}{\mathrm{Pic}}
\DeclareMathOperator{\plim}{\varprojlim}%%\nolimits}%%Causes problems with ArXive
\DeclareMathOperator{\Proj}{\mathrm{Proj}}
\DeclareMathOperator{\proj}{\mathrm{proj}}
\DeclareMathOperator{\pr}{pr}
\DeclareMathOperator{\PSL}{PSL}
\DeclareMathOperator{\PSP}{PSp}
\DeclareMathOperator{\rad}{rad}
\DeclareMathOperator{\rank}{rank}
\DeclareMathOperator{\rec}{rec}
\DeclareMathOperator{\Res}{Res}
\DeclareMathOperator{\Sh}{Sh}
\DeclareMathOperator{\sign}{sign}
\DeclareMathOperator{\SL}{SL}
\DeclareMathOperator{\SO}{SO}
\DeclareMathOperator{\specm}{specm}
\DeclareMathOperator{\Specm}{Specm}
\DeclareMathOperator{\spec}{spec}
\DeclareMathOperator{\Spec}{Spec}
\DeclareMathOperator{\Spin}{Spin}
\DeclareMathOperator{\spm}{spm}
\DeclareMathOperator{\Spm}{Spm}
\DeclareMathOperator{\SP}{Sp} %\Sp already used.
\DeclareMathOperator{\Stab}{Stab}
\DeclareMathOperator{\SU}{SU}
\DeclareMathOperator{\Sym}{Sym}
\DeclareMathOperator{\SMT}{SMT}
\DeclareMathOperator{\Tgt}{Tgt}
\DeclareMathOperator{\Tor}{Tor}
\DeclareMathOperator{\Tr}{Tr}
%%\DeclareMathOperator{\U}{U} This causes weird problems with SWP.

%Some categories (Mac Lane p291).
\newcommand{\Ab}{\mathsf{Ab}}%abelian groups
\newcommand{\Aff}{\mathsf{Aff}}%affine schemes
\newcommand{\AM}{\mathsf{AM}}%Abelian motives
\newcommand{\Art}{\mathsf{Art}}%Artin motives
\newcommand{\AT}{\mathsf{AT}}%Artin-Tate motives
\newcommand{\AV}{\mathsf{AV}}%abelian varieties
\newcommand{\Cat}{\mathsf{CAT}}%Categories and functors
\newcommand{\CMAV}{\mathsf{CMAV}}%abelian varieties of CM-type
\newcommand{\CM}{\mathsf{CM}}%CM-motives
\newcommand{\Comod}{\mathsf{Comod}}%Comodules
\newcommand{\Crys}{\mathsf{Crys}}%crystals
\newcommand{\C}{\mathsf{C}}%general category
\newcommand{\Desc}{\mathsf{Desc}}%Descent
\newcommand{\Ens}{\mathsf{Ens}}%Sets No! better Set
\newcommand{\Et}{\mathsf{Et}}%Etale algebras.
\newcommand{\Hdg}{\mathsf{Hdg}}%Hodge structures
\newcommand{\HMot}{\mathsf{HMot}}
\newcommand{\Hod}{\mathsf{Hod}}%Hodge structures
\newcommand{\Isab}{\mathsf{Isab}}%abelian varieties up to isogeny
\newcommand{\Isoc}{\mathsf{Isoc}}%Isocrystals
\newcommand{\LCM}{\mathsf{LCM}}%Lefschetz motives of CM-type
\newcommand{\LMot}{\mathsf{LMot}}%Lefschetz motives
\newcommand{\LM}{\mathsf{LM}}%Lefschetz motives (shorthand)
\newcommand{\Mdot}{\dot{\mathsf{M}}}
\newcommand{\MF}{\mathsf{MF}}
\newcommand{\Modf}{\mathsf{Modf}}%modules of finite type.
\newcommand{\Mod}{\mathsf{Mod}}%modules
\newcommand{\Mot}{\mathsf{Mot}}%motives
\newcommand{\M}{\mathsf{M}}
\newcommand{\norm}{\Vert}
\newcommand{\PMot}{\mathsf{PMot}}
\newcommand{\QMot}{\mathsf{QMot}}
\newcommand{\Rep}{\mathsf{Rep}}%representations
\newcommand{\Set}{\mathsf{Set}}%Sets Conflicts with braket
\newcommand{\Tat}{\mathsf{Tat}}%Tate motives
\newcommand{\Var}{\mathsf{Var}}
\newcommand{\Vct}{\mathsf{Vec}}
\newcommand{\Vc}{\mathsf{Vec}}%%\Vec would be better, but confuses SWP
\newcommand{\V}{\mathsf{V}}
\renewcommand{\ker}{Ker}

%Some bold expresssions
\newcommand{\bp}{\boldsymbol{\pi}}
\newcommand{\bM}{\bold{M}}
\newcommand{\bL}{\bold{L}} %%\L is already taken.

%Some text boxes
\newcommand{\ab}{\mathrm{ab}}
\newcommand{\al}{\mathrm{al}}
\newcommand{\can}{\mathrm{can}}
\newcommand{\cl}{\mathrm{cl}}
\newcommand{\cm}{\mathrm{cm}}
\newcommand{\crys}{\mathrm{crys}}
\newcommand{\diag}{\mathrm{diag}}
\newcommand{\dR}{\mathrm{dR}}
\newcommand{\der}{\mathrm{der}}
\newcommand{\et}{\mathrm{et}}
\newcommand{\fg}{\mathrm{fg}}
\newcommand{\forget}{\mathrm{forget}}
\newcommand{\lift}{\mathrm{lift}}
\newcommand{\num}{\mathrm{num}}
\newcommand{\rat}{\mathrm{rat}}
%\newcommand{\th}{\mathrm{th}}%%already defined.
\newcommand{\un}{\mathrm{un}}


%Some fancy small caps.
\newcommand{\Tors}{\textsc{Tors}}
\newcommand{\Fib}{\textsc{Fib}}


\begin{document}
\chapterstyle{daleif2}

\pagestyle{empty} \vspace*{0.5in}
\centerline{\fontsize{30}{40}\selectfont Fields and Galois Theory}
\vspace{1in} \centerline{\fontsize{20}{30}\selectfont J.S. Milne}
\vspace{2.0in}

\begin{minipage}{2.5in}
\begin{tikzpicture}[descr/.style={fill=white}]
\matrix(m)[matrix of math nodes, row sep=3.0em, column sep=1.6em,
text height=1.5ex, text depth=0.25ex]
{&\mathbb{Q}[\zeta]\\
\mathbb{Q}[\zeta+\bar{\zeta}]&&\mathbb{Q}[\sqrt{-7}]\\
&\mathbb{Q}\\};
\path[-,font=\scriptsize]
(m-1-2) edge  node[descr] {$\langle\sigma^3 \rangle$}(m-2-1)
edge  node[descr] {$\langle\sigma^2 \rangle$} (m-2-3)
(m-3-2) edge  node[descr] {$\langle\sigma\rangle/\langle\sigma^3	 \rangle$}(m-2-1)
edge  node[descr] {$\langle\sigma\rangle/\langle\sigma^2 \rangle$}(m-2-3);
\end{tikzpicture}
Splitting field of $X^7-1$ over $\mathbb{Q}$.
\end{minipage}\hspace{0.5in} \begin{minipage}{2.0in}
\begin{tikzpicture}[descr/.style={fill=white}]
\matrix(m)[matrix of math nodes, row sep=2.5em, column sep=1em,
text height=1.5ex, text depth=0.25ex]
{&\mathbb{Q}[\zeta,\alpha]\\
\mathbb{Q}[\zeta]&&\mathbb{Q}[\alpha]\\
&\mathbb{Q}\\};
\path[-,font=\scriptsize]
(m-1-2) edge  node[descr] {$N$}(m-2-1)
edge  node[descr] {$H$} (m-2-3)
(m-3-2) edge  node[descr] {$G/N$}(m-2-1)
edge  (m-2-3);
\end{tikzpicture}
Splitting field of $X^5-2$ over $\mathbb{Q}$.
\end{minipage}


\vspace{1in}

\centerline{\begin{minipage}{2.0in}\small{Version 5.00
\newline June 2021}\end{minipage}}



\raggedbottom\clearpage\pagenumbering{arabic}

\vspace*{0.1in} \noindent These
notes give a concise exposition of the theory of fields, including the Galois
theory of finite and infinite extensions and the theory of transcendental
extensions. The first six chapters form a standard course, and the final three
chapters are more advanced.\vfill

\begin{verbatim}
BibTeX information
@misc{milneFT,
  author={Milne, James S.},
  title={Fields and Galois Theory (v5.00)},
  year={2021},
  note={Available at www.jmilne.org/math/},
  pages={142}
}
\end{verbatim}

\vfill

\noindent Please send comments and corrections to me at jmilne at umich.edu.
\begin{description}
\item[v2.01] (August 21, 1996). First version on the web.

\item[v2.02] (May 27, 1998). Fixed many minor errors; 57 pages.

\item[v3.00] (April 3, 2002). Revised notes; minor additions to text; added 82
exercises with solutions, an examination, and an index; 100 pages.

\item[v4.00] (February 19, 2005). Revised notes; added proofs for Infinite
Galois Extensions; expanded Transcendental Extensions; 107 pages.

\item[v4.10] (January 22, 2008). Minor corrections and improvements; added
proofs for Kummer theory; 111 pages.

\item[v4.20] (February 11, 2008). Replaced Maple with PARI; 111 pages.

\item[v4.30] (April 15, 2012). Minor fixes; added sections on \'etale
algebras; 124 pages.

\item[v4.50] (March 18, 2014). Added chapter on the Galois theory of \'etale
algebras (Chapter 8); other improvements; numbering has changed; 138 pages.

\item[v4.61] (April 2020). Minor fixes and additions; numbering little
changed; 138 pages.

\item[v5.00] (June 2021). First version available (with its source code) under a Creative Commons licence.
\end{description}


\vfill

\noindent Version 5.0 is published under a Creative Commons
Attribution-NonCommercial-ShareAlike 4.0 International licence
(CC BY-NC-SA 4.0).

\bigskip

\noindent Licence information:
\href{https://creativecommons.org/licenses/by-nc-sa/4.0/}{https://creativecommons.org/licenses/by-nc-sa/4.0}

\bigskip
\noindent Copyright \copyright 1996--2021 J.S. Milne.

\setcounter{page}{2} \clearpage \pagestyle{plain} \vspace*{0in}

\tableoc\clearpage


\section{Notation.}

\sloppy We use the standard (Bourbaki) notation:
\begin{align*}
\mathbb{N}  &  =\{0,1,2,\ldots\},\\
\mathbb{Z}  &  =\text{ring of integers,}\\
\mathbb{R}{}  &  =\text{field of real numbers,}\\
\mathbb{C}{}  &  =\text{field of complex numbers,}\\
\mathbb{F}_{p}  &  =\mathbb{Z}{}/p\mathbb{Z}{}=\text{field with }p\text{
elements, }p\text{ a prime number.}%
\end{align*}
Given an equivalence relation, $[\ast]$ denotes the equivalence class
containing $\ast$. The cardinality of a set $S$ is denoted by $\left\vert
S\right\vert $ (so $\left\vert S\right\vert $ is the number of elements in
$S$when $S$ is finite). Let $I$ and $A$ be sets. A family of elements of $A$
indexed by $I$, denoted by $(a_{i})_{i\in I}$, is a function $i\mapsto
a_{i}\colon I\rightarrow A$. Throughout the notes, $p$ is a prime number:
$p=2,3,5,7,11,\ldots$.

$%
\begin{array}
[c]{ll}%
X\subset Y & X\text{ is a subset of }Y\text{ (not necessarily proper).}\\
X\overset{\df}{=}Y & X\text{ is defined to be }Y\text{, or
equals }Y\text{ by definition.}\\
X\approx Y & X\text{ is isomorphic to }Y\text{.}\\
X\simeq Y & X\text{ and }Y\text{ are canonically isomorphic}\\
  & \text{(or there is a
given or unique isomorphism).}%
\end{array}
$

\subsection{Prerequisites}

Group theory (for example, GT), basic linear algebra, and some elementary
theory of rings.

\section{References.}

\noindent Jacobson, N., 1964, Lectures in Abstract Algebra, Volume III, van Nostrand.

\noindent Also, the following of my notes (available at www.jmilne.org/math/).

\begin{description}
\item[GT] Group Theory, v4.00, 2021.

\item[ANT] Algebraic Number Theory, v3.08, 2020.

\item[CA] A Primer of Commutative Algebra, v4.03, 2020.

\item[monnnn] Question nnnn on mathoverflow.net.

\item[PARI] An open source computer algebra system that you can run in your
browser. It is freely available \href{http://pari.math.u-bordeaux.fr/}{here}.
\end{description}

\subsection{Acknowledgements}

I thank the following for providing corrections and comments for earlier
versions of the notes: Mike Albert, Terezakis Alexios, Carlos Alberto Ajila Loayza, Lior Bary-Soroker,
Maren Baumann, Leendert Bleijenga, Jin Ce, Tommaso Centeleghe, Sergio Chouhy,
Demetres Christofides, Antoine Chambert-Loir, Dustin Clausen, Keith Conrad,
Daniel Duparc, Hardy Falk, Ralf Goertz, Le Minh Ha, Matin Hajian, Jens Hansen, Albrecht Hess, Tim
Holzschuh, Philip Horowitz, Ivan Ip, Trevor Jarvis, Henry Kim, Martin Klazar,
Jasper Loy Jiabao, Weiyi Liu, Dmitry Lyubshin, Geir Arne Magnussen, John
McKay, Sarah Manski, Georges E.\thinspace Melki, Courtney Mewton, C Nebula,
Shuichi Otsuka, Dmitri Panov, Artem Pelenitsyn, Alain Pichereau, David G.
Radcliffe, Roberto La Scala, Chad Schoen, Ren\'{e} Schoof, Prem L Sharma, Dror
Speiser, Sam Spiro, Bhupendra Nath Tiwari, Mathieu Vienney, Martin Ward (and
class), Yervand Yeghiazarian, Xiande Yang, Wei Xu, and others.

\clearpage
%\thispagestyle{empty}\vspace*{0in}


\pagestyle{ruled} \makeevenfoot{ruled}{}{}{}{} \makeoddfoot{ruled}{}{}{}{}
\makeevenhead{ruled}{\thepage}{\scshape \leftmark}{} \makeoddhead{ruled}{}{\rightmark}{\thepage}

\chapter{Basic Definitions and Results}

\section{Rings}

A \emph{ring}%
\index{ring}
is a set $R$ with two binary operations $+$ and $\cdot$ such that

\begin{enumerate}
\item $(R,+)$ is a commutative group;

\item $\cdot$ is associative, and there exists\footnote{We require that rings
have a $1$, which entails that we require homomorphisms to preserve it.
\par
{}} an element $1_{R}$ such that $a\cdot1_{R}=a=1_{R}\cdot a$ for all $a\in
R;$

\item the distributive law holds: for all $a,b,c\in R$,
\begin{align*}
(a+b)\cdot c  &  =a\cdot c+b\cdot c\\
a\cdot(b+c)  &  =a\cdot b+a\cdot c\text{.}%
\end{align*}

\end{enumerate}

\noindent We usually omit \textquotedblleft$\cdot$\textquotedblright\ and
write $1$ for $1_{R}$ when this causes no confusion. If $1_{R}=0$, then
$R=\{0\}$.

A \emph{subring}%
\index{subring}
of a ring $R$ is a subset $S$ that contains $1_{R}$ and is closed under
addition, passage to the negative, and multiplication. It inherits the
structure of a ring from that on $R$.

A \emph{homomorphism of rings}%
\index{homomorphism!of rings}
$\alpha\colon R\rightarrow R^{\prime}$ is a map such that
\[
\alpha(a+b)=\alpha(a)+\alpha(b),\quad\alpha(ab)=\alpha(a)\alpha(b),\quad
\alpha(1_{R})=1_{R^{\prime}}%
\]


\noindent for all $a,b\in R$. A ring $R$ is said to be \emph{commutative}%
\index{commutative}
if multiplication is commutative:
\[
ab=ba\text{ for all }a,b\in R.
\]
A commutative ring is said to be an \emph{integral domain}%
\index{integral domain}
if $1_{R}\neq0$ and the cancellation law holds for multiplication,
\[
ab=ac\text{, }a\neq0\text{, implies }b=c.
\]
An \emph{ideal}%
\index{ideal}%
\emph{\ }$I$ in a commutative ring $R$ is a subgroup of $(R,+)$ that is closed
under multiplication by elements of $R$,%
\[
r\in R\text{, }a\in I\text{, implies }ra\in I.
\]
The ideal generated by elements $a_{1},\ldots,a_{n}$ is denoted by
$(a_{1},\ldots,a_{n})$. For example, $(a)$ is the principal ideal $aR$.

We assume that the reader has some familiarity with the elementary theory of
rings. For example, in $\mathbb{Z}{}$ (more generally, any Euclidean domain)
an ideal $I$ is generated by any \textquotedblleft smallest\textquotedblright%
\ nonzero element of $I$, and unique factorization into powers of prime
elements holds. We write
\index{gcd@$\gcd$}%
$\gcd(a,b)$ for the greatest common divisor of $a$ and $b$, e.g.,
$\gcd(a,0)=a.$

\section{Fields}

\begin{definition}
\label{ef0}A \emph{field\/}%
\index{field}
is a set $F$ with two composition laws $+$ and $\cdot$ such that

\begin{enumerate}
\item $(F,+)$ is a commutative group;

\item $(F^{\times},\cdot)$, where $F^{\times}=F\smallsetminus\{0\}$, is a
commutative group;

\item the distributive law holds.
\end{enumerate}
\end{definition}

\noindent Thus, a field is a nonzero commutative ring such that every nonzero
element has an inverse. In particular, it is an integral domain. A field
contains at least two distinct elements, $0$ and $1$. The smallest, and one of
the most important, fields is $\mathbb{\mathbb{F}}_{2}=\mathbb{Z}%
/2\mathbb{Z}=\{0,1\}$.

A \emph{subfield}%
\index{subfield}
$S$ of a field $F$ is a subring that is closed under passage to the inverse.
It inherits the structure of a field from that on $F$.

\begin{lemma}
\label{ef1}A nonzero commutative ring $R$ is a field if and only if it has no
ideals other than $(0)$ and $R$.
\end{lemma}

\begin{proof}
Suppose that $R$ is a field, and let $I$ be a nonzero ideal in $R$. If $a$ is
a nonzero element of $I$, then $1=a^{-1}a\in I$, and so $I=R$. Conversely,
suppose that $R$ is a commutative ring with no proper nonzero ideals. If
$a\neq0$, then $(a)=R$, and so there exists a $b$ in $R$ such that $ab=1$.
\end{proof}

\begin{example}
\label{ef2}The following are fields: $\mathbb{Q}$, $\mathbb{R}$, $\mathbb{C}$,
$\mathbb{F}_{p}=\mathbb{Z}/p\mathbb{Z}$ ($p$ prime)$.$
\end{example}

A \emph{homomorphism of fields}%
\index{homomorphism!of fields}
is simply a homomorphism of rings. Such a homomorphism is always injective,
because its kernel is a proper ideal (it doesn't contain $1$), which must
therefore be zero.

Let $F$ be a field. An $F$\emph{-algebra}%
\index{F-algebra@$F$-algebra}
(or \emph{algebra over }$F$)%
\index{algebra over F@algebra over $F$}
is a ring $R$ containing $F$ as a subring (so the inclusion map is a
homomorphism). A \emph{homomorphism of }$F$\emph{-algebras} $\alpha\colon
R\rightarrow R^{\prime}$%
\index{homomorphism!of F-algebras@of $F$-algebras}
is a homomorphism of rings such that $\alpha(c)=c$ for every $c\in F$.

\section{The characteristic of a field}

One checks easily that the map
\[
\mathbb{Z}\rightarrow F,\quad n\mapsto n\cdot1_{F}\overset{\textup{{\tiny def}%
}}{=}1_{F}+1_{F}+\cdots+1_{F}\quad(n\text{ copies of }1_{F}),
\]
is a homomorphism of rings. For example,%
\[
(\underbrace{1_{F}+\cdots+1_{F}}_{m})+(\underbrace{1_{F}+\cdots+1_{F}}%
_{n})=\underbrace{1_{F}+\cdots+1_{F}}_{m+n}%
\]
because of the associativity of addition. Therefore its kernel is an ideal in
$\mathbb{Z}{}$.

\textsc{Case 1:\/} The kernel of the map is $(0)$, so that
\[
n\cdot1_{F}=0\quad\text{(in }F\text{)}\implies n=0\quad\text{(in }%
\mathbb{Z}\text{).}%
\]
Nonzero integers map to invertible elements of $F$ under $n\mapsto n\cdot
1_{F}\colon\mathbb{Z}{}\rightarrow F$, and so this map extends to a
homomorphism
\[
\myfrac[-2pt]{m}{n}\mapsto(m\cdot1_{F})(n\cdot1_{F})^{-1}\colon\mathbb{Q}%
\hookrightarrow F.
\]
In this case, $F$ contains a copy of $\mathbb{Q}$, and we say that it has
\emph{characteristic zero}.%
\index{characteristic!zero}%


\textsc{Case 2:}\emph{\/} The kernel of the map is $\neq(0)$, so that
$n\cdot1_{F}=0$ for some $n\neq0$. The smallest positive such $n$ will be a
prime $p$ (otherwise there will be two nonzero elements in $F$ whose product
is zero), and $p$ generates the kernel. Thus, the map $n\mapsto n\cdot
1_{F}\colon\mathbb{Z}{}\rightarrow F$ defines an isomorphism from
$\mathbb{Z}/p\mathbb{Z}{}$ onto the subring
\[
\{m\cdot1_{F}\mid m\in\mathbb{Z}\}
\]
of $F$. In this case, $F$ contains a copy of $\mathbb{F}_{p}$, and we say that
it has \emph{characteristic} $p$.%
\index{characteristic!p}%


A field isomorphic to one of the fields $\mathbb{F}_{2},\mathbb{F}{}%
_{3},\mathbb{F}{}_{5},\ldots,\mathbb{Q}$ is called a \emph{prime field}.%
\index{field!prime}
Every field contains exactly one prime field (as a subfield).

\begin{plain}
\label{ef3}More generally, a commutative ring $R$ is said to have
\emph{characteristic} $p$ (resp.~$0$) if it contains a prime field (as a
subring) of characteristic $p$ (resp.~$0$).\footnote{A commutative ring has a
characteristic if and only if it contains a field as a subring. For example,
neither $\mathbb{Z}{}$ nor $\mathbb{F}{}_{2}\times\mathbb{F}{}_{3}$ has a
characteristic.} Then the prime field is unique and, by definition, contains
$1_{R}$. Thus, if $R$ has characteristic $p\neq0$, then $1_{R}+\cdots+1_{R}=0$
($p$ terms).

Let $R$ be a nonzero commutative ring. If $R$ has characteristic $p\neq0$,
then%
\[
pa\overset{\df}{=}\underbrace{a+\cdots+a}_{p\text{ terms}%
}=\underbrace{(1_{R}+\cdots+1_{R})}_{p\text{ terms}}a=0a=0
\]
for all $a\in R$. Conversely, if $pa=0$ for all $a\in R$, then $R$ has
characteristic $p$.

Let $R$ be a nonzero commutative ring. The usual proof by induction shows that
the binomial theorem%
\index{theorem!binomial in characteristic $p$}
\[
(a+b)^{m}=a^{m}+\tbinom{m}{1}a^{m-1}b+\tbinom{m}{2}a^{m-2}b^{2}+\cdots+b^{m}%
\]
holds in $R$. If $p$ is prime, then it divides%
\[%
\begin{pmatrix}
p\\
r
\end{pmatrix}
\overset{\df}{=}\frac{p!}{r!(p-r)!}%
\]
for all $r$ with $1\leq r\leq p-1$ because it divides the numerator but not
the denominator. Therefore, when $R$ has characteristic $p$,
\[
(a+b)^{p}=a^{p}+b^{p}\quad\text{for all }a,b\in R,
\]
and so the map $a\mapsto a^{p}\colon R\rightarrow R$ is a homomorphism of
rings (even of $\mathbb{F}{}_{p}$-algebras). It is called the \emph{Frobenius
endomorphism\/}%
\index{Frobenius endomorphism}
of $R$. The map $a\mapsto a^{p^{n}}\colon R\rightarrow R$, $n\geq1$, is the
composite of $n$ copies of the Frobenius endomorphism, and so it also is a
homomorphism. Therefore,%
\[
(a_{1}+\cdots+a_{m})^{p^{n}}=a_{1}^{p^{n}}+\cdots+a_{m}^{p^{n}}%
\]
for all $a_{i}\in R$.

When $F$ is a field, the Frobenius endomorphism is injective, and hence an
automorphism if $F$ is finite.
\end{plain}

The
\index{characteristic exponent}%
\emph{characteristic exponent} of a field $F$ is $1$ if $F$ has characteristic
$0$, and $p$ if $F$ has characteristic $p\neq0$. Thus, if $q$ is the
characteristic exponent of $F$ and $n\geq1$, then $x\mapsto x^{q^{n}}$ is an
isomorphism of $F$ onto a subfield of $F$ (denoted
\index{Fq@$F^{q}$}%
$F^{q^{n}}$).

\section{Review of polynomial rings}

Let $F$ be a field.

\begin{plain}
\label{ef3a}The ring $F[X]$ of polynomials in the symbol (or \textquotedblleft
indeterminate\textquotedblright\ or \textquotedblleft
variable\textquotedblright) $X$ with coefficients in $F$ is an $F$-vector
space with basis $1$, $X$, \ldots\ , $X^{n}$, \ldots\ , and with the
multiplication
\[
\left(  \sum\nolimits_{i}a_{i}X^{i}\right)  \left(  \sum\nolimits_{j}%
b_{j}X^{j}\right)  =\sum\nolimits_{k}\left(  \sum\nolimits_{i+j=k}a_{i}%
b_{j}\right)  X^{k}.
\]
The $F$-algebra $F[X]$ has the following universal property: for any
$F$-algebra $R$ and element $r$ of $R$, there is a unique homomorphism of
$F$-algebras $\alpha\colon F[X]\rightarrow R$ such that $\alpha(X)=r$.
\end{plain}

\begin{plain}
\label{ef3b}\emph{Division algorithm\/}:%
\index{algorithm!division}
given $f(X)$, $g(X)\in F[X]$ with $g\neq0$, there exist $q(X)$, $r(X)\in F[X]$
with $r=0$ or $\deg(r)<\deg(g)$ such that
\[
f=gq+r;
\]
moreover, $q(X)$ and $r(X)$ are uniquely determined. Thus $F[X]$ is a
Euclidean domain with $\deg$ as norm, and so it is a unique factorization domain.
\end{plain}

\begin{plain}
\label{ef3c}Let $f\in F[X]$ be nonconstant, and let $a\in F$. The division
algorithm shows that%
\[
f=(X-a)q+c
\]
with $q\in$ $F[X]$ and $c\in F$. Therefore, if $a$ is a
\index{root!of a polynomial}%
root of $f$ (that is, $f(a)=0$), then $X-a$ divides $f$. From unique
factorization, it now follows that $f$ has at most $\deg(f)$ roots (see also
Exercise \ref{x3}).
\end{plain}

\begin{plain}
\label{ef3d}\emph{Euclid's algorithm}:%
\index{algorithm!Euclid's}
Let $f(X)$, $g(X)\in F[X]$. Euclid's algorithm constructs polynomials $a(X)$,
$b(X)$, and $d(X)$ such that
\[
a(X)\cdot f(X)+b(X)\cdot g(X)=d(X),\quad\deg(a)<\deg(g),\quad\deg(b)<\deg(f)
\]
and $d(X)=\gcd(f,g)$.

Recall how it goes. We may assume that $\deg(f)\geq\deg(g)$ since the argument
is the same in the opposite case. Using the division algorithm, we construct a
sequence of quotients and remainders
\begin{align*}
f  &  =q_{0}g+r_{0}\\
g  &  =q_{1}r_{0}+r_{1}\\
r_{0}  &  =q_{2}r_{1}+r_{2}\\
&  \cdots\\
r_{n-2}  &  =q_{n}r_{n-1}+r_{n}\\
r_{n-1}  &  =q_{n+1}r_{n}%
\end{align*}
with $r_{n}$ the last nonzero remainder. Then, $r_{n}$ divides $r_{n-1}$,
hence $r_{n-2}$,\ldots, hence $g$, and hence $f$. Moreover,
\[
r_{n}=r_{n-2}-q_{n}r_{n-1}=r_{n-2}-q_{n}(r_{n-3}-q_{n-1}r_{n-2})=\cdots=af+bg
\]
and so every common divisor of $f$ and $g$ divides $r_{n}$: we have shown
$r_{n}=\gcd(f,g)$.

Let $af+bg=d$. If $\deg(a)\geq\deg(g)$, write $a=gq+r$ with $\deg(r)<\deg(g)$.
Then
\[
rf+(b+qf)g=d,
\]
and $b+qf$ has degree $<\deg(f)$ because $(b+qf)g=d-rf$, which has degree
$<\deg(g)+\deg(f)$.

PARI%
\index{PARI}
knows how to do Euclidean division: typing \verb|divrem(13,5)| in PARI returns
$[2,3]$, meaning that $13=2\times5+3$, and \verb|gcd(m,n)| returns the
greatest common divisor of $m$ and $n$.
\end{plain}

\begin{plain}
\label{ef3e}Let $I$ be a nonzero ideal in $F[X]$, and let $f$ be a nonzero
polynomial of least degree in $I$; then $I=(f)$ (because $F[X]$ is a Euclidean
domain). When we choose $f$ to be \emph{monic}%
\index{polynomial!monic}%
, i.e., to have leading coefficient one, it is uniquely determined by $I$.
Thus, there is a one-to-one correspondence between the nonzero ideals of
$F[X]$ and the monic polynomials in $F[X]$. The prime ideals correspond to the
irreducible monic polynomials.
\end{plain}

\begin{plain}
\label{ef3f}As $F[X]$ is an integral domain, we can form its field of
fractions $F(X)$. Its elements are quotients $f/g$, $f$ and $g$ polynomials,
$g\neq0.$
\end{plain}

\section{Factoring polynomials}

The following results help in deciding whether a polynomial is reducible, and
in finding its factors.

\begin{proposition}
\label{ef4}Let $r\in\mathbb{Q}{}$ be a root of a polynomial
\[
a_{m}X^{m}+a_{m-1}X^{m-1}+\cdots+a_{0},\quad a_{i}\in\mathbb{Z},
\]
and write $r=c/d$, $c,d\in\mathbb{Z}$, $\gcd(c,d)=1$. Then $c|a_{0}$ and
$d|a_{m}.$
\end{proposition}

\begin{proof}
It is clear from the equation
\[
a_{m}c^{m}+a_{m-1}c^{m-1}d+\cdots+a_{0}d^{m}=0
\]
that $d|a_{m}c^{m}$, and therefore, $d|a_{m}.$ Similarly, $c|a_{0}$.
\end{proof}

\begin{example}
\label{ef5}The polynomial $f(X)=X^{3}-3X-1$ is irreducible in $\mathbb{Q}[X]$
because its only possible roots are $\pm1$, and $f(1)\neq0\neq f(-1)$.
\end{example}

\begin{proposition}
[Gauss's Lemma]\label{ef6}%
\index{Lemma!Gauss's}%
Let $f(X)\in\mathbb{Z}[X]$. If $f(X)$ factors nontrivially in $\mathbb{Q}[X]$,
then it factors nontrivially in $\mathbb{Z}{}[X]$.
\end{proposition}

\begin{proof}
Let $f=gh$ in $\mathbb{Q}{}[X]$ with $g,h$ nonconstant${}$. For suitable
integers $m$ and $n$, $g_{1}\overset{\df}{=}mg$ and
$h_{1}\overset{\df}{=}nh$ have coefficients in $\mathbb{Z}{}%
$, and so we have a factorization%
\[
mnf=g_{1}\cdot h_{1}\text{ in }\mathbb{Z}{}[X]\text{.}%
\]
If a prime $p$ divides $mn$, then, looking modulo $p$, we obtain an equation%
\[
0=\overline{g_{1}}\cdot\overline{h_{1}}\text{ in }\mathbb{F}{}_{p}[X]\text{.}%
\]
Since $\mathbb{F}_{p}[X]$ is an integral domain, this implies that $p$ divides
all the coefficients of at least one of the polynomials $g_{1},h_{1}$, say
$g_{1}$, so that $g_{1}=pg_{2}$ for some $g_{2}\in\mathbb{Z}[X]$. Thus, we
have a factorization%
\[
(mn/p)f=g_{2}\cdot h_{1}\text{ in }\mathbb{Z}{}[X]\text{.}%
\]
Continuing in this fashion, we eventually remove all the prime factors of
$mn$, and so obtain a nontrivial factorization of $f$ in $\mathbb{Z}{}[X]$.
\end{proof}

\begin{proposition}
\label{ef6m}If $f\in\mathbb{Z}[X]$ is monic, then every monic factor of $f$ in
$\mathbb{Q}[X]$ lies in $\mathbb{Z}[X]$.
\end{proposition}

\begin{proof}
Let $g$ be a monic factor of $f$ in $\mathbb{Q}{}[X]$, so that $f=gh$ with
$h\in\mathbb{Q}{}[X]$ also monic. Let $m,n$ be the positive integers with the
fewest prime factors such that $mg,nh\in\mathbb{Z}{}[X]$. As in the proof of
Gauss's Lemma, if a prime $p$ divides $mn$, then it divides all the
coefficients of at least one of the polynomials $mg,nh$, say $mg$, in which
case it divides $m$ because $g$ is monic. Now $\frac{m}{p}g\in\mathbb{Z}[X]$,
which contradicts the definition of $m$.
\end{proof}

\begin{aside}
\label{ef6n}We sketch an alternative proof of Proposition \ref{ef6m}. A
complex number $\alpha$ is said to be an \emph{algebraic integer}%
\index{algebraic integer}
if it is a root of a monic polynomial in $\mathbb{Z}{}[X]$. Proposition
\ref{ef4} shows that every algebraic integer in $\mathbb{Q}{}$ lies in
$\mathbb{Z}$. The algebraic integers form a subring of $\mathbb{C}{}$ --- see
Theorem 6.5 of my notes on Commutative Algebra. Now let $\alpha_{1}%
,\ldots,\alpha_{m}$ be the roots of $f$ in $\mathbb{C}{}$. By definition, they
are algebraic integers, and the coefficients of any monic factor of $f$ are
polynomials in (certain of) the $\alpha_{i}$, and therefore are algebraic
integers. If they lie in $\mathbb{Q}{}$, then they lie in $\mathbb{Z}{}$.
\end{aside}

\begin{proposition}
[Eisenstein's criterion]\label{ef7}%
\index{Eisenstein's criterion}%
Let
\[
f=a_{m}X^{m}+a_{m-1}X^{m-1}+\cdots+a_{0},\quad a_{i}\in\mathbb{Z};
\]
suppose that there is a prime $p$ such that:

\begin{itemize}
\item $p$ does not divide $a_{m}$,

\item $p$ divides $a_{m-1},...,a_{0}$,

\item $p^{2}$ does not divide $a_{0}$.
\end{itemize}

\noindent Then $f$ is irreducible in $\mathbb{Q}[X]$.
\end{proposition}

\begin{proof}
If $f(X)$ factors nontrivially in $\mathbb{Q}[X]$, then it factors
nontrivially in $\mathbb{Z}[X]$, say,
\[
a_{m}X^{m}+a_{m-1}X^{m-1}+\cdots+a_{0}=(b_{r}X^{r}+\cdots+b_{0})(c_{s}%
X^{s}+\cdots+c_{0})
\]
with $b_{i},c_{i}\in\mathbb{Z}$ and $r,s<m$. Since $p$, but not $p^{2}$,
divides $a_{0}=b_{0}c_{0}$, $p$ must divide exactly one of $b_{0}$, $c_{0}$,
say, $b_{0}$. Now from the equation
\[
a_{1}=b_{0}c_{1}+b_{1}c_{0},
\]
we see that $p|b_{1},$ and from the equation
\[
a_{2}=b_{0}c_{2}+b_{1}c_{1}+b_{2}c_{0},
\]
that $p|b_{2}$. By continuing in this way, we find that $p$ divides
$b_{0},b_{1},\ldots,b_{r}$, which contradicts the condition that $p$ does not
divide $a_{m}$.
\end{proof}

The last three propositions hold \textit{mutatis mutandis} with $\mathbb{Z}$
replaced by a unique factorization domain $R$ (replace $\mathbb{Q}{}$ with the
field of fractions of $R$ and $p$ with a prime element of $R$).

\begin{remark}
\label{ef8}There is an algorithm%
\index{algorithm!factoring a polynomial}
for factoring a polynomial in $\mathbb{Q}[X]$. To see this, consider
$f\in\mathbb{Q}[X]$. Multiply $f(X)$ by a rational number so that it is monic,
and then replace it by $D^{\deg(f)}f(\frac{X}{D})$, with $D$ equal to a common
denominator for the coefficients of $f$, to obtain a monic polynomial with
integer coefficients. Thus we need consider only polynomials
\[
f(X)=X^{m}+a_{1}X^{m-1}+\cdots+a_{m},\quad a_{i}\in\mathbb{Z}.
\]


From the fundamental theorem of algebra%
\index{fundamental theorem!of algebra}
(see \ref{ag5} below), we know that $f$ splits completely in $\mathbb{C}[X]$:
\[
f(X)=\prod_{i=1}^{m}(X-\alpha_{i}),\quad\alpha_{i}\in\mathbb{C}.
\]
From the equation
\[
0=f(\alpha_{i})=\alpha_{i}^{m}+a_{1}\alpha_{i}^{m-1}+\cdots+a_{m}\text{,}%
\]
it follows that $|\alpha_{i}|$ is less than some bound depending only on the
degree and coefficients of $f$; in fact,
\[
|\alpha_{i}|\leq\max\{1,mB\}\text{, }B=\max|a_{i}|\text{.}%
\]
Now if $g(X)$ is a monic factor of $f(X)$, then its roots in $\mathbb{C}$ are
certain of the $\alpha_{i}$, and its coefficients are symmetric polynomials in
its roots (see p.\thinspace\pageref{sympol}). Therefore, the absolute values
of the coefficients of $g(X)$ are bounded in terms of the degree and
coefficients of $f$. Since they are also integers (by \ref{ef6m}), we see that
there are only finitely many possibilities for $g(X)$. Thus, to find the
factors of $f(X)$ we (better PARI%
\index{PARI}%
) have to do only a finite amount of checking.\footnote{Of course, there are
much faster methods than this. The Berlekamp--Zassenhaus algorithm factors the
polynomial over certain suitable finite fields $\mathbb{F}{}_{p}$, lifts the
factorizations to rings $\mathbb{Z}{}/p^{m}\mathbb{Z}$ for some $m$, and then
searches for factorizations in $\mathbb{Z}{}[X]$ with the correct form modulo
$p^{m}$.}

Therefore, we need not concern ourselves with the problem of factoring
polynomials in the rings $\mathbb{Q}[X]$ or $\mathbb{F}_{p}[X]$ since PARI%
\index{PARI}
knows how to do it. For example, typing \verb|content(6*X^2+18*X-24)| in PARI
returns 6, and \verb|factor(6*X^2+18*X-24)| returns $X-1$ and $X+4$, showing
that
\[
6X^{2}+18X-24=6(X-1)(X+4)
\]
in $\mathbb{Q}[X]$. Typing \verb|factormod(X^2+3*X+3,7)| returns $X+4$ and
$X+6$, showing that
\[
X^{2}+3X+3=(X+4)(X+6)
\]
in $\mathbb{F}_{7}[X]$.
\end{remark}

\begin{remark}
\label{ef8m}One other observation is useful. Let $f\in\mathbb{Z}[X]$. If the
leading coefficient of $f$ is not divisible by a prime $p$, then a nontrivial
factorization $f=gh$ in $\mathbb{Z}{}[X]$ will give a nontrivial factorization
$\bar{f}=\bar{g}\bar{h}$ in $\mathbb{F}{}_{p}[X]$. Thus, if $f(X)$ is
irreducible in $\mathbb{F}_{p}[X]$ for some prime $p$ not dividing its leading
coefficient, then it is irreducible in $\mathbb{Z}{}[X]$. This test is very
useful, but it is not always effective: for example, $X^{4}-10X^{2}+1$ is
irreducible in $\mathbb{Z}{}[X]$ but it is reducible\footnote{Here is a proof
using only that the product of two nonsquares in $\mathbb{F}{}_{p}^{\times}$
is a square, which follows from the fact that $\mathbb{F}{}_{p}^{\times}$ is
cyclic (see Exercise \ref{x3}). If $2$ is a square in $\mathbb{F}{}_{p}$, then%
\[
X^{4}-10X^{2}+1=(X^{2}-2\sqrt{2}X-1)(X^{2}+2\sqrt{2}X-1).
\]
If $3$ is a square in $\mathbb{F}{}_{p}$, then%
\[
X^{4}-10X^{2}+1=(X^{2}-2\sqrt{3}X+1)(X^{2}+2\sqrt{3}X+1).
\]
If neither $2$ nor $3$ are squares, $6$ will be a square in $\mathbb{F}{}_{p}%
$, and
\par%
\[
X^{4}-10X^{2}+1=(X^{2}-(5+2\sqrt{6}))(X^{2}-(5-2\sqrt{6})).
\]
The general study of such polynomials requires nonelementary methods. See, for
example, the paper Brandl, AMM, \textbf{93} (1986), pp.\ 286--288, which
proves that for every composite integer $n\geq1$, there exists a polynomial in
$\mathbb{Z}{}[X]$ of degree $n$ that is irreducible over $\mathbb{Z}$ but
reducible modulo all primes$.$} modulo every prime $p$.
\end{remark}

\section{Extensions}

Let $F$ be a field. A field containing $F$ is called an
\index{extension}%
\emph{extension} of $F$.\footnote{This is the usual definition of
``extension'' (Wikipedia: field extension), but ``overfield'' would be a
better term because Bourbaki, for example, uses ``extension'' to mean a field
$E$ together with a homomorphism from $F$ to $E$.} In other words, an
extension is an $F$-algebra whose underlying ring is a field. An extension $E$
of $F$ is, in particular, an $F$-vector space, whose dimension is called the
\emph{degree\/}%
\index{degree}
of $E$ over $F$. It is denoted by $[E\colon F]$. An extension is said to
\emph{finite }%
\index{extension!finite}
if its degree is finite, and quadratic, cubic, etc.\ if it is of degree $2$,
$3$, etc.

When $E$ and $E^{\prime}$ are extensions of $F$, an $F$-\emph{homomorphism}%
\index{Fhomomorphism@$F$-homomorphism}
$E\rightarrow E^{\prime}$ is a homomorphism $\varphi\colon E\rightarrow
E^{\prime}$ such that $\varphi(c)=c$ for all $c\in F$.

\begin{example}
\label{ef9}(a) The field of complex numbers $\mathbb{C}$ has degree $2$ over
$\mathbb{R}$ (basis $\{1,i\}).$

(b) The field of real numbers $\mathbb{R}$ has infinite degree over
$\mathbb{Q}$: the field $\mathbb{Q}{}$ is countable, and so every
finite-dimensional $\mathbb{Q}{}$-vector space is also countable, but a famous
argument of Cantor shows that $\mathbb{R}{}$ is not countable.

(c) The field of \emph{Gaussian numbers}%
\index{Gaussian numbers}
\[
\mathbb{Q}(i)\overset{\df}{=}\{a+bi\in\mathbb{C}\mid
a,b\in\mathbb{Q}\}
\]
has degree $2$ over $\mathbb{Q}$ (basis $\{1,i\}$).

(d) The field $F(X)$ has infinite degree over $F$; in fact, even its subspace
$F[X]$ has infinite dimension over $F$ (basis $1,X,X^{2},\ldots$).
\end{example}

\begin{proposition}
[multiplicativity of degrees]\label{ef10} Consider fields $L\supset E\supset
F$. Then $L/F$ is of finite degree if and only if $L/E$ and $E/F$ are both of
finite degree, in which case
\[
\lbrack L\colon F]=[L\colon E][E\colon F].
\]

\end{proposition}

\begin{proof}
If $L$ is finite over $F$, then it is certainly finite over $E$; moreover,
$E$, being a subspace of a finite-dimensional $F$-vector space, is also finite-dimensional.

Thus, assume that $L/E$ and $E/F$ are of finite degree, and let $(e_{i}%
)_{1\leq i\leq m}$ be a basis for $E$ as an $F$-vector space and let
$(l_{j})_{1\leq j\leq n}$ be a basis for $L$ as an $E$-vector space. To
complete the proof of the proposition, it suffices to show that $(e_{i}%
l_{j})_{1\leq i\leq m,1\leq j\leq n}$ is a basis for $L$ over $F$, because
then $L$ will be finite over $F$ of the predicted degree.

First, $(e_{i}l_{j})_{i,j}$ spans $L$. Let $\gamma\in L$. Then, because
$(l_{j})_{j}$ spans $L$ as an $E$-vector space,
\[
\gamma=%
%TCIMACRO{\tsum \nolimits_{j}}%
%BeginExpansion
{\textstyle\sum\nolimits_{j}}
%EndExpansion
\alpha_{j}l_{j},\qquad\text{some }\alpha_{j}\in E,
\]
and because $(e_{i})_{i}$ spans $E$ as an $F$-vector space,
\[
\alpha_{j}=%
%TCIMACRO{\tsum \nolimits_{i}}%
%BeginExpansion
{\textstyle\sum\nolimits_{i}}
%EndExpansion
a_{ij}e_{i},\qquad\text{some $a_{ij}\in F$}.
\]
On putting these together, we find that
\[
\gamma=%
%TCIMACRO{\tsum \nolimits_{i,j}}%
%BeginExpansion
{\textstyle\sum\nolimits_{i,j}}
%EndExpansion
a_{ij}e_{i}l_{j}.
\]


Second, $(e_{i}l_{j})_{i,j}$ is linearly independent. A linear relation $\sum
a_{ij}e_{i}l_{j}=0$, $a_{ij}\in F$, can be rewritten $\sum_{j}(\sum_{i}%
a_{ij}e_{i})l_{j}=0$. The linear independence of the $l_{j}$'s now shows that
$\sum_{i}a_{ij}e_{i}=0$ for each $j$, and the linear independence of the
$e_{i}$'s shows that each $a_{ij}=0$.
\end{proof}

\section{The subring generated by a subset}%

\index{subring!generated by subset}%


An intersection of subrings of a ring is again a ring (this is easy to prove).
Let $F$ be a subfield of a field $E$, and let $S$ be a subset of $E$. The
intersection of all the subrings of $E$ containing $F$ and $S$ is obviously
the smallest subring of $E$ containing both $F$ and $S$. We call it the
subring of $E$ \emph{generated by} $F$ \emph{and} $S$ (or \emph{generated
over} $F$ \emph{by} $S$), and we denote it by $F[S]$. When $S=\{\alpha
_{1},...,\alpha_{n}\}$, we write $F[\alpha_{1},...,\alpha_{n}]$ for $F[S]$.
For example, $\mathbb{C}{}=\mathbb{R}{}[\sqrt{-1}]$.

\begin{lemma}
\label{ef13}The ring $F[S]$ consists of the elements of $E$ that can be
expressed as finite sums of the form
\begin{equation}
\sum a_{i_{1}\cdots i_{n}}\alpha_{1}^{i_{1}}\cdots\alpha_{n}^{i_{n}},\quad
a_{i_{1}\cdots i_{n}}\in F,\quad\alpha_{i}\in S,\quad i_{j}\in\mathbb{N}{}.
\label{eq7}%
\end{equation}

\end{lemma}

\begin{proof}
Let $R$ be the set of all such elements. Obviously, $R$ is a subring of $E$
containing $F$ and $S$ and contained in every other such subring. Therefore it
equals $F[S]$.
\end{proof}

\begin{example}
\label{ef15}The ring $\mathbb{Q}[\pi]$, $\pi=3.14159...$, consists of the real
numbers that can be expressed as a finite sum
\[
a_{0}+a_{1}\pi+a_{2}\pi^{2}+\cdots+a_{n}\pi^{n},\quad a_{i}\in\mathbb{Q}.
\]
The ring $\mathbb{Q}[i]$ consists of the complex numbers of the form $a+bi$,
$a,b\in\mathbb{Q}$.
\end{example}

Note that the expression of an element in the form (\ref{eq7}) will
\textit{not}\emph{\/} be unique in general. This is so already in
$\mathbb{R}{}[i]$.

\begin{lemma}
\label{ef14}Let $R$ be an integral domain containing a subfield $F$ (as a
subring). If $R$ is finite-dimensional when regarded as an $F$-vector space,
then it is a field.
\end{lemma}

\begin{proof}
Let $\alpha$ be a nonzero element of $R$ --- we have to show that $\alpha$ has
an inverse in $R$. The map $x\mapsto\alpha x\colon R\rightarrow R$ is an
injective linear map of finite-dimensional $F$-vector spaces, and is therefore
surjective. In particular, there is an element $\beta\in R$ such that
$\alpha\beta=1$.
\end{proof}

Note that the lemma applies to every subring containing $F$ of a finite
extension of $F$.

\section{The subfield generated by a subset}%

\index{subfield!generated by subset}%


An intersection of subfields of a field is again a field. Let $F$ be a
subfield of a field $E$, and let $S$ be a subset of $E$. The intersection of
all the subfields of $E$ containing $F$ and $S$ is obviously the smallest
subfield of $E$ containing both $F$ and $S$. We call it the subfield of $E$
\emph{generated by} $F$ \emph{and} $S$ (or \emph{generated over} $F$ \emph{by}
$S$), and we denote it $F(S)$. It is the field of fractions of $F[S]$ in $E$
because this is a subfield of $E$ containing $F$ and $S$ and contained in
every other such field. When $S=\{\alpha_{1},...,\alpha_{n}\}$, we write
$F(\alpha_{1},...,\alpha_{n})$ for $F(S)$. Thus, $F[\alpha_{1},\ldots
,\alpha_{n}]$ consists of all elements of $E$ that can be expressed as
polynomials in the $\alpha_{i}$ with coefficients in $F$, and $F(\alpha
_{1},\ldots,\alpha_{n})$ consists of all elements of $E$ that can be expressed
as a quotient of two such polynomials.

Lemma \ref{ef14} shows that $F[S]$ is already a field if it is
finite-dimensional over $F$, in which case $F(S)=F[S]$.

\begin{example}
\label{ef16}(a) The field $\mathbb{Q}(\pi)$, $\pi=3.14\ldots$, consists of the
complex numbers that can be expressed as a quotient
\[
g(\pi)/h(\pi),\quad g(X),h(X)\in\mathbb{Q}[X],\quad h(X)\neq0.
\]


(b) The ring $\mathbb{Q}[i]$ is already a field.
\end{example}

An extension $E$ of $F$ is said to be \emph{simple\/}%
\index{extension!simple}
if $E=F(\alpha)$ some $\alpha\in E$. For example, $\mathbb{Q}(\pi)$ and
$\mathbb{Q}[i]$ are simple extensions of $\mathbb{Q}.$

Let $F$ and $F^{\prime}$ be subfields of a field $E$. The intersection of the
subfields of $E$ containing both $F$ and $F^{\prime}$ is obviously the
smallest subfield of $E$ containing both $F$ and $F^{\prime}$. We call it the
\emph{composite }%
\index{composite of fields}%
of $F$ and $F^{\prime}$ in $E$, and we denote it by $F\cdot F^{\prime}$. It
can also be described as the subfield of $E$ generated over $F$ by $F^{\prime
}$, or the subfield generated over $F^{\prime}$ by $F$:%
\[
F(F^{\prime})=F\cdot F^{\prime}=F^{\prime}(F)\text{.}%
\]


\section{Construction of some extensions}

Let $f(X)\in F[X]$ be a monic polynomial of degree $m$, and let $(f)$ be the
ideal generated by $f$. Consider the quotient ring $F[X]/(f(X))$, and write
$x$ for the image of $X$ in $F[X]/(f(X))$, i.e., $x$ is the coset $X+(f(X))$.

(a) The map
\[
P(X)\mapsto P(x)\colon F[X]\rightarrow F[x]
\]
is a homomorphism sending $f(X)$ to $0$. Therefore, $f(x)=0$.

(b) The division algorithm shows that every element $g$ of $F[X]/(f)$ is
represented by a unique polynomial $r$ of degree $<m$. Hence each element of
$F[x]$ can be expressed uniquely as a sum
\begin{equation}
a_{0}+a_{1}x+\cdots+a_{m-1}x^{m-1},\qquad a_{i}\in F.\hfill\label{eq8}%
\end{equation}


(c) To add two elements, expressed in the form (\ref{eq8}), simply add the
corresponding coefficients.

(d) To multiply two elements expressed in the form (\ref{eq8}), multiply in
the usual way, and use the relation $f(x)=0$ to express the monomials of
degree $\geq m$ in $x$ in terms of lower degree monomials.

(e) \textit{Now assume that} $f(X)$ \textit{is irreducible}. Then every
nonzero $\alpha\in F[x]$ has an inverse, which can be found as follows. Use
(b) to write $\alpha=g(x)$ with $g(X)$ a polynomial of degree $\leq m-1$, and
apply Euclid's algorithm in $F[X]$ to find polynomials $a(X)$ and $b(X)$ such
that
\[
a(X)f(X)+b(X)g(X)=d(X)
\]
with $d(X)$ the gcd of $f$ and $g$. In our case, $d(X)$ is $1$ because $f(X)$
is irreducible and $\deg g(X)<\deg f(X)$. When we replace $X$ with $x$, the
equality becomes
\[
b(x)g(x)=1.
\]
Hence $b(x)$ is the inverse of $g(x)$.

We have proved the following statement.

\begin{E}
\label{ef10m}For a monic irreducible polynomial $f(X)$ of degree $m$ in
$F[X]$,
\[
F[x]\overset{\df}{=}F[X]/(f(X))
\]
is a field of degree $m$ over $F$. Computations in $F[x]$ come down to
computations in $F$.
\end{E}

Note that, because $F[x]$ is a field, $F(x)=F[x]$.\footnote{Thus, we can
denote it by $F(x)$ or by $F[x]$. The former is more common, but I use $F[x]$
to emphasize the fact that its elements are polynomials in $x$.}

\begin{example}
\label{ef11}Let $f(X)=X^{2}+1\in\mathbb{R}[X]$. Then $\mathbb{R}[x]$ has

elements: $a+bx$, $a,b\in\mathbb{R};$

addition: $(a+bx)+(a^{\prime}+b^{\prime}x)=(a+a^{\prime})+(b+b^{\prime})x;$

multiplication: $(a+bx)(a^{\prime}+b^{\prime}x)=(aa^{\prime}-bb^{\prime
})+(ab^{\prime}+a^{\prime}b)x;$

inverses: in this case, it is possible write down the inverse of $a+bx$ directly.

\noindent We usually write $i$ for $x$ and $\mathbb{C}$ for $\mathbb{R}[x].$
\end{example}

\begin{example}
\label{ef12}Let $f(X)=X^{3}-3X-1\in\mathbb{Q}[X]$. We observed in (\ref{ef5})
that this is irreducible over $\mathbb{Q}$, and so $\mathbb{Q}{}[x]$ is a
field. It has basis $\{1,x,x^{2}\}$ as a $\mathbb{Q}$-vector space. Let
\[
\beta=x^{4}+2x^{3}+3\in\mathbb{Q}[x].
\]
Then using that $x^{3}-3x-1=0$, we find that $\beta=3x^{2}+7x+5$. Because
$X^{3}-3X-1$ is irreducible,
\[
\gcd(X^{3}-3X-1,3X^{2}+7X+5)=1.
\]
In fact, Euclid's algorithm gives
\[
\textstyle(X^{3}-3X-1)\left(  \frac{-7}{37}X+\frac{29}{111}\right)
+(3X^{2}+7X+5)\left(  \frac{7}{111}X^{2}-\frac{26}{111}X+\frac{28}%
{111}\right)  =1.
\]
Hence
\[
\textstyle(3x^{2}+7x+5)\left(  \frac{7}{111}x^{2}-\frac{26}{111}x+\frac
{28}{111}\right)  =1,
\]
and we have found the inverse of $\beta.$

We can also do this in PARI%
\index{PARI}%
: \verb|b=Mod(X^4+2*X^3+3,X^3-3*X-1)| reveals that $\beta=3x^{2}+7x+5$ in
$\mathbb{Q}[x]$, and \verb|b^(-1)| reveals that $\beta^{-1}=\frac{7}{111}%
x^{2}-\frac{26}{111}x+\frac{28}{111}$.
\end{example}

\section{Stem fields}

\label{sf}

Let $f$ be a monic irreducible polynomial in $F[X]$. A pair $(E,\alpha)$
consisting of an extension $E$ of $F$ and an $\alpha\in E$ is
called\footnote{Following A.A.\thinspace Albert (Modern Higher Algebra, 1937)
who calls the splitting field of a polynomial its root field.} a%
\index{field!stem}
\emph{stem field for} $f$ if $E=F[\alpha]$ and $f(\alpha)=0$. For example, the
pair $(E,\alpha)$ with $E=F[X]/(f)=F[x]$ and $\alpha=x$ is a stem field for
$f$. Let $(E,\alpha)$ be a stem field, and consider the surjective
homomorphism of $F$-algebras%
\[
g(X)\mapsto g(\alpha)\colon F[X]\rightarrow E\text{.}%
\]
Its kernel is generated by a nonzero monic polynomial, which divides $f$, and
so must equal it. Therefore the homomorphism defines an $F$-isomorphism%
\[
x\mapsto\alpha\colon F[x]\rightarrow E,\quad\text{where }F[x]=F[X]/(f)\text{.}%
\]
In other words, the stem field $(E,\alpha)$ of $f$ is $F$-isomorphic to the
standard\ stem field $(F[X]/(f),x)$. It follows that every element of a stem
field $(E,\alpha)$ for $f$ can be written uniquely in the form%
\[
a_{0}+a_{1}\alpha+\cdots+a_{m-1}\alpha^{m-1},\quad a_{i}\in F,\quad
m=\deg(f)\text{,}%
\]
and that arithmetic in $F[\alpha]$ can be performed using the same rules as in
$F[x]$. If $(E^{\prime},\alpha^{\prime})$ is a second stem field for $f$, then
there is a unique $F$-isomorphism $E\rightarrow E^{\prime}$ sending $\alpha$
to $\alpha^{\prime}$. We sometimes abbreviate \textquotedblleft stem field
$(F[\alpha],\alpha)$\textquotedblright\ to \textquotedblleft stem field
$F[\alpha]$\textquotedblright.

\section{Algebraic and transcendental elements}

Let $F$ be a field. An element $\alpha$ of an extension $E$ of $F$ defines a
homomorphism
\[
f(X)\mapsto f(\alpha)\colon F[X]\rightarrow E.
\]
There are two possibilities.

\textsc{Case 1:\/} The kernel of the map is $(0)$, so that, for $f\in F[X]$,
\[
f(\alpha)=0\implies f=0\text{ (in }F[X]\text{).}%
\]
In this case, we say that $\alpha$ \emph{transcendental over }%
\index{transcendental}%
$F$. The homomorphism $X\mapsto\alpha\colon F[X]\rightarrow F[\alpha]$ is an
isomorphism, and it extends to an isomorphism $F(X)\rightarrow F(\alpha)$ of
the fields of fractions.

\textsc{Case 2:\/} The kernel is $\neq(0)$, so that $g(\alpha)=0$ for some
nonzero $g\in F[X]$. In this case, we say that $\alpha$ is \emph{algebraic
over }$F$.%
\index{algebraic}
The polynomials $g$ such that $g(\alpha)=0$ form a nonzero ideal in $F[X]$,
which is generated by the monic polynomial $f$ of least degree such
$f(\alpha)=0$. We call $f$ the \emph{minimal (}or \emph{minimum) polynomial\/}%
\index{polynomial!minimal}%
\index{polynomial!minimum}%
\label{minimal} of $\alpha$ over $F$.\footnote{When we order the polynomials
by degree, $f$ is a minimal element of the set of polynomials having $\alpha$
as a root. It is also the \textit{unique} minimal (hence least or minimum)
element of the set of \textit{monic} polynomials having $\alpha$ as a root.
See Wikipedia: partially ordered set.} It is irreducible, because otherwise
there would be two nonzero elements of $E$ whose product is zero. The minimal
polynomial is characterized as an element of $F[X]$ by each of the following conditions,

\begin{itemize}
\item $f$ is monic, $f(\alpha)=0$, and $f$ divides every other $g$ in $F[X]$
such that $g(\alpha)=0$;

\item $f$ is the monic polynomial of least degree such that $f(\alpha)=0;$

\item $f$ is monic, irreducible, and $f(\alpha)=0$.
\end{itemize}

\noindent Note that $g(X)\mapsto g(\alpha)$ defines an isomorphism
$F[X]/(f)\rightarrow F[\alpha]$. Since the first is a field\label{field}, so
also is the second,
\[
F(\alpha)=F[\alpha].
\]
Thus, $F[\alpha]$ is a stem field for $f$.

\begin{example}
\label{ef17}Let $\alpha\in\mathbb{C}$ be such that $\alpha^{3}-3\alpha-1=0$.
Then $X^{3}-3X-1$ is monic, irreducible, and has $\alpha$ as a root, and so it
is the minimal polynomial of $\alpha$ over $\mathbb{Q}$. The set
$\{1,\alpha,\alpha^{2}\}$ is a basis for $\mathbb{Q}[\alpha]$ over
$\mathbb{Q}$. The calculations in Example \ref{ef12} show that if $\beta$ is
the element $\alpha^{4}+2\alpha^{3}+3$ of $\mathbb{Q}[\alpha]$, then
$\beta=3\alpha^{2}+7\alpha+5$, and
\[
\textstyle\beta^{-1}=\frac{7}{111}\alpha^{2}-\frac{26}{111}\alpha+\frac
{28}{111}.
\]

\end{example}

\begin{remark}
\label{ef18}PARI%
\index{PARI}
knows how to compute in $\mathbb{Q}[a]$. For example, \verb|factor(X^4+4)|
returns the factorization
\[
X^{4}+4=(X^{2}-2X+2)(X^{2}+2X+2)
\]
in $\mathbb{Q}[X]$. Now type \verb|F=nfinit(a^2+2*a+2)| to define a number
field \textquotedblleft F\textquotedblright\ generated over $\mathbb{Q}$ by a
root $a$ of $X^{2}+2X+2$. Then \verb|nffactor(F,x^4+4)| returns the
factorization
\[
X^{4}+4=(X-a-2)(X-a)(X+a))(X+a+2),
\]
in $\mathbb{Q}[a]$.
\end{remark}

A extension $E$ of $F$ is said to be \emph{algebraic\/}%
\index{extension!algebraic}
(and $E$ is said to be \emph{algebraic over }$F$), if all elements of $E$ are
algebraic over $F$; otherwise it is said to be \emph{transcendental\/}%
\index{extension!transcendental}
(and $E$ is said to be \emph{transcendental over} $F$). Thus, $E/F$ is
transcendental if at least one element of $E$ is transcendental over $F$.

\begin{proposition}
\label{ef19}Let $E\supset F$ be fields. If $E/F$ is finite, then $E$ is
algebraic and finitely generated (as a field) over $F$; conversely, if $E$ is
generated over $F$ by a finite set of algebraic elements, then it is finite
over $F$.
\end{proposition}

\begin{proof}
$\Longrightarrow$: To say that an element $\alpha$ of $E$ is transcendental
over $F$ amounts to saying that its powers $1,\alpha,\alpha^{2},\ldots$ are
linearly independent over $F$. Thus, if $E$ is finite over $F$, then every
element of $E$ is algebraic over $F$. It remains to show that $E$ is finitely
generated over $F$. If $E=F$, then it is generated by the empty set.
Otherwise, there exists an $\alpha_{1}\in E\smallsetminus F$. If $E\neq
F[\alpha_{1}]$, then there exists an $\alpha_{2}\in E\smallsetminus
F[\alpha_{1}]$, and so on. Since
\[
\lbrack F[\alpha_{1}]\colon F]<[F[\alpha_{1},\alpha_{2}]\colon F]<\cdots
<[E\colon F]
\]
this process terminates with $E=F[\alpha_{1},\alpha_{2},\ldots,\alpha_{n}]$.

$\Longleftarrow$: Let $E=F(\alpha_{1},...,\alpha_{n})$ with $\alpha_{1}%
,\alpha_{2},\ldots\alpha_{n}$ algebraic over $F$. The extension $F(\alpha
_{1})/F$ is finite because $\alpha_{1}$ is algebraic over $F$, and the
extension $F(\alpha_{1},\alpha_{2})/F(\alpha_{1})$ is finite because
$\alpha_{2}$ is algebraic over $F$ and hence over $F(\alpha_{1})$. Thus, by
(\ref{ef10}), $F(\alpha_{1},\alpha_{2})$ is finite over $F$. Now repeat the argument.
\end{proof}

\begin{corollary}
\label{ef20}(a) If $E$ is algebraic over $F$, then every subring $R$ of $E$
containing $F$ is a field.

(b) Consider fields $L\supset E\supset F$. If $L$ is algebraic over $E$ and
$E$ is algebraic over $F$, then $L$ is algebraic over $F.$
\end{corollary}

\begin{proof}
(a) If $\alpha\in R$, then $F[\alpha]\subset R$. But $F[\alpha]$ is a field
because $\alpha$ is algebraic (see p.\thinspace\pageref{field}), and so $R$
contains $\alpha^{-1}$.

(b) By assumption, every $\alpha\in L$ is a root of a monic polynomial
\[
X^{m}+a_{m-1}X^{m-1}+\cdots+a_{0}\in E[X].
\]
Each of the extensions
\[
F[a_{0},\ldots,a_{m-1},\alpha]\supset F[a_{0},\ldots,a_{m-1}]\supset
F[a_{0},\ldots,a_{m-2}]\supset\cdots\supset F
\]
is generated by a single algebraic element, and so is finite. Therefore
$F[a_{0},\ldots,a_{m-1},\alpha]$ is finite over $F$ (see \ref{ef10}), which
implies that $\alpha$ is algebraic over $F$.
\end{proof}

\section{Transcendental numbers}

\label{tranum}

A complex number is said to be \emph{algebraic\/}%
\index{algebraic}
or \emph{transcendental\/}%
\index{transcendental}
according as it is algebraic or transcendental over $\mathbb{Q}$. First we
provide a little history.

1844: Liouville showed that certain numbers, now called Liouville numbers, are transcendental.

1873: Hermite showed that $e$ is transcendental.

1874: Cantor showed that the set of algebraic numbers is countable, but that
$\mathbb{R}$ is not countable. Thus most numbers are transcendental (but it is
usually very difficult to prove that any particular number is
transcendental).\footnote{By contrast, when we suspect that a complex number
is algebraic, it is usually possible to prove this, but not always easily.}

1882: Lindemann showed that $\pi$ is transcendental.

1934: Gel'fond and Schneider independently showed that $\alpha^{\beta}$ is
transcendental if $\alpha$ and $\beta$ are algebraic, $\alpha\neq0,1$, and
$\beta\notin\mathbb{Q}$. (This was the seventh of Hilbert's famous problems.)

2020: Euler's constant%

\[
\gamma\overset{\df}{=}\lim_{n\rightarrow\infty}\left(
\sum_{k=1}^{n}1/k-\log n\right)
\]
has not yet been proven to be transcendental or even irrational (see Lagarias,
J., Euler's constant. BAMS 50 (2013), no. 4, 527--628, arXiv:1303:1856).

2020: The numbers $e+\pi$ and $e-\pi$ are surely transcendental, but again
they have not even been proved to be irrational!

\begin{proposition}
\label{ef22}The set of algebraic numbers is countable.
\end{proposition}

\begin{proof}
Every algebraic number is a root of a polynomial%
\[
a_{0}X^{n}+a_{1}X^{n-1}+\cdots+a_{n},\quad a_{0},\ldots,a_{n}\in\mathbb{Z}{}.
\]
For a fixed $N\in\mathbb{N}{}$, there are only finitely many such polynomials
with $n\leq N$ and
%TCIMACRO{\TEXTsymbol{\vert}}%
%BeginExpansion
$\vert$%
%EndExpansion
$a_{0}|,\ldots,|a_{n}|\leq N$, and each polynomial has only finitely many
roots. Thus, the set of algebraic numbers is a countable union of finite sets
$\bigcup_{N\geq1}A(N)$, and any such union is countable --- for example,
choose a bijection from some segment $[0,n(1)]$ of $\mathbb{N}{}$ onto $A(1)$,
extend it to a bijection from a segment $[0,n(2)]$ onto $A(2)$, and so on.
\end{proof}

A typical Liouville number is $\sum_{n=0}^{\infty}\frac{1}{10^{n!}}$ --- in
its decimal expansion there are increasingly long strings of zeros. Since its
decimal expansion is not periodic, the number is not rational. We prove that
the analogue of this number in base $2$ is transcendental.

\begin{theorem}
\label{ef23}%
\index{theorem!Liouville}%
The number $\alpha=\sum\frac{1}{2^{n!}}$ is transcendental.
\end{theorem}

\begin{proof}
\footnote{This proof, which I learnt from David Masser, also works for $%
%TCIMACRO{\tsum }%
%BeginExpansion
{\textstyle\sum}
%EndExpansion
\frac{1}{a^{n!}}$ for every integer $a\geq2$.}Suppose not, and let
\[
f(X)=X^{d}+a_{1}X^{d-1}+\cdots+a_{d},\quad a_{i}\in\mathbb{Q},
\]
be the minimal polynomial of $\alpha$ over $\mathbb{Q}$. Thus $[\mathbb{Q}%
[\alpha]\colon\mathbb{Q}]=d$. Choose a nonzero integer $D$ such that $D\cdot
f(X)\in\mathbb{Z}[X]$.

Let $\Sigma_{N}=\sum_{n=0}^{N}\frac{1}{2^{n!}}$, so that $\Sigma
_{N}\rightarrow\alpha$ as $N\rightarrow\infty$, and let $x_{N}=f(\Sigma_{N})$.
As $\alpha$ is not rational, $f(X)$, being irreducible of degree $>1$, has no
rational root. Since $\Sigma_{N}\neq\alpha$, it can't be a root of $f(X)$, and
so $x_{N}\neq0$. Obviously, $x_{N}\in\mathbb{Q}$; in fact $(2^{N!})^{d}%
Dx_{N}\in\mathbb{Z}$, and so
\begin{equation}
|(2^{N!})^{d}Dx_{N}|\geq1\text{.} \label{eq9}%
\end{equation}


From the fundamental theorem of algebra%
\index{fundamental theorem!of algebra}
(see \ref{ag5} below), we know that $f$ splits in $\mathbb{C}{}[X]$, say,%
\[
f(X)=\prod_{i=1}^{d}(X-\alpha_{i}),\quad\alpha_{i}\in\mathbb{C},\quad
\alpha_{1}=\alpha,
\]
and so
\[
|x_{N}|=\prod_{i=1}^{d}|\Sigma_{N}-\alpha_{i}|\leq|\Sigma_{N}-\alpha
_{1}|(\Sigma_{N}+M)^{d-1},\quad\text{where }M=\max_{i\neq1}\{1,|\alpha
_{i}|\}\text{.}%
\]
But
\[
|\Sigma_{N}-\alpha_{1}|=\sum_{n=N+1}^{\infty}\frac{1}{2^{n!}}\leq\frac
{1}{2^{(N+1)!}}\left(  \sum_{n=0}^{\infty}\frac{1}{2^{n}}\right)  =\frac
{2}{2^{(N+1)!}}.
\]
Hence%
\[
|x_{N}|\leq\frac{2}{2^{(N+1)!}}\cdot(\Sigma_{N}+M)^{d-1}%
\]
and
\[
|(2^{N!})^{d}Dx_{N}|\leq2\cdot\frac{2^{d\cdot N!}D}{2^{(N+1)!}}\cdot
(\Sigma_{N}+M)^{d-1}%
\]
which tends to $0$ as $N\rightarrow\infty$ because $\frac{2^{d\cdot N!}%
}{2^{(N+1)!}}=\left(  \frac{2^{d}}{2^{N+1}}\right)  ^{N!}\rightarrow0$. This
contradicts (\ref{eq9}).
\end{proof}

\section{Constructions with straight-edge and compass.}

The Greeks understood integers and the rational numbers. They were surprised
to find that the length of the diagonal of a square of side $1$, namely,
$\sqrt{2}$, is not rational. They thus realized that they needed to extend
their number system. They then hoped that the \textquotedblleft
constructible\textquotedblright\ numbers would suffice. Suppose that we are
given a length, which we call $1$, a straight-edge, and a compass (device for
drawing circles). A real number (better a length) is \emph{constructible\/}%
\index{constructible}
if it can be constructed by forming successive intersections of

\begin{itemize}
\item lines drawn through two points already constructed, and

\item circles with centre a point already constructed and radius a constructed length.
\end{itemize}

This led them to three famous questions that they were unable to answer: is it
possible to duplicate the cube, trisect an angle, or square the circle by
straight-edge and compass constructions? We'll see that the answer to all
three is negative.

Let $F$ be a subfield of $\mathbb{R}$. For a positive $a\in F$, $\sqrt{a}$
denotes the positive square root of $a$ in $\mathbb{R}{}$. The $F$%
-\emph{plane} is $F\times F\subset\mathbb{R}\times\mathbb{R}$. We make the
following definitions:

\begin{quotation}
\noindent An $F$-\emph{line} is a line in $\mathbb{R}{}\times\mathbb{R}{}$
through two points in the $F$-plane. These are the lines given by equations
\[
ax+by+c=0,\quad a,b,c\in F.
\]


\noindent An $F$-\emph{circle} is a circle in $\mathbb{R}{}\times\mathbb{R}{}$
with centre an $F$-point and radius an element of $F$. These are the circles
given by equations%
\[
(x-a)^{2}+(y-b)^{2}=c^{2},\quad a,b,c\in F.
\]



\end{quotation}

\begin{lemma}
\label{ef24} Let $L\neq L^{\prime}$ be $F$-lines, and let $C\neq C^{\prime} $
be $F$-circles.

\begin{enumerate}
\item $L\cap L^{\prime}=\emptyset$ or consists of a single $F$-point.

\item $L\cap C=\emptyset$ or consists of one or two points in the $F[\sqrt
{e}]$-plane, some $e\in F$, $e>0$.

\item $C\cap C^{\prime}=\emptyset$ or consists of one or two points in the
$F[\sqrt{e}]$-plane, some $e\in F$, $e>0$.
\end{enumerate}
\end{lemma}

\begin{proof}
The points in the intersection are found by solving the simultaneous
equations, and hence by solving (at worst) a quadratic equation with
coefficients in $F$.
\end{proof}

\begin{lemma}
\label{ef25} (a) If $c$ and $d$ are constructible, then so also are $c+d$,
$-c$, $cd$, and $\frac{c}{d}$ $(d\neq0)$.

(b) If $c>0$ is constructible, then so also is $\sqrt c$.
\end{lemma}

\begin{sproof}
First show that it is possible to construct a line perpendicular to a given
line through a given point, and then a line parallel to a given line through a
given point. Hence it is possible to construct a triangle similar to a given
one on a side with given length. By an astute choice of the triangles, one
constructs $cd$ and $c^{-1}$. For (b), draw a circle of radius $\frac{c+1}{2}$
and centre $(\frac{c+1}{2},0)$, and draw a vertical line through the point
$A=(1,0)$ to meet the circle at $P$. The length $AP$ is $\sqrt{c}$. (For more
details, see Artin, M., Algebra, 1991, Chapter 13, Section 4.)
\end{sproof}

\begin{theorem}
\label{ef26}%
\index{theorem!constructible numbers}%


\begin{enumerate}
\item The set of constructible numbers is a field.

\item A number $\alpha$ is constructible if and only if it is contained in a
subfield of $\mathbb{R}{}$ of the form
\[
\mathbb{Q}{}[\sqrt{a_{1}},\ldots,\sqrt{a_{r}}],\quad a_{i}\in\mathbb{Q}%
{}[\sqrt{a_{1}},\ldots,\sqrt{a_{i-1}}],\quad a_{i}>0\text{.}%
\]

\end{enumerate}
\end{theorem}

\begin{proof}
(a) This restates (a) of Lemma \ref{ef25}.

(b) It follows from Lemma \ref{ef24} that every constructible number is
contained in such a field $\mathbb{Q}{}[\sqrt{a_{1}},\ldots,\sqrt{a_{r}}]$.
Conversely, if all the elements of $\mathbb{Q}{}[\sqrt{a_{1}},\ldots
,\sqrt{a_{i-1}}]$ are constructible, then $\sqrt{a_{i}}$ is constructible (by
\ref{ef25}b), and so all the elements of $\mathbb{Q}{}[\sqrt{a_{1}}%
,\ldots,\sqrt{a_{i}}]$ are constructible (by (a)). Applying this for
$i=0,1,\ldots$, we find that all the elements of $\mathbb{Q}{}[\sqrt{a_{1}%
},\ldots,\sqrt{a_{r}}]$ are constructible.
\end{proof}

\begin{corollary}
\label{ef27}If $\alpha$ is constructible, then $\alpha$ is algebraic over
$\mathbb{Q}$, and $[\mathbb{Q}[\alpha]\colon\mathbb{Q}]$ is a power of $2$.
\end{corollary}

\begin{proof}
According to Proposition \ref{ef10}, $[\mathbb{Q}[\alpha]\colon\mathbb{Q}]$
divides
\[
\lbrack\mathbb{Q}[\sqrt{a_{1}}]\cdots\lbrack\sqrt{a_{r}}]\colon\mathbb{Q}]
\]
and $[\mathbb{Q}[\sqrt{a_{1}},\ldots,\sqrt{a_{r}}]\colon\mathbb{Q}]$ is a
power of $2$.
\end{proof}

\begin{corollary}
\label{ef28}It is impossible to duplicate the cube by straight-edge and
compass constructions.
\end{corollary}

\begin{proof}
The problem is to construct a cube with volume $2$. This requires constructing
the real root of the polynomial $X^{3}-2$. But this polynomial is irreducible
(by Eisenstein's criterion \ref{ef7} for example), and so $[\mathbb{Q}%
[\sqrt[3]{2}]\colon\mathbb{Q}]=3$.
\end{proof}

\begin{corollary}
\label{ef29}In general, it is impossible to trisect an angle by straight-edge
and compass constructions.
\end{corollary}

\begin{proof}
Knowing an angle is equivalent to knowing the cosine of the angle. Therefore,
to trisect $3\alpha$, we have to construct a solution to
\[
\cos3\alpha=4\cos^{3}\alpha-3\cos\alpha.
\]
For example, take $3\alpha=60$ degrees. As $\cos60^{\circ}=\frac{1}{2}$, to
construct $\alpha$, we have to solve $8x^{3}-6x-1=0$, which is irreducible
(apply \ref{ef4}), and so $[\mathbb{Q}{}[\alpha]\colon\mathbb{Q}{}]=3$.
\end{proof}

\begin{corollary}
\label{ef30}It is impossible to square the circle by straight-edge and compass constructions.
\end{corollary}

\begin{proof}
A square with the same area as a circle of radius $r$ has side $\sqrt{\pi}r$.
Since $\pi$ is transcendental\footnote{Proofs of this can be found in many
books on number theory, for example, in 11.14 of Hardy, G. H., and Wright, E.
M., An Introduction to the Theory of Numbers, Fourth Edition, Oxford, 1960.},
so also is $\sqrt{\pi}$.
\end{proof}

We next consider another problem that goes back to the ancient Greeks: list
the integers $n$ such that the regular $n$-sided polygon can be constructed
using only straight-edge and compass. Here we consider the question for a
prime $p$ (see \ref{ag11} for the general case). Note that $X^{p}-1$ is not
irreducible; in fact
\[
X^{p}-1=(X-1)(X^{p-1}+X^{p-2}+\cdots+1).
\]


\begin{lemma}
\label{ef31}If $p$ is prime, then $X^{p-1}+\cdots+1$ is irreducible; hence
$\mathbb{Q}[e^{2\pi i/p}]$ has degree $p-1$ over $\mathbb{Q}.$
\end{lemma}

\begin{proof}
Let $f(X)=(X^{p}-1)/(X-1)=X^{p-1}+\cdots+1$; then%
\[
f(X+1)=\frac{(X+1)^{p}-1}{X}=X^{p-1}+\cdots+a_{i}X^{i}+\cdots+p,
\]
with $a_{i}=\tbinom{p}{i+1}$. We know (\ref{ef3}) that $p|a_{i}$ for
$i=1,...,p-2$, and so $f(X+1)$ is irreducible by Eisenstein's criterion
\ref{ef7}. This implies that $f(X)$ is irreducible.
\end{proof}

In order to construct a regular $p$-gon, $p$ an odd prime, we need to
construct
\[
\cos\tfrac{2\pi}{p}=\frac{e^{\frac{2\pi i}{p}}+e^{-\frac{2\pi i}{p}}}{2}.
\]
Note that
\[
\mathbb{Q}[e^{\frac{2\pi i}{p}}]\supset\mathbb{Q}[\cos\tfrac{2\pi}{p}%
]\supset\mathbb{Q}.
\]
The degree of $\mathbb{Q}[e^{\frac{2\pi i}{p}}]$ over $\mathbb{Q}[\cos
\frac{2\pi}{p}]$ is $2$ because the equation
\[
\alpha^{2}-2\cos\tfrac{2\pi}{p}\cdot\alpha+1=0,\quad\alpha=e^{\frac{2\pi i}%
{p}},
\]
shows that it is at most $2$, and it is not $1$ because $e^{\frac{2\pi i}{p}%
}\notin\mathbb{R}{}$. Hence
\[
\lbrack\mathbb{Q}[\cos\tfrac{2\pi}{p}]\colon\mathbb{Q}]=\frac{p-1}{2}.
\]


We deduce that, if the regular $p$-gon is constructible, then $(p-1)/2$ is a
power of $2$; later (\ref{ag11}) we'll prove the converse statement. Thus, the
regular $p$-gon is constructible if and only if $p=2^{r}+1$ for some positive
integer $r$.

A number $2^{r}+1$ can be prime only if $r$ is a power of $2$: if $t$ is odd,
then
\[
Y^{t}+1=(Y+1)(Y^{t-1}-Y^{t-2}+\cdots+1)
\]
and so%
\[
2^{st}+1=(2^{s}+1)((2^{s})^{t-1}-(2^{s})^{t-2}+\cdots+1)\text{.}%
\]


We conclude that the primes $p$ for which the regular $p$-gon is constructible
are exactly those of the form $2^{2^{r}}+1$ for some $r$. Such $p$ are called
\emph{Fermat primes}%
\index{prime!Fermat}
(because Fermat conjectured that all numbers of the form $2^{2^{r}}+1$ are
prime). For $r=0,1,2,3,4$, we have $2^{2^{r}}+1=3,5,17,257,65537$, which are
indeed prime, but Euler showed that $2^{32}+1=(641)(6700417)$, and we don't
know whether there are any more Fermat primes. Thus, we do not know the list
of primes $p$ for which the regular $p$-gon is constructible. See Wikipedia:
Fermat number.

Gauss showed that\footnote{Or perhaps that
\par
$\cos\frac{2\pi}{17}=-\frac{1}{16}+\frac{1}{16}\sqrt{17}+\frac{1}{16}%
\sqrt{34-2\sqrt{17}}+\frac{1}{8}\sqrt{17+3\sqrt{17}-2\sqrt{34-2\sqrt{17}%
}-\sqrt{170-26\sqrt{17}}}$
\par
--- both expressions are correct.}
\[
\cos\frac{2\pi}{17}=-\frac{1}{16}+\frac{1}{16}\sqrt{17}+\frac{1}{16}%
\sqrt{34-2\sqrt{17}}+\frac{1}{8}\sqrt{17+3\sqrt{17}-\sqrt{34-2\sqrt{17}%
}-2\sqrt{34+2\sqrt{17}}}%
\]
when he was 18 years old. This success encouraged him to become a mathematician.

\section{Algebraically closed fields}

Let $F$ be a field. A polynomial is said to
\index{split}%
\emph{split} in $F[X]$ if it is a product of polynomials of degree at most $1$
in $F[X]$.

\begin{proposition}
\label{ac1}For a field $\Omega$, the following statements are equivalent:

\begin{enumerate}
\item Every nonconstant polynomial in $\Omega\lbrack X]$ splits in
$\Omega\lbrack X]$.

\item Every nonconstant polynomial in $\Omega\lbrack X]$ has at least one root
in $\Omega$.

\item The irreducible polynomials in $\Omega\lbrack X]$ are those of degree
$1$.

\item Every field of finite degree over $\Omega$ equals $\Omega$.
\end{enumerate}
\end{proposition}

\begin{proof}
The implications (a)$\Rightarrow$(b)$\Rightarrow$(c) are obvious.

\noindent(c)$\Rightarrow$(a). This follows from the fact that $\Omega\lbrack
X]$ is a unique factorization domain.

\noindent(c)$\Rightarrow$(d). Let $E$ be a finite extension of $\Omega$, and
let $\alpha\in E$. The minimal polynomial of $\alpha$, being irreducible, has
degree $1$, and so $\alpha\in\Omega$.

\noindent(d)$\Rightarrow$(c). Let $f$ be an irreducible polynomial in
$\Omega\lbrack X]$. Then $\Omega\lbrack X]/(f)$ is an extension of $\Omega$ of
degree $\deg(f)$ (see \ref{ef19}), and so $\deg(f)=1$.
\end{proof}

\begin{definition}
\label{ac2}(a) A field $\Omega$ is
\index{algebraically closed}%
\emph{algebraically closed }if it satisfies the equivalent statements of
Proposition \ref{ac1}.

(b) A field $\Omega$ is an
\index{algebraic closure}%
\emph{algebraic closure }of a subfield $F$ if it is algebraically closed and
algebraic over $F$.
\end{definition}

For example, the fundamental theorem of algebra%
\index{theorem!fundamental of algebra}
(see \ref{ag5} below) says that $\mathbb{C}{}$ is algebraically closed. It is
an algebraic closure of $\mathbb{R}{}$.

\begin{proposition}
\label{sf10}If $\Omega$ is algebraic over $F$ and every polynomial $f\in F[X]$
splits in $\Omega\lbrack X]$, then $\Omega$ is algebraically closed (hence an
algebraic closure of $F$).
\end{proposition}

\begin{proof}
Let $f$ be a nonconstant polynomial in $\Omega\lbrack X]$. We have to show
that $f$ has a root in $\Omega$. We know (see \ref{ef10m}) that $f$ has a root
$\alpha$ in some finite extension $\Omega^{\prime}$ of $\Omega$. Set
\[
f=a_{n}X^{n}+\cdots+a_{0},\quad a_{i}\in\Omega,
\]
and consider the fields
\[
F\subset F[a_{0},\ldots,a_{n}]\subset F[a_{0},\ldots,a_{n},\alpha].
\]
Each extension generated by a finite set of algebraic elements, and hence is
finite (\ref{ef19}). Therefore $\alpha$ lies in a finite extension of $F$ (see
\ref{ef10}), and so is algebraic over $F$ --- it is a root of a polynomial $g$
with coefficients in $F$. By assumption, $g$ splits in $\Omega\lbrack X]$, and
so the roots of $g$ in $\Omega^{\prime}$ all lie in $\Omega$. In particular,
$\alpha\in\Omega.$
\end{proof}

\begin{proposition}
\label{sf11}Let $\Omega\supset F$; then
\[
\{\alpha\in\Omega\mid\alpha\text{\textrm{\ algebraic over }}F\}
\]
is a field.
\end{proposition}

\begin{proof}
If $\alpha$ and $\beta$ are algebraic over $F$, then $F[\alpha,\beta]$ is a
field (see \ref{ef20}) of finite degree over $F$ (see \ref{ef19}). Thus, every
element of $F[\alpha,\beta]$ is algebraic over $F$. In particular, $\alpha
\pm\beta$, $\alpha/\beta$, and $\alpha\beta$ are algebraic over $F$.
\end{proof}

The field constructed in the proposition is called the \emph{algebraic closure
of}%
\index{algebraic closure!in an extension}
$F$ \emph{in} $\Omega$.

\begin{corollary}
\label{ac3}Let $\Omega$ be an algebraically closed field. For any subfield $F$
of $\Omega$, the algebraic closure $E$ of $F$ in $\Omega$ is an algebraic
closure of $F.$
\end{corollary}

\begin{proof}
It is algebraic over $F$ by definition. Every polynomial in $F[X]$ splits in
$\Omega\lbrack X]$ and has its roots in $E$, and so splits in $E[X]$. Now
apply Proposition \ref{sf10}.
\end{proof}

Thus, when we admit the fundamental theorem of algebra%
\index{theorem!fundamental of algebra}
(\ref{ag5}), every subfield of $\mathbb{C}{}$ has an algebraic closure (in
fact, a canonical algebraic closure). Later (Chapter 6) we'll prove, using the
axiom of choice, that every field has an algebraic closure.

\begin{aside}
\label{ac3a}Although various classes of field, for example, number fields and
function fields, had been studied earlier, the first systematic account of the
theory of abstract fields was given by Steinitz in 1910 (Algebraische Theorie
der K\"{o}rper, J.\ Reine Angew.\ Math., 137:167--309). Here he introduced the
notion of a prime field, distinguished between separable and inseparable
extensions, and showed that every field can be obtained as an algebraic
extension of a purely transcendental extension. He also proved that every
field has an algebraic closure, unique up to isomorphism. His work influenced
later algebraists (Emmy Noether, van der Waerden, Emil Artin, \ldots) and his
article has been described by Bourbaki as \textquotedblleft\ldots\ a
fundamental work that may be considered as having given birth to the current
conception\footnote{In which objects are to be defined abstractly by axioms.}
of algebra\textquotedblright. See: Roquette, Peter, In memoriam Ernst Steinitz
(1871--1928). J. Reine Angew. Math. 648 (2010), 1--11.
\end{aside}

\section{Exercises}

\begin{exercise}
\label{x1} Let $E={\mathbb{Q}}[\alpha]$, where $\alpha^{3}-\alpha^{2}%
+\alpha+2=0$. Express $(\alpha^{2}+\alpha+1)(\alpha^{2}-\alpha)$ and\linebreak
$(\alpha-1)^{-1}$ in the form $a\alpha^{2}+b\alpha+c$ with $a,b,c\in
{\mathbb{Q}}$.
\end{exercise}

\begin{exercise}
\label{x2} Determine $[{\mathbb{Q}}(\sqrt{2},\sqrt{3})\colon{\mathbb{Q}}]$.
\end{exercise}

\begin{exercise}
\label{x3}Let $F$ be a field, and let $f(X)\in F[X]$.

\begin{enumerate}
\item For every $a\in F$, show that there is a polynomial $q(X )\in F[X]$ such
that
\[
f(X)=q(X)(X-a)+f(a).
\]


\item Deduce that $f(a)=0$ if and only if $(X-a)|f(X)$.

\item Deduce that $f(X)$ can have at most $\deg f$ roots.

\item Let $G$ be a finite abelian group. If $G$ has at most $m$ elements of
order dividing $m$ for each divisor $m$ of $(G\colon1)$, show that $G$ is cyclic.

\item Deduce that every finite subgroup of $F^{\times}$, $F$ a field, is cyclic.
\end{enumerate}
\end{exercise}

\begin{exercise}
\label{x4}Show that with straight-edge, compass, and angle-trisector, it is
possible to construct a regular $7$-gon.
\end{exercise}

\begin{exercise}
\label{x4a}Let $f(X)$ be an irreducible polynomial over $F$ of degree $n$, and
let $E$ be a field extension of $F$ with $[E:F]=m$. If $\gcd(m,n)=1$, show
that $f$ is irreducible over $E$.
\end{exercise}

\begin{exercise}
\label{x4b}Show that there does not exist a polynomial $f(X)\in\mathbb{Z}%
{}[X]$ of degree $>1$ that is irreducible modulo $p$ for all primes $p$.
\end{exercise}

\begin{exercise}
\label{x4c}Let $\alpha=\sqrt[3]{2}$, and let $R$ be the set of complex numbers
of the form $a+b\alpha+c\alpha^{2}$ with $a,b,c\in\mathbb{Q}{}$. Show that $R$
is a field.
\end{exercise}

\clearpage


\chapter{Splitting Fields; Multiple Roots}

\section{Homomorphisms from simple extensions.}

\noindent Let $F$ be a field, and let $E$ and $E^{\prime}$ be fields
containing $F$. Recall that an $F$-homomorphism\/ is a homomorphism
$\varphi\colon E\rightarrow E^{\prime}$ such that $\varphi(a)=a$ for all $a\in
F$. Thus an $F$-homomorphism $\varphi$ maps a polynomial
\[
\sum a_{i_{1}\cdots i_{m}}\alpha_{1}^{i_{1}}\cdots\alpha_{m}^{i_{m}},\quad
a_{i_{1}\cdots i_{m}}\in F,\quad\alpha_{i}\in E,
\]
to
\[
\sum a_{i_{1}\cdots i_{m}}\varphi(\alpha_{1})^{i_{1}}\cdots\varphi(\alpha
_{m})^{i_{m}}.
\]
An $F$-\emph{isomorphism}%
\index{F-isomorphism@$F$-isomorphism}%
\emph{\/} is a bijective $F$-homomorphism.

An $F$-homomorphism $E\rightarrow E^{\prime}$ of fields is, in particular, an
injective $F$-linear map of $F$-vector spaces, and so it is an $F$-isomorphism
if $E$ and $E^{\prime}$ have the same finite degree over $F$.

\begin{proposition}
\label{sf1} Let $F(\alpha)$ be a simple extension of $F$ and $\Omega$ a second
extension of $F$.

\begin{enumerate}
\item Let $\alpha$ be transcendental over $F$. For every $F$-homomorphism
$\varphi\colon F(\alpha)\rightarrow\Omega$, $\varphi(\alpha)$ is
transcendental over $F$, and the map $\varphi\mapsto\varphi(\alpha)$ defines a
one-to-one correspondence
\[
\{F\text{-homomorphisms }F(\alpha)\rightarrow\Omega\}\leftrightarrow
\{\text{elements of }\Omega\text{ transcendental over }F\}.
\]


\item Let $\alpha$ be algebraic over $F$ with minimal polynomial $f(X)$. For
every $F$-homomorphism $\varphi\colon F[\alpha]\rightarrow\Omega$,
$\varphi(\alpha)$ is a root of $f(X)$ in $\Omega$, and the map $\varphi
\mapsto\varphi(\alpha)$ defines a one-to-one correspondence
\[
\{F\text{-homomorphisms }\varphi\colon F[\alpha]\rightarrow\Omega
\}\leftrightarrow\{\text{roots of\textrm{\ }}f\text{\ in }\Omega\}.
\]
In particular, the number of such maps is the number of distinct roots of $f$
in $\Omega$.
\end{enumerate}
\end{proposition}

\begin{proof}
(a) To say that $\alpha$ is transcendental over $F$ means that $F[\alpha]$ is
isomorphic to the polynomial ring in the symbol $\alpha$. Therefore, for every
$\gamma\in\Omega$, there is a unique $F$-homomorphism $\varphi\colon
F[\alpha]\rightarrow\Omega$ such that $\varphi(\alpha)=\gamma$ (see
\ref{ef3a}). This $\varphi$ extends (uniquely) to the field of fractions
$F(\alpha)$ of $F[\alpha]$ if and only if nonzero elements of $F[\alpha]$ are
sent to nonzero elements of $\Omega$, which is the case if and only if
$\gamma$ is transcendental over $F$. Thus we see that there are one-to-one
correspondences between (a) the $F$-homomorphisms $F(\alpha)\rightarrow\Omega
$, (b) the $F$-homomorphisms $\varphi\colon F[\alpha]\rightarrow\Omega$ such
that $\varphi(\alpha)$ is transcendental, (c) the transcendental elements of
$\Omega$.

(b) Let $f(X)=\sum a_{i}X^{i}$, and consider an $F$-homomorphism
$\varphi\colon F[\alpha]\rightarrow\Omega$. On applying $\varphi$ to the
equality $\sum a_{i}\alpha^{i}=0$, we obtain the equality $\sum a_{i}%
\varphi(\alpha)^{i}=0$, which shows that $\varphi(\alpha)$ is a root of $f(X)$
in $\Omega$. Conversely, if $\gamma\in\Omega$ is a root of $f(X)$, then the
map $F[X]\rightarrow\Omega$, $g(X)\mapsto g(\gamma)$, factors through
$F[X]/(f(X))$. When composed with the inverse of the canonical isomorphism
$F[X]/(f(X))\rightarrow F[\alpha]$, this becomes a homomorphism $F[\alpha
]\rightarrow\Omega$ sending $\alpha$ to $\gamma$.
\end{proof}

We shall need a slight generalization of this result.

\begin{proposition}
\label{sf2} Let $F(\alpha)$ be a simple extension of $F$ and $\varphi
_{0}\colon F\rightarrow\Omega$ a homomorphism from $F$ into a second field
$\Omega$.

\begin{enumerate}
\item If $\alpha$ is transcendental over $F$, then the map $\varphi
\mapsto\varphi(\alpha)$ defines a one-to-one correspondence
\[
\{\text{extensions }\varphi\colon F(\alpha)\rightarrow\Omega\text{\textrm{\ of
}}\varphi_{0}\}\leftrightarrow\{\text{elements of }\Omega
\text{\textrm{\ transcendental over }}\varphi_{0}(F)\}.
\]


\item If $\alpha$ is algebraic over $F$, with minimal polynomial $f(X)$, then
the map $\varphi\mapsto\varphi(\alpha)$ defines a one-to-one correspondence
\[
\{\text{extensions }\varphi\colon F[\alpha]\rightarrow\Omega\text{\textrm{\ of
}}\varphi_{0}\}\leftrightarrow\{\text{\textrm{roots of }}\varphi_{0}f\text{ in
}\Omega\}.
\]
In particular, the number of such maps is the number of distinct roots of
$\varphi_{0}f$ in $\Omega$.
\end{enumerate}
\end{proposition}

By $\varphi_{0}f$ we mean the polynomial obtained by applying $\varphi_{0}$ to
the coefficients of $f$. By an extension of $\varphi_{0}$ to $F(\alpha)$ we
mean a homomorphism $\varphi\colon F(\alpha)\rightarrow\Omega$ whose
restriction to $F$ is $\varphi_{0}$. The proof of the proposition is
essentially the same as that of the preceding proposition (indeed, it is
essentially the same proposition).

\section{Splitting fields}

Let $f$ be a polynomial with coefficients in $F$. A field $E$ containing $F$
is said to
\index{split}%
\emph{split}$f$ if $f$ splits in $E[X]$, i.e.,
\[
f(X)=a\prod\nolimits_{i=1}^{m}(X-\alpha_{i})\text{ with all }\alpha_{i}\in E.
\]
If $E$ splits $f$ and is generated by the roots of $f$,
\[
E=F[\alpha_{1},\ldots,\alpha_{m}],
\]
then it is called a \emph{splitting} or \emph{root field}%
\index{splitting field}
for $f$.

Note that $\prod f_{i}(X)^{m_{i}}$ ($m_{i}\geq1$) and $\prod f_{i}(X)$ have
the same splitting fields. Note also that $f$ splits in $E$ if it has
$\deg(f)-1$ roots in $E$ because the sum of the roots of $f$ lies in $F$ (if
$f=aX^{m}+a_{1}X^{m-1}+\cdots$, then $\sum\alpha_{i}=-a_{1}/a$).

\begin{example}
\label{sf3}(a) Let $f(X)=aX^{2}+bX+c\in\mathbb{Q}{}[X]$, and let $\alpha
=\sqrt{b^{2}-4ac}$. The subfield $\mathbb{Q}[\alpha]$ of $\mathbb{C}$ is a
splitting field for $f$.

(b) Let $f(X)=X^{3}+aX^{2}+bX+c\in\mathbb{Q}[X]$ be irreducible, and let
$\alpha_{1},\alpha_{2},\alpha_{3}$ be its roots in $\mathbb{C}$. Then
$\mathbb{Q}[\alpha_{1},\alpha_{2},\alpha_{3}]=\mathbb{Q}[\alpha_{1},\alpha
_{2}]$ is a splitting field for $f(X)$. Note that $[\mathbb{Q}[\alpha
_{1}]\colon\mathbb{Q}]=3$ and that $[\mathbb{Q}[\alpha_{1},\alpha_{2}%
]\colon\mathbb{Q}[\alpha_{1}]]=1$ or $2$, and so $[\mathbb{Q}[\alpha
_{1},\alpha_{2}]\colon\mathbb{Q}]=3$ or $6$. We'll see later (\ref{cg2}) that
the degree is $3$ if and only if the discriminant of $f(X)$ is a square in
$\mathbb{Q}{}$. For example, the discriminant of $X^{3}+bX+c$ is
$-4b^{3}-27c^{2}$, and so the splitting field of $X^{3}+10X+1$ (discriminant
$-4027)$ has degree $6$ over $\mathbb{Q}$.
\end{example}

\begin{proposition}
\label{sf4}Every polynomial $f\in F[X]$ has a splitting field $E_{f}$, and
\[
\lbrack E_{f}\colon F]\leq(\deg f)!\quad(\text{factorial }\deg f).
\]

\end{proposition}

\begin{proof}
Let $F_{1}=F[\alpha_{1}]$ be a stem field for some monic irreducible factor of
$f$ in $F[X]$. Then $f(\alpha_{1})=0$, and we let $F_{2}=F_{1}[\alpha_{2}]$ be
a stem field for some monic irreducible factor of $f(X)/(X-\alpha_{1})$ in
$F_{1}[X]$. Continuing in this fashion, we arrive at a splitting field $E_{f}%
$. Let $n=\deg f$. Then $[F_{1}\colon F]=\deg g_{1}\leq n$, $[F_{2}\colon
F_{1}]\leq n-1,...$, and so $[E_{f}\colon F]\leq n!$.
\end{proof}

\begin{aside}
\label{sf5}Let $F$ be a field. For a given integer $n$, there may or may not
exist polynomials of degree $n$ in $F[X]$ whose splitting field has degree
$n!$ --- this depends on $F$. For example, there do not exist such polynomials
for $n>1$ if $F=\mathbb{C}$ (see \ref{ag5}), nor for $n>2$ if $F=\mathbb{R}$
or $F=\mathbb{F}_{p}$ (see \ref{cg18}). However, later (\ref{cg24}) we'll see
how to write down infinitely many polynomials of degree $n$ in $\mathbb{Q}[X]$
with splitting fields of degree $n!$.
\end{aside}

\begin{example}
\label{sf6}(a) Let $f(X)=(X^{p}-1)/(X-1)\in\mathbb{Q}{}[X]$, $p$ prime. If
$\zeta$ is one root of $f$, then the remaining roots are $\zeta^{2},\zeta
^{3},\ldots,\zeta^{p-1}$, and so the splitting field of $f$ is $\mathbb{Q}%
{}[\zeta]$.

(b) Let $F$ have characteristic $p\neq0$, and let $f=X^{p}-X-a\in F[X]$. If
$\alpha$ is one root of $f$ in some extension of $F$, then the remaining roots
are $\alpha+1,...,\alpha+p-1$, and so the splitting field of $f$ is
$F[\alpha]$.

(c) If $\alpha$ is one root of $X^{n}-a$, then the remaining roots are all of
the form $\zeta\alpha$, where $\zeta^{n}=1$. Therefore, $F[\alpha]$ is a
splitting field for $X^{n}-a$ if and only if $F$ contains all the $n$th roots
of $1$ (by which we mean that $X^{n}-1$ splits in $F[X]$). Note that if $p$ is
the characteristic of $F$, then $X^{p}-1=(X-1)^{p}$, and so $F$ automatically
contains all the $p$th roots of $1$.
\end{example}

\begin{proposition}
\label{sf7}Let $f\in F[X]$. Let $E$ be an extension of $F$ generated by the
roots of $f$ in $E$, and let $\Omega$ be an extension of $F$ splitting $f$.

\begin{enumerate}
\item There exists an $F$-homomorphism $\varphi\colon E\rightarrow\Omega$; the
number of such homomorphisms is at most $[E\colon F]$, and equals $[E\colon
F]$ if $f$ has distinct roots in $\Omega$.

\item If $E$ and $\Omega$ are both splitting fields for $f$, then every
$F$-homomorphism $E\rightarrow\Omega$ is an isomorphism. In particular, any
two splitting fields for $f$ are $F$-isomorphic.
\end{enumerate}
\end{proposition}

\noindent As $f$ splits in $\Omega\lbrack X]$, $f(X)=a\prod\nolimits_{i=1}%
^{\deg(f)}(X-\beta_{i})$ with $\beta_{1},\beta_{2},\ldots\in\Omega$. To say
that $f$ has distinct roots in $\Omega$ means that $\beta_{i}\neq\beta_{j}$ if
$i\neq j$.

\begin{proof}
We may suppose that $f$ is monic.

We begin with an observation: let $F$, $f$, and $\Omega$ be as in the
statement of the proposition, let $L$ be a subfield of $\Omega$ containing
$F$, and let $g$ be a monic factor of $f$ in $L[X]$; as $g$ divides $f$ in
$\Omega\lbrack X]$, it is a product of certain number of the factors
$X-\beta_{i}$ of $f$ in $\Omega\lbrack X]$; in particular, we see that $g$
splits in $\Omega$, and that it has distinct roots in $\Omega$ if $f$ does..

(a) By hypothesis, $E=F[\alpha_{1},...,\alpha_{m}]$ with each $\alpha_{i}$ a
root of $f(X)$ in $E$. The minimal polynomial of $\alpha_{1}$ is an
irreducible polynomial $f_{1}$ dividing $f$. From the initial observation with
$L=F$, we see that $f_{1}$ splits in $\Omega$, and that its roots are distinct
if the roots of $f$ are distinct. According to Proposition \ref{sf1}, there
exists an $F$-homomorphism $\varphi_{1}\colon F[\alpha_{1}]\rightarrow\Omega$,
and the number of such homomorphisms is at most $[F[\alpha_{1}]\colon F]$,
with equality holding when $f$ has distinct roots in $\Omega$.

The minimal polynomial of $\alpha_{2}$ over $F[\alpha_{1}]$ is an irreducible
factor $f_{2}$ of $f$ in $F[\alpha_{1}][X]$. On applying the initial
observation with $L=\varphi_{1}F[\alpha_{1}]$ and $g=\varphi_{1}f_{2}$, we see
that $\varphi_{1}f_{2}$ splits in $\Omega$, and that its roots are distinct if
the roots of $f$ are distinct. According to Proposition \ref{sf2}, each
$\varphi_{1}$ extends to a homomorphism $\varphi_{2}\colon F[\alpha_{1}%
,\alpha_{2}]\rightarrow\Omega$, and the number of extensions is at most
$[F[\alpha_{1},\alpha_{2}]\colon F[\alpha_{1}]]$, with equality holding when
$f$ has distinct roots in $\Omega.$

On combining these statements we conclude that there exists an $F$%
-homomorphism
\[
\varphi\colon F[\alpha_{1},\alpha_{2}]\rightarrow\nolinebreak\Omega,
\]
and that the number of such homomorphisms is at most $[F[\alpha_{1},\alpha
_{2}]\colon F]$, with equality holding if $f$ has distinct roots in $\Omega.$

After repeating the argument $m$ times, we obtain (a).

(b) Every $F$-homomorphism $E\rightarrow\Omega$ is injective, and so, if there
exists such a homomorphism, then $[E\colon F]\leq\lbrack\Omega\colon F]$. If
$E$ and $\Omega$ are both splitting fields for $f$, then (a) shows that there
exist homomorphisms $E\leftrightarrows\Omega$, and so $[E\colon F]=[\Omega
\colon F]$. It follows that every $F$-homomorphism $E\rightarrow\Omega$ is an
$F$-isomorphism.
\end{proof}

\begin{corollary}
\label{sf8}Let $E$ and $L$ be extension of $F$, with $E$ finite over $F$.

\begin{enumerate}
\item The number of $F$-homomorphisms $E\rightarrow L$ is at most $[E\colon
F]$.

\item There exists a finite extension $\Omega/L$ and an $F$-homomorphism
$E\rightarrow\Omega.$
\end{enumerate}
\end{corollary}

\begin{proof}
Write $E=F[\alpha_{1},\ldots,\alpha_{m}]$, and let $f\in F[X]$ be the product
of the minimal polynomials of the $\alpha_{i}$; thus $E$ is generated over $F$
by roots of $f$. Let $\Omega$ be a splitting field for $f$ regarded as an
element of $L[X]$. The proposition shows that there exists an $F$-homomorphism
$E\rightarrow\Omega$, and the number of such homomorphisms is $\leq\lbrack
E\colon F]$. This proves (b), and since an $F$-homomorphism $E\rightarrow L$
can be regarded as an $F$-homomorphism $E\rightarrow\Omega$, it also proves (a).
\end{proof}

\begin{remark}
\label{sf9}(a) Let $E_{1},E_{2},\ldots,E_{m}$ be finite extensions of $F$, and
let $L$ be an extension of $F$. From the corollary we see that there exists a
finite extension $L_{1}/L$ such that $L_{1}$ contains an isomorphic image of
$E_{1}$; then that there exists a finite extension $L_{2}/L_{1}$ such that
$L_{2}$ contains an isomorphic image of $E_{2}$. On continuing in this
fashion, we find that there exists a finite extension $\Omega$/$L$ such that
$\Omega$ contains an isomorphic copy of every $E_{i}$.

(b) Let $f\in F[X]$. If $E$ and $E^{\prime}$ are both splitting fields of $f$,
then we know there exists an $F$-isomorphism $E\rightarrow E^{\prime}$, but
there will in general be no \textit{preferred}\emph{\/} such isomorphism.
Error and confusion can result if the fields are simply identified. Also, it
makes no sense to speak of \textquotedblleft the field $F[\alpha]$ generated
by a root of $f$\textquotedblright\ unless $f$ is irreducible (the fields
generated by the roots of two different factors are unrelated). Even when $f$
is irreducible, it makes no sense to speak of \textquotedblleft the field
$F[\alpha,\beta]$ generated by two roots $\alpha,\beta$ of $f$%
\textquotedblright\ (the extensions of $F[\alpha]$ generated by the roots of
two different factors of $f$ in $F[\alpha][X]$ may be very different).
\end{remark}

\section{Multiple roots}

Even when polynomials in $F[X]$ have no common factor in $F[X]$, one might
expect that they could acquire a common factor in $\Omega\lbrack X]$ for some
$\Omega\supset F$. In fact, this doesn't happen --- greatest common divisors
don't change when the field is extended.

\begin{proposition}
\label{ft1}Let $f$ and $g$ be polynomials in $F[X]$, and let $\Omega$ be an
extension of $F$. If $r(X)$ is the gcd of $f$ and $g$ computed in $F[X]$, then
it is also the gcd of $f$ and $g$ in $\Omega\lbrack X]$. In particular,
distinct monic irreducible polynomials in $F[X]$ do not acquire a common root
in any extension of $F.$
\end{proposition}

\begin{proof}
Let $r_{F}(X)$ and $r_{\Omega}(X)$ be the greatest common divisors of $f$ and
$g$ in $F[X]$ and $\Omega\lbrack X]$ respectively. Certainly $r_{F}%
(X)|r_{\Omega}(X)$ in $\Omega\lbrack X]$, but Euclid's algorithm (\ref{ef3d})
shows that there are polynomials $a$ and $b$ in $F[X]$ such that
\[
a(X)f(X)+b(X)g(X)=r_{F}(X),
\]
and so $r_{\Omega}(X)$ divides $r_{F}(X)$ in $\Omega\lbrack X]$.

For the second statement, note that the hypotheses imply that $\gcd(f,g)=1$
(in $F[X]$), and so $f$ and $g$ can't acquire a common factor in any extension field.
\end{proof}

The proposition allows us to speak of the greatest common divisor of $f$ and
$g$ without reference to a field.

Let $f\in F[X]$. Then $f$ splits into linear factors
\begin{equation}
f(X)=a\prod_{i=1}^{r}(X-\alpha_{i})^{m_{i}},\text{ }\alpha_{i}\text{ distinct,
}m_{i}\geq1\text{, }\sum_{i=1}^{r}m_{i}=\deg(f), \label{eq5}%
\end{equation}
in $E[X]$ for some extension $E$ of $F$ (see \ref{sf4}). We say that
$\alpha_{i}$ is a root of $f$ of \emph{multiplicity}%
\index{multiplicity}
$m_{i}$ in $E$. If $m_{i}>1$, then $\alpha_{i}$ is said to be a \emph{multiple
root }%
\index{root!multiple}
of $f$, and otherwise it is a \emph{simple root}%
\index{root!simple}%
.

I claim that the unordered sequence of integers $m_{1},\ldots,m_{r}$ in
(\ref{eq5}) is independent of the extension $E$ chosen to split $f$.
Certainly, it is unchanged when $E$ is replaced with its subfield
$F[\alpha_{1},\ldots,\alpha_{r}]$, and so we may suppose that $E$ is a
splitting field for $f$. Let $E$ and $E^{\prime}$ be splitting fields for $F$,
and suppose that $f(X)=a\prod_{i=1}^{r}(X-\alpha_{i})^{m_{i}}$ in $E[X]$ and
$f(X)=a^{\prime}\prod_{i=1}^{r^{\prime}}(X-\alpha_{i}^{\prime})^{m_{i}%
^{\prime}}$ in $E^{\prime}[X]$. Let $\varphi\colon E\rightarrow E^{\prime}$ be
an $F$-isomorphism, which exists by (\ref{sf7}b)$,$ and extend it to an
isomorphism $E[X]\rightarrow E^{\prime}[X]$ by sending $X$ to $X$. Then
$\varphi$ maps the factorization of $f$ in $E[X]$ onto a factorization%
\[
f(X)=\varphi(a)\prod_{i=1}^{r}(X-\varphi(\alpha_{i}))^{m_{i}}%
\]
in $E^{\prime}[X]$. By unique factorization, this coincides with the earlier
factorization in $E^{\prime}[X]$ up to a renumbering of the $\alpha_{i}$.
Therefore $r=r^{\prime}$, and%
\[
\{m_{1},\ldots,m_{r}\}=\{m_{1}^{\prime},\ldots,m_{r}^{\prime}\}.
\]


We say that $f$ \emph{has a multiple root} when at least one of the $m_{i}>1$,
and that $f$ has \emph{only simple roots} when all $m_{i}=1$. Thus
\textquotedblleft$f$ has a multiple root\textquotedblright\ means
\textquotedblleft$f$ has a multiple root in one, hence every, extension of $F$
splitting $f$\textquotedblright, and similarly for $``f$ has only simple
roots\textquotedblright.

We wish to determine when a polynomial has a multiple root. If $f$ has a
multiple factor in $F[X]$, say $f=\prod f_{i}(X)^{m_{i}}$ with some $m_{i}>1$,
then obviously it will have a multiple root. If $f=\prod f_{i}$ with the
$f_{i}$ distinct monic irreducible polynomials, then Proposition \ref{ft1}
shows that $f$ has a multiple root if and only if at least one of the $f_{i}$
has a multiple root. Thus, it suffices to determine when an
\textit{irreducible} polynomial has a multiple root.

\begin{example}
\label{ft2}Let $F$ be of characteristic $p\neq0$, and assume that $F$ contains
an element $a$ that is not a $p$th-power, for example, $a=T$ in the field
$\mathbb{F}{}_{p}(T).$ Then $X^{p}-a$ is irreducible in $F[X]$, but by
\ref{ef3} we have $X^{p}-a=(X-\alpha)^{p}$ in its splitting field. Thus an
irreducible polynomial can have multiple roots.
\end{example}

The derivative of a polynomial $f(X)=\sum a_{i}X^{i}$ is defined to be
$f^{\prime}(X)=\sum ia_{i}X^{i-1}$. The usual rules for differentiating sums
and products still hold, but note that in characteristic $p$ the derivative of
$X^{p}$ is zero.

\begin{proposition}
\label{ft3}For a nonconstant irreducible polynomial $f$ in $F[X]$, the
following statements are equivalent:

\begin{enumerate}
\item $f$ has a multiple root;

\item $\gcd(f,f^{\prime})\neq1$;

\item $F$ has nonzero characteristic $p$ and $f$ is a polynomial in $X^{p}$;

\item all the roots of $f$ are multiple.
\end{enumerate}
\end{proposition}

\begin{proof}
(a) $\Rightarrow$ (b). Let $\alpha$ be a multiple root of $f$, and write
$f=(X-\alpha)^{m}g(X)$, $m>1$, in some extension field. Then
\begin{equation}
f^{\prime}(X)=m(X-\alpha)^{m-1}g(X)+(X-\alpha)^{m}g^{\prime}(X). \label{eq2}%
\end{equation}
Hence $f$ and $f^{\prime}$ have $X-\alpha$ as a common factor.

(b) $\Rightarrow$ (c). As $f$ is irreducible and $\deg(f^{\prime})<\deg(f)$,
\[
\gcd(f,f^{\prime})\neq1\implies f^{\prime}=0.
\]
Let $f=a_{0}+\cdots+a_{d}X^{d}$, $d\geq1$. Then $f^{\prime}=a_{1}%
+\cdots+ia_{i}X^{i-1}+\cdots+da_{d}X^{d-1}$, which is the zero polynomial if
only if $F$ has characteristic $p\neq0$ and $a_{i}=0$ for all $i$ not
divisible by $p$.

(c) $\Rightarrow$ (d). By hypothesis, $f(X)=g(X^{p})$ with $g(X)\in F[X]$. Let
$g(X)=\prod_{i}(X-a_{i})^{m_{i}}$ in some extension field. Then each $a_{i}$
becomes a $p$th power, say, $a_{i}=\alpha_{i}^{p}$, in some possibly larger
extension field. Now
\[
f(X)=g(X^{p})=\prod\nolimits_{i}(X^{p}-a_{i})^{m_{i}}=\prod\nolimits_{i}%
(X-\alpha_{i})^{pm_{i}}%
\]
which shows that every root of $f(X)$ has multiplicity at least $p$.

(d) $\Rightarrow$ (a). Obvious.
\end{proof}

\begin{proposition}
\label{ft3a}The following conditions on a nonzero polynomial $f\in F[X]$ are equivalent:

\begin{enumerate}
\item $\gcd(f,f^{\prime})=1$ in $F[X]$;

\item $f$ has only simple roots.
\end{enumerate}
\end{proposition}

\begin{proof}
Let $\Omega$ be an extension of $F$ splitting $f$. From (\ref{eq2}),
p.\thinspace\pageref{eq2}, we see that a root $\alpha$ of $f$ in $\Omega$ is
multiple if and only if it is also a root of $f^{\prime}$.

If $\gcd(f,f^{\prime})=1$, then $f$ and $f^{\prime}$ have no common factor in
$\Omega\lbrack X]$ (see \ref{ft1}). In particular, they have no common root,
and so $f$ has only simple roots.

If $f$ has only simple roots, then $\gcd(f,f^{\prime})$ must be the constant
polynomial, because otherwise it would have a root in $\Omega$ which would
then be a common root of $f$ and $f^{\prime}$.
\end{proof}

\begin{definition}
\label{ft4}A polynomial is
\index{polynomial!separable}%
\emph{separable} if it is nonzero and satisfies the equivalent conditions on
(\ref{ft3a}).\footnote{This is Bourbaki's definition. Often (e.g., in the
books of Jacobson and in earlier versions of these notes) a polynomial $f$ is
said to be separable if each of its irreducible factors has only simple
roots.}
\end{definition}

Thus a nonconstant irreducible polynomial $f$ is not separable if and only if
$F$ has characteristic $p\neq0$ and $f$ is a polynomial in $X^{p}$ (see
\ref{ft3}). Let $f=\prod f_{i}$ with $f$ and the $f_{i}$ monic and the $f_{i}$
irreducible; then $f$ is separable if and only if the $f_{i}$ are distinct and
separable. If $f$ is separable as a polynomial in $F[X]$, then it is separable
as a polynomial in $E[X]$ for every extension $E$ of $F$.

\begin{definition}
\label{ft4m}A field $F$ is \emph{perfect\/}%
\index{perfect field}%
\index{field!perfect}
if it has characteristic zero or it has characteristic $p$ and every every
element of $F$ is a $p$th power.
\end{definition}

Thus, $F$ is perfect if and only if $F=F^{q}$, where $q$ is the characteristic
exponent of $F$.

\begin{proposition}
\label{ft5}A field $F$ is perfect if and only if every irreducible polynomial
in $F[X]$ is separable.
\end{proposition}

\begin{proof}
If $F$ has characteristic zero, the statement is obvious, and so we may
suppose $F$ has characteristic $p\neq0$. If $F$ contains an element $a$ that
is not a $p$th power, then $X^{p}-a$ is irreducible in $F[X]$ but not
separable (see \ref{ft2}). Conversely, if every element of $F$ is a $p$th
power, then every polynomial in $X^{p}$ with coefficients in $F$ is a $p$th
power in $F[X]$,%
\[
\tstyle\sum a_{i}X^{ip}=\left(  \sum b_{i}X^{i}\right)  ^{p}\quad\text{if
}\quad a_{i}=b_{i}^{p}\text{,}%
\]
and so it is not irreducible.
\end{proof}

\begin{example}
\label{ft6}

\begin{enumerate}
\item A finite field $F$ is perfect, because the Frobenius endomorphism%
\index{Frobenius!endomorphism}
$a\mapsto a^{p}\colon F\rightarrow F$ is injective and therefore surjective
(by counting).

\item A field that can be written as a union of perfect fields is perfect.
Therefore, every field algebraic over $\mathbb{F}{}_{p}$ is perfect.

\item Every algebraically closed field is perfect.

\item If $F_{0}$ has characteristic $p\neq0$, then $F=F_{0}(X)$ is not
perfect, because $X$ is not a $p$th power.
\end{enumerate}
\end{example}

\begin{aside}
Let $F$ be a perfect field. We'll see later (\ref{ag1}) that every finite
extension $E/F$ is simple, i.e., $E=F[\alpha]$ with $\alpha$ a root of a
(separable) polynomial $f\in F[X]$ of degree $[E\colon F]$. Thus it follows
directly from (\ref{sf2}b) that, for any extension $\Omega$ of $F$, the number
of $F$-homomorphisms $E\rightarrow\Omega$ is $\leq\lbrack E\colon F]$, with
equality if and only if $f$ splits in $\Omega$. We can't use this argument
here because it would make the exposition circular.
\end{aside}

\section{Exercises}

\begin{exercise}
\label{x5} Let $F$ be a field of characteristic $\neq2$.

\begin{enumerate}
\item Let $E$ be quadratic extension of $F$; show that
\[
S(E)=\{a\in F^{\times}\mid a\text{\textrm{\ }is a square in\textrm{\ }}E\}
\]
is a subgroup of $F^{\times}$ containing $F^{\times2}$.

\item Let $E$ and $E^{\prime}$ be quadratic extensions of $F$; show that there
exists an $F$-isomorphism $\varphi\colon E\rightarrow E^{\prime}$ if and only
if $S(E)=S(E^{\prime})$.

\item Show that there is an infinite sequence of fields $E_{1},E_{2},\ldots$
with $E_{i}$ a quadratic extension of ${\mathbb{Q}}$ such that $E_{i}$ is not
isomorphic to $E_{j}$ for $i\neq j$.

\item Let $p$ be an odd prime. Show that, up to isomorphism, there is exactly
one field with $p^{2}$ elements.
\end{enumerate}
\end{exercise}

\begin{exercise}
\label{x6} (a) Let $F$ be a field of characteristic $p$. Show that if
$X^{p}-X-a$ is reducible in $F[X]$, then it splits into distinct factors in
$F[X]$.

(b) For every prime $p$, show that $X^{p}-X-1$ is irreducible in ${\mathbb{Q}%
}[X]$.
\end{exercise}

\begin{exercise}
\label{x7} Construct a splitting field for $X^{5}-2$ over ${\mathbb{Q}}$. What
is its degree over ${\mathbb{Q}}$?
\end{exercise}

\begin{exercise}
\label{x8} Find a splitting field of $X^{p^{m}}-1\in\mathbb{F}_{p}[X]$. What
is its degree over $\mathbb{F}_{p}$?
\end{exercise}

\begin{exercise}
\label{x9} Let $f\in F[X]$, where $F$ is a field of characteristic $0$. Let
$d(X)=\gcd(f,f^{\prime})$. Show that $g(X)=f(X)d(X)^{-1}$ has the same roots
as $f(X)$, and these are all simple roots of $g(X)$.
\end{exercise}

\begin{exercise}
\label{x10} Let $f(X)$ be an irreducible polynomial in $F[X]$, where $F$ has
characteristic $p$. Show that $f(X)$ can be written $f(X)=g(X^{p^{e}})$ where
$g(X)$ is irreducible and separable. Deduce that every root of $f(X)$ has the
same multiplicity $p^{e}$ in any splitting field.
\end{exercise}

\clearpage


\chapter{The Fundamental Theorem of Galois Theory}

In this chapter, we prove the fundamental theorem of Galois theory, which
classifies the subfields of the splitting field of a separable polynomial $f$
in terms of the Galois group of~$f$.

\section{Groups of automorphisms of fields}

Consider fields $E\supset F$. An $F$-isomorphism $E\rightarrow E$ is called an
$F$\emph{-automorphism}%
\index{automorphism}
of $E$. The $F$-automorphisms of $E$ form a group, which we denote $\Aut(E/F)$.

\begin{example}
\label{ft7}(a) There are two obvious automorphisms of $\mathbb{C}$, namely,
the identity map and complex conjugation. We'll see later (\ref{te16}) that by
using the Axiom of Choice we can construct uncountably many more.

(b) Let $E=\mathbb{C}(X)$. A $\mathbb{C}{}$-automorphism of $E{}$ sends $X$ to
another generator of $E$ over $\mathbb{C}{}$. It follows from (\ref{te17a})
below that these are exactly the elements $\frac{aX+b}{cX+d}$, $ad-bc\neq0$.
Therefore $\Aut(E/\mathbb{C})$ consists of the maps $f(X)\mapsto f\left(
\frac{aX+b}{cX+d}\right)  $, $ad-bc\neq0$, and so
\[
\Aut(E/\mathbb{C})\simeq\PGL_{2}(\mathbb{C}),
\]
the group of invertible $2\times2$ matrices with complex coefficients modulo
its centre. Analysts will note that this is the same as the automorphism group
of the Riemann sphere. Here is the explanation. The field $E$ of meromorphic
functions on the Riemann sphere $\mathbb{P}_{\mathbb{C}}^{1}$ consists of the
rational functions in $z$, i.e., $E=\mathbb{C}(z)\simeq\mathbb{C}(X)$, and the
natural map $\Aut(\mathbb{P}_{\mathbb{C}}^{1})\rightarrow\Aut(E/\mathbb{C})$
is an isomorphism.

(c) The group $\Aut(\mathbb{C}(X_{1},X_{2})/\mathbb{C})$ is quite complicated
--- there is a map
\[
\PGL_{3}(\mathbb{C})=\Aut(\mathbb{P}_{\mathbb{C}}^{2})\hookrightarrow
\Aut(\mathbb{C}(X_{1},X_{2})/\mathbb{C}),
\]
but this is very far from being surjective. When there are even more variables
$X$, the group is not known. The group $\Aut(\mathbb{C}(X_{1},\ldots
,X_{n})/\mathbb{C})$ is the group of birational\emph{ } automorphisms of
projective $n$-space $\mathbb{P}_{\mathbb{C}}^{n}$, and is called the
\emph{Cremona group.}%
\index{group!Cremona}
Its study is part of algebraic geometry (Wikipedia: Cremona group).
\end{example}

In this section, we'll be concerned with the groups $\Aut(E/F)$ when $E$ is a
finite extension of $F$.

\begin{proposition}
\label{ft8} Let $E$ be a splitting field of a separable polynomial $f$ in
$F[X]$; then $\Aut(E/F)$ has order $[E\colon F].$
\end{proposition}

\begin{proof}
As $f$ is separable, it has $\deg f$ distinct roots in $E$. Therefore
Proposition \ref{sf7} shows that the number of $F$-homomorphisms $E\rightarrow
E$ is $[E\colon F]$. Because $E$ is finite over $F$, all such homomorphisms
are isomorphisms.
\end{proof}

\begin{example}
\label{ft9}Consider a simple extension $E=F[\alpha]$, and let $f$ be a
polynomial in $F[X]$ having $\alpha$ as a root. If $\alpha$ is the only root
of $f$ in $E$, then $\Aut(E/F)=1$ by (\ref{sf1}b). For example, if
$\sqrt[3]{2}$ is the real cube root of $2$, then $\Aut(\mathbb{Q}[\sqrt[3]%
{2}]/\mathbb{Q})=1$. As another example, let $F$ be a field of characteristic
$p\neq0$, and let $a$ be an element of $F$ that is not a $p$th power. Let $E$
be a splitting field of $f=X^{p}-a.$ Then $f$ has only one root in $E$ (see
\ref{ft2}), and so $\Aut(E/F)=1$.

These examples show that, in the statement of the proposition, is necessary
that $E$ be a \textit{splitting} field of a \textit{separable}\emph{ }polynomial.
\end{example}

When $G$ is a group of automorphisms of a field $E$, we set
\[
E^{G}=\Inv(G)=\{\alpha\in E\mid\sigma\alpha=\alpha\text{, all }\sigma\in G\}.
\]
It is a subfield of $E$, called the subfield of $G$-\emph{invariants\/}%
\index{invariants}
of $E$ or the \emph{fixed field }%
\index{fixed field}%
of $G$.

In this section, we'll show that, when $E$ is the splitting field of a
separable polynomial in $F[X]$ and $G=\Aut(E/F)$, then the maps
\[
M\mapsto\Aut(E/M),\quad H\mapsto\Inv(H)
\]
give a one-to-one correspondence between the set of intermediate fields $M$,
$F\subset M\subset E$, and the set of subgroups $H$ of $G$.

\begin{theorem}
[E.~Artin]\label{ft10}%
\index{theorem!Artin's}%
Let $G$ be a finite group of automorphisms of a field $E$, then
\[
\lbrack E\colon E^{G}]\leq(G\colon1).
\]

\end{theorem}

\begin{proof}
Let $F=E^{G}$, and let $G=\{\sigma_{1},\ldots,\sigma_{m}\}$ with $\sigma_{1}$
the identity map. It suffices to show that every set $\{\alpha_{1}%
,\ldots,\alpha_{n}\}$ of elements of $E$ with $n>m$ is linearly dependent over
$F$. For such a set, consider the system of linear equations%
\begin{align}
\sigma_{1}(\alpha_{1})X_{1}+\cdots+\sigma_{1}(\alpha_{n})X_{n}  &
=0\nonumber\\
\vdots\qquad\qquad & \label{eq10}\\
\sigma_{m}(\alpha_{1})X_{1}+\cdots+\sigma_{m}(\alpha_{n})X_{n}  &  =0\nonumber
\end{align}
with coefficients in $E$. There are $m$ equations and $n>m$ unknowns, and
hence there are nontrivial solutions in $E$. We choose one $(c_{1}%
,\ldots,c_{n})$ having the fewest possible nonzero elements. After renumbering
the $\alpha_{i}$, we may suppose that $c_{1}\neq0$, and then, after
multiplying by a scalar, that $c_{1}\in F$. With these normalizations, we'll
show that all $c_{i}\in F$, and so the first equation
\[
\alpha_{1}c_{1}+\cdots+\alpha_{n}c_{n}=0
\]
(recall that $\sigma_{1}$ is the identity map) is a linear relation on the
$\alpha_{i}$.

If not all $c_{i}$ are in $F$, then $\sigma_{k}(c_{i})\neq c_{i}$ for some
$k\neq1$ and $i\neq1$. On applying $\sigma_{k}$ to the system of linear
equations
\begin{align*}
\sigma_{1}(\alpha_{1})c_{1}+\cdots+\sigma_{1}(\alpha_{n})c_{n}  &  =0\\
\vdots\qquad\qquad & \\
\sigma_{m}(\alpha_{1})c_{1}+\cdots+\sigma_{m}(\alpha_{n})c_{n}  &  =0
\end{align*}
and using that $\{\sigma_{k}\sigma_{1},\ldots,\sigma_{k}\sigma_{m}%
\}=\{\sigma_{1},\ldots,\sigma_{m}\}$ ($\sigma_{k}$ merely permutes the
$\sigma_{i}$), we find that
\[
(c_{1},\sigma_{k}(c_{2}),\ldots,\sigma_{k}(c_{i}),\ldots)
\]
is also a solution to the system of equations (\ref{eq10}). On subtracting it
from the first solution, we obtain a solution $(0,\ldots,c_{i}-\sigma
_{k}(c_{i}),\ldots)$, which is nonzero (look at the $i$th entry), but has more
zeros than the first solution (look at the first entry) --- contradiction.
\end{proof}

\begin{corollary}
\label{ft10d}Let $G$ be a finite group of automorphisms of a field $E$; then
\[
G=\Aut(E/E^{G}).
\]

\end{corollary}

\begin{proof}
\noindent As $G\subset\Aut(E/E^{G})$, we have inequalities
\[
\lbrack E\colon E^{G}]\overset{\text{\ref{ft10}}}{\leq}(G\colon1)\leq
(\Aut(E/E^{G})\colon1)\overset{\text{\ref{sf8}a}}{\leq}[E\colon E^{G}].
\]
All the inequalities must be equalities, and so $G=\Aut(E/E^{G}).$
\end{proof}

\section{Separable, normal, and Galois extensions}

\begin{definition}
\label{ft10m}An algebraic extension $E/F$ is \emph{separable\/}%
\index{extension!separable}
if the minimal polynomial of every element of $E$ is separable; otherwise, it
is \emph{inseparable}%
\index{extension!inseparable}%
.
\end{definition}

Thus, an algebraic extension $E/F$ is separable if every irreducible
polynomial in $F[X]$ having at least one root in $E$ is separable, and it is
inseparable if

\begin{itemize}
\item $F$ is nonperfect, and in particular has characteristic $p\neq0$,
\textit{and }

\item there is an element $\alpha$ of $E$ whose minimal polynomial is of the
form $g(X^{p})$, $g\in F[X]$.
\end{itemize}

\noindent See \ref{ft4} \textit{et seq}. For example, the extension
$\mathbb{F}_{p}(T)$ of $\mathbb{F}_{p}(T^{p})$ is inseparable extension
because $T$ has minimal polynomial $X^{p}-T^{p}$.

\begin{definition}
\label{ft10n}An extension $E/F$ is \emph{normal\/}%
\index{extension!normal}%
\footnote{Bourbaki says \textquotedblleft quasi-galoisienne\textquotedblright%
.} if it is algebraic and the minimal polynomial of every element of $E$
splits in $E[X]$.
\end{definition}

In other words, an algebraic extension $E/F$ is normal if and only if every
irreducible polynomial $f\in F[X]$ having at least one root in $E$ splits in
$E[X]$.

Let $f$ be a monic irreducible polynomial of degree $m$ in $F[X]$, and let $E$
be an algebraic extension of $F$. If $f$ has a root in $E$, so that it is the
minimal polynomial of an element of $E$, then
\[
\renewcommand{\arraystretch}{1.3}\left.
\begin{array}
[c]{lcl}%
E/F\text{\ separable } & \implies & \text{\ }f\text{\ has only simple roots}\\
E/F\text{\ normal } & \implies & f\text{\ splits in }E
\end{array}
\right\}  \implies f\text{\ has }m\text{\ distinct roots in }E.
\]
It follows that $E/F$ is separable and normal if and only if the minimal
polynomial of every element $\alpha$ of $E$ has $[F[\alpha]\colon F]$ distinct
roots in $E$.

\begin{example}
\label{ft11}(a) The polynomial $X^{3}-2{}$ has one real root $\sqrt[3]{2}$ and
two nonreal roots in $\mathbb{C}{}$. Therefore the extension $\mathbb{Q}%
[\sqrt[3]{2}]/\mathbb{Q}{}$ (which is separable) is not normal.

(b) The extension $\mathbb{F}_{p}(T)/\mathbb{F}_{p}(T^{p})$ (which is normal)
is not separable because the minimal polynomial of $T$ is not separable.
\end{example}

\begin{theorem}
\label{ft12}%
\index{theorem!Galois extensions}%
For an extension $E/F$, the following statements are equivalent:

\begin{enumerate}
\item $E$ is the splitting field of a separable polynomial $f\in F[X]$;

\item $E$ is finite over $F$ and $F=E^{\Aut(E/F)}$;

\item $F=E^{G}$ for some finite group $G$ of automorphisms of $E$;

\item $E$ is normal, separable, and finite over $F$.
\end{enumerate}
\end{theorem}

\begin{proof}
(a) $\Rightarrow$ (b). Certainly, $E$ is finite over $F$. Let $F^{\prime
}=E^{\Aut(E/F)}\supset F$. We have to show that $F^{\prime}=F$. Note that $E$
is also the splitting field of $f$ regarded as a polynomial with coefficients
in $F^{\prime}$, and that $f$ is still separable when it is regarded in this
way. Hence
\[
\left\vert \Aut(E/F^{\prime})\right\vert \overset{\text{\ref{ft8}}}{=}[E\colon
F^{\prime}]\leq\lbrack E\colon F]\overset{\text{\ref{ft8}}}{=}\left\vert
\Aut(E/F)\right\vert .
\]
According to Corollary \ref{ft10d}, $\Aut(E/F)=\Aut(E/F^{\prime})$, and so
$[E\colon F^{\prime}]=[E\colon F]$ and $F^{\prime}=F$.

(b) $\Rightarrow$ (c). Let $G=\Aut(E/F)$. We are given that $F=E^{G}$, and $G$
is finite because $E$ is finite over $F$ (apply \ref{sf8}a).

(c) $\Rightarrow$ (d). According to Theorem \ref{ft10}, $[E\colon
F]\leq(G\colon1)$; in particular, $E/F$ is finite. Let $\alpha\in E$, and let
$f$ be the minimal polynomial of $\alpha$; we have to show that $f$ splits
into distinct factors in $E[X]$. Let $\{\alpha_{1}=\alpha,\alpha
_{2},...,\alpha_{m}\}$ be the orbit of $\alpha$ under the action of $G$ on $E$
(so the $\alpha_{i}$ are distinct elements of $E$), and let
\[
g(X)=\prod\nolimits_{i=1}^{m}(X-\alpha_{i})=X^{m}+a_{1}X^{m-1}+\cdots+a_{m}.
\]
The coefficients $a_{j}$ are symmetric polynomials in the $\alpha_{i}$, and
each $\sigma\in G$ permutes the $\alpha_{i}$, and so $\sigma a_{j}=a_{j}$ for
all $j$. Thus $g(X)\in F[X]$. As it is monic and $g(\alpha)=0$, it is
divisible by $f$ (see the definition of minimal polynomial, p.\thinspace
\pageref{minimal}). Let $\alpha_{i}=\sigma\alpha$; on applying $\sigma$ to the
equation $f(\alpha)=0$ we find that $f(\alpha_{i})=0$. Therefore every
$\alpha_{i}$ is a root of $f$, and so $g$ divides $f$. Hence $f=g$, and we
conclude that $f(X)$ splits into distinct factors in $E$.

(d) $\Rightarrow$ (a). Because $E$ has finite degree over $F$, it is generated
over $F$ by a finite number of elements, say, $E=F[\alpha_{1},...,\alpha_{m}%
]$, $\alpha_{i}\in E$, $\alpha_{i}$ algebraic over $F$. Let $f_{i}$ be the
minimal polynomial of $\alpha_{i}$ over $F$, and let $f$ be the product of the
distinct $f_{i}$. Because $E$ is normal over $F$, each $f_{i}$ splits in $E$,
and so $E$ is the splitting field of $f.$ Because $E$ is separable over $F$,
each $f_{i}$ is separable, and so $f$ is separable.
\end{proof}

\begin{definition}
\label{ft11m}An extension $E/F$ of fields is \emph{Galois}%
\index{extension!Galois}
if it satisfies the equivalent conditions of (\ref{ft12}). When $E/F$ is
Galois, $\Aut(E/F)$ is called the \emph{Galois group }%
\index{Galois group}
of $E$ over $F$, and it is denoted by $\Gal(E/F)$.
\end{definition}

\begin{remark}
\label{ft13}(a) Let $E$ be Galois over $F$ with Galois group $G$, and let
$\alpha\in E$. The elements $\alpha_{1}$, $\alpha_{2},...,\alpha_{m}$ of the
orbit of $\alpha$ under $G$ are called the \emph{conjugates\/}%
\index{conjugates}
of $\alpha$. In the course of proving the theorem we showed that the minimal
polynomial of $\alpha$ is $\prod(X-\alpha_{i})$, i.e., the conjugates of
$\alpha$ are exactly the roots of its minimal polynomial in $E$.

(b) Let $G$ be a finite group of automorphisms of a field $E$, and let
$F=E^{G}$. By definition, $E$ is Galois over $F$. Moreover, $\Gal(E/F)=G$
(apply \ref{ft10d}) and $[E\colon F]=|\Gal(E/F)|$ (apply \ref{ft8}).
\end{remark}

\begin{corollary}
\label{ft14}Every finite separable extension $E$ of $F$ is contained in a
Galois extension.
\end{corollary}

\begin{proof}
Let $E=F[\alpha_{1},...,\alpha_{m}]$, and let $f_{i}$ be the minimal
polynomial of $\alpha_{i}$ over $F$. The product of the distinct $f_{i}$ is a
separable polynomial in $F[X]$ whose splitting field is a Galois extension of
$F$ containing $E$.
\end{proof}

\begin{corollary}
\label{ft15}Let $E\supset M\supset F$; if $E$ is Galois over $F$, then it is
Galois over $M.$
\end{corollary}

\begin{proof}
We know $E$ is the splitting field of some separable $f\in F[X]$; it is also
the splitting field of $f$ regarded as an element of $M[X].$
\end{proof}

\begin{remark}
\label{ft16}An element $\alpha$ of an algebraic extension of $F$ is said to be
\emph{separable\/}%
\index{element!separable}
over $F$ if its minimal polynomial over $F$ is separable. The proof of
Corollary \ref{ft14} shows that every finite extension generated by separable
elements is separable. Therefore, the elements of an algebraic extension $E$
of $F$ that are separable over $F$ form a subfield $E_{\text{sep}}$ of $E$
that is separable over $F$. When $E$ is finite over $F$, we let $[E\colon
F]_{\text{sep}}=[E_{\text{sep}}\colon F]$ and call it the \emph{separable
degree}%
\index{degree!separable}%
\emph{\/} of $E$ over $F$.

An algebraic extension $E$ is \emph{purely inseparable} over $F$ if the only
elements of $E$ separable over $F$ are the elements of $F$. If $E$ is a finite
extension of $F$, then $E$ is purely inseparable over $E_{\mathrm{sep}}$. See
Jacobson 1964, Chap. I, Section 10, for more on this topic.
\end{remark}

\begin{definition}
\label{ft21}An extension $E$ of $F$ is
\index{extension!cyclic}%
\index{extension!abelian}%
\index{extension!solvable}%
\emph{cyclic }(resp.\ \emph{abelian}, resp.\ \emph{solvable\/}, etc.$)$ if it
is Galois with cyclic (resp.\ abelian, resp.\ solvable, etc.) Galois group.
\end{definition}

\section{The fundamental theorem of Galois theory}

Let $E$ be an extension of $F$. A \emph{subextension} of $E/F$ is an extension
$M/F$ with $M\subset E$, i.e., a field $M$ with $F\subset M\subset E$. When
$E$ is Galois over $F$, the subextensions of $E/F$ are in one-to-one
correspondence with the subgroups of $\Gal(E/F)$. More precisely, there is the
following statement.

\begin{theorem}
[Fundamental theorem of Galois theory]\label{ft17}%
\index{theorem!fundamental of Galois theory}%
Let $E$ be a Galois extension of $F$ with Galois group $G$. The map $H\mapsto
E^{H}$ is a bijection from the set of subgroups of $G$ to the set of
subextensions of $E/F$,
\[
\{\text{subgroups }H\text{ of }G\}\overset{1\colon1}{\leftrightarrow
}\{\text{subextensions }F\subset M\subset E\},
\]
with inverse $M\mapsto\Gal(E/M)$. Moreover,

\begin{enumerate}
\item the correspondence is inclusion-reversing: $H_{1}\supset H_{2}\iff
E^{H_{1}}\subset E^{H_{2}};$

\item indexes equal degrees: $(H_{1}\colon H_{2})=[E^{H_{2}}\colon E^{H_{1}}]$;

\item $\sigma H\sigma^{-1}\leftrightarrow\sigma M$, i.e., $E^{\sigma
H\sigma^{-1}}=\sigma(E^{H})$; $\Gal(E/\sigma M)=\sigma\Gal(E/M)\sigma^{-1}.$

\item $H$ is normal in $G\iff E^{H}$ is normal (hence Galois) over $F$, in
which case
\[
\Gal(E^{H}/F)\simeq G/H.
\]

\end{enumerate}
\end{theorem}

\begin{proof}
For the first statement, we have to show that $H\mapsto E^{H}$ and
$M\mapsto\Gal(E/M)$ are inverse maps. Let $H$ be a subgroup of $G$. Then,
Corollary \ref{ft10d} shows that $\Gal(E/E^{H})=H$. Let $M/F$ be a
subextension. Then $E$ is Galois over $M$ by (\ref{ft15}), which means that
$E^{\Gal(E/M)}=M\,$.

(a) We have the obvious implications,
\[
H_{1}\supset H_{2}\implies E^{H_{1}}\subset E^{H_{2}}\implies\Gal(E/E^{H_{1}%
})\supset\Gal(E/E^{H_{2}}).
\]
As $\Gal(E/E^{H_{i}})=H_{i}$, this proves (a).

(b) Let $H$ be a subgroup of $G$. According to \ref{ft13}b, \qquad\
\[
(\Gal(E/E^{H})\colon1)=[E\colon E^{H}].
\]
This proves (b) in the case $H_{2}=1$, and the general case follows, using
that
\begin{align*}
(H_{1}\colon1)  &  \overset{\phantom{1.20}}{=}(H_{1}\colon H_{2})(H_{2}%
\colon1)\quad\\
\lbrack E\colon E^{H_{1}}]  &  \overset{\text{\ref{ef10}}}{=}[E\colon
E^{H_{2}}][E^{H_{2}}\colon E^{H_{1}}].
\end{align*}


(c) For $\tau\in G$ and $\alpha\in E$,
\[
\tau\alpha=\alpha\iff\sigma\tau\sigma^{-1}(\sigma\alpha)=\sigma\alpha.
\]
Therefore, $\tau$ fixes $M$ if and only if $\sigma\tau\sigma^{-1}$ fixes
$\sigma M\,$, and so $\sigma\Gal(E/M)\sigma^{-1}=\Gal(E/\sigma M)$. This shows
that $\sigma\Gal(E/M)\sigma^{-1}$ corresponds to $\sigma M.$

(d) Let $H$ be a normal subgroup of $G$. Because $\sigma H\sigma^{-1}=H$ for
all $\sigma\in G$, we must have $\sigma E^{H}=E^{H}$ for all $\sigma\in G$,
i.e., the action of $G$ on $E$ stabilizes $E^{H}$. We therefore have a
homomorphism
\[
\sigma\mapsto\sigma|E^{H}\colon G\rightarrow\Aut(E^{H}/F)
\]
whose kernel is $H$. As $(E^{H})^{G/H}=F$, we see that $E^{H}$ is Galois over
$F$ (by Theorem \ref{ft12}) and that $G/H\simeq\Gal(E^{H}/F)$ (by \ref{ft13}b).

Conversely, suppose that $M$ is normal over $F$, and let $\alpha_{1}%
,\ldots,\alpha_{m}$ generate $M$ over $F$. For $\sigma\in G$, $\sigma
\alpha_{i}$ is a root of the minimal polynomial of $\alpha_{i}$ over $F$, and
so lies in $M$. Hence $\sigma M=M$, and this implies that $\sigma H\sigma
^{-1}=H$ (by (c)).
\end{proof}

\begin{remark}
\label{ft18}Let $E/F$ be a Galois extension, so that there is an order
reversing bijection between the subextensions of $E/F$ and the subgroups of
$G$. From this, we can read off the following results.

(a) Let $M_{1},M_{2},\ldots,M_{r}$ be subextensions of $E/F$, and let $H_{i}$
be the subgroup corresponding to $M_{i}$ (i.e., $H_{i}=\Gal(E/M_{i})$). Then
(by definition) $M_{1}M_{2}\cdots M_{r}$ is the smallest field containing all
$M_{i}$; hence it must correspond to the largest subgroup contained in all
$H_{i}$, which is $\bigcap H_{i}$. Therefore
\[
\Gal(E/M_{1}\cdots M_{r})=H_{1}\cap...\cap H_{r}.
\]


(b) Let $H$ be a subgroup of $G$ and let $M=E^{H}$. The largest normal
subgroup contained in $H$ is $N=\bigcap\nolimits_{\sigma\in G}\sigma
H\sigma^{-1}$ (see GT, 4.1), and so $E^{N}$ is the smallest normal
extension of $F$ containing $M$. Note that, by (a), $E^{N}$ is the composite
of the fields $\sigma M$. It is called the \emph{normal}, or \emph{Galois},
closure of $M$ in $E$.%
\index{normal closure}%
\index{Galois closure}%

\end{remark}

\begin{proposition}
\label{ft18f}\ Let $E$ and $L$ be extensions of $F$ contained in some common
field. If $E/F$ is Galois, then $EL/L$ and $E/E\cap L$ are Galois, and the
map
\[
\sigma\mapsto\sigma|E\colon\Gal(EL/L)\rightarrow\Gal(E/E\cap L)
\]
is an isomorphism.
\end{proposition}

\begin{proof}
Because $E$ is Galois over $F$, it is the splitting field of a separable polynomial

\noindent\begin{minipage}[t]{4.5in}
$f\in F[X]$. Then $EL$ is the splitting field of $f$ over
$L$, and $E$ is the splitting field of $f$ over $E\cap L$. Hence $EL/L$ and
$E/E\cap L$ are Galois.
Every automorphism $\sigma$ of $EL$ fixing the elements of $L$ maps roots of $f$
to roots of $f$, and so $\sigma E=E$. There is therefore a homomorphism%
\[
\sigma\mapsto\sigma|E\colon\Gal(EL/L)\rightarrow\Gal(E/E\cap L)\text{.}%
\]
If $\sigma\in\Gal(EL/L)$ fixes the elements of $E$, then it fixes the elements
of $EL$, and hence is the identity map. Thus, $\sigma\mapsto\sigma|E$ is injective.
If $\alpha\in$ $E$ is fixed by all $\sigma\in\Gal(EL/L)$, then $\alpha\in
E\cap L$. By Corollary \ref{ft10d},
\end{minipage}
\begin{minipage}[t]{1.5in}
\begin{tikzpicture}[baseline=(current bounding box.north)]
\matrix(m)[matrix of math nodes, row sep=1.5em, column sep=0.5em,
text height=1.5ex, text depth=0.25ex]
{&EL\\
E&&L\\
&E\cap L\\
&F\\};
\path[-,font=\scriptsize]
(m-1-2) edge  (m-2-1)
edge node[right] {$=$} (m-2-3)
(m-3-2) edge node[right] {$=$} (m-2-1)
edge  (m-2-3)
edge  (m-4-2);
\end{tikzpicture}
\end{minipage}


\noindent this implies that the image of $\sigma\mapsto\sigma|E$ is
$\Gal(E/E\cap L)$.
\end{proof}

\begin{corollary}
\label{ft18g}Suppose, in the proposition, that $L$ is finite over $F$. Then%
\[
\lbrack EL\colon F]=\frac{[E\colon F][L\colon F]}{[E\cap L\colon F]}\text{.}%
\]

\end{corollary}

\begin{proof}
According to Proposition \ref{ef10},%
\[
\lbrack EL\colon F]=[EL\colon L][L\colon F],
\]
but%
\[
\lbrack EL\colon L]\overset{\ref{ft18f}}{=}[E\colon E\cap
L]\overset{\ref{ef10}}{=}\frac{[E\colon F]}{[E\cap L\colon F]}\text{.}%
\]

\end{proof}

\begin{proposition}
\label{ft18h}Let $E_{1}$ and $E_{2}$ be extensions of $F$ contained in some
common field. If $E_{1}$ and $E_{2}$ are Galois over $F$, then $E_{1}E_{2}$
and $E_{1}\cap E_{2}$ are Galois over $F$, and the map
\[
\sigma\mapsto(\sigma|E_{1},\sigma|E_{2})\colon\Gal(E_{1}E_{2}/F)\rightarrow
\Gal(E_{1}/F)\times\!\Gal(E_{2}/F)
\]
is an isomorphism of $\Gal(E_{1}E_{2}/F)$ onto the subgroup%
\[
H=\{(\sigma_{1},\sigma_{2})\mid\sigma_{1}|E_{1}\cap E_{2}=\sigma_{2}|E_{1}\cap
E_{2}\}
\]
of $\Gal(E_{1}/F)\times\!\Gal(E_{2}/F)$.
\end{proposition}

\noindent\textsc{Proof: }Let $a\in E_{1}\cap E_{2}$, and let $f$ be its
minimal polynomial over $F$. Then $f$ has

\noindent\begin{minipage}[t]{4.0in}
$\deg f$ distinct roots in $E_{1}$ and $\deg
f$ distinct roots in $E_{2}$. Since $f$ can have at most $\deg f$ roots in
$E_{1}E_{2}$, it follows that it has $\deg f$ distinct roots in $E_{1}\cap
E_{2}$. This shows that $E_{1}\cap E_{2}$ is normal and separable over $F$,
and hence Galois (\ref{ft12}).
As $E_{1}$ and $E_{2}$ are Galois over $F$, they are splitting fields for
separable polynomials $f_{1},f_{2}\in F[X]$. Now $E_{1}E_{2}$ is a splitting
field for $\textrm{lcm}(f_{1},f_{2})$, and hence it also is Galois over $F$.
The map $\sigma\mapsto(\sigma|E_{1},\sigma|E_{2})$ is clearly an injective
homomorphism, and its image is contained in $H$. We'll prove that the image is
the whole of $H$ by counting.
\end{minipage}
\begin{minipage}[t]{1.5in}
\begin{tikzpicture}[baseline=(current bounding box.north)]
\matrix(m)[matrix of math nodes, row sep=1.5em, column sep=0.1em,
text height=1.5ex, text depth=0.25ex]
{&E_1E_2\\
E_1&&E_2\\
&E_1\cap E_2\\
&F\\};
\path[-,font=\scriptsize]
(m-1-2) edge  (m-2-1)
edge  (m-2-3)
(m-3-2) edge  (m-2-1)
edge  (m-2-3)
edge  (m-4-2);
\end{tikzpicture}
\end{minipage}


From the fundamental theorem,
\[
\frac{\Gal(E_{2}/F)}{\Gal(E_{2}/E_{1}\cap E_{2})}\simeq\Gal(E_{1}\cap
E_{2}/F)\text{,}%
\]
and so, for each $\sigma_{1}\in\Gal(E_{1}/F)$, $\sigma_{1}|E_{1}\cap E_{2}$
has exactly $[E_{2}\colon E_{1}\cap E_{2}]$ extensions to an element of
$\Gal(E_{2}/F)$. Therefore,
\[
(H\colon1)=[E_{1}\colon F][E_{2}\colon E_{1}\cap E_{2}]=\frac{[E_{1}\colon
F]\cdot\lbrack E_{2}\colon F]}{[E_{1}\cap E_{2}\colon F]},
\]
which equals $[E_{1}E_{2}\colon F]$ by (\ref{ft18g})$.\hfill\square$

\section{Examples}

\begin{example}
\label{ft19}We analyse the extension $\mathbb{Q}[\zeta]/\mathbb{Q}$, where
$\zeta$ is a primitive $7$th root of $1$, say $\zeta=e^{2\pi i/7}$.

\smallskip\noindent\begin{minipage}{3.0in}
$\quad$Note that $\mathbb{Q}[\zeta]$ is the splitting field of the polynomial
$X^{7}-1$, and that $\zeta$ has minimal polynomial
\[
X^{6}+X^{5}+X^{4}+X^{3}+X^{2}+X+1
\]
(see \ref{ef31}). Therefore, $\mathbb{Q}{}[\zeta]$ is Galois of degree $6$
over $\mathbb{Q}$. For any $\sigma\in \Gal(\mathbb{Q}[\zeta]/\mathbb{Q})$, $\sigma\zeta=\zeta^{i}$, some $i$,
$1\leq i\leq6$, and the map $\sigma\mapsto i$ defines an isomorphism
$\Gal(\mathbb{Q}[\zeta]/\mathbb{Q})\rightarrow(\mathbb{Z}/7\mathbb{Z}%
)^{\times}$. Let $\sigma$ be the element of $\Gal(\mathbb{Q}[\zeta
]/\mathbb{Q})$ such that $\sigma\zeta=\zeta^{3}$. Then $\sigma$ generates
$\Gal(\mathbb{Q}[\zeta]/\mathbb{Q})$ because the class of $3$ in
$(\mathbb{Z}/7\mathbb{Z})^{\times}$ generates it (the powers of $3$ mod $7$
are $3,2,6,4,5,1$). We investigate the subfields of $\mathbb{Q}[\zeta]$
corresponding to the subgroups $\langle{}\sigma^{3}\rangle$ and $\langle
{}\sigma^{2}\rangle$.
\end{minipage}
\begin{minipage}{2.5in}
\begin{tikzpicture}[descr/.style={fill=white}]
\matrix(m)[matrix of math nodes, row sep=3.0em, column sep=1.6em,
text height=1.5ex, text depth=0.25ex]
{&\mathbb{Q}[\zeta]\\
\mathbb{Q}[\zeta+\bar{\zeta}]&&\mathbb{Q}[\sqrt{-7}]\\
&\mathbb{Q}\\};
\path[-,font=\scriptsize]
(m-1-2) edge  node[descr] {$\langle\sigma^3 \rangle$}(m-2-1)
edge  node[descr] {$\langle\sigma^2 \rangle$} (m-2-3)
(m-3-2) edge  node[descr] {$\langle\sigma\rangle/\langle\sigma^3	 \rangle$}(m-2-1)
edge  node[descr] {$\langle\sigma\rangle/\langle\sigma^2 \rangle$}(m-2-3);
\end{tikzpicture}
\end{minipage}


Note that $\sigma^{3}\zeta=\zeta^{6}=\bar{\zeta}$ (complex conjugate of
$\zeta)$, and so $\zeta+\bar{\zeta}=2\cos\frac{2\pi}{7}$ is fixed by
$\sigma^{3}$. Now $\mathbb{Q}{}[\zeta]\supset\mathbb{Q}{}[\zeta]^{\langle
\sigma^{3}\rangle}\supset\mathbb{Q}{}[\zeta+\bar{\zeta}]\neq\mathbb{Q}{}$, and
so $\mathbb{Q}{}[\zeta]^{\langle\sigma^{3}\rangle}=\mathbb{Q}{}[\zeta
+\bar{\zeta}]$ (look at degrees). As $\langle{}\sigma^{3}\rangle$ is a normal
subgroup of $\langle{}\sigma\rangle$, $\mathbb{Q}[\zeta+\bar{\zeta}]$ is
Galois over $\mathbb{Q}$, with Galois group $\langle{}\sigma\rangle/\langle
{}\sigma^{3}\rangle.$ The conjugates of $\alpha_{1}%
\overset{\df}{=}\zeta+\bar{\zeta}$ are $\alpha_{3}=\zeta
^{3}+\zeta^{-3}$, $\alpha_{2}=\zeta^{2}+\zeta^{-2}$. Direct calculation shows
that
\begin{align*}
\alpha_{1}+\alpha_{2}+\alpha_{3}  &  =\sum\nolimits_{i=1}^{6}\zeta^{i}=-1,\\
\alpha_{1}\alpha_{2}+\alpha_{1}\alpha_{3}+\alpha_{2}\alpha_{3}  &  =-2,\\
\alpha_{1}\alpha_{2}\alpha_{3}  &  =(\zeta+\zeta^{6})(\zeta^{2}+\zeta
^{5})(\zeta^{3}+\zeta^{4})\\
&  =(\zeta+\zeta^{3}+\zeta^{4}+\zeta^{6})(\zeta^{3}+\zeta^{4})\\
&  =(\zeta^{4}+\zeta^{6}+1+\zeta^{2}+\zeta^{5}+1+\zeta+\zeta^{3})\\
&  =1.
\end{align*}
Hence the minimal polynomial\footnote{More directly, on setting $X=\zeta
+\bar{\zeta}$ in%
\[
(X^{3}-3X)+(X^{2}-2)+X+1
\]
one obtains $1+\zeta+\zeta^{2}+\cdots+\zeta^{6}=0$.} of $\zeta+\bar{\zeta}$
is
\[
g(X)=X^{3}+X^{2}-2X-1.
\]
The minimal polynomial of $\cos\frac{2\pi}{7}=\frac{\alpha_{1}}{2}$ is
therefore
\[
\frac{g(2X)}{8}=X^{3}+X^{2}/2-X/2-1/8.
\]


The subfield of $\mathbb{Q}[\zeta]$ corresponding to $\langle{}\sigma
^{2}\rangle$ is generated by $\beta=\zeta+\zeta^{2}+\zeta^{4}$. Let
$\beta^{\prime}=\sigma\beta$. Then $(\beta-\beta^{\prime})^{2}=-7$. Hence the
field fixed by $\langle{}\sigma^{2}\rangle$ is $\mathbb{Q}[\sqrt{-7}].$
\end{example}

\begin{example}
\label{ft20}We compute the Galois group of a splitting field $E$ of
$X^{5}-2\in\mathbb{Q}[X]$.

\noindent\begin{minipage}{4.0in}
\smallskip Recall from Exercise \ref{x7} that $E=\mathbb{Q}[\zeta
,\alpha]$ where $\zeta$ is a primitive $5$th root of $1$, and
$\alpha$ is a root of $X^{5}-2$. For example, we could take $E$ to be the
splitting field of $X^{5}-2$ in $\mathbb{C}$, with $\zeta=e^{2\pi i/5}$ and
$\alpha$ equal to the real $5$th root of $2$. We have the picture at
right, and
\[
\lbrack\mathbb{Q}[\zeta]:\mathbb{Q}]=4,\quad\lbrack\mathbb{Q}[\alpha
]:\mathbb{Q}]=5.
\]
Because $4$ and $5$ are relatively prime,
\[
\lbrack\mathbb{Q}[\zeta,\alpha]:\mathbb{Q}]=20.
\]
{}
\end{minipage}
\begin{minipage}{2.0in}
\begin{tikzpicture}[descr/.style={fill=white}]
\matrix(m)[matrix of math nodes, row sep=2.5em, column sep=1em,
text height=1.5ex, text depth=0.25ex]
{&\mathbb{Q}[\zeta,\alpha]\\
\mathbb{Q}[\zeta]&&\mathbb{Q}[\alpha]\\
&\mathbb{Q}\\};
\path[-,font=\scriptsize]
(m-1-2) edge  node[descr] {$N$}(m-2-1)
edge  node[descr] {$H$} (m-2-3)
(m-3-2) edge  node[descr] {$G/N$}(m-2-1)
edge  (m-2-3);
\end{tikzpicture}
\end{minipage}
Hence $G=\Gal(\mathbb{Q}[\zeta,\alpha]/\mathbb{Q})$ has order $20$, and the
subgroups $N$ and $H$ fixing $\mathbb{Q}[\zeta]$ and $\mathbb{Q}[\alpha]$ have
orders $5$ and $4$ respectively. Because $\mathbb{Q}[\zeta]$ is normal over
$\mathbb{Q}$ (it is the splitting field of $X^{5}-1$), $N$ is normal in $G$.
Because $\mathbb{Q}[\zeta]\cdot\mathbb{Q}[\alpha]=\mathbb{Q}[\zeta,\alpha]$,
we have $H\cap N=1$, and so $G=N\rtimes_{\theta}H$. Moreover, $H\simeq
G/N\simeq(\mathbb{Z}/5\mathbb{Z})^{\times}$, which is cyclic, being generated
by the class of $2$. Let $\tau$ be the generator of $H$ corresponding to $2$
under this isomorphism, and let $\sigma$ be a generator of $N$. Thus
$\sigma(\alpha)$ is another root of $X^{5}-2$, which we can take to be
$\zeta\alpha$ (after possibly replacing $\sigma$ by a power). Hence:
\[
\left\{
\begin{array}
[c]{rcl}%
\tau\zeta & = & \zeta^{2}\\
\tau\alpha & = & \alpha
\end{array}
\right.  \left\{
\begin{array}
[c]{rcl}%
\sigma\zeta & = & \zeta\\
\sigma\alpha & = & \zeta\alpha.
\end{array}
\right.
\]
Note that $\tau\sigma\tau^{-1}(\alpha)=\tau\sigma\alpha=\tau(\zeta
\alpha)=\zeta^{2}\alpha$ and it fixes $\zeta$; therefore $\tau\sigma\tau
^{-1}=\sigma^{2}$. Thus $G$ has generators $\sigma$ and $\tau$ and defining
relations
\[
\sigma^{5}=1,\quad\tau^{4}=1,\quad\tau\sigma\tau^{-1}=\sigma^{2}.
\]
The subgroup $H$ has five conjugates, which correspond to the five fields
$\mathbb{Q}[\zeta^{i}\alpha]$,
\[
\sigma^{i}H\sigma^{-i}\leftrightarrow\sigma^{i}\mathbb{Q}[\alpha
]=\mathbb{Q}[\zeta^{i}\alpha],\qquad1\leq i\leq5.
\]

\end{example}

\section{Constructible numbers revisited}

Earlier (\ref{ef26}) we showed that a real number $\alpha$ is constructible%
\index{constructible}
if and only if it is contained in a subfield of $\mathbb{R}{}$ of the form
$\mathbb{Q}[\sqrt{a_{1}},\ldots,\sqrt{a_{r}}]$ with each $a_{i}$ a positive
element of $\mathbb{Q}{}[\sqrt{a_{1}},\ldots,\sqrt{a_{i-1}}]$. In particular
\begin{equation}
\alpha\text{\ constructible }\implies\lbrack\mathbb{Q}[\alpha]\colon
\mathbb{Q}]=2^{s}\text{\ some }s. \label{e35}%
\end{equation}
Now we can prove a partial converse to this last statement.

\begin{theorem}
\label{ft22}%
\index{theorem!constructible numbers}%
If $\alpha$ is contained in a subfield of $\mathbb{R}{}$ that is Galois of
degree $2^{r}$ over $\mathbb{Q}$, then it is constructible.
\end{theorem}

\begin{proof}
Suppose $\alpha\in E\subset\mathbb{R}{}$ where $E$ is Galois of degree $2^{r}$
over $\mathbb{Q}$, and let $G=\Gal(E/\mathbb{Q})$. Because finite $p$-groups
are solvable (GT, 6.7), there exists a sequence of groups
\[
\{1\}=G_{0}\subset G_{1}\subset G_{2}\subset\cdots\subset G_{r}=G
\]
with $G_{i}/G_{i-1}$ of order $2$. Correspondingly, there will be a sequence
of fields,
\[
E=E_{0}\supset E_{1}\supset E_{2}\supset\cdots\supset E_{r}=\mathbb{Q}{}%
\]
with $E_{i-1}$ of degree $2$ over $E_{i}$. The next lemma shows that
$E_{i}=E_{i-1}[\sqrt{a_{i}}]$ for some $a_{i}\in E_{i-1}$, and $a_{i}>0$
because otherwise $E_{i}$ would not be real. This proves the theorem.
\end{proof}

\begin{lemma}
\label{ft23}Let $E/F$ be a quadratic extension of fields of characteristic
$\neq2$. Then $E=F[\sqrt{d}]$ for some $d\in F$.
\end{lemma}

\begin{proof}
Let $\alpha\in E$, $\alpha\notin F$, and let $X^{2}+bX+c$ be the minimal
polynomial of $\alpha$. Then $\alpha=\frac{-b\pm\sqrt{b^{2}-4c}}{2}$, and so
$E=F[\sqrt{b^{2}-4c}]$.
\end{proof}

\begin{corollary}
\label{ft24}If $p$ is a prime of the form $2^{k}+1$, then $\cos\frac{2\pi}{p}$
is constructible.
\end{corollary}

\begin{proof}
The field $\mathbb{Q}[e^{2\pi i/p}]$ is Galois over $\mathbb{Q}$ with Galois
group $G\simeq(\mathbb{Z}/p\mathbb{Z})^{\times}$, which has order $p-1=2^{k}$.
The field $\mathbb{Q}{}[\cos\frac{2\pi}{p}]$ is contained in $\mathbb{Q}%
[e^{2\pi i/p}]$, and therefore is Galois of degree dividing $2^{k}$
(fundamental theorem \ref{ft17} and \ref{ef10}). As $\mathbb{Q}{}[\cos
\frac{2\pi}{p}]$ is a subfield of $\mathbb{R}{}$, we can apply the theorem.
\end{proof}

Thus a regular $p$-gon, $p$ prime, is constructible if and only if $p$ is a
Fermat prime, i.e., of the form $2^{2^{r}}+1$. For example, we have proved
that the regular $65537$-polygon is constructible, without (happily) having to
exhibit an explicit formula for $\cos\frac{2\pi}{65537}$.

\begin{remark}
\label{ft23r}The converse to (\ref{e35}) is false; in particular, there are
nonconstructible algebraic numbers of degree $4$ over $\mathbb{Q}{}$. The
polynomial $f(X)=$ $X^{4}-4X+2\in\mathbb{Q}{}[X]$ is irreducible, and we'll
show below (\ref{cg8a}) that the Galois group of a splitting field $E{}$ for
$f$ is $S_{4}$. Each root of $f(X)$ lies in an extension of degree $2^{2}$ of
$\mathbb{Q}{}$ . If the four roots of $f(X)$ were constructible, then all the
elements of $E$ would be constructible (\ref{ef26}a), but if $H$ denotes a
Sylow $2$-subgroup of $S_{4}$, then $E^{H}$ has odd degree over $\mathbb{Q}{}%
$, and so no element of $E^{H}\smallsetminus\mathbb{Q}{}$ is
constructible.\footnote{It is possible to prove this without appealing to the
Sylow theorems. If a root $\alpha$ of $f(X)$ were constructible, then there
would exist a tower of quadratic extensions $\mathbb{Q}[\alpha]\supset
M\supset\mathbb{Q}$. By Galois theory, the groups $\Gal(E/M)\supset
\Gal(E/\mathbb{Q}[\alpha])$ have orders $12$ and $6$ respectively. As
$\Gal(E/\mathbb{Q})=S_{4}$, $\Gal(E/M)$ would be $A_{4}$. But $A_{4}$ has no
subgroup of order $6$, a contradiction. Thus no root of $f(X)$ is
constructible. (Actually $\Gal(E/\mathbb{Q}[\alpha])=S_{3}$, but that does not
matter here.)}
\end{remark}

\section{The Galois group of a polynomial}

\label{ggp}

If a polynomial $f\in F[X]$ is separable, then its splitting field $F_{f}$ is
Galois over $F$, and we call $\Gal(F_{f}/F)$ the \emph{Galois group}%
\index{Galois group!of a polynomial}
$G_{f}$ of $f.$

Let $f(X)=\prod_{i=1}^{n}(X-\alpha_{i})$ in a splitting field $F_{f}$. We know
that the elements of $\Gal(F_{f}/F)$ map roots of $f$ to roots of $f$, i.e.,
they map the set $\{\alpha_{1},\alpha_{2},\ldots,\alpha_{n}\}$ into itself.
Being automorphisms, they act as permutations on $\{\alpha_{1},\alpha
_{2},\ldots,\alpha_{n}\}$. As the $\alpha_{i}$ generate $F_{f}$ over $F$, an
element of $\Gal(F_{f}/F)$ is uniquely determined by the permutation it
defines. Thus $G_{f}$ can be identified with a subset of $\Sym(\{\alpha
_{1},\alpha_{2},\ldots,\alpha_{n}\})\approx S_{n}$
\index{Sn@$S_{n}$}%
(symmetric group on $n$ symbols). In fact, $G_{f}$ consists exactly of the
permutations $\sigma$ of $\{\alpha_{1},\alpha_{2},\ldots,\alpha_{n}\}$ such
that, for $P\in F[X_{1},\ldots,X_{n}]$,%
\begin{equation}
P(\alpha_{1},\ldots,\alpha_{n})=0\implies P(\sigma\alpha_{1},\ldots
,\sigma\alpha_{n})=0. \label{eq4}%
\end{equation}
To see this, note that the kernel of the map%
\begin{equation}
F[X_{1},\ldots,X_{n}]\rightarrow F_{f},\quad X_{i}\mapsto\alpha_{i},
\label{eq11}%
\end{equation}
consists of the polynomials $P(X_{1},\ldots,X_{n})$ such that $P(\alpha
_{1},\ldots,\alpha_{n})=0$. Let $\sigma$ be a permutation of the $\alpha_{i}$
satisfying the condition (\ref{eq4}). Then the map%
\[
F[X_{1},\ldots,X_{n}]\rightarrow F_{f},\quad X_{i}\mapsto\sigma\alpha_{i},
\]
factors through the map (\ref{eq11}), and defines an $F$-isomorphism
$F_{f}\rightarrow F_{f}$, i.e., an element of the Galois group. This shows
that every permutation satisfying the condition (\ref{eq4}) extends uniquely
to an element of $G_{f}$, and it is obvious that every element of $G_{f}$
arises in this way.

This gives a description of $G_{f}$ not mentioning fields or abstract groups,
neither of which were available to Galois. Note that it shows again that
$(G_{f}\colon1)$, hence $[F_{f}\colon F]$, divides $\deg(f)!.$

\section{Solvability of equations}

For a polynomial $f\in F[X]$, we say that $f(X)=0$ is \emph{solvable in
radicals }%
\index{solvable in radicals}%
if its solutions can be obtained by the algebraic operations of addition,
subtraction, multiplication, division, and the extraction of $m$th roots, or,
more precisely, if there exists a tower of fields%
\[
F=F_{0}\subset F_{1}\subset F_{2}\subset\cdots\subset F_{m}%
\]
such that

\begin{enumerate}
\item $F_{i}=F_{i-1}[\alpha_{i}]$, $\alpha_{i}^{m_{i}}\in F_{i-1}$;

\item $F_{m}$ contains a splitting field for $f.$
\end{enumerate}

\begin{theorem}
[Galois, 1832]\label{ft25}%
\index{theorem!Galois 1832}%
Let $F$ be a field of characteristic zero, and let $f\in F[X]$. The equation
$f(X)=0$ is solvable in radicals if and only if the Galois group of $f$ is solvable.
\end{theorem}

We'll prove this later (\ref{ag23}). Also we'll exhibit polynomials
$f(X)\in\mathbb{Q}[X]$ with Galois group $S_{n}$, which are therefore not
solvable when $n\geq5$ by GT, 4.37.

\begin{remark}
\label{ft26}When $F$ has characteristic $p$, the theorem fails for two reasons,

\begin{enumerate}
\item $f$ need not be separable, and so not have a Galois group;

\item $X^{p}-X-a=0$ need not be solvable in radicals even though it is
separable with abelian Galois group (cf.\ Exercise \ref{x6}).
\end{enumerate}

\noindent If the definition of solvable is changed to allow extensions defined
by polynomials of the type in (b) in the chain, then the theorem holds for
fields $F$ of characteristic $p\neq0$ and separable $f\in F[X]$.
\end{remark}

\begin{nt}
Much of what has been written about Galois is unreliable --- see Tony Rothman,
\textquotedblleft Genius and Biographers: The Fictionalization of Evariste
Galois,\textquotedblright\ Amer. Math. Mon. 89, 84 (1982). For a careful
explanation of Galois's \textquotedblleft Premier
M\'{e}moire\textquotedblright, see Edwards, Harold M., Galois for 21st-century
readers. Notices A.M.S.~59 (2012), no. 7, 912--923.
\end{nt}

\section{Exercises}

\begin{exercise}
\label{x11} Let $F$ be a field of characteristic $0$. Show that $F(X^{2})\cap
F(X^{2}-X)=F$ (intersection inside $F(X)$). [Hint: Find automorphisms $\sigma$
and $\tau$ of $F(X)$, each of order $2$, fixing $F(X^{2})$ and $F(X^{2}-X)$
respectively, and show that $\sigma\tau$ has infinite order.]
\end{exercise}

\begin{exercise}
\label{x12}\footnote{This problem shows that every quadratic extension of
${\mathbb{Q}}$ is contained in a cyclotomic extension of ${\mathbb{Q}}$. The
Kronecker-Weber theorem says that \textit{every} abelian extension of
${\mathbb{Q}}$ is contained in a cyclotomic extension.} Let $p$ be an odd
prime, and let $\zeta$ be a primitive $p$th root of $1$ in $\mathbb{C}$. Let
$E={\mathbb{Q}}[\zeta]$, and let $G=\Gal(E/{\mathbb{Q}})$; thus $G=(\mathbb{Z}%
/(p))^{\times}$. Let $H$ be the subgroup of index $2$ in $G$. Put $\alpha
=\sum_{i\in H}\zeta^{i}$ and $\beta=\sum_{i\in G\setminus H}\zeta^{i}$. Show:

\begin{enumerate}
\item $\alpha$ and $\beta$ are fixed by $H$;

\item if $\sigma\in G\setminus H$, then $\sigma\alpha=\beta$, $\sigma
\beta=\alpha$.
\end{enumerate}

\noindent Thus $\alpha$ and $\beta$ are roots of the polynomial $X^{2}%
+X+\alpha\beta\in{\mathbb{Q}}[X]$. Compute\footnote{Schoof suggests computing
$\alpha-\beta$ instead.} $\alpha\beta$ and show that the fixed field of $H$ is
${\mathbb{Q}}[\sqrt{p}]$ when $p\equiv1\,\mod4$ and ${\mathbb{Q}}[\sqrt{-p}]$
when $p\equiv3\mod4$.
\end{exercise}

\begin{exercise}
\label{x13} Let $M={\mathbb{Q}}[\sqrt{2},\sqrt{3}]$ and $E=M[\sqrt{(\sqrt
{2}+2)(\sqrt{3}+3)}]$ (subfields of $\mathbb{R}$).

\begin{enumerate}
\item Show that $M$ is Galois over ${\mathbb{Q}}$ with Galois group the
$4$-group $C_{2}\times C_{2}$.

\item Show that $E$ is Galois over ${\mathbb{Q}}$ with Galois group the
quaternion group.
\end{enumerate}
\end{exercise}

\begin{exercise}
\label{x13a}Let $E$ be a Galois extension of $F$ with Galois group $G$, and
let $L$ be the fixed field of a subgroup $H$ of $G$. Show that the
automomorphism group of $L/F$ is $N/H$ where $N$ is the normalizer of $H$ in
$G$.
\end{exercise}

\begin{exercise}
\label{x13b}Let $E$ be a finite extension of $F$. Show that the order of
$\Aut(E/F)$ divides the degree $[E\colon F].$
\end{exercise}

\clearpage


\chapter{Computing Galois Groups}

In this chapter, we investigate general methods for computing Galois groups.

\section{When is \texorpdfstring{$G_{f}\subset A_{n}$}{Gf in An}?}

Let $\sigma$ be a permutation of the set $\{1,2,\ldots,n\}$. The pairs $(i,j)$
with $i<j$ but $\sigma(i)>\sigma(j)$ are called the \emph{inversions} of
$\sigma$, and $\sigma$ is said to be \emph{even} or \emph{odd} according as
the number of inversions is even or odd. The \emph{signature} of $\sigma$,
$\sign(\sigma)$, is $+1$ or $-1$ according as $\sigma$ is even or odd. We can
define the signature of a permutation $\sigma$ of any set $S$ of $n$ elements
by choosing a numbering of the set and identifying $\sigma$ with a permutation
of $\{1,\ldots,n\}$. Then $\sign$ is the unique homomorphism
$\Sym(S)\rightarrow\{\pm1\}$ such that $\sign(\sigma)=-1$ for every
transposition. In particular, it is independent of the choice of the
numbering. See GT, 4.25.

Now consider a monic polynomial
\[
f(X)=X^{n}+a_{1}X^{n-1}+\cdots+a_{n}%
\]
and let $f(X)=\prod_{i=1}^{n}(X-\alpha_{i})$ in some splitting field. Set
\[
\Delta(f)=\prod_{1\leq i<j\leq n}(\alpha_{i}-\alpha_{j}),\qquad D(f)=\Delta
(f)^{2}=\prod_{1\leq i<j\leq n}(\alpha_{i}-\alpha_{j})^{2}.
\]
The \emph{discriminant }%
\index{discriminant}%
of\emph{\ }$f$ is defined to be $D(f)$. Note that $D(f)$ is nonzero if and
only if $f$ has only simple roots, i.e., is separable. Let $G_{f}$ be the
Galois group of $f$, and identify it with a subgroup of $\Sym(\{\alpha
_{1},\ldots,\alpha_{n}\})$ (as on p.~\pageref{ggp}).

\begin{proposition}
\label{cg1}Let $f\in F[X]$ be a separable polynomial, and let $\sigma\in
G_{f}$.

\begin{enumerate}
\item $\sigma\Delta(f)=\sign(\sigma)\Delta(f)$, where $\sign(\sigma)$ is the
signature of $\sigma.$

\item $\sigma D(f)=D(f).$
\end{enumerate}
\end{proposition}

\begin{proof}
Each inversion of $\sigma$ introduces a negative sign into $\sigma\Delta(f)$,
and so (a) follows from the definition of $\sign(\sigma)$. The equation in (b)
is obtained by squaring that in (a).
\end{proof}

While $\Delta(f)$ depends on the choice of the numbering of the roots of $f$,
$D(f)$ does not.

\begin{corollary}
\label{cg2}Let $f(X)\in F[X]$ be separable of degree $n$. Let $F_{f}$ be a
splitting field for $f$ and let $G_{f}=\Gal(F_{f}/F)$.

\begin{enumerate}
\item The discriminant $D(f)\in F$.

\item Assume that $F$ has characteristic $\neq 2$. The subfield of $F_{f}$ corresponding to $A_{n}\cap G_{f}$ is
$F[\Delta(f)]$. Hence
\[
G_{f}\subset A_{n}\iff\Delta(f)\in F\iff D(f)\text{\ is a square in }F.
\]

\end{enumerate}
\end{corollary}

\begin{proof}
(a) The discriminant of $f$ is an element of $F_{f}$ fixed by $G_{f}%
\overset{\df}{=}\Gal(F_{f}/F)$, and hence lies in $F$ (by the
fundamental theorem).

(b) Because $f$ has simple roots, $\Delta(f)\neq0$, and so the formula
$\sigma\Delta(f)=\sign(\sigma)\Delta(f)$ shows that an element of $G_{f}$
fixes $\Delta(f)$ if and only if it lies in $A_{n}$. Thus, under the Galois
correspondence,%
\[
G_{f}\cap A_{n}\leftrightarrow F[\Delta(f)]\text{.}%
\]
Hence,%
\[
G_{f}\cap A_{n}=G_{f}\iff F[\Delta(f)]=F.
\]

\end{proof}

The roots of $X^{2}\allowbreak+bX+c$ are $\frac{-b\pm\sqrt{b^{2}-4c}}{2}$ and
so%
\begin{align*}
\Delta(X^{2}\allowbreak+bX+c)  &  =\sqrt{b^{2}-4c}\text{ (or }-\sqrt{b^{2}%
-4c}\text{),}\\
D(X^{2}\allowbreak+bX+c)  &  =b^{2}-4c.
\end{align*}
Similarly, \qquad%
\[
D(X^{3}+bX+c)=-4b^{3}-27c^{2}.
\]
By completing the cube, one can put any cubic polynomial in this form (in
characteristic $\neq3$).

Although there is a not a universal formula for the roots of $f$ in terms of
its coefficients when the $\deg(f)>4$, there is for its discriminant. However,
the formulas for the discriminant rapidly become very complicated, for
example, that for $X^{5}+aX^{4}+bX^{3}+cX^{2}+dX+e$ has $59$ terms.
Fortunately, PARI%
\index{PARI}
knows them. For example, typing \verb|poldisc(X^3+a*X^2+b*X+c,X)| returns the
discriminant of $X^{3}+aX^{2}+bX+c$, namely,
\[
-4ca^{3}+b^{2}a^{2}+18cba+(-4b^{3}-27c^{2}).
\]


\begin{remark}
\label{cg3}Suppose $F\subset\mathbb{R}$. Then $D(f)$ will not be a square if
it is negative. It is known that the sign of $D(f)$ is $(-1)^{s}$ where $2s$
is the number of nonreal roots of $f$ in $\mathbb{C}$ (see ANT 2.40). Thus if
$s$ is odd, then $G_{f}$ is not contained in $A_{n}$. This can be proved more
directly by noting that complex conjugation acts on the roots as the product
of $s$ disjoint transpositions.

The converse is not true: when $s$ is even, $G_{f}$ is not necessarily
contained in $A_{n}$.
\end{remark}

\section{When does \texorpdfstring{$G_{f}$}{Gf} act transitively on the
roots?}

\begin{proposition}
\label{cg4} Let $f(X)\in F[X]$ be separable. Then $f(X)$ is irreducible if and
only if $G_{f}$ permutes the roots of $f$ transitively.
\end{proposition}

\begin{proof}
$\implies\colon$ If $\alpha$ and $\beta$ are two roots of $f(X)$ in a
splitting field $F_{f}$ for $f$, then they both have $f(X)$ as their minimal
polynomial, and so $F[\alpha]$ and $F[\beta]$ are both stem fields for $f$.
Hence, there is an $F$-isomorphism
\[
F[\alpha]\simeq F[\beta],\qquad\alpha\leftrightarrow\beta.
\]
Write $F_{f}=F[\alpha_{1},\alpha_{2},...]$ with $\alpha_{1}=\alpha$ and
$\alpha_{2},\alpha_{3},\ldots$ the other roots of $f(X)$. Then the
$F$-homomorphism $\alpha\mapsto\beta\colon F[\alpha]\rightarrow F_{f}$ extends
(step by step) to an $F$-homomorphism $F_{f}\rightarrow F_{f}$ (use
\ref{sf2}b), which is an $F$-isomorphism sending $\alpha$ to $\beta$.

$\impliedby\colon$ Let $g(X)\in F[X]$ be an irreducible factor of $f$, and let
$\alpha$ be one of its roots. If $\beta$ is a second root of $f$, then (by
assumption) $\beta=\sigma\alpha$ for some $\sigma\in G_{f}$. Now, because $g$
has coefficients in $F$,
\[
g(\sigma\alpha)=\sigma g(\alpha)=0,
\]
and so $\beta$ is also a root of $g$. Therefore, every root of $f$ is also a
root of $g$, and so $f(X)=g(X).$
\end{proof}

Note that when $f(X)$ is irreducible of degree $n$, $n|(G_{f}\colon1)$ because
$[F[\alpha]\colon F]=n$ and $[F[\alpha]\colon F]$ divides $[F_{f}\colon
F]=(G_{f}\colon1)$. Thus $G_{f}$ is a transitive subgroup of $S_{n}$ whose
order is divisible by $n$.

\section{Polynomials of degree at most three}

\begin{example}
\label{cg5}Let $f(X)\in F[X]$ be a polynomial of degree $2$. Then $f$ is
inseparable $\iff$ $F$ has characteristic $2$ and $f(X)=X^{2}-a$ for some
$a\in F\smallsetminus F^{2}$. If $f$ is separable, then $G_{f}=1(=A_{2})$ or
$S_{2}$ according as $D(f)$ is a square in $F$ or not.
\end{example}

\begin{example}
\label{cg6}Let $f(X)\in F[X]$ be a polynomial of degree $3$. We can assume $f$
to be irreducible, for otherwise we are essentially back in the previous case.
Then $f$ is inseparable if and only if $F$ has characteristic $3$ and
$f(X)=X^{3}-a$ for some $a\in F\smallsetminus F^{3}$. If $f$ is separable,
then $G_{f}$ is a transitive subgroup of $S_{3}$ whose order is divisible by
$3$. There are only two possibilities: $G_{f}=A_{3}$ or $S_{3}$ according as
$D(f)$ is a square in $F$ or not. Note that $A_{3}$ is generated by the cycle
$(123)$.

For example, $X^{3}-3X+1$ is irreducible in $\mathbb{Q}{}[X]$ (see \ref{ef5}).
Its discriminant is $-4(-3)^{3}-27=81=9^{2}$, and so its Galois group is
$A_{3}$.

On the other hand, $X^{3}+3X+1\in\mathbb{Q}[X]$ is also irreducible (apply
\ref{ef4}), but its discriminant is $-135$ which is not a square in
$\mathbb{Q}$, and so its Galois group is $S_{3}$.
\end{example}

\section{Quartic polynomials}

Let $f(X)$ be a separable quartic polynomial. In order to determine $G_{f}$
we'll exploit the fact that $S_{4}$ has
\[
V=\{1,(12)(34),(13)(24),(14)(23)\}
\]
as a normal subgroup --- it is normal because it contains all elements of type
$2+2$ (GT, 4.29). Let $E$ be a splitting field of $f$, and let
$f(X)=\prod(X-\alpha_{i})$ in $E$. We identify the Galois group $G_{f}$ of $f$
with a subgroup of the symmetric group $\Sym(\{\alpha_{1},\alpha_{2}%
,\alpha_{3},\alpha_{4}\})$. Consider the partially symmetric elements
\begin{align*}
\alpha &  =\alpha_{1}\alpha_{2}+\alpha_{3}\alpha_{4}\\
\beta &  =\alpha_{1}\alpha_{3}+\alpha_{2}\alpha_{4}\\
\gamma &  =\alpha_{1}\alpha_{4}+\alpha_{2}\alpha_{3}.
\end{align*}
They are distinct because the $\alpha_{i}$ are distinct; for example,
\[
\alpha-\beta=\alpha_{1}(\alpha_{2}-\alpha_{3})+\alpha_{4}(\alpha_{3}%
-\alpha_{2})=(\alpha_{1}-\alpha_{4})(\alpha_{2}-\alpha_{3}).
\]
The group $\Sym(\{\alpha_{1},\alpha_{2},\alpha_{3},\alpha_{4}\})$ permutes
$\{\alpha,\beta,\gamma\}$ transitively. The stabilizer of each of
$\alpha,\beta,\gamma$ must therefore be a subgroup of index $3$ in $S_{4}$,
and hence has order $8$. For example, the stabilizer of $\beta$ is $\langle
{}(1234),(13)\rangle$. Groups of order $8$ in $S_{4}$ are Sylow $2$-subgroups.
There are three of them, all isomorphic to $D_{4}$. By the Sylow theorems, $V$
is contained in a Sylow $2$-subgroup; in fact, because the Sylow $2$-subgroups
are conjugate and $V$ is normal, it is contained in all three. It follows that
$V$ is the intersection of the three Sylow $2$-subgroups. Each Sylow
$2$-subgroup fixes exactly one of $\alpha,\beta,$ or $\gamma$, and therefore
their intersection $V$ is the subgroup of $\Sym(\{\alpha_{1},\alpha_{2}%
,\alpha_{3},\alpha_{4}\})$ fixing $\alpha$, $\beta$, and $\gamma$.\bigskip

\noindent\begin{minipage}{4.0in}
\begin{lemma}
\label{cg7}The fixed field of $G_{f}\cap V$ is $F[\alpha,\beta,\gamma]$. Hence
$F[\alpha,\beta,\gamma]$ is Galois over $F$ with Galois group $G_{f}/G_{f}\cap
V$.
\end{lemma}
\begin{proof}
The above discussion shows that the subgroup of $G_{f}$ of elements fixing
$F[\alpha,\beta,\gamma]$ is $G_{f}\cap V$, and so $E^{G_{f}\cap V}%
=F[\alpha,\beta,\gamma]$ by the fundamental theorem of Galois theory. The
remaining statements follow from the fundamental theorem using that $V$ is normal.
\end{proof}{\smallskip}
\end{minipage}\hspace{0.3in} \begin{minipage}{1in}
\begin{tikzpicture}
\matrix(m)[matrix of math nodes, row sep=2em, column sep=2.5em,
text height=1.5ex, text depth=0.25ex]
{E\\
F[\alpha,\beta,\gamma]\\
F\\};
\path[-,font=\scriptsize]
(m-1-1) edge node[right] {$G_f\cap V$} (m-2-1)
(m-2-1) edge node[right] {$G_f/G_f\cap V$} (m-3-1);
\end{tikzpicture}
\end{minipage}


\medskip Let $M=F[\alpha,\beta,\gamma]$, and let $g(X)=(X-\alpha
)(X-\beta)(X-\gamma)\in M[X]$ --- it is called the \emph{resolvent cubic\/}
\index{cubic!resolvent}%
of $f$. Every permutation of the $\alpha_{i}$ (\textit{a fortiori}, every
element of $G_{f}$) merely permutes $\alpha,\beta,\gamma$, and so fixes
$g(X)$. Therefore (by the fundamental theorem) $g(X)$ has coefficients in $F$.
More explicitly, we have:

\begin{lemma}
\label{cg8}The resolvent cubic of $f=X^{4}+bX^{3}+cX^{2}+dX+e$ is
\[
g=X^{3}-cX^{2}+(bd-4e)X-b^{2}e+4ce-d^{2}.
\]
The discriminants of $f$ and $g$ are equal.
\end{lemma}

\begin{sproof}
Expand $f=(X-\alpha_{1})(X-\alpha_{2})(X-\alpha_{3})(X-\alpha_{4})$ to express
$b,c,d,e$ in terms of $\alpha_{1},\alpha_{2},\alpha_{3},\alpha_{4}$. Expand
$g=(X-\alpha)(X-\beta)(X-\gamma)$ to express the coefficients of $g$ in terms
of $\alpha_{1},\alpha_{2},\alpha_{3},\alpha_{4}$, and substitute to express
them in terms of $b,c,d,e$.
\end{sproof}

Now let $f$ be an irreducible separable quartic. Then $G=G_{f}$ is a
transitive subgroup of $S_{4}$ whose order is divisible by $4$. There are the
following possibilities for $G$:\hfill\break

\begin{center}
\renewcommand{\arraystretch}{1.2}%
\begin{tabular}
[c]{|c|c|c|}\hline
{$G$} & {$(G\cap V\colon1)$} & {$(G\colon V\cap G)$}\\\hline
$S_{4}$ & {$4$} & {$6$}\\\hline
$A_{4}$ & {$4$} & {$3$}\\\hline
$V$ & {$4$} & {$1$}\\\hline
{$D_{4}$} & {$4$} & {$2$}\\\hline
{$C_{4}$} & {$2$} & {$2$}\\\hline
\end{tabular}
$\quad%
\begin{array}
[c]{c}%
(G\cap V\colon1)=[E\colon M]\\
(G\colon V\cap G)=[M\colon F]
\end{array}
$
\end{center}

\noindent The groups of type $D_{4}$ are the Sylow $2$-subgroups discussed
above, and the groups of type $C_{4}$ are those generated by cycles of length
$4$.

We can compute $(G\colon V\cap G)$ from the resolvent cubic $g$, because
$G/V\cap G=\Gal(M/F)$ and $M$ is the splitting field of $g$. Once we know
$(G\colon V\cap G)$, we can deduce $G$ except in the case that it is $2$. If
$[M\colon F]=2$, then $G\cap V=V$ or $C_{2}$. Only the first group acts
transitively on the roots of $f$, and so (from \ref{cg4}) we see that in this
case $G=D_{4}$ or $C_{4}$ according as $f$ is irreducible or not in $M[X]$.

\begin{example}
\label{cg8a}Consider $f(X)=X^{4}-4X+2\in\mathbb{Q}{}[X]$. It is irreducible by
Eisenstein's criterion (\ref{ef7}), and its resolvent cubic is $g(X)=X^{3}%
-8X-16$, which is irreducible because it has no roots in $\mathbb{F}{}_{5}$.
The discriminant of $g(X)$ is $-4864$, which is not a square, and so the
Galois group of $g(X)$ is $S_{3}$. From the table, we see that the Galois
group of $f(X)$ is $S_{4}$.
\end{example}

\begin{example}
\label{cg9}Consider $f(X)=X^{4}+4X^{2}+2\in\mathbb{Q}[X]$. It is irreducible
by Eisenstein's criterion (\ref{ef7}), and its resolvent cubic is
$(X-4)(X^{2}-8)$; thus $M=\mathbb{Q}[\sqrt{2}]$. From the table we see that
$G_{f}$ is of type $D_{4}$ or $C_{4}$, but $f$ factors over $M$ (even as a
polynomial in $X^{2}$), and hence $G_{f}$ is of type $C_{4}$.
\end{example}

\begin{example}
\label{cg10}Consider $f(X)=X^{4}-10X^{2}+4\in\mathbb{Q}[X]$. It is irreducible
in $\mathbb{Q}{}[X]$ because (by inspection) it is irreducible in
$\mathbb{Z}{}[X]$. Its resolvent cubic is $(X+10)(X+4)(X-4)$, and so $G_{f}$
is of type $V$.
\end{example}

\begin{example}
\label{cg11}Consider $f(X)=X^{4}-2\in\mathbb{Q}[X]$. It is irreducible by
Eisenstein's criterion (\ref{ef7}), and its resolvent cubic is $g(X)=X^{3}%
+8X$. Hence $M=\mathbb{Q}[i\sqrt{2}]$. One can check that $f$ is irreducible
over $M$, and $G_{f}$ is of type $D_{4}$.

Alternatively, analyse the equation as in (\ref{ft20}).
\end{example}

As we explained in (\ref{ef18}), PARI%
\index{PARI}
knows how to factor polynomials with coefficients in $\mathbb{Q}[\alpha]$.

\begin{example}
\label{cg11a}(From the web, sci.math.research, search for \textquotedblleft
final analysis\textquotedblright.) Consider $f(X)=X^{4}-2cX^{3}-dX^{2}%
+2cdX-dc^{2}\in\mathbb{Z}{}[X]$ with $a>0$, $b>0$, $c>0$, $a>b$ and
$d=a^{2}-b^{2}$. Let $r=d/c^{2}$ and let $w$ be the unique positive real
number such that $r=w^{3}/(w^{2}+4)$. Let $m$ be the number of roots of $f(X)$
in $\mathbb{Z}{}$ (counted with multiplicities). The Galois group of $f$ is as follows:

\begin{itemize}
\item If $m=0$ and $w$ not rational, then $G$ is $S_{4}$.

\item If $m=1$ and $w$ not rational then $G$ is $S_{3}$.

\item If $w$ is rational and $w^{2}+4$ is not a square then $G=D_{4}$.

\item If $w$ is rational and $w^{2}+4$ is a square then $G=V=C_{2}\times
C_{2}.$
\end{itemize}

\noindent This covers all possible cases. The hard part was to establish that
$m=2$ could never happen.
\end{example}

\begin{aside}
\label{cg11b}For a discussion of whether the method of solving a quartic by
reducing to a cubic generalizes to other even degrees, see mo149099.
\end{aside}

\section{Examples of polynomials with \texorpdfstring{$S_{p}$}{Sp} as Galois
group over \texorpdfstring{$\mathbb{Q}$}{Q}}

\noindent The next lemma gives a criterion for a subgroup of $S_{p}$ to be the
whole of $S_{p}$.

\begin{lemma}
\label{cg12}For $p$ prime, the symmetric group $S_{p}$ is generated by any
transposition and any $p$-cycle.
\end{lemma}

\begin{proof}
After renumbering, we may assume that the transposition is $\tau=(12)$, and we
may write the $p$-cycle $\sigma$ so that $1$ occurs in the first position,
$\sigma=(1\,i_{2}\cdots i_{p})$. Now some power of $\sigma$ will map $1$ to
$2$ and will still be a $p$-cycle (here is where we use that $p$ is prime).
After replacing $\sigma$ with the power, we have $\sigma=(1\,2\,j_{3}\,\ldots
j_{p})$, and after renumbering again, we have $\sigma=(1\,2\,3\ldots p).$ Now%
\[
(i\,\,i+1)=\sigma^{i}(12)\sigma^{-i}%
\]
(see GT, 4.29) and so lies in the subgroup generated by $\sigma$ and
$\tau$. These transpositions generate $S_{p}$.
\end{proof}

\begin{proposition}
\label{cg13} Let $f$ be an irreducible polynomial of prime degree $p$ in
$\mathbb{Q}[X]$. If $f$ splits in $\mathbb{C}$ and has exactly two nonreal
roots, then $G_{f}=S_{p}.$
\end{proposition}

\begin{proof}
Let $E$ be the splitting field of $f$ in $\mathbb{C}{}$, and let $\alpha\in E$
be a root of $f$. Because $f$ is irreducible, $[\mathbb{Q}[\alpha
]\colon\mathbb{Q}]=\deg f=p$, and so $p|[E\colon\mathbb{Q}]=(G_{f}\colon1)$.
Therefore $G_{f}$ contains an element of order $p$ (Cauchy's theorem, GT,
4.13), but the only elements of order $p$ in $S_{p}$ are $p$-cycles
(here we use that $p$ is prime again).

Let $\sigma$ be complex conjugation on $\mathbb{C}$. Then $\sigma$ transposes
the two nonreal roots of $f(X)$ and fixes the rest. Therefore $G_{f}\subset
S_{p}$ and contains a transposition and a $p$-cycle, and so is the whole of
$S_{p}$.
\end{proof}

It remains to construct polynomials satisfying the conditions of the Proposition.

\begin{example}
\label{cg14}Let $p$$\geq5$ be a prime number. Choose a positive even integer
$m$ and even integers
\[
n_{1}<n_{2}<\cdots<n_{p-2},
\]
and let
\[
g(X)=(X^{2}+m)(X-n_{1})...(X-n_{p-2}).
\]
The graph of $g$ crosses the $x$-axis exactly at the points $n_{1}%
,\ldots,n_{p-2}$, and it doesn't have a local maximum or minimum at any of
those points (because the $n_{i}$ are simple roots). Thus $e=\min_{g^{\prime
}(x)=0}|g(x)|>0$, and we can choose an odd positive integer $n$ such that
$\frac{2}{n}<e$.

Consider%
\[
f(X)=g(X)-\frac{2}{n}\text{.}%
\]
As $\frac{2}{n}<e$, the graph of $f$ also crosses the $x$-axis at exactly
$p-2$ points, and so $f$ has exactly two nonreal roots. On the other hand,
when we write%
\[
nf(X)=nX^{p}+a_{1}X^{p-1}+\cdots+a_{p},
\]
the $a_{i}$ are all even and $a_{p}$ is not divisible by $2^{2}$, and so
Eisenstein's criterion implies that $f$ is irreducible. Over $\mathbb{R}{}$,
$f$ has $p-2$ linear factors and one irreducible quadratic factor, and so it
certainly splits over $\mathbb{C}{}$ (high school algebra). Therefore, the
proposition applies to $f$.\footnote{If $m$ is taken sufficiently large, then
$g(X)-2$ will have exactly two nonreal roots, i.e., we can take $n=1$, but the
proof is longer (see Jacobson 1964, p.\thinspace107, who credits the example
to Brauer). The shorter argument in the text was suggested to me by Martin
Ward.}
\end{example}

\begin{example}
\label{cg14a}The reader shouldn't think that, in order to have Galois group
$S_{p}$, a polynomial must have exactly two nonreal roots. For example, the
polynomial $X^{5}-5X^{3}+4X-1$ has Galois group $S_{5}$ but all of its roots
are real.
\end{example}

\section{Finite fields}

Let $\mathbb{F}_{p}=\mathbb{Z}/p\mathbb{Z}$, the field of $p$ elements. As we
noted in \S 1, every field $E$ of characteristic $p$ contains a copy of
$\mathbb{F}_{p}$, namely, $\{m1_{E}\mid m\in\mathbb{Z}\}$. No harm results if
we identify $\mathbb{F}_{p}$ with this subfield of $E$.

Let $E$ be a field of degree $n$ over $\mathbb{F}_{p}$. Then $E$ has $q=p^{n}$
elements, and so $E^{\times}$ is a group of order $q-1$. Therefore the nonzero
elements of $E$ are roots of $X^{q-1}-1$, and \textit{all} elements of $E$ are
roots of $X^{q}-X$. Hence $E$ is a splitting field for $X^{q}-X$, and so any
two fields with $q$ elements are isomorphic.

\begin{proposition}
\label{cg17}Every extension of finite fields is simple.
\end{proposition}

\begin{proof}
Consider $E\supset F$. Then $E^{\times}$ is a finite subgroup of the
multiplicative group of a field, and hence is cyclic (see Exercise \ref{x3}).
If $\zeta$ generates $E^{\times}$ as a multiplicative group, then certainly
$E=F[\zeta]$.
\end{proof}

Now let $E$ be a splitting field of $f(X)=X^{q}-X$, $q=p^{n}$. The derivative
$f^{\prime}(X)=-1$, which is relatively prime to $f(X)$ (in fact, to every
polynomial), and so $f(X)$ has $q$ distinct roots in $E$. Let $S$ be the set
of its roots. Then $S$ is obviously closed under multiplication and the
formation of inverses, but it is also closed under subtraction: if $a^{q}=a$
and $b^{q}=b$, then
\[
(a-b)^{q}=a^{q}-b^{q}=a-b.
\]
Hence $S$ is a field, and so $S=E$. In particular, $E$ has $p^{n}$ elements.

\begin{proposition}
\label{cg15}For each power $q=p^{n}$ of $p$ there exists a field
$\mathbb{F}_{q}$ with $q$ elements. Every such field is a splitting field for
$X^{q}-X$, and so any two are isomorphic. Moreover, $\mathbb{F}_{q}$ is Galois
over $\mathbb{F}_{p}$ with cyclic Galois group generated by the Frobenius
automorphism $\sigma(a)=a^{p}$.
\end{proposition}

\begin{proof}
Only the final statement remains to be proved. The field $\mathbb{F}_{q}$ is
Galois over $\mathbb{F}_{p}$ because it is the splitting field of a separable
polynomial. We noted in \ref{ef3} that $x\overset{\sigma}{\mapsto}x^{p}$ is an
automorphism of $\mathbb{F}_{q}$. An element $a$ of $\mathbb{F}_{q}$ is fixed
by $\sigma$ if and only if $a^{p}=a$, but $\mathbb{F}_{p}$ consists exactly of
such elements, and so the fixed field of $\langle{}\sigma\rangle$ is
$\mathbb{F}_{p}$. This proves that $\mathbb{F}_{q}$ is Galois over
$\mathbb{F}_{p}$ and that $\langle{}\sigma\rangle=\Gal(\mathbb{F}%
_{q}/\mathbb{F}_{p})$ (see \ref{ft13}b).
\end{proof}

\begin{corollary}
\label{cg16}Let $E$ be a field with $p^{n}$ elements. For each divisor $m$ of
$n$, $m\geq0$, $E$ contains exactly one field with $p^{m}$ elements.
\end{corollary}

\begin{proof}
We know that $E$ is Galois over $\mathbb{F}_{p}$ and that $\Gal(E/\mathbb{F}%
_{p})$ is the cyclic group of order $n$ generated by $\sigma$. The group
$\langle{}\sigma\rangle$ has one subgroup of order $n/m$ for each $m$ dividing
$n$, namely, $\langle{}\sigma^{m}\rangle$, and so $E$ has exactly one subfield
of degree $m$ over $\mathbb{F}{}_{p}$ for each $m$ dividing $n$, namely,
$E^{\langle\sigma^{m}\rangle}$. Because it has degree $m$ over $\mathbb{F}%
{}_{p}$, $E^{\langle\sigma^{m}\rangle}$ has $p^{m}$ elements.
\end{proof}

\begin{corollary}
\label{cg18}Each monic irreducible polynomial $f$ of degree $d|n$ in
$\mathbb{F}_{p}[X]$ occurs exactly once as a factor of $X^{p^{n}}-X$; hence,
the degree of the splitting field of $f$ is $\leq d$.
\end{corollary}

\begin{proof}
First, the factors of $X^{p^{n}}-X$ are distinct because it has no common
factor with its derivative. If $f(X)$ is irreducible of degree $d$, then
$f(X)$ has a root in a field of degree $d$ over $\mathbb{F}_{p}$. But the
splitting field of $X^{p^{n}}-X$ contains a copy of every field of degree $d$
over $\mathbb{F}_{p}$ with $d|n$. Hence some root of $X^{p^{n}}-X$ is also a
root of $f(X)$, and therefore $f(X)|X^{p^{n}}-X$. In particular, $f$ divides
$X^{p^{d}}-X$, and therefore it splits in its splitting field, which has
degree $d$ over $\mathbb{F}{}_{p}$.
\end{proof}

\begin{proposition}
\label{cg18m}Let $\mathbb{F}$ be an algebraic closure of $\mathbb{F}_{p}$.
Then $\mathbb{F}$ contains exactly one field $\mathbb{F}_{p^{n}}$ with $p^{n}$
elements for each integer $n\geq1$, and $\mathbb{F}{}_{p^{n}}$ consists of the
roots of $X^{p^{n}}-X$. Moreover,
\[
\mathbb{F}_{p^{m}}\subset\mathbb{F}_{p^{n}}\iff m|n.
\]
The partially ordered set of finite subfields of $\mathbb{F}$ is isomorphic to
the set of integers $n\geq1$ partially ordered by divisibility.
\end{proposition}

\begin{proof}
In fact, the set of roots of $X^{p^{n}}-X$ is a field (see above), with
$p^{n}$ elements, and is the only such subfield. If $\mathbb{F}_{p^{m}}%
\subset\mathbb{F}_{p^{n}}$, say, $[\mathbb{F}{}_{p^{n}}\colon\mathbb{F}%
{}_{p^{m}}]=d$, then $p^{n}=(p^{m})^{d}=p^{md}$, and so $m|n$; the converse
follows from the first statement. The final statement follows from the second statement.
\end{proof}

\begin{proposition}
\label{cg18n}The field $\mathbb{F}{}_{p}$ has an algebraic closure
$\mathbb{F}{}$.
\end{proposition}

\begin{proof}
Choose a sequence of integers $1=n_{1}<n_{2}<n_{3}<\cdots$ such that
$n_{i}|n_{i+1}$ for all $i$, and every integer $n$ divides some $n_{i}$. For
example, let $n_{i}=i!$. Define the fields $\mathbb{F}{}_{p^{n_{i}}}$
inductively as follows: $\mathbb{F}{}_{p^{n_{1}}}=\mathbb{F}{}_{p}$;
$\mathbb{F}{}_{p^{n_{i}}}$ is the splitting field of $X^{p^{n_{i}}}-X$ over
$\mathbb{F}{}_{p^{n_{i-1}}}$. Then, $\mathbb{F}{}_{p^{n_{1}}}\subset
\mathbb{F}{}_{p^{n_{2}}}\subset\mathbb{F}{}_{p^{n_{3}}}\subset\cdots$, and we
define $\mathbb{F}{}=\bigcup\mathbb{F}{}_{p^{n_{i}}}$. As a union of a chain
of fields algebraic over $\mathbb{F}{}_{p}$, it is again a field algebraic
over $\mathbb{F}{}_{p}$. Moreover, every polynomial in $\mathbb{F}{}_{p}[X]$
splits in $\mathbb{F}{}$, and so it is an algebraic closure of $\mathbb{F}{}$
(by \ref{sf10}).
\end{proof}

\begin{remark}
\label{sf13m}Since the $\mathbb{F}{}_{p^{n}}$ are not subsets of a fixed set,
forming the union requires explanation. Let $S$ be the disjoint union of the
$\mathbb{F}{}_{p^{n}}$. For $a,b\in S$, set $a\sim b$ if $a=b$ in one of the
$\mathbb{F}{}_{p^{n}}$. Then $\sim$ is an equivalence relation, and we let
$\mathbb{F}{}=S/\sim$.
\end{remark}

Any two fields with $q$ elements are isomorphic, but not necessarily
\textit{canonically} isomorphic. However, once we have chosen an algebraic
closure $\mathbb{F}{}$ of $\mathbb{F}{}_{p}$, there is a \textit{unique}
subfield of $\mathbb{F}{}$ with $q$ elements.

PARI%
\index{PARI}
factors polynomials modulo $p$ very quickly. Recall that the syntax is
\newline\texttt{factormod(f(X),p)}. For example, to obtain a list of all monic
polynomials of degree $1,2,$ or $4$ over $\mathbb{F}_{5}$, ask PARI to factor
$X^{625}-X$ modulo $5$ (note that $625=5^{4}$).

\begin{aside}
\label{cg18a}In one of the few papers published during his lifetime, Galois
defined finite fields of arbitrary prime power order and established their
basic properties, for example, the existence of a primitive element (Notices
A.M.S., Feb. 2003, p.~198). For this reason finite fields are often called
\emph{Galois fields}%
\index{Galois field}
and the field with $q$ elements is often denoted by $\mathrm{GF}(q)$.
\end{aside}

\section{Computing Galois groups over \texorpdfstring{$\mathbb{Q}$}{Q}}

In the remainder of this chapter, I describe a practical method for computing
Galois groups over $\mathbb{Q}$ and similar fields. Recall that for a
separable polynomial $f\in F[X]$, $F_{f}$ denotes a splitting field for $F$,
and $G_{f}=\Gal(F_{f}/F)$ denotes the Galois group of $f$. Moreover, $G_{f}$
permutes the roots $\alpha_{1},\ldots,\alpha_{m}$, $m=\deg f$, of $f$ in
$F_{f}$:%
\[
G\subset\Sym\{\alpha_{1},\ldots,\alpha_{m}\}\text{.}%
\]
The first result generalizes Proposition \ref{cg4}.

\begin{proposition}
\label{cg19}Let $f(X)$ be a separable polynomial in $F[X]$, and suppose that
the orbits of $G_{f}$ acting on the roots of $f$ have $m_{1},\ldots,m_{r}$
elements respectively. Then $f$ factors as $f=f_{1}\cdots f_{r}$ with $f_{i}$
irreducible of degree $m_{i}$.
\end{proposition}

\begin{proof}
We may suppose that $f$ is monic. Let $\alpha_{1},\ldots,\alpha_{m}$, be the
roots of $f(X)$ in $F_{f}$. The monic factors of $f(X)$ in $F_{f}[X]$
correspond to subsets $S$ of $\{\alpha_{1},\ldots,\alpha_{m}\}$,
\[
S\leftrightarrow f_{S}=\prod_{\alpha\in S}(X-\alpha)\text{,}%
\]
and $f_{S}$ is fixed under the action of $G_{f}$ (and hence has coefficients
in $F$) if and only if $S$ is stable under $G_{f}$. Therefore the irreducible
factors of $f$ in $F[X]$ are the polynomials $f_{^{S}}$ corresponding to
minimal subsets $S$ of $\{\alpha_{1},\ldots,\alpha_{m}\}$ stable under $G_{f}%
$, but these subsets $S$ are precisely the orbits of $G_{f}$ in $\{\alpha
_{1},\ldots,\alpha_{m}\}$.
\end{proof}

\begin{remark}
\label{cg19m}Note that the proof shows the following: let $\{\alpha_{1}%
,\ldots,\alpha_{m}\}=\bigcup O_{i}$ be the decomposition of $\{\alpha
_{1},\ldots,\alpha_{m}\}$ into a disjoint union of orbits for the group
$G_{f}$; then%
\[
f=\prod f_{i},\quad\text{where }f_{i}=\prod_{\alpha_{j}\in O_{i}}(X-\alpha
_{j}),
\]
is the decomposition of $f$ into a product of irreducible polynomials in
$F[X]$.
\end{remark}

Now suppose that $F$ is finite, with $p^{n}$ elements say. Then $G_{f}$ is a
cyclic group generated by the Frobenius automorphism $\sigma\colon x\mapsto
x^{p^{n}}$. When we regard $\sigma$ as a permutation of the roots of $f$, then
the orbits of $\sigma$ correspond to the factors in its cycle decomposition
(GT, 4.26). Hence, if the degrees of the distinct irreducible factors
of $f$ are $m_{1},m_{2},\ldots,m_{r}$, then $\sigma$ has a cycle decomposition
of type
\[
m_{1}+\cdots+m_{r}=\deg f.
\]


\begin{proposition}
\label{cg20}Let $R$ be a unique factorization domain with field of fractions
$F$, and let $f$ be a monic polynomial in $R[X]$. Let $P$ be a prime ideal in
$R$, let $\bar{F}=R/P$, and let $\bar{f}$ be the image of $f$ in $\bar{F}[X]$.
Assume that $\bar{f}$ is separable. Then $f$ is separable, and its roots
$\alpha_{1},\ldots,\alpha_{m}$ lie in some finite extension $R^{\prime}$ of
$R$. Their reductions $\bar{\alpha}_{i}$ modulo $PR^{\prime}$ are the roots of
$\bar{f}$, and $G_{\bar{f}}\subset G_{f}$ when both are identified with
subgroups of $\Sym\{\alpha_{1},\ldots,\alpha_{m}\}=\Sym\{\bar{\alpha}%
_{1},\ldots,\bar{\alpha}_{m}\}$.
\end{proposition}

We defer the proof to the end of this section.

On combining these results, we obtain the following theorem.

\begin{theorem}
[Dedekind]\label{cg21}%
\index{theorem!Dedekind}%
Let $f(X)\in\mathbb{Z}[X]$ be a monic polynomial of degree $m$, and let $p$ be
a prime such that $f\mod p$ has simple roots (equivalently, $D(f)$ is not
divisible by $p$). Suppose that $\bar{f}=\prod f_{i}$ with $f_{i}$ irreducible
of degree $m_{i}$ in $\mathbb{F}_{p}[X]$. Then $G_{f}$ contains an element
whose cycle decomposition is of type
\[
m=m_{1}+\cdots+m_{r}.
\]

\end{theorem}

\begin{example}
\label{cg22}Consider $X^{5}-X-1$. Modulo $2$, this factors as
\[
(X^{2}+X+1)(X^{3}+X^{2}+1),
\]
and modulo $3$ it is irreducible. The theorem shows that $G_{f}$ contains
permutations $(ik)(lmn)$ and $(12345)$, and so also $((ik)(lmn))^{3}=(ik)$.
Therefore $G_{f}=S_{5}$ by (\ref{cg12}).
\end{example}

\begin{lemma}
\label{cg23}A transitive subgroup of $H\subset S_{n}$ containing a
transposition and an $(n-1)$-cycle is equal to $S_{n}$.
\end{lemma}

\begin{proof}
After renumbering, we may suppose that the $(n-1)$-cycle is $(123\ldots n-1)$.
Because of the transitivity, the transposition can be transformed into $(in)$,
some $1\leq i\leq n-1$. Conjugating $(in)$ by $(123\ldots n-1)$ and its powers
will transform it into $(1n)$, $(2n)$, $\ldots$, $(n-1\,n)$, and these
elements obviously generate $S_{n}.$
\end{proof}

\begin{example}
\label{cg24}Select separable monic polynomials of degree $n$, $f_{1}%
,f_{2},f_{3}$ with coefficients in $\mathbb{Z}$ with the following factorizations:

\begin{enumerate}
\item $f_{1}$ is irreducible modulo $2$;

\item $f_{2}=(\text{degree }1)(\text{irreducible of degree }n-1)\mod3$;

\item $f_{3}=($irreducible of degree $2$)(product of $1$ or $2$ irreducible
polynomials of odd degree) mod $5$.
\end{enumerate}

\noindent Take
\[
f=-15f_{1}+10f_{2}+6f_{3}.
\]
Then

\begin{itemize}
\item[(i)] $G_{f}$ is transitive (it contains an $n$-cycle because $f\equiv
f_{1}$ mod $2$);

\item[(ii)] $G_{f}$ contains a cycle of length $n-1$ (because $f\equiv f_{2}$
mod $3$);

\item[(iii)] $G_{f}$ contains a transposition (because $f\equiv f_{3}$ mod $5
$, and so it contains the product of a transposition with a commuting element
of odd order; on raising this to an appropriate odd power, we are left with
the transposition). Hence $G_{f}$ is $S_{n}.$
\end{itemize}
\end{example}

The above results give the following strategy for computing the Galois group
of an irreducible polynomial $f\in\mathbb{Q}[X]$. Factor $f$ modulo a sequence
of primes $p$ not dividing $D(f)$ to determine the cycle types of the elements
in $G_{f}$ --- a difficult theorem in number theory, the effective Chebotarev
density theorem, says that if a cycle type occurs in $G_{f}$, then this will
be seen by looking modulo a set of prime numbers of positive density, and will
occur for a prime less than some bound. Now look up a table of transitive
subgroups of $S_{n}$ with order divisible by $n$ and their cycle types. If
this doesn't suffice to determine the group, then look at its action on the
set of subsets of $r$ roots for some $r$.

See, Butler and McKay, \textit{The transitive groups of degree up to
eleven}\emph{,\/} Comm. Algebra 11 (1983), 863--911. This lists all transitive
subgroups of $S_{n}$, $n\leq11$, and gives the cycle types of their elements
and the orbit lengths of the subgroup acting on the $r$-sets of roots. With
few exceptions, these invariants are sufficient to determine the subgroup up
to isomorphism.

PARI%
\index{PARI}
can compute Galois groups for polynomials of degree $\leq11$ over $\mathbb{Q}%
$. The syntax is \verb|polgalois(f)| where $f$ is an irreducible polynomial of
degree $\leq11$ (or $\leq7$ depending on your setup), and the output is
$(n,s,k,$name$)$ where $n$ is the order of the group, $s$ is $+1$ or $-1$
according as the group is a subgroup of the alternating group or not, and
\textquotedblleft name\textquotedblright\ is the name of the group. For
example, \verb|polgalois(X^5-5*X^3+4*X-1)| (see \ref{cg14a}) returns the
symmetric group $S_{5}$, which has order $120$,
\verb|polgalois(X^11-5*X^3+4*X-1)| returns the symmetric group $S_{11}$, which
has order $39916800$, and \newline\verb|polgalois(X^12-5*X^3+4*X-1)| returns
an apology. The reader should use PARI to check the examples \ref{cg8a}%
--\ref{cg11}.

See also, Soicher and McKay, \textit{Computing Galois groups over the
rationals}, J. Number Theory, 20 (1985) 273--281.

\subsection{Proof of Proposition \ref{cg20}}

We follow the elegant argument in van der Waerden, Modern Algebra, I, \S 61.

Let $f(X)$ be a separable polynomial in $F[X]$ and $\alpha_{1},\ldots
,\alpha_{m}$ its roots. Let $T_{1},\ldots,T_{m}$ be symbols. For a permutation
$\sigma$ of $\{1,\ldots,m\}$, we let $\sigma_{\alpha}$ and $\sigma_{T}$
respectively denote the corresponding permutations of $\{\alpha_{1}%
,\ldots,\alpha_{m}\}$ and $\{T_{1},\ldots,T_{m}\}$.

Let%
\[
\theta=T_{1}\alpha_{1}+\cdots+T_{m}\alpha_{m}%
\]
and%
\[
f(X,T)=\prod_{\sigma\in S_{m}}(X-\sigma_{T}\theta).
\]
Clearly $f(X,T)$ is symmetric in the $\alpha_{i}$, and so its coefficients lie
in $F$. Let%
\begin{equation}
f(X,T)=f_{1}(X,T)\cdots f_{r}(X,T) \label{eq13}%
\end{equation}
be the factorization of $f(X,T)$ into a product of irreducible monic
polynomials. Here we use that $F[X,T_{1},\ldots,T_{m}]$ is a unique
factorization domain (CA 4.10). The permutations $\sigma$ such that
$\sigma_{T}$ carries any one of the factors, say $f_{1}(X,T)$, into itself
form a subgroup $G$ of $S_{m}$.

\begin{lemma}
\label{cg25}The map $\sigma\mapsto\sigma_{\alpha}$ is an isomorphism from $G$
onto $G_{f}$.
\end{lemma}

\begin{proof}
In any $F$-algebra containing the roots of $f$, the polynomial $f_{1}(X,T)$ is
a product of factors of the form $X-\sigma\theta$. After possibly renumbering
the roots of $f$, we may suppose that $f_{1}(X,T)$ contains the factor
$X-\theta$. Note that $s_{T}s_{\alpha}$ leaves $\theta$ invariant, i.e.,
$s_{T}s_{\alpha}\theta=\theta$, and so
\begin{equation}
s_{\alpha}\theta=s_{T}^{-1}\theta. \label{eq12}%
\end{equation}


Let $\sigma$ be a permutation of $\{1,\ldots,m\}$. If $\sigma_{T}$ leaves
$f_{1}(X,T)$ invariant, then it permutes its roots. Therefore, it maps
$X-\theta$ into a linear factor of $f_{1}(X,T)$. Conversely, if $\sigma_{T}$
maps $X-\theta$ into a linear factor of $f_{1}(X,T)$, then this linear factor
will be a common factor of $f_{1}(X,T)$ and the image of $f_{1}(X,T)$ under
$\sigma_{T}$, which implies that the two are equal, and so $\sigma_{T}$ leaves
$f_{1}(X,T)$ invariant. We conclude that $\sigma_{T}$ leaves $f_{1}(X,T)$
invariant if and only if $\sigma_{T}$ maps $X-\theta$ into a linear factor of
$f_{1}(X,T)$.

!!In the third paragraph of the proof of Lemma 4.34, $\theta$ is algebraic
over the field $F(T)=_{def} F(T_{1},\ldots,T_{m})$ with minimal polynomial
equal to $f(X,T)$ (regarded as a polynomial in $X$ with coefficients in the
field $F(T)$).!!

Again, let $\sigma$ be a permutation of $\{1,\ldots,m\}$. Then $\sigma
_{\alpha}\in G_{f}$ if and only if it maps $F(T)[\theta]$ isomorphically onto
$F(T)[\sigma_{\alpha}\theta]$, i.e., if and only if $\theta$ and
$\sigma_{\alpha}\theta$ have the same minimal polynomial. The minimal
polynomial of $\theta$ is $f_{1}(X,T)$, and so this shows that $s_{\alpha}$
lies in $G_{f}$ if and only if $\sigma_{\alpha}$ leaves $f_{1}(X,T)$
invariant, i.e., if and only if $\sigma_{\alpha}$ maps $X-\theta$ into a
linear factor of $f_{1}(X,T)$.

From the last two paragraphs and (\ref{eq12}), we see that the condition for
$\sigma$ to lie in $G$ is the same as the condition for $\sigma_{\alpha}$ to
lie in $G_{f}$, which concludes the proof.
\end{proof}

After these preliminaries, we prove Lemma \ref{cg20}. With the notation of the
lemma, let $R^{\prime}=R[\alpha_{1},\ldots,\alpha_{m}]$. Then $R^{\prime}$ is
generated by a finite number of elements, each integral over $R$, and so it is
finite as an $R$-algebra (CA 6.2). Clearly, the map $a\mapsto\bar{a}\colon
R^{\prime}\rightarrow R^{\prime}/PR^{\prime}$ sends the roots of $f$ onto the
roots of $\bar{f}$. As the latter are distinct, so are the former, and the map
is bijective.

A general form of Proposition \ref{ef6m} shows that, in the factorization
(\ref{eq13}), the $f_{i}$ lie in $R[X,T]$. Hence (\ref{eq13}) gives a
factorization%
\[
\bar{f}(X,T)=\bar{f}_{1}(X,T)\cdots\bar{f}_{r}(X,T)
\]
in $\bar{F}[X,T]$. Let $\bar{f}_{1}(X,T)_{1}$ be an irreducible factor of
$\bar{f}_{1}(X,T)$. According to Lemma \ref{cg25}, $G_{f}$ is the set of
permutations $\sigma_{\alpha}$ such that $\sigma_{T}$ leaves $f_{1}(X,T)$
invariant, and $G_{\bar{f}}$ is the set of permutations $\sigma_{\alpha}$ such
that $\sigma_{T}$ leaves $\bar{f}_{1}(X,T)_{1}$ invariant. Clearly $G_{\bar
{f}}\subset G_{f}$.

\begin{aside} For a monic polynomial $f$ of degree $n$ with bounded integers
as coefficients, it is expected that the Galois group of $f$ equals $S_n$
with probability $1$ as $n\rightarrow\infty$. See
Bary-Soroker, Kozma, and Gady, Duke Math. J. 169 (2020), 579--598,
for precise statements.
\end{aside}

\section{Exercises}

\begin{exercise}
\label{x14} Find the splitting field of $X^{m}-1\in\mathbb{F}_{p}[X]$.
\end{exercise}

\begin{exercise}
\label{x15} Find the Galois group of $X^{4}-2X^{3}-8X-3$ over ${\mathbb{Q}}$.
\end{exercise}

\begin{exercise}
\label{x16} Find the degree of the splitting field of $X^{8}- 2$ over
${\mathbb{Q}}$.
\end{exercise}

\begin{exercise}
\label{x17} Give an example of a field extension $E/F$ of degree $4$ such that
there does not exist a field $M$ with $F\subset M\subset E$, $[M\colon F]=2$.
\end{exercise}

\begin{exercise}
\label{x18} List all irreducible polynomials of degree $3 $ over
$\mathbb{F}_{7}$ in 10 seconds or less (there are 112).
\end{exercise}

\begin{exercise}
\label{x19} ``It is a thought-provoking question that few graduate students
would know how to approach the question of determining the Galois group of,
say,
\[
X^{6}+2X^{5}+3X^{4}+4X^{3}+5X^{2}+6X+7.\text{\textrm{''}}%
\]
[over ${\mathbb{Q}}$].

\begin{enumerate}
\item Can you find it?

\item Can you find it without using the \textquotedblleft\texttt{polgalois}%
\textquotedblright\ command in PARI%
\index{PARI}%
?
\end{enumerate}
\end{exercise}

\begin{exercise}
\label{x20} Let $f(X)=X^{5}+aX+b$, $a,b\in{\mathbb{Q}}$. Show that
$G_{f}\approx D_{5}$ (dihedral group) if and only if

\begin{enumerate}
\item $f(X)$ is irreducible in ${\mathbb{Q}}[X]$, and

\item the discriminant $D(f)=4^{4}a^{5}+5^{5}b^{4}$ of $f(X)$ is a square, and

\item the equation $f(X)=0$ is solvable by radicals.
\end{enumerate}
\end{exercise}

\begin{exercise}
\label{x20a} Show that a polynomial $f$ of degree $n=\prod_{i=1}^{k}%
p_{i}^{r_{i}}$ (the $p_{i}$ are distinct primes) is irreducible over
$\mathbb{F}_{p}$ if and only if (a) $\mathrm{gcd}(f(X),X^{p^{n/p_{i}}}-X)=1$
for all $1\leq i\leq k$ and (b) $f$ divides $X^{p^{n}}-X$ (Rabin
irreducibility test\footnote{Rabin, Michael O. Probabilistic algorithms in
finite fields. SIAM J. Comput. 9 (1980), no. 2, 273--280.}).
\end{exercise}

\begin{exercise}
\label{x20b}Let $f(X)$ be an irreducible polynomial in $\mathbb{Q}{}[X]$ with
both real and nonreal roots. Show that its Galois group is nonabelian. Can the
condition that $f$ is irreducible be dropped?
\end{exercise}

\begin{exercise}
\label{x20c}Let $F$ be a Galois extension of $\mathbb{Q}{}$, and let $\alpha$
be an element of $F$ such that $\alpha F^{\times2}$ is not fixed by the action
of $\Gal(F/\mathbb{Q}{})$ on $F^{\times}/F^{\times2}$. Let $\alpha=\alpha
_{1},\ldots,\alpha_{n}$ be the orbit of $\alpha$ under $\Gal(F/\mathbb{Q}{})$. Show:

\begin{enumerate}
\item $F[\sqrt{\alpha_{1}},\ldots,\sqrt{\alpha_{n}}]/F$ is Galois with
commutative Galois group contained in $\left(  \mathbb{Z}{}/2\mathbb{Z}%
{}\right)  ^{n}$.

\item $F[\sqrt{\alpha_{1}},\ldots,\sqrt{\alpha_{n}}]/\mathbb{Q}{}$ is Galois
with noncommutative Galois group contained in $(\mathbb{Z}{}/2\mathbb{Z}%
)^{n}\rtimes\Gal(F/\mathbb{Q}{})$. (Cf. mo113794.)
\end{enumerate}
\end{exercise}

\clearpage


\chapter{Applications of Galois Theory}

In this chapter, we apply the fundamental theorem of Galois theory to obtain
other results about polynomials and extensions of fields.

\section{Primitive element theorem.}%

\index{theorem!primitive element}%


\noindent Recall that a finite extension of fields $E/F$ is simple if
$E=F[\alpha]$ for some element $\alpha$ of $E$. Such an $\alpha$ is called a
\emph{primitive element\/}%
\index{primitive element}
of $E$. We'll show that (at least) all separable extensions have primitive elements.

Consider for example $\mathbb{Q}[\sqrt{2},\sqrt{3}]/\mathbb{Q}$. We know (see
Exercise \ref{x13}) that its Galois group over $\mathbb{Q}$ is a $4$-group
$\langle{}\sigma,\tau\rangle,$ where
\[
\left\{
\begin{array}
[c]{rrr}%
\sigma\sqrt{2} & = & -\sqrt{2}\\
\sigma\sqrt{3} & = & \sqrt{3}%
\end{array}
,\quad\left\{
\begin{array}
[c]{rrr}%
\tau\sqrt{2} & = & \sqrt{2}\\
\tau\sqrt{3} & = & -\sqrt{3}%
\end{array}
.\right.  \right.
\]
Note that
\[%
\begin{array}
[c]{rcl}%
\sigma(\sqrt{2}+\sqrt{3}) & = & -\sqrt{2}+\sqrt{3},\\
\quad\tau(\sqrt{2}+\sqrt{3}) & = & \sqrt{2}-\sqrt{3},\\
\quad(\sigma\tau)(\sqrt{2}+\sqrt{3}) & = & -\sqrt{2}-\sqrt{3}.
\end{array}
\]
These all differ from $\sqrt{2}+\sqrt{3}$, and so only the identity element of
$\Gal(\mathbb{Q}[\sqrt{2},\sqrt{3}]/\mathbb{Q})$ fixes the elements of
$\mathbb{Q}[\sqrt{2}+\sqrt{3}]$. According to the fundamental theorem, this
implies that $\sqrt{2}+\sqrt{3}$ is a primitive element:
\[
\mathbb{Q}[\sqrt{2},\sqrt{3}]=\mathbb{Q}[\sqrt{2}+\sqrt{3}].
\]
It is clear that this argument should work much more generally.

Recall that an element $\alpha$ algebraic over a field $F$ is separable\/%
\index{separable}
over $F$ if its minimal polynomial over $F$ has no multiple roots.

\begin{theorem}
\label{ag1} Let $E=F[\alpha_{1},...,\alpha_{r}]$ be a finite extension of $F$,
and assume that $\alpha_{2},...,\alpha_{r}$ are separable over $F$ (but not
necessarily $\alpha_{1}$). Then there is an element $\gamma\in E$ such that
$E=F[\gamma]$.
\end{theorem}

\begin{proof}
For finite fields, we proved this in \ref{cg17}. Hence we may assume $F$ to be
infinite. It suffices to prove the statement for $r=2$, for then%
\[
F[\alpha_{1},\alpha_{2},\ldots,\alpha_{r}]=F[\alpha_{1}^{\prime},\alpha
_{3},\ldots,\alpha_{r}]=F[\alpha_{1}^{\prime\prime},\alpha_{4},\ldots
,\alpha_{r}]=\cdots.
\]
Thus let $E=F[\alpha,\beta]$ with $\beta$ separable over $F$. Let $f$ and $g$
be the minimal polynomials of $\alpha$ and $\beta$ over $F$, and let $L$ be a
splitting field for $fg$ containing $E$. Let $\alpha_{1}=\alpha,\ldots
,\alpha_{s}$ be the roots of $f$ in $L$, and let $\beta_{1}=\beta$, $\beta
_{2},\ldots,\beta_{t}$ be the roots of $g$. For $j\neq1$, $\beta_{j}\neq\beta
$, and so the the equation
\[
\alpha_{i}+X\beta_{j}=\alpha+X\beta,
\]
has exactly one solution, namely, $X=\frac{\alpha_{i}-\alpha}{\beta-\beta_{j}%
}$. If we choose a $c\in F$ different from any of these solutions (using that
$F$ is infinite), then
\[
\alpha_{i}+c\beta_{j}\neq\alpha+c\beta\text{\ unless }i=1=j.
\]
Let $\gamma=\alpha+c\beta$. I claim that%
\[
F[\alpha,\beta]=F[\gamma]\text{.}%
\]
The polynomials $g(X)$ and $f(\gamma-cX)$ have coefficients in $F[\gamma]$,
and have $\beta$ as a root:
\[
g(\beta)=0,\quad f(\gamma-c\beta)=f(\alpha)=0.
\]
In fact, $\beta$ is their only common root, because we chose $c$ so that
$\gamma-c\beta_{j}\neq\alpha_{i}$ unless $i=1=j$. Therefore
\[
\gcd(g(X),f(\gamma-cX))=X-\beta\text{.}%
\]
Here we computed the $\gcd$ in $L[X]$, but this is equal to the $\gcd$
computed in $F[\gamma][X]$ (Proposition \ref{ft1}). Hence $\beta\in F[\gamma
]$, and this implies that $\alpha=\gamma-c\beta$ also lies in $F[\gamma]$.
This proves the claim.
\end{proof}

\begin{remark}
\label{ag2}When $F$ is infinite, the proof shows that $\gamma$ can be chosen
to be of the form
\[
\gamma=\alpha_{1}+c_{2}\alpha_{2}+\cdots+c_{r}\alpha_{r},\quad c_{i}\in F.
\]
If $F[\alpha_{1},\ldots,\alpha_{r}]$ is Galois over $F$, then an element of
this form will be a primitive element provided it is moved by every nontrivial
element of the Galois group. This remark makes it very easy to write down
primitive elements.
\end{remark}

Our hypotheses are minimal: if \textit{two\/ }of the $\alpha$ are not
separable, then the extension need not be simple. Before giving an example to
illustrate this, we need another result.

\begin{proposition}
\label{ag3}Let $E=F[\gamma]$ be a simple algebraic extension of $F$. Then
there are only finitely many intermediate fields $M$,
\[
F\subset M\subset E.
\]

\end{proposition}

\begin{proof}
Let $M$ be such a field, and let $g(X)$ be the minimal polynomial of $\gamma$
over $M$. Let $M^{\prime}$ be the subfield of $E$ generated over $F$ by the
coefficients of $g(X)$. Clearly $M^{\prime}\subset M$, but (equally clearly)
$g(X)$ is the minimal polynomial of $\gamma$ over $M^{\prime}$. Hence
\[
\lbrack E\colon M^{\prime}]=\deg(g)=[E\colon M],
\]
and so $M=M^{\prime}$; we have shown that $M$ is generated by the coefficients
of $g(X)$.

Let $f(X)$ be the minimal polynomial of $\gamma$ over $F$. Then $g(X)$ divides
$f(X)$ in $M[X]$, and hence also in $E[X]$. Therefore, there are only finitely
many possible $g$, and consequently only finitely many possible $M$.
\end{proof}

\begin{remark}
\label{ag4}(a) Note that the proof in fact gives a description of all the
intermediate fields: each is generated over $F$ by the coefficients of a
factor $g(X)$ of $f(X)$ in $E[X]$. The coefficients of such a $g(X)$ are
partially symmetric polynomials in the roots of $f(X)$ (that is, fixed by
some, but not necessarily all, of the permutations of the roots).

(b) The proposition has a converse: if $E$ is a finite extension of $F$ and
there are only finitely many intermediate fields $M$, $F\subset M\subset E$,
then $E$ is a simple extension of $F$. This gives another proof of Theorem
\ref{ag1} in the case that $E$ is separable over $F$, because Galois theory
shows that there are only finitely many intermediate fields in this case (even
the Galois closure of $E$ over $F$ has only finitely many intermediate fields).
\end{remark}

\begin{example}
\label{ag4m}The simplest nonsimple algebraic extension is $k(X,Y)\supset
k(X^{p},Y^{p})$, where $k$ is an algebraically closed field of characteristic
$p$. Let $F=k(X^{p},Y^{p})$. For all $c\in k$, we have
\[
k(X,Y)=F[X,Y]\supset F[X+cY]\supset F
\]
with the degree of each extension equal to $p$. If
\[
F[X+cY]=F[X+c^{\prime}Y],\quad c\neq c^{\prime},
\]
then $F[X+cY]$ would contain both $X$ and $Y$, which is impossible because
$[k(X,Y)\colon F]=p^{2}$. Hence there are infinitely many distinct
intermediate fields.\footnote{Zariski showed that there is even an
intermediate field $M$ that is not isomorphic to $F(X,Y)$, and Piotr Blass
showed, using the methods of algebraic geometry, that there is an infinite
sequence of intermediate fields, no two of which are isomorphic.}

Alternatively, note that the degree of $k(X,Y)$ over $k(X^{p},Y^{p})$ is
$p^{2}$, but if $\alpha\in k(X,Y)$, then $\alpha^{p}\in k(X^{p},Y^{p})$, and
so $\alpha$ generates a field of degree at most $p$ over $k(X^{p},Y^{p})$.
\end{example}

\section{Fundamental Theorem of Algebra}%

\index{theorem!fundamental of algebra}%


We finally prove the misnamed\footnote{Because it is not strictly a theorem in
algebra: it is a statement about $\mathbb{R}$ whose construction is part of
analysis (or maybe topology). In fact, I prefer the proof based on Liouville's
theorem in complex analysis to the more algebraic proof given in the text: if
$f(z)$ is a polynomial without a root in $\mathbb{C}$, then $f(z)^{-1}$ is
bounded and holomorphic on the whole complex plane, and hence (by Liouville)
constant. The Fundamental Theorem was quite difficult to prove. Gauss gave a
proof in his doctoral dissertation in 1798 in which he used some geometric
arguments which he didn't justify. He gave the first rigorous proof in 1816.
The elegant argument given here is a simplification by Emil Artin of earlier
proofs (see Artin, E., Algebraische Konstruction reeller K\"{o}rper, Hamb.
Abh., Bd. 5 (1926), 85-90; translation available in Artin, Emil. Exposition by
Emil Artin: a selection. AMS; LMS 2007).} fundamental theorem of algebra.%
\index{theorem!fundamental of algebra}%


\begin{theorem}
\label{ag5}The field $\mathbb{C}$ of complex numbers is algebraically closed.
\end{theorem}

\begin{proof}
We define $\mathbb{C}$ to be the splitting field of $X^{2}+1$ over
$\mathbb{R}{}$, and we let $i$ denote a root of $X^{2}+1$ in $\mathbb{C}$.
Thus $\mathbb{C}=\mathbb{R}[i]$. We have to show (see \ref{sf10}) that every
$f(X)\in\mathbb{R}[X]$ has a root in $\mathbb{C}$. We may suppose that $f$ is
monic, irreducible, and $\neq X^{2}+1$.

We'll need to use the following two facts about $\mathbb{R}$:

\begin{itemize}
\item positive real numbers have square roots;

\item every polynomial of odd degree with real coefficients has a real root.
\end{itemize}

\noindent Both are immediate consequences of the Intermediate Value Theorem,
which says that a continuous function on a closed interval takes every value
between its maximum and minimal values (inclusive). (Intuitively, this says
that, unlike the rationals, the real line has no \textquotedblleft
holes\textquotedblright.)

We first show that every element of $\mathbb{C}$ has a square root. Write
$\alpha=a+bi$, with $a,b\in\mathbb{R}$, and choose $c,d$ to be real numbers
such that
\[
c^{2}=\frac{(a+\sqrt{a^{2}+b^{2}})}{2},\quad d^{2}=\frac{(-a+\sqrt{a^{2}%
+b^{2}})}{2}.
\]
Then $c^{2}-d^{2}=a$ and $(2cd)^{2}=b^{2}$. If we choose the signs of $c$ and
$d$ so that $cd$ has the same sign as $b$, then $(c+di)^{2}=\alpha$ and so
$c+di$ is a square root of $\alpha$.

Let $f(X)\in\mathbb{R}[X]$, and let $E$ be a splitting field for
$f(X)(X^{2}+1)$. Then $E$ contains $\mathbb{C}{}$, and we have to show that it
equals $\mathbb{C}{}$. Since $\mathbb{R}$ has characteristic zero, the
polynomial is separable, and so $E$ is Galois over $\mathbb{R}$ (see
\ref{ft12}). Let $G$ be its Galois group, and let $H$ be a Sylow $2$-subgroup
of $G$.

Let $M=E^{H}$ and let $\alpha\in M$. Then $M$ has of degree $(G\colon H)$ over
$\mathbb{R}$, which is odd, and so the minimal polynomial of $\alpha$ over
$\mathbb{R}{}$ has odd degree (by the multiplicativity of degrees,
\ref{ef10}). This implies that it has a real root, and so is of degree $1$.
Hence $\alpha\in\mathbb{R}{}$, and so $M=\mathbb{R}{}$ and $G=H$.

We deduce that $\Gal(E/\mathbb{C})$ is a 2-group. If it is $\neq1$, then it
has a subgroup $N$ of index 2 (GT, 4.17). The field $E^{N}$ has
degree $2$ over $\mathbb{C}$, and so it is generated by the square root of an
element of $\mathbb{C}$ (see \ref{ft23}), but all square roots of elements of
$\mathbb{C}{}$ lie in $\mathbb{C}$. Hence $E^{N}=\mathbb{C}$, which is a
contradiction. Thus $\Gal(E/\mathbb{C})=1$ and $E=\mathbb{C}$.
\end{proof}

\begin{corollary}
\label{ag6}(a) The field $\mathbb{C}$ is the algebraic closure of $\mathbb{R}
$.

(b) The set of all algebraic numbers is an algebraic closure of $\mathbb{Q}.$
\end{corollary}

\begin{proof}
Part (a) is obvious from the definition of \textquotedblleft algebraic
closure\textquotedblright\ (\ref{ac2}), and (b) follows from Corollary
\ref{ac3}.
\end{proof}

\section{Cyclotomic extensions}

A \emph{primitive} $n$th root%
\index{primitive root of 1}
of $1$ in $F$ is an element of order $n$ in $F^{\times}$. Such an element can
exist only if $F$ has characteristic $0$ or if its characteristic $p$ does not
divide $n$.

\begin{proposition}
\label{ag7} Let $F$ be a field of characteristic $0$ or characteristic $p$ not
dividing $n$, and let $E$ be the splitting field of $X^{n}-1$.

\begin{enumerate}
\item There exists a primitive $n$th root of $1$ in $E$.

\item If $\zeta$ is a primitive $n$th root of $1$ in $E$, then $E=F[\zeta]$.

\item The field $E$ is Galois over $F$; for each $\sigma\in\Gal(E/F)$, there
is an $i\in(\mathbb{Z}{}/n\mathbb{Z}{})^{\times}$ such that $\sigma\zeta
=\zeta^{i}$ for all $\zeta$ with $\zeta^{n}=1$; the map $\sigma\mapsto\lbrack
i]$ is an injective homomorphism
\[
\Gal(E/F)\rightarrow(\mathbb{Z}/n\mathbb{Z})^{\times}\text{.}%
\]

\end{enumerate}
\end{proposition}

\begin{proof}
(a) The roots of $X^{n}-1$ are distinct, because its derivative $nX^{n-1}$ has
only zero as a root (here we use the condition on the characteristic), and so
$E$ contains $n$ distinct $n$th roots of $1$. The $n$th roots of $1$ form a
finite subgroup of $E^{\times}$, and so (see Exercise 3) they form a cyclic
group. Every generator has order $n$, and hence is a primitive $n$th root of
$1$.

(b) The roots of $X^{n}-1$ are the powers of $\zeta$, and $F[\zeta]$ contains
them all.

(c) The extension $E/F$ is Galois because $E$ is the splitting field of a
separable polynomial. If $\zeta_{0}$ is one primitive $n$th root of $1$, then
the remaining primitive $n$th roots of $1$ are the elements $\zeta_{0}^{i}$
with $i$ relatively prime to $n$. Since, for any automorphism $\sigma$ of $E$,
$\sigma\zeta_{0}$ is again a primitive $n$th root of $1$, it equals $\zeta
_{0}^{i}$ for some $i$ relatively prime to $n$, and the map $\sigma\mapsto
i\mod n$ is injective because $\zeta_{0}$ generates $E$ over $F$. It obviously
is a homomorphism. Moreover, for any other $n$th root of $1$, say,
$\zeta=\zeta_{0}^{m}$, we have
\[
\sigma\zeta=(\sigma\zeta_{0})^{m}=\zeta_{0}^{im}=\zeta^{i},
\]
and so the homomorphism does not depend on the choice of $\zeta_{0}$.
\end{proof}

The map $\sigma\mapsto\lbrack i]\colon\Gal(F[\zeta]/F)\rightarrow
(\mathbb{Z}/n\mathbb{Z})^{\times}$ need not be surjective. For example, if
$F=\mathbb{C}$, then its image is $\{1\}$, and if $F=\mathbb{R}$, it is either
$\{[1]\}$ or $\{[-1],[1]\}$. On the other hand, when $n=p$ is prime, we showed
in (\ref{ef31}) that $[\mathbb{Q}[\zeta]\colon\mathbb{Q}]=p-1$, and so the map
is surjective. We now prove that the map is surjective for all $n$ when
$F=\mathbb{Q}$.

The polynomial $X^{n}-1$ has some obvious factors in $\mathbb{Q}[X]$, namely,
the polynomials $X^{d}-1$ for any $d|n$. When we remove all factors of
$X^{n}-1$ of this form with $d<n$, the polynomial we are left with is called
the $n$th \emph{cyclotomic polynomial}%
\index{cyclotomic polynomial}
$\Phi_{n}$. Thus
\[
\Phi_{n}=\prod(X-\zeta)\qquad\text{(product over the primitive }n{\text{th}%
}\text{\ roots of }1).
\]
It has degree%
\index{phi(n)@$\varphi(n)$}
$\varphi(n)$, the order of $(\mathbb{Z}/n\mathbb{Z})^{\times}$. Since every
$n$th root of $1$ is a primitive $d$th root of $1$ for exactly one $d$
dividing $n$, we see that
\[
X^{n}-1=\prod_{d|n}\Phi_{d}(X).
\]
For example, $\Phi_{1}(X)=X-1$, $\Phi_{2}(X)=X+1$, $\Phi_{3}(X)=X^{2}+X+1$,
and
\[
\Phi_{6}(X)=\frac{X^{6}-1}{(X-1)(X+1)(X^{2}+X+1)}=X^{2}-X+1.
\]
This gives an easy inductive method of computing the cyclotomic polynomials.
Alternatively type
\index{PARI}%
\texttt{polcyclo(n,X)} in PARI.

Because $X^{n}-1$ has coefficients in $\mathbb{Z}$ and is monic, every monic
factor of it in $\mathbb{Q}{}[X]$ has coefficients in $\mathbb{Z}$ (see
\ref{ef6m}). In particular, the cyclotomic polynomials lie in $\mathbb{Z}[X]$.

\begin{lemma}
\label{ag8}Let $F$ be a field of characteristic $0$ or $p$ not dividing $n$,
and let $\zeta$ be a primitive $n$th root of $1$ in some extension of $F$. The
following are equivalent:

\begin{enumerate}
\item the $n$th cyclotomic polynomial $\Phi_{n}$ is irreducible;

\item the degree $[F[\zeta]\colon F]=\varphi(n)$;

\item the homomorphism
\[
\Gal(F[\zeta]/F)\rightarrow(\mathbb{Z}/n\mathbb{Z})^{\times}%
\]
is an isomorphism.
\end{enumerate}
\end{lemma}

\begin{proof}
Because $\zeta$ is a root of $\Phi_{n}$, the minimal polynomial of $\zeta$
divides $\Phi_{n}$. It equals it if and only if $[F[\zeta]\colon
F]=\varphi(n)$, which is true if and only if the injection $\Gal(F[\zeta
]/F)\hookrightarrow(\mathbb{Z}/n\mathbb{Z})^{\times}$ is onto.
\end{proof}

\begin{theorem}
\label{ag9}%
\index{theorem!cyclotomic polynomials}%
The $n$th cyclotomic polynomial $\Phi_{n}$ is irreducible in $\mathbb{Q}[X]$.
\end{theorem}

\begin{proof}
Let $f(X)$ be a monic irreducible factor of $\Phi_{n}$ in $\mathbb{Q}[X]$. Its
roots will be primitive $n$th roots of $1$, and we have to show they include
\textit{all} primitive $n$th roots of $1$. For this it suffices to show that
\[
\zeta\text{\ a root of }f(X)\implies\zeta^{i}\text{\ a root of }%
f(X)\text{\ for all }i\text{\ such that }\gcd(i,n)=1.
\]
Such an $i$ is a product of primes not dividing $n$, and so it suffices to
show that
\[
\zeta\text{\ a root of }f(X)\implies\zeta^{p}\text{\ a root of $f(X)$ for all
primes }p\ \text{not dividing }n.
\]


Write
\[
\Phi_{n}(X)=f(X)g(X)\text{.}%
\]
Proposition \ref{ef6m} shows that $f(X)$ and $g(X)$ lie in $\mathbb{Z}[X]$.
Suppose that $\zeta$ is a root of $f$ but that, for some prime $p$ not
dividing $n$, $\zeta^{p}$ is not a root of $f$. Then $\zeta^{p}$ is a root of
$g(X)$, $g(\zeta^{p})=0$, and so $\zeta$ is a root of $g(X^{p})$. As $f(X)$
and $g(X^{p})$ have a common root, they have a nontrivial common factor in
$\mathbb{Q}{}[X]$ (\ref{ft1}), which automatically lies in $\mathbb{Z}{}[X]$
(\ref{ef6m}).

Write $h(X)\mapsto\bar{h}(X)$ for the quotient map $\mathbb{Z}[X]\rightarrow
\mathbb{F}_{p}[X]$, and note that, because $f(X)$ and $g(X^{p})$ have a common
factor of degree $\geq1$ in $\mathbb{Z}{}[X]$, so also do $\bar{f}(X)$ and
$\bar{g}(X^{p})$ in $\mathbb{F}{}_{p}[X]$. The mod $p$ binomial theorem shows
that
\[
\bar{g}(X)^{p}=\bar{g}(X^{p})
\]
(recall that $a^{p}=a$ for all $a\in\mathbb{F}{}_{p}$), and so $\bar{f}(X)$
and $\bar{g}(X)$ have a common factor of degree $\geq1$ in $\mathbb{F}{}%
_{p}[X]$. Hence $X^{n}-1$, when regarded as an element of $\mathbb{F}_{p}[X]$,
has multiple roots, but we saw in the proof of Proposition \ref{ag7} that it
doesn't. Contradiction.
\end{proof}

\begin{remark}
\label{ag10}This proof is very old --- in essence it goes back to Dedekind in
1857 --- but its general scheme has recently become popular: take a statement
in characteristic zero, reduce modulo $p$ (where the statement may no longer
be true), and exploit the existence of the Frobenius automorphism $a\mapsto
a^{p}$ to obtain a proof of the original statement. For example, commutative
algebraists use this method to prove results about commutative rings, and
there are theorems about complex manifolds that were first proved by reducing
things to characteristic $p.$

There are some beautiful relations between what happens in characteristic $0$
and in characteristic $p$. For example, let $f(X_{1},...,X_{n})\in
\mathbb{Z}[X_{1},...,X_{n}]$. We can

\begin{enumerate}
\item look at the solutions of $f=0$ in $\mathbb{C}$, and so get a topological space;

\item reduce mod $p$, and look at the solutions of $\bar{f}=0$ in
$\mathbb{F}_{p^{n}}$.
\end{enumerate}

\noindent The Weil conjectures (Weil 1949; proved in part by Grothendieck in
the 1960s and completely by Deligne in 1973) assert that the Betti numbers of
the space in (a) control the cardinalities of the sets in (b).
\end{remark}

\begin{theorem}
\label{ag11}%
\index{theorem!constructibility of n-gons}
The regular $n$-gon%
\index{regular n-gon}
is constructible if and only if $n=2^{k}p_{1}\cdots p_{s}$ where the $p_{i}$
are distinct Fermat primes.
\end{theorem}

\begin{proof}
The regular $n$-gon is constructible if and only if $\cos\frac{2\pi}{n}$
(equivalently, $\zeta=e^{2\pi i/n}$) is constructible. We know that
$\mathbb{Q}[\zeta]$ is Galois over $\mathbb{Q}$, and so (according to
\ref{ef27} and \ref{ft22}) $\zeta$ is constructible if and only if
$[\mathbb{Q}[\zeta]\colon\mathbb{Q}]$ is a power of $2$. When we write
$n=\prod p^{n(p)}$,
\[
\varphi(n)=\prod_{p|n}(p-1)p^{n(p)-1},
\]
(GT, 3.5), and this is a power of $2$ if and only if $n$ has the
required form.
\end{proof}

\begin{remark}
\label{ag12}(a) As mentioned earlier, the Fermat primes are those of the form
$2^{2^{r}}+1$. It is known that these numbers are prime when $r=0,1,2,3,4$,
but it is not known whether or not there are more Fermat primes. Thus the
problem of listing the $n$ for which the regular $n$-gon is constructible is
not yet solved (Wikipedia: Fermat numbers).

(b) The final section of Gauss's, \textit{Disquisitiones Arithmeticae}%
\emph{\/} (1801) is titled \textquotedblleft Equations defining sections of a
Circle\textquotedblright. In it Gauss proves that the $n$th roots of $1$ form
a cyclic group, that $X^{n}-1$ is solvable (this was before the theory of
abelian groups had been developed, and before Galois), and that the regular
$n$-gon is constructible when $n$ is as in the Theorem. He also claimed to
have proved the converse statement. This leads some people to credit him with
the above proof of the irreducibility of $\Phi_{n}$, but in the absence of
further evidence, I'm sticking with Dedekind.
\end{remark}

\section{Dedekind's theorem on the independence of characters}

\begin{theorem}
[Dedekind]\label{ag13}%
\index{theorem!independence of characters}%
Let $F$ be a field and $G$ a group. Every finite set $\{\chi_{1},\ldots
,\chi_{m}\}$ of group homomorphisms $G\rightarrow F^{\times}$ is linearly
independent over $F$, i.e.,
\[
\sum a_{i}\chi_{i}=0\text{\ (as a function }G\rightarrow F)\implies
a_{1}=0,\ldots,a_{m}=0.
\]

\end{theorem}

\begin{proof}
We use induction on $m$. For $m=1$, the statement is obvious. Assume it for
$m-1$, and suppose that, for some set $\{\chi_{1},\ldots,\chi_{m}\}$ of
homomorphisms $G\rightarrow F^{\times}$ and $a_{i}\in F$,%
\[
a_{1}\chi_{1}(x)+a_{2}\chi_{2}(x)+\cdots+a_{m}\chi_{m}(x)=0\quad\text{for all
}x\in G.
\]
We have to show that the $a_{i}$ are zero. As $\chi_{1}$ and $\chi_{2}$ are
distinct, they will take distinct values on some $g\in G$. On replacing $x$
with $gx$ in the equation, we find that
\[
a_{1}\chi_{1}(g)\chi_{1}(x)+a_{2}\chi_{2}(g)\chi_{2}(x)+\cdots+a_{m}\chi
_{m}(g)\chi_{m}(x)=0\quad\text{for all }x\in G.
\]
On multiplying the first equation by $\chi_{1}(g)$ and subtracting it from the
second, we obtain the equation
\[
a_{2}^{\prime}\chi_{2}+\cdots+a_{m}^{\prime}\chi_{m}=0,\qquad a_{i}^{\prime
}=a_{i}(\chi_{i}(g)-\chi_{1}(g)).
\]
The induction hypothesis shows that $a_{i}^{\prime}=0$ for $i=2,3,\ldots$. As
$\chi_{2}(g)-\chi_{1}(g)\neq0$, this implies that $a_{2}=0$, and so%
\[
a_{1}\chi_{1}+a_{3}\chi_{3}+\cdots+a_{m}\chi_{m}=0.
\]
The induction hypothesis now shows that the remaining $a_{i}$ are also zero.
\end{proof}

\begin{corollary}
\label{ag14}Let $F$ and $E$ be fields, and let $\sigma_{1},...,\sigma_{m}$ be
distinct homomorphisms $F\rightarrow E$. Then $\sigma_{1},...,\sigma_{m}$ are
linearly independent over $E.$
\end{corollary}

\begin{proof}
Apply the theorem to $\chi_{i}=\sigma_{i}|F^{\times}$.
\end{proof}

\begin{corollary}
\label{ag14m}Let $E$ be a finite separable extension of $F$ of degree $m$. Let
$\alpha_{1},\ldots,\alpha_{m}$ be a basis for $E$ as an $F$-vector space, and
let $\sigma_{1},\ldots,\sigma_{m}$ be distinct $F$-homomorphisms from $E$ into
a field $\Omega$. Then the matrix whose $(i,j)$th-entry is $\sigma_{i}%
\alpha_{j}$ is invertible.
\end{corollary}

\begin{proof}
If not, there exist $c_{i}\in\Omega$ such that $\sum\nolimits_{i=1}^{m}%
c_{i}\sigma_{i}(\alpha_{j})=0$ for all $j$. But the map $\sum\nolimits_{i=1}%
^{m}c_{i}\sigma_{i}\colon E\rightarrow\Omega$ is $F$-linear, and so this
implies that $\sum\nolimits_{i=1}^{m}c_{i}\sigma_{i}(\alpha)=0$ for all
$\alpha\in E$, which contradicts Corollary \ref{ag14}.
\end{proof}

\section{The normal basis theorem}%

\index{theorem!normal basis}%


\begin{definition}
\label{ag14a}Let $E$ be a finite Galois extension of $F$. A basis for $E$ as
an $F$-vector space is called a \emph{normal basis }%
\index{normal basis}
if it consists of the conjugates of a single element of $E$.
\end{definition}

In other words, a normal basis is one of the form%
\[
\{\sigma\alpha\mid\sigma\in\Gal(E/F)\}
\]
for some $\alpha\in E$.

\begin{theorem}
[Normal basis theorem]\label{ag14b}Every Galois extension has a normal basis.
\end{theorem}

The%
\index{group algebra}
\emph{group algebra} $FG$ of a group $G$ is the $F$-vector space with basis
the elements of $G$ endowed with the multiplication extending that of $G$.
Thus an element of $FG$ is a sum $\sum_{\sigma\in G}a_{\sigma}\sigma$,
$a_{\sigma}\in F$, and
\[
\tstyle\left(  \sum\nolimits_{\sigma}a_{\sigma}\sigma\right)  \left(
\sum\nolimits_{\sigma}b_{\sigma}\sigma\right)  =\sum\nolimits_{\sigma}\left(
\sum\nolimits_{\sigma_{1}\sigma_{2}=\sigma}a_{\sigma_{1}}b_{\sigma_{2}%
}\right)  \sigma.
\]
Every $F$-linear action of $G$ on an $F$-vector space $V$ extends uniquely to
an action of $FG$ on $V$.

Let $E/F$ be a Galois extension with Galois group $G$. Then $E$ is an
$FG$-module, and Theorem \ref{ag14b} says that there exists an element
$\alpha\in E$ such that the map%
\[
\tstyle\sum\nolimits_{\sigma}a_{\sigma}\sigma\mapsto\sum\nolimits_{\sigma
}a_{\sigma}\sigma\alpha\colon FG\rightarrow E
\]
is an isomorphism of $FG$-modules, i.e., that $E$ is a free $FG$-module of
rank $1.$

We give three proofs of Theorem \ref{ag14b}. The first assumes that $F$ is
infinite and the second that $G$ is cyclic. Since every Galois extension of a
finite field is cyclic (\ref{cg15}), this covers all cases. The third proof
applies to both finite and infinite fields, but uses the Krull-Schmidt theorem.

\subsection{Proof for infinite fields}

\begin{lemma}
\label{ag14c}Let $f\in F[X_{1},\ldots,X_{m}]$, and let $S$ be an infinite
subset of $F$. If $f(a_{1},\ldots,a_{m})=0$ for all $a_{1},\ldots,a_{m}\in S$,
then $f$ is the zero polynomial (i.e., $f=0$ in $F[X_{1},\ldots,X_{m}]$).
\end{lemma}

\begin{proof}
We prove this by induction on $m$. For $m=1$, the lemma becomes the statement
that a nonzero polynomial in one symbol has only finitely many roots (see
\ref{ef3c}). For $m>1$, write $f$ as a polynomial in $X_{m}$ with coefficients
in $F[X_{1},\ldots,X_{m-1}]$, say,
\[
f=\sum c_{i}(X_{1},\ldots,X_{m-1})X_{m}^{i}.
\]
For any $(m-1)$-tuple $a_{1},\ldots,a_{m-1}$ of elements of $S$,%
\[
f(a_{1},\ldots,a_{m-1},X_{m})
\]
is a polynomial in $X_{m}$ having every element of $S$ as a root. Therefore,
each of its coefficients is zero: $c_{i}(a_{1},\ldots,a_{m-1})=0$ for all $i$.
Since this holds for all $(a_{1},\ldots,a_{m-1})$, the induction hypothesis
shows that $c_{i}(X_{1},\ldots,X_{m-1})$ is the zero polynomial.
\end{proof}

We now prove \ref{ag14b} in the case that $F$ is infinite. Number the elements
of $G$ as $\sigma_{1},\ldots,\sigma_{m}$ with $\sigma_{1}$ the identity map.

Let $f\in F[X_{1},\ldots,X_{m}]$ have the property that
\[
f(\sigma_{1}\alpha,\ldots,\sigma_{m}\alpha)=0
\]
for all $\alpha\in E$. For a basis $\alpha_{1},\ldots,\alpha_{m}$ of $E$ over
$F$, let
\[
g(Y_{1},\ldots,Y_{m})=f(\tstyle\sum\nolimits_{i=1}^{m}Y_{i}\sigma_{1}%
\alpha_{i}, \tstyle\sum\nolimits_{i=1}^{m}Y_{i}\sigma_{2}\alpha_{i},\ldots)\in
E[Y_{1},\ldots,Y_{m}]\text{.}%
\]
The hypothesis on $f$ implies that $g(a_{1},\ldots,a_{m})=0$ for all $a_{i}\in
F$, and so $g=0$ (because $F$ is infinite). But the matrix $(\sigma_{i}%
\alpha_{j})$ is invertible (\ref{ag14m}). Since $g$ is obtained from $f$ by an
invertible linear change of variables, $f$ can be obtained from $g$ by the
inverse linear change of variables. Therefore it also is zero.

Write $X_{i}=X(\sigma_{i})$, and let $A=(X(\sigma_{i}\sigma_{j}))$, i.e., $A$
is the $m\times m$ matrix having $X_{k}$ in the $(i,j)$th place if $\sigma
_{i}\sigma_{j}=\sigma_{k}$. Then $\det(A)$ is a polynomial in $X_{1}%
,\ldots,X_{m}$, say, $\det(A)=h(X_{1},\ldots,X_{m})$. Clearly, $h(1,0,\ldots
,0)$ is the determinant of a matrix having exactly one $1$ in each row and
each column and its remaining entries $0$. Hence the rows of the matrix are a
permutation of the rows of the identity matrix, and so its determinant is
$\pm1$. In particular, $h$ is not identically zero, and so there exists an
$\alpha\in E^{\times}$ such that $h(\sigma_{1}\alpha,\ldots,\sigma_{m}\alpha)$
$(=$ $\det(\sigma_{i}\sigma_{j}\alpha)$) is nonzero. We'll show that
$\{\sigma_{i}\alpha\}$ is a normal basis. For this, it suffices to show that
the $\sigma_{i}\alpha$ are linearly independent over $F$. Suppose that
\[
\sum\nolimits_{j=1}^{m}a_{j}\sigma_{j}\alpha=0
\]
for some $a_{j}\in F$. On applying $\sigma_{1},\ldots,\sigma_{m}$
successively, we obtain a system of $m$-equations
\[
\sum a_{j}\sigma_{i}\sigma_{j}\alpha=0
\]
in the $m$ \textquotedblleft unknowns\textquotedblright\ $a_{j}$. Because this
system of equations is nonsingular, the $a_{j}$ are zero. This completes the
proof of the theorem in the case that $F$ is infinite.

\subsection{Proof when $G$ is cyclic.}

Assume that $G$ is generated by an element $\sigma_{0}$ of order $n$. Then
$[E\colon F]=n$. The minimal polynomial of $\sigma_{0}$ regarded as an
endomorphism of the $F$-vector space $E$ is the monic polynomial in $F[X]$ of
least degree such that $P(\sigma_{0})=0$ (as an endomorphism of $E$). It has
the property that it divides every polynomial $Q(X)\in F[X]$ such that
$Q(\sigma_{0})=0$. Since $\sigma_{0}^{n}=1$, $P(X)$ divides $X^{n}-1$. On the
other hand, Dedekind's theorem on the independence of characters (\ref{ag13})
implies that $1,\sigma_{0},\ldots,\sigma_{0}^{n-1}$ are linearly independent
over $F$, and so $\deg P(X)>n-1$. We conclude that $P(X)=X^{n}-1$. Therefore,
as an $F[X]$-module with $X$ acting as $\sigma_{0}$, $E$ is isomorphic to
$F[X]/(X^{n}-1)$. For any generator $\alpha$ of $E$ as an $F[X]$-module,
$\alpha,\sigma_{0}\alpha,\ldots,\sigma_{0}\alpha^{n-1}$ is an $F$-basis for
$E$.

When $F$ is finite, it is possible to replace the use of Dedekind's theorem
(\ref{ag13}) with a counting argument.

\subsection{Uniform proof}

Recall that a module is indecomposable if it is nonzero and cannot be written
as a direct sum of two nonzero submodules. The Krull-Schmidt theorem says that
every nonzero module $M$ of finite length over a ring can be written as a
direct sum of indecomposable modules and that the indecomposable modules
occurring in a decomposition are unique up to order and isomorphism. Thus
$M=\bigoplus_{i}m_{i}M_{i}$ where $M_{i}$ is indecomposable and $m_{i}M_{i}$
denotes the direct sum of $m_{i}$ copies of $M_{i}$; the set of isomorphism
classes of the $M_{i}$ is uniquely determined and, when we choose the $M_{i}$
to be pairwise nonisomorphic, each $m_{i}$ is uniquely determined. From this
it follows that two modules $M$ and $M^{\prime}$ of finite length over a ring
are isomorphic if $mM\approx mM^{\prime}$ for some $m\geq1$.

Consider the $F$-vector space $E\otimes_{F}E$. We let $E$ act on the first
factor, and $G$ act on the second factor (so $a(x\otimes y)=ax\otimes y$,
$a\in E$, and $\sigma(x\otimes y)=x\otimes\sigma y$, $\sigma\in G$). We'll
prove Theorem \ref{ag14b} by showing that%
\[
\underbrace{FG\oplus\cdots\oplus FG}_{n}\approx E\otimes_{F}E\approx
\underbrace{E\oplus\cdots\oplus E}_{n}%
\]
as $FG$-modules ($n=[E\colon F]$).

For $\sigma\in G$, let $\lambda_{\sigma}\colon E\otimes_{F}E\rightarrow E$
denote the map $x\otimes y\mapsto x\cdot\sigma y$. Then $\lambda_{\sigma}$ is
obviously $E$-linear, and $\lambda_{\sigma}(\tau z)=\lambda_{\sigma\tau}(z)$
for all $\tau\in G$ and $z\in E\otimes_{F}E$. I claim that $\{\lambda_{\sigma
}\mid\sigma\in G\}$ is an $E$-basis for $\Hom_{E\text{-linear}}(E\otimes
_{F}E,E)$. As this space has dimension $n$, it suffices to show that the set
is linearly independent. But if $\sum_{\sigma}c_{\sigma}\lambda_{\sigma}=0$,
$c_{\sigma}\in E$, then%
\[
0=\sum\nolimits_{\sigma}c_{\sigma}(\lambda_{\sigma}(1\otimes y))=\sum
\nolimits_{\sigma}c_{\sigma}\cdot\sigma y
\]
for all $y\in E$, which implies that all $c_{\sigma}=0$ by Dedekind's theorem
\ref{ag13}.

Consider the map%
\[
\phi\colon E\otimes_{F}E\rightarrow EG,\quad z\mapsto\sum\nolimits_{\sigma
}\lambda_{\sigma}(z)\cdot\sigma^{-1}.
\]
Then $\phi$ is $E$-linear. If $\phi(z)=0$, then $\lambda_{\sigma}(z)=0$ for
all $\sigma\in G$, and so $z=0$ in $E\otimes_{F}E$ (because the $\lambda
_{\sigma}$ span the dual space). Therefore $\phi$ is injective, and as
$E\otimes_{F}E\ $and $EG$ both have dimension $n$ over $E$, it is an
isomorphism. For $\tau\in G$,%
\begin{align*}
\phi(\tau z)  &  =\sum\nolimits_{\sigma}\lambda_{\sigma}(\tau z)\cdot
\sigma^{-1}\\
&  =\sum\nolimits_{\sigma}\lambda_{\sigma\tau}(z)\cdot\tau(\sigma\tau)^{-1}\\
&  =\tau\phi(z),
\end{align*}
and so $\phi$ is an isomorphism of $EG$-modules. Thus%
\[
E\otimes_{K}E\simeq EG\approx FG\oplus\cdots\oplus FG
\]
as an $FG$-module.

On the other hand, for any basis $\{e_{1},\ldots,e_{n}\}$ for $E$ as an
$F$-vector space,
\[
E\otimes_{F}E=(e_{1}\otimes E)\oplus\cdots\oplus(e_{n}\otimes E)\simeq
E\oplus\cdots\oplus E
\]
as $FG$-modules. This completes the proof.

\begin{nt}
The normal basis theorem was stated for finite fields by Eisenstein in 1850,
and proved for finite fields by Hensel in 1888. Dedekind used normal bases in
number fields in his work on the discriminant in 1880, but he had no general
proof. Emmy Noether gave a proof for some infinite fields (1932) and Deuring
gave a uniform proof (also 1932). The above uniform proof simplifies that of
Deuring --- see Blessenohl, Dieter. On the normal basis theorem. Note Mat.~27
(2007), 5--10. According to the Wikipedia, normal bases are frequently used in
cryptographic applications that are based on the discrete logarithm problem
such as elliptic curve cryptography.
\end{nt}

\section{Hilbert's Theorem 90}

Let $G$ be a group. A $G$\emph{-module\/}%
\index{module!G-}
is an abelian group $M$ together with an
\index{action of a group}%
\emph{action} of $G$, i.e., a map $G\times M\rightarrow M$ such that

\begin{enumerate}
\item $\sigma(m+m^{\prime})=\sigma m+\sigma m^{\prime}$ for all $\sigma\in G$,
$m,m^{\prime}\in M$;

\item $(\sigma\tau)(m)=\sigma(\tau m)$ for all $\sigma,\tau\in G$, $m\in M$;

\item $1_{G}m=m$ for all $m\in M$.
\end{enumerate}

\noindent Thus, to give an action of $G$ on $M$ is the same as giving a
homomorphism $G\rightarrow\Aut(M)$. A
\index{G-module@$G$-module}%
$G$\emph{-module} is an abelian group together with an action of $G$.

\begin{example}
\label{ag15}Let $E$ be a Galois extension of $F$ with Galois group $G$. Then
$(E,+)$ and $(E^{\times},\cdot)$ are $G$-modules.
\end{example}

Let $M$ be a $G$-module. A \emph{crossed homomorphism\/}%
\index{homomorphism!crossed}
is a map $f\colon G\rightarrow M$ such that
\[
f(\sigma\tau)=f(\sigma)+\sigma f(\tau)\text{ for all }\sigma,\tau\in G\text{.}%
\]
Note that the condition implies that $f(1)=f(1\cdot1)=f(1)+f(1)$, and so
$f(1)=0.$

\begin{example}
\label{ag16} (a) Let $f\colon G\rightarrow M$ be a crossed homomorphism. For
any $\sigma\in G$,
\begin{align*}
f(\sigma^{2})  &  =f(\sigma)+\sigma f(\sigma),\\
f(\sigma^{3})  &  =f(\sigma\cdot\sigma^{2})=f(\sigma)+\sigma f(\sigma
)+\sigma^{2}f(\sigma)\\
&  \cdots\\
f(\sigma^{n})  &  =f(\sigma)+\sigma f(\sigma)+\cdots+\sigma^{n-1}f(\sigma).
\end{align*}
Thus, if $G$ is a cyclic group of order $n$ generated by $\sigma$, then a
crossed homomorphism $f\colon G\rightarrow M$ is determined by its value, $x$
say, on $\sigma$, and $x$ satisfies the equation
\begin{equation}
x+\sigma x+\cdots+\sigma^{n-1}x=0, \label{eq6}%
\end{equation}
Moreover, if $x\in M$ satisfies (\ref{eq6}), then the formulas $f(\sigma
^{i})=x+\sigma x+\cdots+\sigma^{i-1}x$ define a crossed homomorphism $f\colon
G\rightarrow M$. Thus, for a finite cyclic group $G=\langle\sigma\rangle$,
there is a one-to-one correspondence
\[
\{\text{crossed homs }f\colon G\rightarrow M\}\overset{f\leftrightarrow
f(\sigma)}{\longleftrightarrow}\{x\in M\text{\ satisfying (\ref{eq6})}\}.
\]


(b) For every $x\in M$, we obtain a crossed homomorphism by putting
\[
f(\sigma)=\sigma x-x,\qquad\text{all }\sigma\in G.
\]
Such a crossed homomorphism is said to be \emph{principal}.
\index{homomorphism!principal crossed}%


(c) If $G$ acts trivially on $M$, i.e., $\sigma m=m$ for all $\sigma\in G$ and
$m\in M$, then a crossed homomorphism is simply a homomorphism, and there are
no nonzero principal crossed homomorphisms.
\end{example}

The sum and difference of two crossed homomorphisms is again a crossed
homomorphism, and the sum and difference of two principal crossed
homomorphisms is again principal. Thus we can define
\[
H^{1}(G,M)=\frac{\{\text{crossed homomorphisms}\}}{\{\text{principal crossed
homomorphisms}\}}%
\]
(quotient abelian group). There are also cohomology groups%
\index{cohomology group}
$H^{n}(G,M)$ for $n>1$, but they were not inroduced until the twentieth
century, and so will not be discussed in this course. An exact sequence of
$G$-modules%
\[
0\rightarrow M^{\prime}\rightarrow M\rightarrow M^{\prime\prime}\rightarrow0
\]
gives rise to an exact sequence%
\[
0\longrightarrow M^{\prime G}\longrightarrow M^{G}\longrightarrow
M^{\prime\prime G}\overset{d}{\longrightarrow}H^{1}(G,M^{\prime}%
)\longrightarrow H^{1}(G,M)\longrightarrow H^{1}(G,M^{\prime\prime}).
\]
Let $m^{\prime\prime}\in M^{\prime\prime G}$, and let $m\in M$ map to
$m^{\prime\prime}$. For all $\sigma\in G$, $\sigma m-m$ lies in the submodule
$M^{\prime}$ of $M$, and $\sigma\mapsto\sigma m-m\colon G\rightarrow
M^{\prime}$ is a crossed homomorphism, whose class we define to be
$d(m^{\prime\prime})$. We leave it as an exercise for the reader to check the exactness.

\begin{example}
\label{ag17}Let $\pi\colon\tilde{X}\rightarrow X$ be the universal covering
space of a topological space $X$, and let $\Gamma$ be the group of covering
transformations. Under some fairly general hypotheses, a $\Gamma$-module $M$
will define a sheaf $\mathcal{M}$ on $X$, and $H^{1}(X,\mathcal{M})\simeq
H^{1}(\Gamma,M)$. For example, when $M=\mathbb{Z}$ with the trivial action of
$\Gamma$, this becomes the isomorphism $H^{1}(X,\mathbb{Z})\simeq H^{1}%
(\Gamma,\mathbb{Z})=\Hom(\Gamma,\mathbb{Z})$.
\end{example}

\begin{theorem}
\label{ag18}Let $E$ be a Galois extension of $F$ with group $G$; then
$H^{1}(G,E^{\times})=0$, i.e., every crossed homomorphism $G\rightarrow
E^{\times}$ is principal.
\end{theorem}

\begin{proof}
Let $f$ be a crossed homomorphism $G\rightarrow E^{\times}$. In multiplicative
notation, this means that
\[
f(\sigma\tau)=f(\sigma)\cdot\sigma(f(\tau)),\quad\sigma,\tau\in G,
\]
and we have to find a $\gamma\in E^{\times}$ such that $f(\sigma)=\frac
{\sigma\gamma}{\gamma}$ for all $\sigma\in G$. Because the $f(\tau)$ are
nonzero, Corollary \ref{ag14} implies that
\[
\sum\nolimits_{\tau\in G}f(\tau)\tau\colon E\rightarrow E
\]
is not the zero map, i.e., there exists an $\alpha\in E$ such that
\[
\beta\overset{\df}{=}\sum\nolimits_{\tau\in G}f(\tau
)\tau\alpha\neq0.
\]
But then, for $\sigma\in G$,
\begin{align*}
\sigma\beta &  =\sum\nolimits_{\tau\in G}\sigma(f(\tau))\cdot\sigma\tau
(\alpha)\\
&  =\sum\nolimits_{\tau\in G}f(\sigma)^{-1}\ f(\sigma\tau)\cdot\sigma
\tau(\alpha)\\
&  =f(\sigma)^{-1}\ \sum\nolimits_{\tau\in G}f(\sigma\tau)\sigma\tau(\alpha),
\end{align*}
which equals $f(\sigma)^{-1}\beta$ because, as $\tau$ runs over $G$, so also
does $\sigma\tau$. Therefore, $f(\sigma)=\frac{\beta}{\sigma(\beta)}%
=\frac{\sigma(\beta^{-1})}{\beta^{-1}}$.
\end{proof}

Let $E$ be a Galois extension of $F$ with Galois group $G$. We define the
\emph{norm}%
\index{norm}%
\emph{\/} of an element $\alpha\in E$ to be
\[
\Nm\alpha=\prod\nolimits_{\sigma\in G}\sigma\alpha.
\]
For $\tau\in G$,
\[
\tau(\Nm\alpha)=\prod\nolimits_{\sigma\in G}\tau\sigma\alpha=\Nm\alpha,
\]
and so $\Nm\alpha\in F$. The map
\[
\alpha\mapsto\Nm\alpha\colon E^{\times}\rightarrow F^{\times}%
\]
is a obviously a homomorphism.

\begin{example}
\label{ag18a}The norm map $\mathbb{C}^{\times}\rightarrow\mathbb{R}^{\times}$
is $\alpha\mapsto|\alpha|^{2}$ and the norm map $\mathbb{Q}[\sqrt{d}]^{\times
}\rightarrow\mathbb{Q}^{\times}$ is $a+b\sqrt{d}\mapsto a^{2}-db^{2}$.
\end{example}

We are interested in determining the kernel of the norm map. Clearly an
element of the form $\frac{\beta}{\tau\beta}$ has norm $1$, and our next
result shows that, for cyclic extensions, all elements with norm $1$ are of
this form.

\begin{corollary}
[Hilbert's theorem 90]\label{ag19}Let $E$ be a finite cyclic extension of $F$,
and let $\sigma$ generate $\Gal(E/F)$. Let $\alpha\in E^{\times}$; if
$\Nm_{E/F}\alpha=1$, then $\alpha=\beta/\sigma\beta$ for some $\beta\in E$.
\end{corollary}

\begin{proof}
Let $m=[E\colon F]$. The condition on $\alpha$ is that $\alpha\cdot
\sigma\alpha\cdots\sigma^{m-1}\alpha=1$, and so (see \ref{ag16}a) there is a
crossed homomorphism $f\colon\langle\sigma\rangle\rightarrow E^{\times}$ with
$f(\sigma)=\alpha$. Theorem \ref{ag18} now shows that $f$ is principal, which
means that there is a $\beta$ with $f(\sigma)=\beta/\sigma\beta.$
\end{proof}

\begin{nt}
The corollary is Satz 90 in Hilbert's book, Theorie der Algebraischen
Zahlk\"{o}rper, 1897. The theorem was discovered by Kummer in the special case
of $\mathbb{Q}{}[\zeta_{p}]/\mathbb{Q}{}$, and generalized to Theorem
\ref{ag18} by Emmy Noether. Theorem \ref{ag18}, as well as various vast
generalizations of it, are also referred to as Hilbert's Theorem 90. For an
illuminating discussion of Hilbert's book, see the introduction to its English
translation, written by Lemmermeyer and Schappacher.
\end{nt}

\begin{nt}
With the obvious notion of morphism, the $G$-modules form a category. This is
essentially the same as the category of $\mathbb{Z}G$-modules, where
$\mathbb{Z}G$ is the group ring of $G$ (Wikipedia: Group ring). Thus, the
category has enough injectives, and the $H^{1}$ is the first right derived
functor of $M\rightsquigarrow M^{G}$.
\end{nt}

\section{Cyclic extensions}

Let $F$ be a field containing a primitive $n$th root of $1$, some $n\geq2$,
and write $\mu_{n}$ for the group of $n$th roots of $1$ in $F$. Then $\mu_{n}$
is a cyclic subgroup of $F^{\times}$ of order $n$ with generator $\zeta$ say.
In this section, we classify the cyclic extensions of degree $n$ of $F$.

Consider a field $E=F[\alpha]$ generated by an element $\alpha$ whose $n$th
power (but no smaller power) is in $F$. Then $\alpha$ is a root of $X^{n}-a$,
and the remaining roots are the elements $\zeta^{i}\alpha$, $1\leq i\leq n-1$.
Since these all lie in $E$, $E$ is a Galois extension of $F$, with Galois
group $G$ say. For every $\sigma\in G$, $\sigma\alpha$ is also a root of
$X^{n}-a$, and so $\sigma\alpha=\zeta^{i}\alpha$ for some $i$. Hence
$\sigma\alpha/\alpha\in\mu_{n}$. The map
\[
\sigma\mapsto\sigma\alpha/\alpha\colon G\rightarrow\mu_{n}%
\]
doesn't change when $\alpha$ is replaced by a conjugate, and it follows that
the map is a homomorphism:%
\[
\frac{\sigma\tau\alpha}{\alpha}=\frac{\sigma(\tau\alpha)}{\tau\alpha}%
\frac{\tau\alpha}{\alpha}.
\]
If $\sigma$ lies in the kernel of the map $G\rightarrow\mu_{n}$, then
$\sigma\alpha=\alpha$, and so $\sigma$ acts trivially on $E=F[\alpha]$;
therefore $\sigma$ is the identity element, and the map is injective. If it is
not surjective, then $G$ maps into a subgroup $\mu_{d}$ of $\mu_{n}$, some
$d|n$, $d<n$. In this case, $(\sigma\alpha/\alpha)^{d}=1$, i.e., $\sigma
\alpha^{d}=\alpha^{d}$, for all $\sigma\in G$, and so $\alpha^{d}\in F$,
contradicting the hypothesis on $\alpha$. Thus the map is surjective. We have
proved the first part of the following statement.

\begin{proposition}
\label{ag19b} Let $F$ be a field containing a primitive $n$th root of $1$. Let
$E=F[\alpha]$ where $\alpha^{n}\in F$ and no smaller power of $\alpha$ is in
$F$. Then $E$ is a Galois extension of $F$ with cyclic Galois group of order
$n$. Conversely, if $E$ is a cyclic extension of $F$ of degree $n$, then
$E=F[\alpha]$ for some $\alpha$ with $\alpha^{n}\in F$.
\end{proposition}

\begin{proof}
It remains to prove the last statement. Let $\sigma$ generate $G$ and let
$\zeta$ generate $\mu_{n}$. It suffices to find an element $\alpha\in
E^{\times}$ such that $\sigma\alpha=\zeta^{-1}\alpha$, for then $\alpha^{n}$
is the smallest power of $\alpha$ lying in $F$. As $1,\sigma,\ldots
,\sigma^{n-1}$ are distinct homomorphisms $F^{\times}\rightarrow F^{\times}$,
Dedekind's Theorem \ref{ag13} shows that $\sum_{i=0}^{n-1}\zeta^{i}\sigma^{i}$
is not the zero function, and so there exists a $\gamma$ such that
$\alpha\overset{\df}{=}$ $\sum\zeta^{i}\sigma^{i}\gamma\neq
0$. Now $\sigma\alpha=\zeta^{-1}\alpha$.
\end{proof}

\begin{aside}
\label{ag21}(a) It is not difficult to show that the polynomial $X^{n}-a$ is
irreducible in $F[X]$ if $a$ is not a $p$th power for any prime $p$ dividing
$n$. When we drop the condition that $F$ contains a primitive $n$th root of
$1$, this is still true except that, if $4|n$, we need to add the condition
that $a\notin-4F^{4}$. See Lang, Algebra, Springer, 2002, VI, \S 9, Theorem
9.1, p.~297.

(b) If $F$ has characteristic $p$ (hence has no $p$th roots of $1$ other than
$1$), then $X^{p}-X-a$ is irreducible in $F[X]$ unless $a=b^{p}-b$ for some
$b\in F$, and when it is irreducible, its Galois group is cyclic of order $p$
(generated by $\alpha\mapsto\alpha+1$ where $\alpha$ is a root). Moreover,
every cyclic extension of $F$ of degree $p$ is the splitting field of such a polynomial.
\end{aside}

\begin{proposition}
\label{ag20} Let $F$ be a field containing a primitive $n$th root of $1$. Two
cyclic extensions $F[a^{\frac{1}{n}}]$ and $F[b^{\frac{1}{n}}]$ of $F$ of
degree $n$ are equal if and only if $a=b^{r}c^{n}$ for some $r\in\mathbb{Z}$
relatively prime to $n$ and some $c\in F^{\times}$, i.e., if and only if $a$
and $b$ generate the same subgroup of $F^{\times}/F^{\times n}$.
\end{proposition}

\begin{proof}
Only the \textquotedblleft only if\textquotedblright\ part requires proof. We
are given that $F[\alpha]=F[\beta]$ with $\alpha^{n}=a$ and $\beta^{n}=b$. Let
$\sigma$ be the generator of the Galois group with $\sigma\alpha=\zeta\alpha$,
and let $\sigma\beta=\zeta^{i}\beta$, $(i,n)=1$. We can write
\[
\beta=\sum_{j=0}^{n-1}c_{j}\alpha^{j},\quad c_{j}\in F,
\]
and then
\[
\sigma\beta=\sum_{j=0}^{n-1}c_{j}\zeta^{j}\alpha^{j}.
\]
On comparing this with $\sigma\beta=\zeta^{i}\beta$, we find that $\zeta
^{i}c_{j}=\zeta^{j}c_{j}$ for all $j$. Hence $c_{j}=0$ for $j\neq i$, and
therefore $\beta=c_{i}\alpha^{i}$.
\end{proof}

\section{Kummer theory}

Throughout this section, $F$ is a field and $\zeta$ is a primitive $n$th root
of $1$ in $F$.\emph{\/} In particular, $F$ either has characteristic $0$ or
characteristic $p$ not dividing $n$.

The last two proposition give us a complete classification of the cyclic
extensions of $F$ of degree $n$. We now extend this to a classification of all
abelian extensions of $F$ whose Galois group has exponent $n$. (Recall that
the
\index{exponent}%
\emph{exponent} of a group $G$ is the smallest integer $n\geq1$ such that
$\sigma^{n}=1$ for all $\sigma\in G$. A finite abelian group of exponent $n$
is isomorphic to a subgroup of $(\mathbb{Z}/n\mathbb{Z})^{r}$ for some $r$.)

Let $E/F$ be a finite Galois extension with Galois group $G$. From the exact
sequence
\[
1\rightarrow\mu_{n}\xrightarrow{\phantom{x\mapsto x^{n}}} E^{\times
}\xrightarrow{x\mapsto x^{n}}E^{\times n}\rightarrow1
\]
we obtain a cohomology sequence
\[
1\rightarrow\mu_{n}\rightarrow F^{\times}\xrightarrow{x\mapsto x^{n}}F^{\times
}\cap E^{\times n}\rightarrow H^{1}(G,\mu_{n})\rightarrow1.
\]
The $1$ at the right is because of Hilbert's Theorem 90. Thus we obtain an
isomorphism
\[
F^{\times}\cap E^{\times n}/F^{\times n}\rightarrow\Hom(G,\mu_{n}).
\]
This map can be described as follows: let $a$ be an element of $F^{\times}$
that becomes an $n$th power in $E$, say $a=\alpha^{n}$; then $a$ maps to the
homomorphism $\sigma\mapsto\frac{\sigma\alpha}{\alpha}$. If $G$ is abelian of
exponent $n$, then
\[
\left\vert \Hom(G,\mu_{n})\right\vert =(G\colon1).
\]


\begin{theorem}
\label{ag20a} The map
\[
E\mapsto F^{\times}\cap E^{\times n}%
\]
defines a one-to-one correspondence between the sets of

\begin{enumerate}
\item finite abelian extensions of $F$ of exponent $n$ contained in some fixed
algebraic closure $\Omega$ of $F,$ and

\item subgroups $B$ of $F^{\times}$ containing $F^{\times n}$ as a subgroup of
finite index.
\end{enumerate}

\noindent The extension corresponding to $B$ is $F[B^{\frac{1}{n}}]$, the
smallest subfield of $\Omega$ containing $F$ and an $n$th root of each element
of $B$. If $E\leftrightarrow B$, then $[E\colon F]=(B\colon F^{\times n})$.
\end{theorem}

\begin{proof}
For any finite Galois extension $E$ of $F$, define $B(E)=F^{\times}\cap
E^{\times n}$. Then $E\supset F[B(E)^{\frac{1}{n}}]$, and for any group $B$
containing $F^{\times n}$ as a subgroup of finite index, $B(F[B^{\frac{1}{n}%
}])\supset B$. Therefore,
\[
\lbrack E\colon F]\geq\lbrack F[B(E)^{\frac{1}{n}}]\colon F]=(B(F[B(E)^{\frac
{1}{n}}])\colon F^{\times n})\geq(B(E)\colon F^{\times n}).
\]
If $E/F$ is abelian of exponent $n$, then $[E\colon F]=(B(E)\colon F^{\times
n})$, and so equalities hold throughout: $E=F[B(E)^{\frac{1}{n}}]$.

Next consider a group $B$ containing $F^{\times n}$ as a subgroup of finite
index, and let $E=F[B^{\frac{1}{n}}]$. Then $E$ is a composite of the
extensions $F[a^{\frac{1}{n}}]$ for $a$ running through a set of generators
for $B/F^{\times n}$, and so it is a finite abelian extension of exponent $n$.
Therefore
\[
a\mapsto\left(  \sigma\mapsto\frac{\sigma a^{\frac{1}{n}}}{a^{\frac{1}{n}}%
}\right)  \colon B(E)/F^{\times n}\rightarrow\Hom(G,\mu_{n}),\quad
G=\Gal(E/F),
\]
is an isomorphism. This map sends $B/F^{\times n}$ isomorphically onto the
subgroup $\Hom(G/H,\mu_{n})$ of $\Hom(G,\mu_{n})$ where $H$ consists of the
$\sigma\in G$ such that $\sigma a^{\frac{1}{n}}/a^{\frac{1}{n}}=1$ for all
$a\in B$. But such a $\sigma$ fixes all $a^{\frac{1}{n}}$ for $a\in B$, and
therefore is the identity automorphism on $E=F[B^{\frac{1}{n}}]$. This shows
that $B(E)=B$, and hence $E\mapsto B(E)$ and $B\mapsto F[B^{\frac{1}{n}}]$ are
inverse bijections.
\end{proof}

\begin{example}
\label{ag20b}(a) The theorem says that the abelian extensions of $\mathbb{R}$
of exponent $2$ are indexed by the subgroups of $\mathbb{R}^{\times
}/\mathbb{R}^{\times2}=\{\pm1\}$. This is certainly true.

(b) The theorem says that the finite abelian extensions of $\mathbb{Q}$ of
exponent $2$ are indexed by the finite subgroups of $\mathbb{Q}^{\times
}/\mathbb{Q}^{\times2}$. Modulo squares, every nonzero rational number has a
unique representative of the form $\pm p_{1}\cdots p_{r}$ with the $p_{i}$
prime numbers. Therefore $\mathbb{Q}^{\times}/\mathbb{Q}^{\times2}$ is a
direct sum of cyclic groups of order $2$ indexed by the prime numbers plus
$\infty$. The extension corresponding to the subgroup generated by the primes
$p_{1},\ldots,p_{r}$ (and $-1$) is obtained by adjoining the square roots of
$p_{1},\ldots,p_{r}$ (and $-1$) to $\mathbb{Q}{}$.
\end{example}

\begin{remark}
\label{ag22}Let $E$ be an abelian extension of $F$ of exponent $n$, and let
\[
B(E)=\{a\in F^{\times}\mid a\text{\ becomes an }n\text{th power in }E\}.
\]
There is a perfect pairing
\[
(a,\sigma)\mapsto\frac{\sigma a^{\frac{1}{n}}}{a^{\frac{1}{n}}}\colon
\frac{B(E)}{F^{\times n}}\times\Gal(E/F)\rightarrow\mu_{n}.
\]
Cf. Exercise \ref{x5} for the case $n=2$.
\end{remark}

\section{Proof of Galois's solvability theorem}

\begin{lemma}
\label{ag24}Let $f\in F[X]$ be separable, and let $F^{\prime}$ be a field
containing $F$. Then the Galois group of $f$ as an element of $F^{\prime}[X]$
is a subgroup of the Galois group of $f$ as an element of $F[X].$
\end{lemma}

\begin{proof}
Let $E^{\prime}$ be a splitting field for $f$ over $F^{\prime}$, and let
$\alpha_{1},\ldots,\alpha_{m}$ be the roots of $f(X)$ in $E^{\prime}$. Then
$E=F[\alpha_{1},...,\alpha_{m}]$ is a splitting field of $f$ over $F$. Every
element of $\Gal(E^{\prime}/F^{\prime})$ permutes the $\alpha_{i}$ and so maps
$E$ into itself. The map $\sigma\mapsto\sigma|E$ is an injection
$\Gal(E^{\prime}/F^{\prime})\rightarrow\Gal(E/F).$
\end{proof}

\begin{theorem}
\label{ag23} Let $F$ be a field of characteristic $0$. A polynomial in $F[X]$
is solvable if and only if its Galois group is solvable.
\end{theorem}

\begin{proof}
$\Longleftarrow$: Let $f\in F[X]$ have solvable Galois group $G_{f}$. Let
$F^{\prime}=F[\zeta]$ where $\zeta$ is a primitive $n$th root of $1$ for some
large $n$ --- for example, $n=(\deg f)!$ will do. The lemma shows that the
Galois group $G$ of $f$ as an element of $F^{\prime}[X]$ is a subgroup of
$G_{f}$, and hence is also solvable (GT, 6.6a). This means that
there is a sequence of subgroups
\[
G=G_{0}\supset G_{1}\supset\cdots\supset G_{m-1}\supset G_{m}=\{1\}
\]
such that each $G_{i}$ is normal in $G_{i-1}$ and $G_{i-1}/G_{i}$ is cyclic.
Let $E$ be a splitting field of $f(X)$ over $F^{\prime}$, and let
$F_{i}=E^{G_{i}}$. We have a sequence of fields
\[
F\subset F[\zeta]=F^{\prime}=F_{0}\subset F_{1}\subset F_{2}\subset
\cdots\subset F_{m}=E
\]
with $F_{i}$ cyclic over $F_{i-1}$. Theorem \ref{ag19b} shows that
$F_{i}=F_{i-1}[\alpha_{i}]$ with $\alpha_{i}^{[F_{i}\colon F_{i-1}]}\in
F_{i-1}$, each $i$, and this shows that $f$ is solvable.

$\Longrightarrow$: It suffices to show that $G_{f}$ is a quotient of a
solvable group (GT, 6.6a). Hence it suffices to find a solvable
extension $\tilde{E}$ of $F$ such that $f(X)$ splits in $\tilde{E}[X]$.

We are given that there exists a tower of fields%
\[
F=F_{0}\subset F_{1}\subset F_{2}\subset\cdots\subset F_{m}%
\]
such that

\begin{enumerate}
\item $F_{i}=F_{i-1}[\alpha_{i}]$, $\alpha_{i}^{r_{i}}\in F_{i-1}$;

\item $F_{m}$ contains a splitting field for $f.$
\end{enumerate}

Let $n=r_{1}\cdots r_{m}$, and let $\Omega$ be a field Galois over $F$ and
containing (a copy of) $F_{m}$ and a primitive $n$th root $\zeta$ of $1.$ For
example, choose a primitive element $\gamma$ for $F_{m}\ $over $F$ (see
\ref{ag1}), and take $\Omega$ to be a splitting field of $g(X)(X^{n}-1)$ where
$g(X)$ is the minimal polynomial of $\gamma$ over $F$. Alternatively, apply
\ref{sf9}a.

Let $G$ be the Galois group of $\Omega/F$, and let $\tilde{E}$ be the Galois
closure of $F_{m}[\zeta]$ in $\Omega$. According to (\ref{ft18}a), $\tilde{E}$
is the composite of the fields $\sigma F_{m}[\zeta]$, $\sigma\in G$, and so it
is generated over $F$ by the elements
\[
\zeta,\alpha_{1},\alpha_{2},\ldots,\alpha_{m},\sigma\alpha_{1},\ldots
,\sigma\alpha_{m},\sigma^{\prime}\alpha_{1},\ldots.
\]
We adjoin these elements to $F$ one by one to get a sequence of fields
\[
F\subset F[\zeta]\subset F[\zeta,\alpha_{1}]\subset\cdots\subset F^{\prime
}\subset F^{\prime\prime}\subset\cdots\subset\tilde{E}%
\]
in which each field $F^{\prime\prime}$ is obtained from its predecessor
$F^{\prime}$ by adjoining an $r$th root of an element of $F^{\prime}$
($r=r_{1},\ldots,r_{m},$ or $n$). According to (\ref{ag7}) and (\ref{ag19b}),
each of these extensions is abelian (and even cyclic after the first), and so
$\tilde{E}/F$ is a solvable extension.
\end{proof}

\begin{aside}
One of Galois's major achievements was to show that an irreducible polynomial
of prime degree in $\mathbb{Q}{}[X]$ is solvable by radicals if and only if
its splitting field is generated by any two roots of the
polynomial.\footnote{Pour qu'une \'{e}quation de degr\'{e} premier soit
r\'{e}soluble par radicaux, il faut et il suffit que deux quelconques de ces
racines \'{e}tant connues, les autres s'en d\'{e}duisent rationnellement
(\'{E}variste Galois, Bulletin de M. F\'{e}russac, XIII (avril 1830), p.
271).} This theorem of Galois answered a question on mathoverflow in 2010
(mo24081). For a partial generalization of Galois's theorem, see mo110727.
\end{aside}

\section{Symmetric polynomials\label{sympol}}

Let $R$ be a commutative ring (with $1$). A polynomial $P(X_{1},...,X_{n})\in
R[X_{1},\ldots,X_{n}]$ is said to be \emph{symmetric\/}%
\index{symmetric polynomial}
if it is unchanged when its variables are permuted, i.e., if
\[
P(X_{\sigma(1)},\ldots,X_{\sigma(n)})=P(X_{1},\ldots,X_{n}),\quad\text{all
}\sigma\in\text{$S_{n}$}.
\]
For example
\[
\renewcommand{\arraystretch}{1.3}%
\begin{array}
[c]{rcll}%
p_{1} & = & \sum_{i}X_{i} & =X_{1}+X_{2}+\cdots+X_{n},\\
p_{2} & = & \sum_{i<{}j}X_{i}X_{j} & =X_{1}X_{2}+X_{1}X_{3}+\cdots+X_{1}%
X_{n}+X_{2}X_{3}+\cdots+X_{n-1}X_{n},\\
p_{3} & = & \sum_{i<{}j<{}k}X_{i}X_{j}X_{k}, & =X_{1}X_{2}X_{3}+\cdots\\
& \cdots &  & \\
p_{r} & = & \sum_{i_{1}<{}\cdots<{}i_{r}}X_{i_{1}}...X_{i_{r}} & \\
& \cdots &  & \\
p_{n} & = & X_{1}X_{2}\cdots X_{n} &
\end{array}
\]
are each symmetric because $p_{r}$ is the sum of \textit{all\/} monomials of
degree $r$ made up out of distinct $X_{i}$. These particular polynomials are
called the \emph{elementary symmetric polynomials}.%
\index{symmetric polynomial!elementary}%


\begin{theorem}
[Symmetric polynomials theorem]\label{ag25}Every symmetric polynomial
$P(X_{1},...,X_{n})$ in $R[X_{1},...,X_{n}]$ is equal to a polynomial in the
elementary symmetric polynomials with coefficients in $R$, i.e., $P\in
R[p_{1},...,p_{n}].$
\end{theorem}

\begin{proof}
We define an ordering on the monomials in the $X_{i}$ by requiring that
\[
X_{1}^{i_{1}}X_{2}^{i_{2}}\cdots X_{n}^{i_{n}}>X_{1}^{j_{1}}X_{2}^{j_{2}%
}\cdots X_{n}^{j_{n}}%
\]
if either
\[
i_{1}+i_{2}+\cdots+i_{n}>j_{1}+j_{2}+\cdots+j_{n}%
\]
or equality holds and, for some $s$,
\[
i_{1}=j_{1},\,\,\ldots,\,\,i_{s}=j_{s},\text{\ but }i_{s+1}>j_{s+1}.
\]
For example,
\[
X_{1}X_{2}X_{3}^{3}>X_{1}X_{2}^{2}X_{3}>X_{1}X_{2}X_{3}^{2}.
\]


Let $P(X_{1},\ldots,X_{n})$ be a symmetric polynomial, and let $X_{1}^{i_{1}%
}\cdots X_{n}^{i_{n}}$ be the highest monomial occurring in $P$ with a nonzero
coefficient, so%
\[
P=cX_{1}^{i_{1}}\cdots X_{n}^{i_{n}}+\text{lower terms,}\quad c\neq0.
\]
Because $P$ is symmetric, it contains all monomials obtained from
$X_{1}^{i_{1}}\cdots X_{n}^{i_{n}}$ by permuting the $X$. Hence $i_{1}\geq
i_{2}\geq\cdots\geq i_{n}$.

The highest monomial in $p_{i}$ is $X_{1}\cdots X_{i}$, and it follows that
the highest monomial in $p_{1}^{d_{1}}\cdots p_{n}^{d_{n}}$ is
\begin{equation}
X_{1}^{d_{1}+d_{2}+\cdots+d_{n}}X_{2}^{d_{2}+\cdots+d_{n}}\cdots X_{n}^{d_{n}%
}. \label{e2}%
\end{equation}
Therefore the highest monomial of
\begin{equation}
P(X_{1},\ldots,X_{n})-cp_{1}^{i_{1}-i_{2}}p_{2}^{i_{2}-i_{3}}\cdots
p_{n}^{i_{n}} \label{e3}%
\end{equation}
is strictly less than the highest monomial in $P(X_{1},\ldots,X_{n})$. We can
repeat this argument with the polynomial (\ref{e3}), and after a finite number
of steps, we will arrive at a representation of $P$ as a polynomial in
$p_{1},\ldots,p_{n}$.
\end{proof}

\begin{remark}
\label{ag27a}(a) The proof is algorithmic. Consider, for
example,\footnote{From the Wikipedia: elementary symmetric polynomials.}
\begin{align*}
P(X_{1},X_{2})  &  =(X_{1}+7X_{1}X_{2}+X_{2})^{2}\\
&  =X_{1}^{2}+2X_{1}X_{2}+14X_{1}^{2}X_{2}+X_{2}^{2}+14X_{1}X_{2}^{2}%
+49X_{1}^{2}X_{2}^{2}.
\end{align*}
\noindent The highest monomial is $49X_{1}^{2}X_{2}^{2}$, and so we subtract
$49p_{2}^{2}$, getting{}%
\[
P-49p_{2}^{2}=X_{1}^{2}+2X_{1}X_{2}+14X_{1}^{2}X_{2}+X_{2}^{2}+14X_{1}%
X_{2}^{2}.
\]
Continuing, we get%
\[
P-49p_{2}^{2}-14p_{1}p_{2}=X_{1}^{2}+2X_{1}X_{2}+X_{2}^{2}%
\]
and finally,%
\[
P-49p_{2}^{2}-14p_{1}p_{2}-p_{1}^{2}=0\text{.}%
\]


(b) The expression of $P$ as a polynomial in the $p_{i}$ in (\ref{ag25}) is
unique. Otherwise, by subtracting, we would get a nontrivial polynomial
$Q(p_{1},\ldots,p_{n})$ in the $p_{i}$ which is zero when expressed as a
polynomial in the $X_{i}$. But the highest monomials (\ref{e2}) in the
polynomials $p_{1}^{d_{1}}\cdots p_{n}^{d_{n}}$ are distinct (the map
$(d_{1},\ldots,d_{n})\mapsto(d_{1}+\cdots+d_{n},\ldots,d_{n})$ is injective),
and so they can't cancel.
\end{remark}

Let
\[
f(X)=X^{n}+a_{1}X^{n-1}+\cdots+a_{n}\in R[X],
\]
and suppose that $f$ splits over some ring $S$ containing $R$:
\[
f(X)=\tstyle\prod\nolimits_{i=1}^{n}(X-\alpha_{i}),\quad\alpha_{i}\in
S\text{.}%
\]
Then
\[
a_{1}=-p_{1}(\alpha_{1},\ldots,\alpha_{n}),\quad a_{2}=p_{2}(\alpha_{1}%
,\ldots,\alpha_{n}),\quad\ldots,\quad a_{n}=(-1)^{n}p_{n}(\alpha_{1}%
,\ldots,\alpha_{n}).
\]
Thus the \textit{elementary}\emph{\/} symmetric polynomials in the roots of
$f(X)$ lie in $R$, and so the theorem implies that \textit{every}\emph{\/}
symmetric polynomial in the roots of $f(X)$ lies in $R$. For example, the
discriminant
\[
D(f)=\prod_{i<j}(\alpha_{i}-\alpha_{j})^{2}%
\]
of $f$ lies in $R$.

\begin{theorem}
[Symmetric functions theorem]\label{ag26}Let $F$ be a field. When $S_{n}$ acts
on $F(X_{1},...,X_{n})$ by permuting the $X_{i}$, the field of invariants is
$F(p_{1},...,p_{n}).$
\end{theorem}

\begin{proof}
Let $f\in F(X_{1},\ldots,X_{n})$ be symmetric (i.e., fixed by $S_{n})$. Set
$f=g/h$, $g,h\in F[X_{1},\ldots,X_{n}]$. The polynomials $H=\prod_{\sigma\in
S_{n}}\sigma h$ and $Hf$ are symmetric, and therefore lie in $F[p_{1}%
,\ldots,p_{n}]$ by \ref{ag25}. Hence their quotient $f=Hf/H$ lies in
$F(p_{1},\ldots,p_{n})$.
\end{proof}

\begin{corollary}
\label{ag27}The field $F(X_{1},...,X_{n})$ is Galois over $F(p_{1},...,p_{n})$
with Galois group $S_{n}$ (acting by permuting the $X_{i}$).
\end{corollary}

\begin{proof}
We have shown that $F(p_{1},\ldots,p_{n})=F(X_{1},\ldots,X_{n})^{S_{n}}$, and
so this follows from (\ref{ft12}).
\end{proof}

The field $F(X_{1},\ldots,X_{n})$ is the splitting field over $F(p_{1}%
,\ldots,p_{n})$ of%
\[
g(T)=(T-X_{1})\cdots(T-X_{n})=X^{n}-p_{1}X^{n-1}+\cdots+(-1)^{n}p_{n}.
\]
Therefore, the Galois group of $g(T)\in F(p_{1},\ldots,p_{n})[T]$ is $S_{n}$.

\begin{aside}
\label{ag27b}Symmetric polynomials played an important role in the work of
Galois. In his \textit{M\'{e}moire sur les conditions de r\'{e}solubilit\'{e}
des \'{e}quations par radicaux}, he prove the following proposition:\bquote
Let $f$ be a polynomial with coefficients $\sigma_{1},\ldots,\sigma_{n}$. Let
$x_{1},\ldots,x_{n}$ be its roots, and let $U,V,\ldots$ be certain numbers
that are rational functions in the $x_{i}$. Then there exists a group $G$ of
permutations of the $x_{i}$ such that the rational functions in the $x_{i}$
that are fixed under all permutations in $G$ are exactly those that are
rationally expressible in terms of $\sigma_{1},\ldots,\sigma_{n}$ and
$U,V,\ldots$\equote When we take $U,V,\ldots$ to be the elements of a field
$E$ intermediate between the field of coefficients of $f$ and the splitting
field of $f$, this says that the exists a group $G$ of permutations of the
$x_{i}$ whose fixed field (when $G$ acts on the splitting field) is exactly
$E$.
\end{aside}

\section{The general polynomial of degree \texorpdfstring{$n$}{n}}

When we say that the roots of
\[
aX^{2}+bX+c
\]
are
\[
\frac{-b\pm\sqrt{b^{2}-4ac}}{2a}%
\]
we are thinking of $a,b,c$ as symbols: for any particular values of $a,b,c$,
the formula gives the roots of the particular equation. We'll prove in this
section that there is no similar formula for the roots of the
\textquotedblleft general polynomial\textquotedblright\ of degree $\geq5$.

We define the \emph{general polynomial of degree}%
\index{general polynomial}
$n$ to be
\[
f(X)=X^{n}-t_{1}X^{n-1}+\cdots+(-1)^{n}t_{n}\in F[t_{1},...,t_{n}][X]
\]
where the $t_{i}$ are symbols. We'll show that, when we regard $f$ as a
polynomial in $X$ with coefficients in the field $F(t_{1},\ldots,t_{n})$, its
Galois group is $S_{n}$. Then Theorem \ref{ag23} proves the above remark (at
least in characteristic zero).

\begin{theorem}
\label{ag28}The Galois group of the general polynomial of degree $n$ is
$S_{n}$.
\end{theorem}

\begin{proof}
Let $f(X)$ be the general polynomial of degree $n$,
\[
f(X)=X^{n}-t_{1}X^{n-1}+\cdots+(-1)^{n}t_{n}\in F[t_{1},...,t_{n}][X].
\]
If we can show that the homomorphism
\[
t_{i}\mapsto p_{i}\colon F[t_{1},\ldots,t_{n}]\rightarrow F[p_{1},\ldots
,p_{n}]
\]
is injective, then it will extend to an isomorphism%
\[
F(t_{1},\ldots,t_{n})\rightarrow F(p_{1},\ldots,p_{n})
\]
sending $f(X)$ to%
\[
g(X)=X^{n}-p_{1}X^{n-1}+\cdots+(-1)^{n}p_{n}\in F(p_{1},\ldots,p_{n})[X].
\]
Then the statement will follow from Corollary \ref{ag27}.

We now prove that the homomorphism is injective.\footnote{To say that the
homomorphism is injective means that the $p_{i}$ are algebraically independent
over $F$ (see p.~\pageref{ai}). This can be proved by noting that, because
$F(X_{1},\ldots,X_{n})$ is algebraic over $F(p_{1},\ldots,p_{n})$, the latter
must have transcendence degree $n$ (see \S 8).} Suppose on the contrary that
there exists a $P(t_{1},\ldots,t_{n})$ such that $P(p_{1},\ldots,p_{n})=0$.
Equation (\ref{e2}), p.~\pageref{e2}, shows that if $m_{1}(t_{1},\ldots
,t_{n})$ and $m_{2}(t_{1},\ldots,t_{n})$ are distinct monomials, then
$m_{1}(p_{1},\ldots,p_{n})$ and $m_{2}(p_{1},\ldots,p_{n})$ have distinct
highest monomials. Therefore, cancellation can't occur, and so $P(t_{1}%
,\ldots,t_{n})$ must be the zero polynomial.
\end{proof}

\begin{remark}
\label{ag30}Since $S_{n}$ occurs as a Galois group over $\mathbb{Q}$, and
every finite group occurs as a subgroup of some $S_{n}$, it follows that every
finite group occurs as a Galois group over some finite extension of
$\mathbb{Q}$, but does every finite Galois group occur as a Galois group over
$\mathbb{Q}$ itself? This is known as the inverse Galois problem.

The Hilbert-Noether program for proving this was the following. Hilbert proved
that if $G$ occurs as the Galois group of an extension $E\supset
\mathbb{Q}(t_{1},...,t_{n})$ (the $t_{i}$ are symbols), then it occurs
infinitely often as a Galois group over $\mathbb{Q}$. For the proof, realize
$E$ as the splitting field of a polynomial $f(X)\in k[t_{1},\ldots,t_{n}][X]$
and prove that for infinitely many values of the $t_{i}$, the polynomial you
obtain in $\mathbb{Q}[X]$ has Galois group $G$. (This is quite a difficult
theorem --- see Serre, J.-P., \textit{Lectures on the Mordell-Weil
Theorem,}\emph{\/} 1989, Chapter 9.) Emmy Noether conjectured the following:
Let $G\subset S_{n}$ act on $F(X_{1},...,X_{n})$ by permuting the $X_{i}$;
then $F(X_{1},\ldots,X_{n})^{G}\approx F(t_{1},...,t_{n})$ (for symbols
$t_{i}$). However, Swan proved in 1969 that the conjecture is false for $G$
the cyclic group of order $47$. Hence this approach can not lead to a proof
that all finite groups occur as Galois groups over $\mathbb{Q}$, but it
doesn't exclude other approaches. For more information on the problem, see
Serre, ibid., Chapter 10; Serre, J.-P., \textit{Topics in Galois Theory},
1992; and the Wikipedia: inverse Galois problem.
\end{remark}

\begin{remark}
\label{ag31}Take $F=\mathbb{C}$, and consider the subset of $\mathbb{C}^{n+1}
$ defined by the equation%
\[
X^{n}-T_{1}X^{n-1}+\cdots+(-1)^{n}T_{n}=0.
\]
It is a beautiful complex manifold $S$ of dimension $n$. Consider the
projection
\[
\pi\colon S\rightarrow\mathbb{C}^{n},\quad(x,t_{1},\ldots,t_{n})\mapsto
(t_{1},\ldots,t_{n}).
\]
Its fibre over a point $(a_{1},\ldots,a_{n})$ is the set of roots of the
polynomial
\[
X^{n}-a_{1}X^{n-1}+\cdots+(-1)^{n}a_{n}.
\]
The discriminant $D(f)$ of $f(X)=X^{n}-T_{1}X^{n-1}+\cdots+(-1)^{n}T_{n}$ is a
polynomial in $\mathbb{C}[T_{1},\ldots,T_{n}]$. Let $\Delta$ be the zero set
of $D(f)$ in $\mathbb{C}^{n}$. Then over each point of $\mathbb{C}%
^{n}\smallsetminus\Delta$, there are exactly $n$ points of $S$, and
$S\smallsetminus\pi^{-1}(\Delta)$ is a covering space over $\mathbb{C}%
^{n}\smallsetminus\Delta$.
\end{remark}

\subsection{A brief history}

As far back as 1500 BC, the Babylonians (at least) knew a general formula for
the roots of a quadratic polynomial. Cardan (about 1515 AD) found a general
formula for the roots of a cubic polynomial. Ferrari (about 1545 AD) found a
general formula for the roots of a quartic polynomial (he introduced the
resolvent cubic, and used Cardan's result). Over the next 275 years there were
many fruitless attempts to obtain similar formulas for higher degree
polynomials, until, in about 1820, Ruffini and Abel proved that there are none.

\section{Norms and traces}

Recall that, for an $n\times n$ matrix $A=(a_{ij})$
\[
\renewcommand{\arraystretch}{1.3}
\begin{array}
[c]{rcll}%
\Tr(A) & = & \sum\nolimits_{i}a_{ii}\quad\quad & \text{(trace of }A\text{)}\\
\det(A) & = & \sum\nolimits_{\sigma\in S_{n}}\text{sign}(\sigma)a_{1\sigma
(1)}\cdots a_{n\sigma(n)},\quad\quad & \text{(determinant of }A\text{)}\\
c_{A}(X) & = & \det(XI_{n}-A)\quad\quad & \text{(characteristic polynomial of
}A\text{).}%
\end{array}
\]
Moreover,%
\[
c_{A}(X)=X^{n}-\Tr(A)X^{n-1}+\cdots+(-1)^{n}\det(A)\text{.}%
\]
None of these is changed when $A$ is replaced by its conjugate $UAU^{-1}$ by
an invertible matrix $U$. Therefore, for any endomorphism $\alpha$ of a
finite-dimensional vector space $V$, we can define\footnote{The coefficients
of the characteristic polynomial, $c_{\alpha}(X)=X^{n}+c_{1}X^{n-1}%
+\cdots+c_{n}$, of $\alpha$ have the following description: $c_{i}%
=(-1)^{i}\Tr(\alpha|\bigwedge\nolimits^{i}V)$ --- see Bourbaki, N., Algebra,
Chapter 3, 8.11.}%
\[
\Tr(\alpha)=\Tr(A),\quad\det(\alpha)=\det(A),\quad c_{\alpha}(X)=c_{A}(X)
\]
where $A$ is the matrix of $\alpha$ with respect to a basis of $V$. If $\beta$
is a second endomorphism of $V$,
\begin{align*}
\Tr(\alpha+\beta)  &  =\Tr(\alpha)+\Tr(\beta);\\
\quad\det(\alpha\beta)  &  =\det(\alpha)\det(\beta).
\end{align*}


Now let $E$ be a finite field extension of $F$ of degree $n.$ An element
$\alpha$ of $E$ defines an $F$-linear map
\[
\alpha_{L}\colon E\rightarrow E,\quad x\mapsto\alpha x,
\]
and we define%
\index{trace}%
\index{norm}%
\[
\renewcommand{\arraystretch}{1.3}
\begin{array}
[c]{llll}%
\Tr_{E/F}(\alpha) & = & \Tr(\alpha_{L}) & \text{(trace of }\alpha\text{)}\\
\Nm_{E/F}(\alpha) & = & \det(\alpha_{L})\quad\quad & \text{(norm of }%
\alpha\text{)}\\
c_{\alpha,E/F}(X) & = & c_{\alpha_{L}}(X) & \text{(characteristic polynomial
of }\alpha\text{)}.
\end{array}
\]
Thus, $\Tr_{E/F}$\ is a homomorphism $(E,+)\rightarrow(F,+)$, and $\Nm_{E/F}$
is a homomorphism $(E^{\times},\cdot)\rightarrow(F^{\times},\cdot)$.

\begin{example}
\label{ag32}(a) Consider the field extension $\mathbb{C}\supset\mathbb{R}$.
For $\alpha=a+bi$, the matrix of $\alpha_{L}$ with respect to the basis
$\left\{  1,i\right\}  $ is $\left(
\begin{smallmatrix}
a & -b\\
b & a
\end{smallmatrix}
\right)  $, and so
\[
\Tr_{\mathbb{C}/\mathbb{R}}(\alpha)=2\Re(\alpha)\text{, }\Nm_{\mathbb{C}%
/\mathbb{R}}(\alpha)=|\alpha|^{2}.
\]


(b) For $a\in F$, $a_{L}$ is multiplication by the scalar $a$. Therefore%
\[
\Tr_{E/F}(a)=na\quad\Nm_{E/F}(a)=a^{n}\quad c_{a,E/F}(X)=(X-a)^{n}%
\]
where $n=[E\colon F].$
\end{example}

Let $E=\mathbb{Q}[\alpha,i]$ be the splitting field of $X^{8}-2$ (see Exercise
\ref{x16}). Then $E$ has degree $16$ over $\mathbb{Q}{}$, and so to compute
the trace and norm an element of $E$, the definition requires us to compute
the trace and norm of a $16\times16$ matrix. The next proposition gives us a
quicker method.

\begin{proposition}
\label{ag33}Let $E/F$ be a finite extension of fields, and let $f(X)$ be the
minimal polynomial of $\alpha\in E$. Then
\[
c_{\alpha,E/F}(X)=f(X)^{[E\colon F[\alpha]]}.
\]

\end{proposition}

\begin{proof}
Suppose first that $E=F[\alpha]$. In this case, we have to show that
$c_{\alpha}(X)=f(X)$. Note that $\alpha\mapsto\alpha_{L}$ is an
\emph{injective\/} homomorphism from $E$ into the ring of endomorphisms of $E$
as a vector space over $F$. The Cayley-Hamilton theorem shows that $c_{\alpha
}(\alpha_{L})=0$, and therefore $c_{\alpha}(\alpha)=0$. Hence $f|c_{\alpha}$,
but they are monic of the same degree, and so they are equal.

For the general case, let $\beta_{1},...,\beta_{n}$ be a basis for $F[\alpha]$
over $F$, and let $\gamma_{1},...,\gamma_{m}$ be a basis for $E$ over
$F[\alpha]$. As we saw in the proof of (\ref{ef10}), $\{\beta_{i}\gamma_{k}\}$
is a basis for $E$ over $F$. Write $\alpha\beta_{i}=\sum a_{ji}\beta_{j}$.
Then, according to the first case proved, $A\overset{\df%
}{=}(a_{ij})$ has characteristic polynomial $f(X)$. But $\alpha\beta_{i}%
\gamma_{k}=\sum a_{ji}\beta_{j}\gamma_{k}$, and so the matrix of $\alpha_{L}$
with respect to $\{\beta_{i}\gamma_{k}\}$ breaks up into $n\times n$ blocks
with $A$'s down the diagonal and zero matrices elsewhere, from which it
follows that $c_{\alpha_{L}}(X)=c_{A}(X)^{m}=$ $f(X)^{m}.$
\end{proof}

\begin{corollary}
\label{ag34}Suppose that the roots of the minimal polynomial of $\alpha$ are
$\alpha_{1},\ldots,\alpha_{n}$ (in some splitting field containing $E$), and
that $[E\colon F[\alpha]]=m$. Then
\[
\Tr(\alpha)=m%
%TCIMACRO{\tsum \nolimits_{i=1}^{n}}%
%BeginExpansion
{\textstyle\sum\nolimits_{i=1}^{n}}
%EndExpansion
\alpha_{i},\qquad\Nm_{E/F}\alpha=\left(
%TCIMACRO{\tprod \nolimits_{i=1}^{n}}%
%BeginExpansion
{\textstyle\prod\nolimits_{i=1}^{n}}
%EndExpansion
\alpha_{i}\right)  ^{m}.
\]

\end{corollary}

\begin{proof}
Write the minimal polynomial of $\alpha$ as
\[
f(X)=X^{n}+a_{1}X^{n-1}+\cdots+a_{n}=%
%TCIMACRO{\tprod }%
%BeginExpansion
{\textstyle\prod}
%EndExpansion
(X-\alpha_{i}),
\]
so that%
\begin{align*}
a_{1}  &  =-%
%TCIMACRO{\tsum }%
%BeginExpansion
{\textstyle\sum}
%EndExpansion
\alpha_{i}\text{, and}\\
a_{n}  &  =(-1)^{n}%
%TCIMACRO{\tprod }%
%BeginExpansion
{\textstyle\prod}
%EndExpansion
\alpha_{i}\text{.}%
\end{align*}
Then
\[
c_{\alpha}(X)=(f(X))^{m}=X^{mn}+ma_{1}X^{mn-1}+\cdots+a_{n}^{m},
\]
so that
\begin{align*}
\Tr_{E/F}(\alpha)  &  =-ma_{1}=m%
%TCIMACRO{\tsum }%
%BeginExpansion
{\textstyle\sum}
%EndExpansion
\alpha_{i}\text{, and }\\
\Nm_{E/F}(\alpha)  &  =(-1)^{mn}a_{n}^{m}=(%
%TCIMACRO{\tprod }%
%BeginExpansion
{\textstyle\prod}
%EndExpansion
\alpha_{i})^{m}\text{.}%
\end{align*}

\end{proof}

\begin{example}
\label{ag35}(a) Consider the extension $\mathbb{C}\supset\mathbb{R}$. If
$\alpha\in\mathbb{C}\smallsetminus\mathbb{R}$, then
\[
c_{\alpha}(X)=f(X)=X^{2}-2\Re(\alpha)X+|\alpha|^{2}.
\]
If $\alpha\in\mathbb{R}$, then $c_{\alpha}(X)=(X-a)^{2}$.

(b) Let $E$ be the splitting field of $X^{8}-2$. Then $E$ has degree $16$ over
$\mathbb{Q}{}$ and is generated by $\alpha=\sqrt[8]{2}$ and $i=\sqrt{-1}$ (see
Exercise \ref{x16}). The minimal polynomial of $\alpha$ is $X^{8}-2$, and so
\[
\renewcommand{\arraystretch}{1.3}
\begin{array}
[c]{rcllrcl}%
c_{\alpha,\mathbb{Q}{}[\alpha]/\mathbb{Q}{}}(X) & = & X^{8}-2, & \quad &
c_{\alpha,E/\mathbb{Q}{}}(X) & = & (X^{8}-2)^{2}\\
\Tr_{\mathbb{Q}[\alpha]/\mathbb{Q}}\alpha & = & 0,\quad &  & \Tr_{E/\mathbb{Q}%
}\alpha & = & 0\\
\Nm_{\mathbb{Q}[\alpha]/\mathbb{Q}}\alpha & = & -2,\quad &  &
\Nm_{E/\mathbb{Q}}\alpha & = & 4
\end{array}
\]

\end{example}

\begin{remark}
\label{ag36}Let $E$ be a separable extension of $F$, and let $\varSigma$ be
the set of $F$-homomorphisms of $E$ into an algebraic closure $\Omega$ of $F$.
Then
\begin{align*}
\Tr_{E/F}\alpha &  =%
%TCIMACRO{\tsum \nolimits_{\sigma\in\varSigma}}%
%BeginExpansion
{\textstyle\sum\nolimits_{\sigma\in\varSigma}}
%EndExpansion
\sigma\alpha\\
\Nm_{E/F}\alpha &  =%
%TCIMACRO{\tprod \nolimits_{\sigma\in\varSigma}}%
%BeginExpansion
{\textstyle\prod\nolimits_{\sigma\in\varSigma}}
%EndExpansion
\sigma\alpha.
\end{align*}
When $E=F[\alpha]$, this follows from \ref{ag34} and the observation
(cf.\ \ref{sf1}b) that the $\sigma\alpha$ are the roots of the minimal
polynomial $f(X)$ of $\alpha$ over $F$. In the general case, the $\sigma
\alpha$ are still roots of $f(X)$ in $\Omega$, but now each root of $f(X)$
occurs $[E\colon F[\alpha]]$ times (because each $F$-homomorphism
$F[\alpha]\rightarrow\Omega$ has $[E\colon F[\alpha]]$ extensions to $E$). For
example, if $E$ is Galois over $F$ with Galois group $G$, then
\begin{align*}
\Tr_{E/F}\alpha &  =%
%TCIMACRO{\tsum \nolimits_{\sigma\in G}}%
%BeginExpansion
{\textstyle\sum\nolimits_{\sigma\in G}}
%EndExpansion
\sigma\alpha\\
\Nm_{E/F}\alpha &  =%
%TCIMACRO{\tprod \nolimits_{\sigma\in G}}%
%BeginExpansion
{\textstyle\prod\nolimits_{\sigma\in G}}
%EndExpansion
\sigma\alpha.
\end{align*}

\end{remark}

\begin{proposition}
\label{ag37}For finite extensions $E\supset M\supset F$, we have
\begin{align*}
\Tr_{M/F}\circ\Tr_{E/M}  &  =\Tr_{E/F},\\
\Nm_{M/F}\circ\Nm_{E/M}  &  =\Nm_{E/F}.
\end{align*}

\end{proposition}

\begin{proof}
If $E$ is separable over $F$, then this can be proved fairly easily using the
descriptions in the above remark. We omit the proof in the general case.
\end{proof}

\begin{proposition}
\label{ag38}Let $f(X)$ be a monic irreducible polynomial with coefficients in
$F$, and let $\alpha$ be a root of $f$ in some splitting field of $f$. Then
\[
\disc f(X)=(-1)^{m(m-1)/2}\Nm_{F[\alpha]/F}f^{\prime}(\alpha)
\]
where $f^{\prime}$ is the formal derivative $\frac{df}{dX}$ of $f$.
\end{proposition}

\begin{proof}
Let $f(X)=\prod_{i=1}^{m}(X-\alpha_{i})$ be the factorization of $f$ in the
given splitting field, and number the roots so that $\alpha=\alpha_{1}$.
Compute that
\begin{align*}
\disc f(X)  &  \overset{\df}{=}\prod_{i<j}(\alpha_{i}%
-\alpha_{j})^{2}\\
&  =(-1)^{m(m-1)/2}\cdot\prod_{i}(\prod_{j\neq i}(\alpha_{i}-\alpha_{j}))\\
&  =(-1)^{m(m-1)/2}\cdot\prod_{i}f^{\prime}(\alpha_{i})\\
&  =(-1)^{m(m-1)/2}\Nm_{F[\alpha]/F}(f^{\prime}(\alpha))\qquad\text{(by
\ref{ag36})}.
\end{align*}

\end{proof}

\begin{example}
\label{ag39}We compute the discriminant of
\[
f(X)=X^{n}+aX+b,\quad a,b\in F,
\]
assumed to be irreducible and separable, by computing the norm of
\[
\gamma\overset{\df}{=}f^{\prime}(\alpha)=n\alpha
^{n-1}+a,\quad f(\alpha)=0\text{.}%
\]
On multiplying the equation
\[
\alpha^{n}+a\alpha+b=0
\]
by $n\alpha^{-1}$ and rearranging, we obtain the equation
\[
n\alpha^{n-1}=-na-nb\alpha^{-1}.
\]
Hence
\[
\gamma=n\alpha^{n-1}+a=-(n-1)a-nb\alpha^{-1}.
\]
Solving for $\alpha$ gives
\[
\alpha=\frac{-nb}{\gamma+(n-1)a}.
\]
From the last two equations, it is clear that $F[\alpha]=F[\gamma]$, and so
the minimal polynomial of $\gamma$ over $F$ has degree $n$ also. If we write
\begin{align*}
f\left(  \frac{-nb}{X+(n-1)a}\right)   &  =\frac{P(X)}{Q(X)}\\
P(X)  &  =(X+(n-1)a)^{n}-na(X+(n-1)a)^{n-1}+(-1)^{n}n^{n}b^{n-1}\\
Q(X)  &  =(X+(n-1)a)^{n}/b,
\end{align*}
then
\[
P(\gamma)=f(\alpha)\cdot Q(\gamma)=0.
\]
As%
\[
Q(\gamma)=\frac{(\gamma+(n-1)a)^{n}}{b}=\frac{(-nb)^{n}}{\alpha^{n}b}\neq0
\]
and $P(X)$ is monic of degree $n$, it must be the minimal polynomial of
$\gamma$. Therefore $\Nm\gamma$ is $(-1)^{n}$ times the constant term of
$P(X)$, namely,
\[
\Nm\gamma=n^{n}b^{n-1}+(-1)^{n-1}(n-1)^{n-1}a^{n}.
\]
Therefore,
\[
\disc(X^{n}+aX+b)=(-1)^{n(n-1)/2}(n^{n}b^{n-1}+(-1)^{n-1}(n-1)^{n-1}a^{n}),
\]
which is something PARI%
\index{PARI}
doesn't know (because it doesn't understand symbols as exponents). For
example,
\[
\disc(X^{5}+aX+b)=5^{5}b^{4}+4^{4}a^{5}.
\]

\end{example}

\section{Exercises}

\begin{exercise}
\label{x21} For $a\in{\mathbb{Q}}$, let $G_{a}$ be the Galois group of
$X^{4}+X^{3}+X^{2}+X+a$. Find integers $a_{1},a_{2},a_{3},a_{4}$ such that
$i\neq j\implies G_{a_{i}}$ is not isomorphic to $G_{a_{j}}$.
\end{exercise}

\begin{exercise}
\label{x22} Prove that the rational solutions $a,b\in{\mathbb{Q}}$ of
Pythagoras's equation $a^{2}+b^{2}=1$ are of the form
\[
a=\frac{s^{2}-t^{2}}{s^{2}+t^{2}},\quad b=\frac{2st}{s^{2}+t^{2}},\qquad
s,t\in{\mathbb{Q}},
\]
and deduce that every right triangle with integer sides has sides of length
\[
d(m^{2}-n^{2},2mn,m^{2}+n^{2})
\]
for some integers $d$, $m$, and $n$ (Hint: Apply Hilbert's Theorem 90 to the
extension ${\mathbb{Q}}[i]/{\mathbb{Q}}$.)
\end{exercise}

\begin{exercise}
\label{x23} Prove that a finite extension of ${\mathbb{Q}}$ can contain only
finitely many roots of $1$.
\end{exercise}

\clearpage


\chapter{Algebraic Closures}

In this chapter, we use Zorn's lemma to show that every field $F$ has an
algebraic closure $\Omega$. Recall that if $F$ is a subfield $\mathbb{C}{}$,
then the algebraic closure of $F$ in $\mathbb{C}{}$ is an algebraic closure of
$F$ (\ref{ac3}). If $F$ is countable, then the existence of $\Omega$ can be
proved as in the finite field case (\ref{cg18n}), namely, the set of monic
irreducible polynomials in $F[X]$ is countable, and so we can list them
$f_{1},f_{2},\ldots$; define $E_{i}$ inductively by, $E_{0}=F$, $E_{i}=$ a
splitting field of $f_{i}$ over $E_{i-1}$; then $\Omega=\bigcup E_{i}$ is an
algebraic closure of $F$.

The difficulty in showing the existence of an algebraic closure of an
arbitrary field $F$ is in the set theory. Roughly speaking, we would like to
take a union of a family of splitting fields indexed by the monic irreducible
polynomials in $F[X]$, but we need to find a way of doing this that is allowed
by the axioms of set theory. After reviewing the statement of Zorn's lemma, we
sketch three solutions\footnote{There do exist naturally occurring uncountable
fields not contained in $\mathbb{C}$. For example, the field of formal Laurent
series $F((T))$ over a field $F$ is uncountable even when $F$ is finite.} to
the problem.

\section{Zorn's lemma}

\begin{definition}
(a) A relation $\leq$ on a set $S$ is a \emph{partial ordering}%
\index{ordering!partial}%
\emph{ }if it reflexive, transitive, and anti-symmetric ($a\leq b$ and $b\leq
a\implies a=b$).

(b) A partial ordering is a \emph{total ordering}%
\index{ordering!total}%
\emph{ }if, for all $s,t\in T$, either $s\leq t$ or $t\leq s$.

(c) An \emph{upper bound}%
\index{bound!upper}
for a subset $T$ of a partially ordered set $(S,\leq)$ is an element $s\in S$
such that $t\leq s$ for all $t\in T$.

(d) A \emph{maximal element}%
\index{element!maximal}%
\emph{ }of a partially ordered set $S$ is an element $s$ such that $s\leq
s^{\prime}\implies s=s^{\prime}$.
\end{definition}

A partially ordered set need not have any maximal elements, for example, the
set of finite subsets of an infinite set is partially ordered by inclusion,
but it has no maximal elements.

\begin{lemma}
[Zorn]\label{sf14}Let $(S,\leq)$ be a nonempty partially ordered set for which
every totally ordered subset has an upper bound in $S$. Then $S$ has a maximal element.
\end{lemma}

Zorn's lemma\footnote{The following is quoted from A.J. Berrick and M.E.
Keating, An Introduction to Rings and Modules, 2000: The name of the
statement, although widely used (allegedly first by Lefschetz), has attracted
the attention of historians (Campbell 1978). As a `maximum principle', it was
first brought to prominence, and used for algebraic purposes in Zorn 1935,
apparently in ignorance of its previous usage in topology, most notably in
Kuratowski 1922. Zorn attributed to Artin the realization that the `lemma' is
in fact equivalent to the Axiom of Choice (see Jech 1973). Zorn's contribution
was to observe that it is more suited to algebraic applications like ours.} is
equivalent to the Axiom of Choice, and hence independent of the axioms of set theory.

\begin{remark}
\label{sf14m}The set $S$ of finite subsets of an infinite set doesn't
contradict Zorn's lemma, because it contains totally ordered subsets with no
upper bound in $S$.
\end{remark}

The following proposition is a typical application of Zorn's lemma --- we
shall use a * to signal results that depend on Zorn's lemma (equivalently, the
Axiom of Choice).

\begin{proposition}
[*]\label{sf15}Every nonzero commutative ring $A$ has a maximal ideal
(meaning, maximal among \textnf{proper} ideals).
\end{proposition}

\begin{proof}
Let $S$ be the set of all proper ideals in $A$, partially ordered by
inclusion. If $T$ is a totally ordered set of ideals, then $J=\bigcup_{I\in
T}I$ is again an ideal, and it is proper because if $1\in J$ then $1\in I$ for
some $I$ in $T$, and $I$ would not be proper. Thus $J$ is an upper bound for
$T$. Now Zorn's lemma implies that $S$ has a maximal element, which is a
maximal ideal in $A$.
\end{proof}

\section{First proof of the existence of algebraic closures}

(Bourbaki, Alg\`{e}bre, Chap. V, \S 4.) Recall that an $F$-algebra is a ring
containing $F$ as a subring. Let $(A_{i})_{i\in I}$ be a family of commutative
$F$-algebras, and define $\bigotimes_{F}A_{i}$ to be the quotient of the
$F$-vector space with basis $\prod\nolimits_{i\in I}A_{i}$ by the subspace
generated by elements of the form:

$(x_{i})+(y_{i})-(z_{i})$ with $x_{j}+y_{j}=z_{j}$ for one $j\in I$ and
$x_{i}=y_{i}=z_{i}$ for all $i\neq j$;

$(x_{i})-a(y_{i})$ with $x_{j}=ay_{j}$ for one $j\in I$ and $x_{i}=y_{i}$ for
all $i\neq j$,

\noindent(ibid., Chap. II, 3.9). It can be made into a commutative $F$-algebra
in an obvious fashion, and there are canonical homomorphisms $A_{i}%
\rightarrow\bigotimes\nolimits_{F}A_{i}$ of $F$-algebras.

For each polynomial $f\in F[X]$, choose a splitting field $E_{f}$, and let
$\Omega=(\bigotimes\nolimits_{F}E_{f})/M$ where $M$ is a maximal ideal in
$\bigotimes_{F}E_{f}$ (whose existence is ensured by Zorn's lemma). Note that
$F\subset\bigotimes\nolimits_{F}E_{f}$ and $M\cap F=0$. As $\Omega$ has no
ideals other than $(0)$ and $\Omega$, it is a field (see \ref{ef1}). The
composite of the $F$-homomorphisms $E_{f}\rightarrow\bigotimes\nolimits_{F}%
E_{f}\rightarrow\Omega$, being a homomorphism of fields, is injective. Since
$f$ splits in $E_{f}$, it must also split in the larger field $\Omega$. The
algebraic closure of $F$ in $\Omega$ is therefore an algebraic closure of $F$
(by \ref{sf10}).

\begin{aside}
In fact, it suffices to take $\Omega=(\bigotimes\nolimits_{F}E_{f})/M$ where
$f$ runs over the monic irreducible polynomials in $F[X]$ and $E_{f}$ is the
stem field $F[X]/(f)$ of $f$ (apply the statement in \ref{sf15a} below).
\end{aside}

\section{Second proof of the existence of algebraic closures}

(Jacobson 1964, p.\,144.) After \ref{cg18n} we may assume $F$ to be infinite.
This implies that the cardinality of every field algebraic over $F$ is the
same as that of $F$ (cf.\ the proof of \ref{ef22}). Choose an uncountable set
$\Xi$ of cardinality greater than that of $F$, and identify $F$ with a subset
of $\Xi$. Let $S$ be the set of triples $(E,+,\cdot)$ with $E\subset\Xi$ and
$(+,\cdot)$ a field structure on $E$ such that $(E,+,\cdot)$ contains $F$ as a
subfield and is algebraic over it. Write $(E,+,\cdot)\leq(E^{\prime}%
,+^{\prime},\cdot^{\prime})$ if the first is a subfield of the second. Apply
Zorn's lemma to show that $S$ has maximal elements, and then show that a
maximal element is algebraically closed.

\section{Third proof of the existence of algebraic closures}

(Emil Artin.) Consider the polynomial ring $F[\ldots,x_{f},\ldots]$ in a
family of symbols $x_{f}$ indexed by the nonconstant monic polynomials $f\in
F[X]$. If $1$ lies in the ideal $I$ of $F[\ldots,x_{f},\ldots]$ generated by
the polynomials $f(x_{f})$, then%
\[
g_{1}f_{1}(x_{f_{1}})+\cdots+g_{n}f_{n}(x_{f_{n}})=1\qquad(\text{in }%
F[\ldots,x_{f},\ldots])
\]
for some $g_{i}\in F[\ldots,x_{f},\ldots]$ and some nonconstant monic
$f_{i}\in F[X]$. Let $E$ be an extension of $F$ such that each $f_{i}$,
$i=1,\ldots,n$, has a root $\alpha_{i}$ in $E$. Under the $F$-homomorphism
$F[\ldots,x_{f},\ldots]\rightarrow E$ sending
\[
\left\{
\begin{array}
[c]{l}%
x_{f_{i}}\mapsto\alpha_{i}\\
x_{f}\mapsto0,\quad f\notin\{f_{1},\ldots,f_{n}\}
\end{array}
\right.
\]
the above relation becomes $0=1$. From this contradiction, we deduce that $1$
does not lie in $I$, and so Proposition \ref{sf15} applied to $F[\ldots
,x_{f},\ldots]/I$ shows that $I$ is contained in a maximal ideal $M$ of
$F[\ldots,x_{f},\ldots]$. Let $\Omega=F[\ldots,x_{f},\ldots]/M$. Then $\Omega$
is a field containing (a copy of) $F$ in which every nonconstant polynomial in
$F[X]$ has at least one root. Repeat the process starting with $E_{1}$ instead
of $F$ to obtain a field $E_{2}$. Continue in this fashion to obtain a
sequence of fields%
\[
F=E_{0}\subset E_{1}\subset E_{2}\subset\cdots,
\]
and let $E=\bigcup\nolimits_{i}E_{i}$. Then $E$ is algebraically closed
because the coefficients of any nonconstant polynomial $g$ in $E[X]$ lie in
$E_{i}$ for some $i$, and so $g$ has a root in $E_{i+1}$. Therefore, the
algebraic closure of $F$ in $E$ is an algebraic closure of $F$ (\ref{ac3}).

\begin{aside}
\label{sf15b}In fact, $E$ is algebraic over $F$. To see this, note that
$E_{1}$ is generated by algebraic elements over $F$, and so is algebraic over
$F$ (apply \ref{sf11}). Similarly, $E_{2}$ is algebraic over $E_{1}$, and
hence over $F$ (apply \ref{ef20}b). Continuing in this fashion, we find that
every element of every $E_{i}$ is algebraic over $F$.
\end{aside}

\begin{aside}
\label{sf15a}In fact, $E_{1}$ is already algebraically closed (hence the
algebraic closure of $F$). This follows from the statement:\bquote Let
$\Omega$ be a field. If $\Omega$ is algebraic over a subfield $F$ and every
nonconstant polynomial in $F[X]$ has a root in $\Omega$, then $\Omega$ is
algebraically closed.\equote In order to prove this, it suffices to show that
every irreducible polynomial $f$ in $F[X]$ splits in $\Omega\lbrack X]$ (see
\ref{sf10}). Suppose first that $f$ is separable, and let $E$ be a splitting
field for $f$. According to Theorem \ref{ag1}, $E=F[\gamma]$ for some
$\gamma\in E$. Let $g(X)$ be the minimal polynomial of $\gamma$ over $F$. Then
$g(X)$ has coefficients in $F$, and so it has a root $\beta$ in $\Omega$. Both
of $F[\gamma]$ and $F[\beta]$ are stem fields for $g$, and so there is an
$F$-isomorphism $F[\gamma]\rightarrow F[\beta]\subset\Omega$. As $f$ splits
over $F[\gamma]$, it must split over $\Omega$.

This completes the proof when $F$ is perfect. Otherwise, $F$ has
characteristic $p\neq0$, and we let $F^{\prime}$ be the set of elements $x$ of
$\Omega$ such that $x^{p^{m}}\in F$ for some $m$. It is easy to see that
$F^{\prime}$ is a field, and we'll complete the proof of the lemma by showing
that (a) $F^{\prime}$ is perfect, and (b) every polynomial in $F^{\prime}[X]$
has a root in $\Omega$.

\textsc{Proof of \textup{(a)}.} Let $a\in F^{\prime}$, so that
$b\overset{\df}{=}a^{p^{m}}\in F$ for some $m$. The
polynomial $X^{p^{m+1}}-b$ has coefficients in $F$, and so it has a root
$\alpha\in\Omega$, which automatically lies in $F^{\prime}$. Now
$\alpha^{p^{m+1}}=a^{p^{m}}$, which implies that $\alpha^{p}=a$, because the
$p$th power map is injective on fields of characteristic $p$.

Before continuing, we note that, because $\Omega$ is algebraic over a perfect
field $F^{\prime}$, it is itself perfect: let $a\in\Omega$, and let $g$ be the
minimal polynomial of $a$ over $F^{\prime}$; if $X^{p}-a$ is irreducible in
$\Omega\lbrack X]$, then $g(X^{p})$ is irreducible in $F^{\prime}[X]$, but it
is not separable, which is a contradiction.

\textsc{Proof of \textup{(b)}.} Let $f(X)\in F^{\prime}[X]$, say,
$f(X)=\sum_{i}a_{i}X^{i}$, $a_{i}\in F^{\prime}$. For some $m$, the polynomial
$\sum_{i}a_{i}^{p^{m}}X^{i}$ has coefficients in $F$, and therefore has a root
$\alpha\in\Omega$. As $\Omega$ is perfect, we can write $\alpha=\beta^{p^{m}}$
with $\beta\in\Omega$. Now%
\[
\left(  f(\beta)\right)  ^{p^{m}}=\left(  \sum\nolimits_{i}a_{i}\beta
^{i}\right)  ^{p^{m}}=\sum\nolimits_{i}a_{i}^{p^{m}}\alpha^{i}=0,
\]
and so $\beta$ is a root of $f$.
\end{aside}

\section{(Non)uniqueness of algebraic closures}

\begin{theorem}
[*]\label{sf16}Let $\Omega$ be an algebraic closure of $F$, and let $E$ be an
algebraic extension of $F$. There exists an $F$-homomorphism $E\rightarrow
\Omega$, and, if $E$ is also an algebraic closure of $F$, then every such
homomorphism is an isomorphism.
\end{theorem}

\begin{proof}
Suppose first that $E$ is countably generated over $F$, i.e., $E=F[\alpha
_{1},...,\alpha_{n},\ldots]$. Then we can extend the inclusion map
$F\rightarrow\Omega$ to $F[\alpha_{1}]$ (map $\alpha_{1}$ to any root of its
minimal polynomial in $\Omega)$, then to $F[\alpha_{1},\alpha_{2}],$ and so on
(see \ref{sf2}).

In the uncountable case, we use Zorn's lemma. Let $S$ be the set of pairs
$(M,\varphi_{M})$ with $M$ a field $F\subset M\subset E$ and $\varphi_{M}$ an
$F$-homomorphism $M\rightarrow\Omega$. Write $(M,\varphi_{M})\leq
(N,\varphi_{N})$ if $M\subset N$ and $\varphi_{N}|M=\varphi_{M}$. This makes
$S$ into a partially ordered set. Let $T$ be a totally ordered subset of $S$.
Then $M^{\prime}=\bigcup_{M\in T}M$ is a subfield of $E$, and we can define a
homomorphism $\varphi^{\prime}\colon M^{\prime}\rightarrow\Omega$ by requiring
that $\varphi^{\prime}(x)=\varphi_{M}(x)$ if $x\in M$. The pair $(M^{\prime
},\varphi^{\prime})$ is an upper bound for $T$ in $S$. Hence Zorn's lemma
gives us a maximal element $(M,\varphi)$ in $S$. Suppose that $M\neq E$. Then
there exists an element $\alpha\in E$, $\alpha\notin M$. Since $\alpha$ is
algebraic over $M$, we can apply (\ref{sf2}) to extend $\varphi$ to
$M[\alpha]$, contradicting the maximality of $M$. Hence $M=E$, and the proof
of the first statement is complete.

If $E$ is algebraically closed, then every polynomial $f\in F[X]$ splits in
$E[X]$ and hence in $\varphi(E)[X]$. Let $\alpha\in\Omega$, and let $f(X)$ be
the minimal polynomial of $\alpha$. Then $X-\alpha$ is a factor of $f(X) $ in
$\Omega\lbrack X]$, but, as we just observed, $f(X)$ splits in $\varphi
(E)[X]$. Because of unique factorization, this implies that $\alpha\in
\varphi(E)$.
\end{proof}

The above proof is a typical application of Zorn's lemma: once we know how to
do something in a finite (or countable) situation, Zorn's lemma allows us to
do it in general.

\begin{remark}
\label{sf17}Even for a finite field $F$, there will exist uncountably many
isomorphisms from one algebraic closure to a second, none of which is to be
preferred over any other. Thus it is (uncountably) sloppy to say that the
algebraic closure of $F$ is unique. All one can say is that, given two
algebraic closures $\Omega$, $\Omega^{\prime}$ of $F$, then, thanks to Zorn's
lemma, there exists an $F$-isomorphism $\Omega\rightarrow\Omega^{\prime}$.
\end{remark}

\section{Separable closures}

Let $\Omega$ be a field containing $F$, and let $\mathcal{E}{}$ be a set of
intermediate fields $F\subset E\subset\Omega$ with the following property:

\begin{quote}
(*) for all $E_{1},E_{2}\in\mathcal{E}{}$, there exists an $E\in\mathcal{E}{}$
such that $E_{1},E_{2}\subset E$.
\end{quote}

\noindent Then $E(\mathcal{E}{})=\bigcup_{E\in\mathcal{E}{}}E$ is a subfield
of $\Omega$ (and we call $\bigcup_{E\in\mathcal{E}{}}E$ a \emph{directed
}union), because (*) implies that every finite set of elements of
$E(\mathcal{E}{})$ is contained in a common $E\in\mathcal{E}{}$, and therefore
their product, sum, etc., also lie in $E(\mathcal{E}{})$.

We apply this remark to the set of subfields $E$ of $\Omega$ that are finite
and separable over $F$. As the composite of any two such subfields is again
finite and separable over $F$ (cf.\ \ref{ft16}), we see that the union $L$ of
all such $E$ is a subfield of $\Omega$. We call $L$ the \emph{separable
closure }of $F$ in $\Omega$ --- clearly, it is separable over $F$ and every
element of $\Omega$ separable over $F$ lies in $L$. Moreover, because a
separable extension of a separable extension is separable, $\Omega$ is purely
inseparable over $L$.

\begin{definition}
\label{sf18}(a) A field $\Omega$ is
\index{separably closed}
\emph{separably closed }if every nonconstant separable polynomial in
$\Omega\lbrack X]$ splits in $\Omega$.

(b) A field $\Omega$ is a%
\index{closure!separable}
\emph{separable closure }of a subfield $F$ if it is separable and algebraic
over $F$ and it is separably closed.
\end{definition}

\begin{theorem}
[*]\label{sf19}(a) Every field has a separable closure.

(b) Let $E$ be a separable algebraic extension of $F$, and let $\Omega$ be a
separable algebraic closure of $F$. There exists an $F$-homomorphism
$E\rightarrow\Omega$, and, if $E$ is also a separable closure of $F$, then
every such homomorphism is an isomorphism.
\end{theorem}

\begin{proof}
Replace \textquotedblleft polynomial\textquotedblright\ with \textquotedblleft
separable polynomial\textquotedblright\ in the proofs of the corresponding
theorems for algebraic closures. Alternatively, define $\Omega$ to be the
separable closure of $F$ in an algebraic closure, and apply the preceding theorems.
\end{proof}

\begin{aside}
\label{sf19a}It is not necessary to assume the full axiom of choice to prove
the existence of algebraic closures and their uniqueness up to isomorphism,
but only a weaker axiom. See Banaschewski, Bernhard. Algebraic closure without
choice. Z. Math. Logik Grundlag. Math. 38 (1992), no. 4, 383--385.
\end{aside}

\clearpage


\chapter{Infinite Galois Extensions}

In this chapter, we make free use of the axiom of choice.\footnote{It is
necessary to assume some choice axiom in order to have a sensible Galois
theory of infinite extensions. For example, it is consistent with
Zermelo-Fraenkel set theory that there exist an algebraic closure $L$ of the
$\mathbb{Q}$ with no nontrivial automorphisms. See: Hodges, Wilfrid,
L\"{a}uchli's algebraic closure of $\mathbb{Q}$. Math. Proc. Cambridge Philos.
Soc. 79 (1976), no. 2, 289--297.} We also assume the reader is familiar with
infinite topological products, including Tychonoff's theorem.

As in the finite case, an algebraic extension $\Omega$ of a field $F$ is said
to be Galois if it is normal and separable. For each finite Galois
subextension $M/F$ of $\Omega$, we have a restriction map $\Gal(\Omega
/F)\rightarrow\Gal(M/F)$, and hence a homomorphism $\Gal(\Omega/F)\rightarrow
\prod\nolimits_{M}\Gal(M/F)$, where the product is over all such
subextensions. Clearly every element of $\Omega$ lies in some $M$, and so this
homomorphism is injective. When we endow each group $\Gal(M/F)$ with the
discrete topology, the product acquires a topology for which it is compact.
The image of the homomorphism is closed, and so $\Gal(\Omega/F)$ also acquires
a compact topology. Now, all of the Galois theory of finite extensions holds
for infinite extensions\footnote{One difference: it need no longer be true
that the order of $\Gal(\Omega/F)$ equals the degree $[\Omega\colon F]$.
Certainly, $\Gal(\Omega/F)$ is infinite if and only if $[\Omega\colon F]$ is
infinite, but $\Gal(\Omega/F)$ is always uncountable when infinite whereas
$[\Omega\colon F]$ need not be.} provided \textquotedblleft
subgroup\textquotedblright\ is replaced everywhere with \textquotedblleft
closed subgroup\textquotedblright. The reader prepared to accept this, can
skip to the examples and exercises.

\section{Topological groups}

\begin{definition}
\label{ig1}A set $G$ together with a group structure and a topology is a%
\index{group!topological}
\emph{topological group }if the maps
\begin{align*}
(g,h)\mapsto gh  &  \colon G\times G\rightarrow G,\quad\\
g\mapsto g^{-1}  &  \colon G\rightarrow G
\end{align*}
are both continuous.
\end{definition}

Let $a$ be an element of a topological group $G$. Then $a_{L}\colon
G\xrightarrow{g\mapsto ag}G$ is continuous because it is the composite of%
\[
G\xrightarrow{g\mapsto(a,g)}G\times G\xrightarrow{(g,h)\mapsto gh}G.
\]
In fact, it is a homeomorphism with inverse $(a^{-1})_{L}$. Similarly
$a_{R}\colon g\mapsto ga$ and $g\mapsto g^{-1}$ are both homeomorphisms. In
particular, for any subgroup $H$ of $G$, the coset $aH$ of $H$ is open or
closed if $H$ is open or closed. As the complement of $H$ in $G$ is a union of
such cosets, this shows that $H$ is closed if it is open, and it is open if it
is closed and of finite index.

Recall that a%
\index{base!neighbourhood}
\emph{neighbourhood base} for a point $x$ of a topological space $X$ is a set
of neighbourhoods $\mathcal{N}$ such that every open subset $U$ of $X$
containing $x$ contains an $N$ from $\mathcal{N}{}$.

\begin{proposition}
\label{ig2}Let $G$ be a topological group, and let $\mathcal{N}{}$ be a
neighbourhood base for the identity element $e$ of $G$. Then\footnote{For
subsets $S$ and $S^{\prime}$ of $G$, we let $SS^{\prime}=\{ss^{\prime}\mid
s\in S$, $s^{\prime}\in S^{\prime}\}$ and $S^{-1}=\{s^{-1}\mid s\in S\}$.}

\begin{enumerate}
\item for all $N_{1},N_{2}\in\mathcal{N}{}$, there exists an $N^{\prime}%
\in\mathcal{N}{}$ such that $e\in N^{\prime}\subset N_{1}\cap N_{2}$;

\item for all $N\in\mathcal{N}{}$, there exists an $N^{\prime}\in\mathcal{N}%
{}$ such that $N^{\prime}N^{\prime}\subset N$;

\item for all $N\in\mathcal{N}{}$, there exists an $N^{\prime}\in\mathcal{N}%
{}$ such that $N^{\prime}\subset N^{-1}$;

\item for all $N\in\mathcal{N}{}$ and all $g\in G$, there exists an
$N^{\prime}\in\mathcal{N}{}$ such that $N^{\prime}\subset gNg^{-1};$

\item for all $g\in G$, $\{gN\mid N\in\mathcal{N}{}\}$ is a neighbourhood base
for $g$.
\end{enumerate}

\noindent Conversely, if $G$ is a group and $\mathcal{N}{}$ is a nonempty set
of subsets of $G$ satisfying (a,b,c,d), then there is a (unique) topology on
$G$ for which (e) holds.
\end{proposition}

\begin{proof}
If $\mathcal{N}{}$ is a neighbourhood base at $e$ in a topological group $G$,
then (b), (c), and (d) are consequences of the continuity of $(g,h)\mapsto
gh$, $g\mapsto g^{-1}$, and $h\mapsto ghg^{-1}$ respectively. Moreover, (a) is
a consequence of the definitions and (e) of the fact that $g_{L}$ is a homeomorphism.

Conversely, let $\mathcal{N}{}$ be a nonempty collection of subsets of a group
$G$ satisfying the conditions (a)--(d). Note that (a) implies that $e$ lies in
all the $N$ in $\mathcal{N}{}$. Define $\mathcal{U}{}$ to be the collection of
subsets $U$ of $G$ such that, for every $g\in U$, there exists an
$N\in\mathcal{N}{}$ with $gN\subset U$. Clearly, the empty set and $G$ are in
$\mathcal{U}{}$, and unions of sets in $\mathcal{U}{}$ are in $\mathcal{U}{}$.
Let $U_{1},U_{2}\in\mathcal{U}{}$, and let $g\in U_{1}\cap U_{2}$; by
definition there exist $N_{1},N_{2}\in\mathcal{N}{}$ with $gN_{1}%
,gN_{2}\subset U$; on applying (a) we obtain an $N^{\prime}\in\mathcal{N}{}$
such that $gN^{\prime}\subset U_{1}\cap U_{2}$, which shows that $U_{1}\cap
U_{2}\in\mathcal{U}{}$. It follows that the elements of $\mathcal{U}{}$ are
the open sets of a topology on $G$. In fact, one sees easily that it is the
unique topology for which (e) holds.

We next use (b) and (d) to show that $(g,g^{\prime})\mapsto gg^{\prime}$ is
continuous. Note that the sets $g_{1}N_{1}\times g_{2}N_{2}$ form a
neighbourhood base for $(g_{1},g_{2})$ in $G\times G$. Therefore, given an
open $U\subset G$ and a pair $(g_{1},g_{2})$ such that $g_{1}g_{2}\in U$, we
have to find $N_{1},N_{2}\in\mathcal{N}{}$ such that $g_{1}N_{1}g_{2}%
N_{2}\subset U$. As $U$ is open, there exists an $N\in\mathcal{N}{}$ such that
$g_{1}g_{2}N\subset U$. Apply (b) to obtain an $N^{\prime}$ such that
$N^{\prime}N^{\prime}\subset N$; then $g_{1}g_{2}N^{\prime}N^{\prime}\subset
U$. But $g_{1}g_{2}N^{\prime}N^{\prime}=g_{1}(g_{2}N^{\prime}g_{2}^{-1}%
)g_{2}N^{\prime}$, and it remains to apply (d) to obtain an $N_{1}%
\in\mathcal{N}{}$ such that $N_{1}\subset g_{2}N^{\prime}g_{2}^{-1}$.

Finally, we use (c) and (d) to show that $g\mapsto g^{-1}$ is continuous.
Given an open $U\subset G$ and a $g\in G$ such that $g^{-1}\in U$, we have to
find an $N\in\mathcal{N}{}$ such that $gN\subset U^{-1}$. By definition, there
exists an $N\in\mathcal{N}{}$ such that $g^{-1}N\subset U$. Now $N^{-1}%
g\subset U^{-1}$, and we use (c) to obtain an $N^{\prime}\in\mathcal{N}{}$
${}$ such that $N^{\prime}g\subset U^{-1}$, and (d) to obtain an
$N^{\prime\prime}\in\mathcal{N}{}$ such that $gN^{\prime\prime}\subset
g(g^{-1}N^{\prime}g)\subset U^{-1}$.
\end{proof}

\section{ The Krull topology on the Galois group}

Recall (\ref{ft11m}) that a finite extension $\Omega$ of $F$ is Galois over
$F$ if it is normal and separable, i.e., if every irreducible polynomial $f\in
F[X]$ having a root in $\Omega$ has $\deg f$ distinct roots in $\Omega$.
Similarly, we define an algebraic extension $\Omega$ of $F$ to be
\emph{Galois}%
\index{Galois}%
\emph{\/} over $F$ if it is normal and separable. For example,
$F^{\mathrm{sep}}$ is a Galois extension of $F$. Clearly, $\Omega$ is Galois
over $F$ if and only if it is a union of finite Galois extensions.

\begin{proposition}
\label{ag1a}If $\Omega$ is Galois over $F$, then it is Galois over every
intermediate field $M$.
\end{proposition}

\begin{proof}
Let $f(X)$ be an irreducible polynomial in $M[X]$ having a root $a$ in
$\Omega$. The minimal polynomial $g(X)$ of $a$ over $F$ splits into distinct
degree-one factors in $\Omega\lbrack X]$. As $f$ divides $g$ (in $M[X]$), it
also must split into distinct degree-one factors in $\Omega\lbrack X]$.
\end{proof}

\begin{proposition}
\label{ig2a}Let $\Omega$ be a Galois extension of $F$ and let $E$ be a
subfield of $\Omega$ containing $F$. Then every $F$-homomorphism
$E\rightarrow\Omega$ extends to an $F$-isomorphism $\Omega\rightarrow\Omega$.
\end{proposition}

\begin{proof}
The same Zorn's lemma argument as in the proof of Theorem \ref{sf16} shows
that every $F$-homomorphism $E\rightarrow\Omega$ extends to an $F$%
-homomorphism $\alpha\colon\Omega\rightarrow\Omega$. Let $a\in\Omega$, and let
$f$ be its minimal polynomial over $F$. Then $\Omega$ contains exactly
$\deg(f)$ roots of $f$, and so therefore does $\alpha(\Omega)$. Hence
$a\in\alpha(\Omega)$, which shows that $\alpha$ is surjective.
\end{proof}

\begin{corollary}
\label{ig2b}Let $\Omega\supset E\supset F$ be as in the proposition. If $E$ is
stable under $\Aut(\Omega/F)$, then $E$ is Galois over $F$.
\end{corollary}

\begin{proof}
Let $f(X)$ be an irreducible polynomial in $F[X]$ having a root $a$ in $E$.
Because $\Omega$ is Galois over $F$, $f(X)$ has $n=\deg(f)$ distinct roots
$a_{1},\ldots,a_{n}$ in $\Omega$. There is an $F$-isomorphism $F[a]\rightarrow
F[a_{i}]\subset\Omega$ sending $a$ to $a_{i}$ (they are both stem fields for
$f$), which extends to an $F$-isomorphism $\Omega\rightarrow\Omega$. As $E$ is
stable under $\Aut(\Omega/F)$, this shows that $a_{i}\in E$.
\end{proof}

Let $\Omega$ be a Galois extension of $F$, and let $G=\Aut(\Omega/F)$. For any
finite subset $S$ of $\Omega$, let%
\[
G(S)=\{\sigma\in G\mid\sigma s=s\text{ for all }s\in S\}.
\]


\begin{proposition}
\label{ig3}There is a unique structure of a topological group on $G$ for which
the sets $G(S)$ form an open neighbourhood base of $1$. For this topology, the
sets $G(S)$ with $S$ $G$-stable form a neighbourhood base of $1$ consisting of
open normal subgroups.
\end{proposition}

\begin{proof}
We show that the collection of sets $G(S)$ satisfies (a,b,c,d) of (\ref{ig2}).
It satisfies (a) because $G(S_{1})\cap G(S_{2})=G(S_{1}\cup S_{2})$. It
satisfies (b) and (c) because each set $G(S)$ is a group. Let $S$ be a finite
subset of $\Omega$. Then $F(S)$ is a finite extension of $F$, and so there are
only finitely many $F$-homomorphisms $F(S)\rightarrow\Omega$. Since $\sigma
S=\tau S$ if $\sigma|F(S)=\tau|F(S)$, this shows that $\bar{S}=\bigcup
\nolimits_{\sigma\in G}\sigma S$ is finite. Now $\sigma\bar{S}=\bar{S}$ for
all $\sigma\in G$, and it follows that $G(\bar{S})$ is normal in $G$.
Therefore, $\sigma G(\bar{S})\sigma^{-1}=G(\bar{S})\subset G(S)$, which proves
(d). It also proves the second statement.
\end{proof}

The topology on $\Aut(\Omega/F)$ defined in the proposition is called the%
\index{topology!Krull}
\emph{Krull topology}. We write $\Gal(\Omega/F)$ for $\Aut(\Omega/F)$ endowed
with the Krull topology, and call it the%
\index{Galois group!infinite}
\emph{Galois group} of $\Omega/F$. The Galois group of $F^{\mathrm{sep}}$ over
$F$ is called the \emph{absolute Galois group}\footnote{But note that the
absolute Galois group of $F$ is only defined up to an inner automorphism: let
$F^{\prime}$ be a second separable algebraic closure of $F$; the choice of an
isomorphism $F^{\prime}\rightarrow F^{\mathrm{sep}}$ determines an isomorphism
$\Gal(F^{\prime}/F)\rightarrow\Gal(F^{\mathrm{sep}}/F)$; a second isomorphism
$F^{\prime}\rightarrow F^{\mathrm{sep}}$ will differ from the first by an
element $\sigma$ of $\Gal(F^{\mathrm{sep}}/F)$, and the isomorphism
$\Gal(F^{\prime}/F)\rightarrow\Gal(F^{\mathrm{sep}}/F)$ it defines differs
from the first by $\inn(\sigma)$.} of $F$%
\index{Galois group!absolute}%
.

If $S$ is a finite set stable under $G$, then $F(S)$ is a finite extension of
$F$ stable under $G$, and hence Galois over $F$ (\ref{ig2b}). Therefore,
\[
\left\{  \Gal(\Omega/E)\mid E\text{ finite and Galois over }F\right\}
\]
is a neighbourhood base of $1$ consisting of open normal subgroups.

\begin{proposition}
\label{ig3b}Let $\Omega$ be Galois over $F$. For every intermediate field $E$
finite and Galois over $F$, the map
\[
\sigma\mapsto\sigma|E\colon\Gal(\Omega/F)\rightarrow\Gal(E/F)
\]
is a continuous surjection (discrete topology on $\Gal(E/F)$).
\end{proposition}

\begin{proof}
Let $\sigma\in\Gal(E/F)$, and regard it as an $F$-homomorphism $E\rightarrow
\Omega$. Then $\sigma$ extends to an $F$-isomorphism $\Omega\rightarrow\Omega$
(see \ref{ig2a}), which shows that the map is surjective. For every finite set
$S$ of generators of $E$ over $F$, $\Gal(\Omega/E)=G(S)$, which shows that the
inverse image of $1_{\Gal(E/F)}$ is open in $G$. By homogeneity, the same is
true for every element of $\Gal(E/F)$.
\end{proof}

\begin{proposition}
\label{ig3d}The Galois group $G$ of a Galois extension $\Omega/F$ is compact
and totally disconnected.\footnote{Following Bourbaki, we require compact
spaces to be Hausdorff. A topological space is \emph{totally disconnected }if
its connected components are the one-point sets.}
\end{proposition}

\begin{proof}
We first show that $G$ is Hausdorff. If $\sigma\neq\tau$, then $\sigma
^{-1}\tau\neq1_{G}$, and so it moves some element of $\Omega$, i.e., there
exists an $a\in\Omega$ such that $\sigma(a)\neq\tau(a)$. For any $S$
containing $a$, $\sigma G(S)$ and $\tau G(S)$ are disjoint because their
elements act differently on $a$. Hence they are disjoint open subsets of $G$
containing $\sigma$ and $\tau$ respectively.

We next show that $G$ is compact. As we noted above, if $S$ is a finite set
stable under $G$, then $G(S)$ is a normal subgroup of $G$, and it has finite
index because it is the kernel of%
\[
G\rightarrow\mathrm{Sym}(S).
\]
Since every finite set is contained in a stable finite set,\footnote{Each
element of $\Omega$ is algebraic over $F$, and its orbit is the set of its
conjugates (roots of its minimal polynomial over $F$).} the argument in the
last paragraph shows that the map%
\[
G\rightarrow\prod_{S\text{ finite stable under }G}G/G(S)
\]
is injective. When we endow $\prod G/G(S)$ with the product topology, the
induced topology on $G$ is that for which the $G(S)$ form an open
neighbourhood base of $e$, i.e., it is the Krull topology. According to the
Tychonoff theorem, $\prod G/G(S)$ is compact, and so it remains to show that
$G$ is closed in the product. For each $S_{1}\subset S_{2}$, there are two
continuous maps $\prod G/G(S)\rightarrow G/G(S_{1})$, namely, the projection
onto $G/G(S_{1})$ and the projection onto $G/G(S_{2})$ followed by the
quotient map $G/G(S_{2})\rightarrow G/G(S_{1})$. Let $E(S_{1},S_{2})$ be the
closed subset of $\prod G/G(S)$ on which the two maps agree. Then
$\bigcap_{S_{1}\subset S_{2}}E(S_{1},S_{2})$ is closed, and equals the image
of $G$.

Finally, for each finite set $S$ stable under $G$, $G(S)$ is a subgroup that
is open and hence closed. Since $\bigcap G(S)=\{1_{G}\}$, this shows that the
connected component of $G$ containing $1_{G}$ is just $\{1_{G}\}$. By
homogeneity, a similar statement is true for every element of $G$.
\end{proof}

\begin{proposition}
\label{ig3c}For every Galois extension $\Omega/F$, $\Omega^{\Gal(\Omega/F)}=F$.
\end{proposition}

\begin{proof}
Every element of $\Omega\smallsetminus F$ lies in a finite Galois extension of
$F$, and so this follows from the surjectivity in Proposition \ref{ig3b}.
\end{proof}

\begin{aside}
\label{ig4}There is a converse to Proposition \ref{ig3d}: every compact
totally disconnected group arises as the Galois group of some Galois extension
of fields of characteristic zero (Douady, A., Cohomologie des groupes compact
totalement discontinus (d'apr\`{e}s J. Tate), S\'{e}minaire Bourbaki 1959/60,
no. 189; Waterhouse, Proc.\ AMS, 1973). However, not all such groups arise as
the \textit{absolute} Galois group of a field. In fact, absolute Galois groups
of fields of characterist zero, if finite, must have order $1$ or $2$. More
precisely, there is the following theorem of Artin and Schreier (1927): let
$F$ be a field, not algebraically closed, but of finite index in its algebraic
closure; then $F$ is real-closed and $E=F[\sqrt{-1}]$ (Jacobson 1964, Chapter
VI, Theorem 17).
\end{aside}

\section{The fundamental theorem of infinite Galois theory}

\begin{proposition}
\label{ig5}Let $\Omega$ be Galois over $F$, with Galois group $G$.

\begin{enumerate}
\item Let $M$ be a subfield of $\Omega$ containing $F$. Then $\Omega$ is
Galois over $M$, the Galois group $\Gal(\Omega/M)$ is closed in $G$, and
$\Omega^{\Gal(\Omega/M)}=M$.

\item For every subgroup $H$ of $G$, $\Gal(\Omega/\Omega^{H})$ is the closure
of $H$.
\end{enumerate}
\end{proposition}

\begin{proof}
(a) The first assertion was proved in (\ref{ag1a}). For each finite subset
$S\subset M$, $G(S)$ is an open subgroup of $G$, and hence it is closed. But
$\Gal(\Omega/M)=\bigcap_{S\subset M}G(S)$, and so it also is closed. The final
statement now follows from (\ref{ig3c}).

(b) Since $\Gal(\Omega/\Omega^{H})$ contains $H$ and is closed, it certainly
contains the closure $\bar{H}$ of $H$. On the other hand, let $\sigma\in
G\smallsetminus\bar{H}$; we have to show that $\sigma$ moves some element of
$\Omega^{H}$. Because $\sigma$ is not in the closure of $H$,
\[
\sigma\Gal(\Omega/E)\cap H=\emptyset
\]
for some finite Galois extension $E$ of $F$ in $\Omega$ (because the sets
$\Gal(\Omega/E)$ form a neighbourhood base of $1$; see above). Let $\phi$
denote the surjective map $\Gal(\Omega/F)\rightarrow\Gal(E/F)$. Then
$\sigma|E\notin\phi H$, and so $\sigma$ moves some element of $E^{\phi
H}\subset\Omega^{H}$ (apply \ref{ft13}b).
\end{proof}

\begin{theorem}
\label{ig6}Let $\Omega$ be Galois over $F$ with Galois group $G$. The maps
\[
H\mapsto\Omega^{H},\quad M\mapsto\Gal(\Omega/M)
\]
are inverse bijections between the set of closed subgroups of $G$ and the set
of intermediate fields between $\Omega$ and $F$:%
\[
\{\text{closed subgroups of }G\}\leftrightarrow\{\text{intermediate fields
}F\subset M\subset\Omega\}.
\]
Moreover,

\begin{enumerate}
\item the correspondence is inclusion-reversing: $H_{1}\supset H_{2}\iff
\Omega^{H_{1}}\subset\Omega^{H_{2}}$;

\item a closed subgroup $H$ of $G$ is open if and only if $\Omega^{H}$ has
finite degree over $F$, in which case $(G\colon H)=[\Omega^{H}\colon F]$;

\item $\sigma H\sigma^{-1}\leftrightarrow\sigma M$, i.e., $\Omega^{\sigma
H\sigma^{-1}}=\sigma(\Omega^{H})$; $\Gal(\Omega/\sigma M)=\sigma
\Gal(\Omega/M)\sigma^{-1}$;

\item a closed subgroup $H$ of $G$ is normal if and only if $\Omega^{H}$ is
Galois over $F$, in which case $\Gal(\Omega^{H}/F)\simeq G/H$.
\end{enumerate}
\end{theorem}

\begin{proof}
For the first statement, we have to show that $H\mapsto\Omega^{H}$ and
$M\mapsto\Gal(\Omega/M)$ are inverse maps.

Let $H$ be a closed subgroup of $G$. Then $\Omega$ is Galois over $\Omega^{H}$
and $\Gal(\Omega/\Omega^{H})=H$ (see \ref{ig5}).

Let $M$ be an intermediate field. Then $\Gal(\Omega/M)$ is a closed subgroup
of $G$ and $\Omega^{\Gal(\Omega/M)}=M\,$(see \ref{ig5}).

(a) We have the obvious implications:
\[
H_{1}\supset H_{2}\implies\Omega^{H_{1}}\subset\Omega^{H_{2}}\implies
\Gal(\Omega/\Omega^{H_{1}})\supset\Gal(\Omega/\Omega^{H_{2}}).
\]
But $\Gal(\Omega/\Omega^{H_{i}})=H_{i}$ (see \ref{ig5}).

(b) As we noted earlier, a closed subgroup of finite index in a topological
group is always open. Because $G$ is compact, conversely an open subgroup of
$G$ is always of finite index. Let $H$ be such a subgroup. The map
$\sigma\mapsto\sigma|\Omega^{H}$ defines a bijection%
\[
G/H\rightarrow\Hom_{F}(\Omega^{H},\Omega)
\]
(apply \ref{ig2a}) from which the statement follows.

(c) For $\tau\in G$ and $\alpha\in\Omega$, $\tau\alpha=\alpha\iff\sigma
\tau\sigma^{-1}(\sigma\alpha)=\sigma\alpha$. Therefore, $\Gal(\Omega/\sigma
M)=\sigma\Gal(\Omega/M)\sigma^{-1}\,$, and so $\sigma\Gal(\Omega/M)\sigma
^{-1}\leftrightarrow\sigma M.$

(d) Let $H\leftrightarrow M$. It follows from (c) that $H$ is normal if and
only if $M$ is stable under the action of $G$. But $M$ is stable under the
action of $G$ if and only it is a union of finite extensions of $F$ stable
under $G$, i.e., of finite Galois extensions of $G$. We have already observed
that an extension is Galois if and only if it is a union of finite Galois extensions.
\end{proof}

\begin{remark}
\label{ig7}As in the finite case (\ref{ft18}), we can deduce the following statements.

(a) Let $(M_{i})_{i\in I}$ be a (possibly infinite) family of intermediate
fields, and let $H_{i}\leftrightarrow M_{i}$. Let $%
%TCIMACRO{\tprod }%
%BeginExpansion
{\textstyle\prod}
%EndExpansion
M_{i}$ be the smallest field containing all the $M_{i}$; then because
$\bigcap_{i\in I}H_{i}$ is the largest (closed) subgroup contained in all the
$H_{i}$,%
\[
\Gal(\Omega/%
%TCIMACRO{\tprod }%
%BeginExpansion
{\textstyle\prod}
%EndExpansion
M_{i})=\bigcap_{i\in I}H_{i}.
\]


(b) Let $M\leftrightarrow H$. The largest (closed) normal subgroup contained
in $H$ is $N=\bigcap_{\sigma}\sigma H\sigma^{-1}$ (cf.\ GT, 4.10),
and so $\Omega^{N}$, which is the composite of the fields $\sigma M$, is the
smallest normal extension of $F$ containing $M$.
\end{remark}

\noindent\begin{minipage}{4.0in}
\begin{proposition}
\label{ig8}Let $E$ and $L$ be field extensions of $F$ contained in some common
field. If $E/F$ is Galois, then $EL/L$ and $E/E\cap L$ are Galois, and the map%
\[
\sigma\mapsto\sigma|E\colon\Gal(EL/L)\rightarrow\Gal(E/E\cap L)
\]
is an isomorphism of topological groups.
\end{proposition}
\end{minipage}
\begin{minipage}{1.5in}
\begin{tikzpicture}
\matrix(m)[matrix of math nodes, row sep=1.5em, column sep=0.5em,
text height=1.5ex, text depth=0.25ex]
{&EL\\
E&&L\\
&E\cap L\\
&F\\};
\path[-,font=\scriptsize]
(m-1-2) edge  (m-2-1)
edge node[right] {$=$} (m-2-3)
(m-3-2) edge node[right] {$=$} (m-2-1)
edge  (m-2-3)
edge  (m-4-2);
\end{tikzpicture}
\end{minipage}


\begin{proof}
We first prove that the map is continuous. Let $G_{1}=\Gal(EL/L)$ and let
$G_{2}=\Gal(E/E\cap L)$. For any finite set $S$ of elements of $E$, the
inverse image of $G_{2}(S)$ in $G_{1}$ is $G_{1}(S)$.

We next show that the map is an isomorphism of groups (neglecting the
topology). As in the finite case, it is an injective homomorphism
(\ref{ft18f}). Let $H$ be the image of the map. Then the fixed field of $H$ is
$E\cap L$, which implies that $H$ is dense in $\Gal(E/E\cap L)$. But $H$ is
closed because it is the continuous image of a compact space in a Hausdorff
space, and so $H=\Gal(E/E\cap L)$.

Finally, we prove that it is open. An open subgroup of $\Gal(EL/L)$ is closed
(hence compact) of finite index; therefore its image in $\Gal(E/E\cap L)$ is
compact (hence closed) of finite index, and hence open.
\end{proof}

\begin{corollary}
\label{ig9}Let $\Omega$ be an algebraically closed field containing $F$, and
let $E$ and $L$ be as in the proposition. If $\rho\colon E\rightarrow\Omega$
and $\sigma\colon L\rightarrow\Omega$ are $F$-homomorphisms such that
$\rho|E\cap L=\sigma|E\cap L$, then there exists an $F$-homomorphism
$\tau\colon EL\rightarrow\Omega$ such that $\tau|E=\rho$ and $\tau|L=\sigma$.
\end{corollary}

\begin{proof}
According to (\ref{ig2a}), $\sigma$ extends to an $F$-homomorphism $s\colon
EL\rightarrow\Omega$. As $s|E\cap L=\rho|E\cap L$, we can write $s|E=\rho
\circ\varepsilon$ for some $\varepsilon\in\Gal(E/E\cap L)$. According to the
proposition, there exists a unique $e\in\Gal(EL/L)$ such that $e|E=\varepsilon
$. Define $\tau=s\circ e^{-1}$.
\end{proof}

\begin{example}
\label{ig10}Let $\Omega$ be an algebraic closure of the finite field
$\mathbb{F}_{p}$. Then $G=\Gal(\Omega/\mathbb{F}_{p})$ contains a canonical
Frobenius element, $\sigma=(a\mapsto a^{p})$, and it is generated by it as a
topological group, i.e., $G$ is the closure of $\langle\sigma\rangle$. We now
determine the structure of $G$.

Endow $\mathbb{Z}$ with the topology for which the groups $n\mathbb{Z}$,
$n\geq1$, form a fundamental system of neighbourhoods of $0$. Thus two
integers are close if their difference is divisible by a large integer.

As for any topological group, we can complete $\mathbb{Z}$ for this topology.
A Cauchy sequence in $\mathbb{Z}$ is a sequence $(a_{i})_{i\geq1}$, $a_{i}%
\in\mathbb{Z}$, satisfying the following condition: for all $n\geq1$, there
exists an $N$ such that $a_{i}\equiv a_{j}\mod n$ for $i,j>N$. Call a Cauchy
sequence in $\mathbb{Z}$ trivial if $a_{i}\rightarrow0$ as $i\rightarrow
\infty$, i.e., if for all $n\geq1$, there exists an $N$ such that $a_{i}%
\equiv0\mod n$ for all $i>N$. The Cauchy sequences form a commutative group,
and the trivial Cauchy sequences form a subgroup. We define $\hat{\mathbb{Z}}$
to be the quotient of the first group by the second. It has a ring structure,
and the map sending $m\in\mathbb{Z}$ to the constant sequence $m,m,m,\ldots$
identifies $\mathbb{Z}$ with a subgroup of $\hat{\mathbb{Z}}$.

Let $\alpha\in\hat{\mathbb{Z}}$ be represented by the Cauchy sequence
$(a_{i})$. The restriction of the Frobenius element $\sigma$ to $\mathbb{F}%
_{p^{n}}$ has order $n$. Therefore $(\sigma|\mathbb{F}_{p^{n}})^{a_{i}}$ is
independent of $i$ provided it is sufficiently large, and we can define
$\sigma^{\alpha}\in\Gal(\Omega/\mathbb{F}_{p})$ to be such that, for each $n$,
$\sigma^{\alpha}|\mathbb{F}_{p^{n}}=(\sigma|\mathbb{F}_{p^{n}})^{a_{i}}$ for
all $i$ sufficiently large (depending on $n$). The map $\alpha\mapsto
\sigma^{\alpha}\colon\hat{\mathbb{Z}}\rightarrow\Gal(\Omega/\mathbb{F}_{p})$
is an isomorphism.

The group $\hat{\mathbb{Z}}$ is uncountable. To most analysts, it is a little
weird---its connected components are one-point sets. To number theorists it
will seem quite natural --- the Chinese remainder theorem implies that it is
isomorphic to $\prod_{p\text{\ prime}}\mathbb{Z}_{p}$ where $\mathbb{Z}_{p}$
is the ring of $p$-adic integers.
\end{example}

\begin{example}
\label{ig11}Let $\mathbb{Q}{}^{\mathrm{al}}$ be the algebraic closure of
$\mathbb{Q}$ in $\mathbb{C}$. Then $\Gal(\mathbb{Q}{}^{\mathrm{al}}%
/\mathbb{Q})$ is one of the most basic, and intractable, objects in
mathematics. It is expected that \textit{every}\emph{\/} finite group occurs
as a quotient of it. This is known, for example, for $S_{n}$ and for every
sporadic simple group except possibly $M_{23}$. See (\ref{ag30}) and mo80359.

On the other hand, we do understand $\Gal(F^{\text{ab}}/F)$ where
$F\subset\mathbb{Q}{}^{\mathrm{al}}$ is a finite extension of $\mathbb{Q}$ and
$F$$^{\text{{a}{b}}}$ is the union of all finite abelian extensions of $F$
contained in $\mathbb{Q}{}^{\mathrm{al}}$. For example, $\Gal(\mathbb{Q}%
^{\text{ab}}/\mathbb{Q})\simeq\hat{\mathbb{Z}}^{\times}$. This is abelian
class field theory --- see my notes Class Field Theory\textit{.}
\end{example}

\begin{aside}
\label{ig11a}A%
\index{Galois correspondence}
\emph{simple Galois correspondence }is a system consisting of two partially
ordered sets $P$ and $Q$ and order reversing maps $f\colon P\rightarrow Q$ and
$g\colon Q\rightarrow P$ such that $gf(p)\geq p$ for all $p\in P$ and
$fg(q)\geq q$ for all $q\in Q$. Then $fgf=f$, because $fg(fp)\geq fp$ and
$gf(p)\geq p$ implies $f(gfp)\leq f(p)$ for all $p\in P$. Similarly, $gfg=g$,
and it follows that $f$ and $g$ define a one-to-one correspondence between the
sets $g(Q)$ and $f(P)$.

From a Galois extension $\Omega$ of $F$ we get a simple Galois correspondence
by taking $P$ to be the set of subgroups of $\Gal(\Omega/F)$ and $Q$ to be the
set of subsets of $\Omega$, and by setting $f(H)=\Omega^{H}$ and $g(S)=G(S)$.
Thus, to prove the one-to-one correspondence in the fundamental theorem, it
suffices to identify the closed subgroups as exactly those in the image of $g$
and the intermediate fields as exactly those in the image of $f$. This is
accomplished by (\ref{ig5}).
\end{aside}

\section{Galois groups as inverse limits}

\begin{definition}
\label{ig12}A partial ordering $\leq$ on a set $I$ is
\index{directed}
\emph{directed}, and the pair $(I,\leq)$ is a \emph{directed set}, if for all
$i,j\in I$ there exists a $k\in I$ such that $i,j\leq k$.
\end{definition}

\begin{definition}
\label{ig13}Let $(I,\leq)$ be a directed set, and let $\mathsf{C}$ be a
category (for example, the category of groups and homomorphisms, or the
category of topological groups and continuous homomorphisms).

\begin{enumerate}
\item An%
\index{inverse system}
\emph{inverse system }in $\mathsf{C}$ indexed by $(I,\leq)$ is a family
$(A_{i})_{i\in I}$ of objects of $\mathsf{C}$ together with a family
$(p_{i}^{j}\colon A_{j}\rightarrow A_{i})_{i\leq j}$ of morphisms such that
$p_{i}^{i}=\id_{A_{i}}$ and $p_{i}^{j}\circ p_{j}^{k}=p_{i}^{k}$ all $i\leq
j\leq k$.

\item An object $A$ of $\mathsf{C}$ together with a family $(p_{j}\colon
A\rightarrow A_{j})_{j\in I}$ of morphisms satisfying $p_{i}^{j}\circ
p_{j}=p_{i}$ all $i\leq j$ is an%
\index{inverse limit}
\emph{inverse limit }of the system in (a) if it has the following universal
property: for any other object $B$ and family $(q_{j}\colon B\rightarrow
A_{j})$ of morphisms such $p_{i}^{j}\circ q_{j}=q_{i}$ all $i\leq j$, there
exists a unique morphism $r\colon B\rightarrow A$ such that $p_{j}\circ
r=q_{j}$ for $j$,%
\[
\begin{tikzpicture}[descr/.style={fill=white}]
\matrix(m)[matrix of math nodes, row sep=3em, column sep=2.5em,
text height=1.5ex, text depth=0.25ex]
{B&A\\
&&A_j\\
&A_i\\};
\path[->,font=\scriptsize,>=angle 90]
(m-1-1) edge node[descr,sloped]{$q_j$} (m-2-3)
(m-1-1) edge node[descr]{$q_i$} (m-3-2)
(m-1-2) edge node[descr]{$p_i$} (m-3-2)
(m-1-2) edge node[descr]{$p_j$} (m-2-3)
(m-2-3) edge node[descr]{$p_i^j$} (m-3-2);
\path[dashed,->,font=\scriptsize,>=angle 90]
(m-1-1) edge node[auto]{$r$} (m-1-2);
\end{tikzpicture}
\]

\end{enumerate}
\end{definition}

\noindent Clearly, the inverse limit (if it exists), is uniquely determined by
this condition up to a unique isomorphism. We denote it by $\varprojlim
(A_{i},p_{i}^{j})$, or just $\varprojlim A_{i}$.

\begin{example}
\label{ig14}Let $(G_{i},p_{i}^{j}\colon G_{j}\rightarrow G_{i})$ be an inverse
system of groups. Let
\[
G=\{(g_{i})\in\prod G_{i}\mid p_{i}^{j}(g_{j})=g_{i}\text{ all }i\leq j\},
\]
and let $p_{i}\colon G\rightarrow G_{i}$ be the projection map. Then
$p_{i}^{j}\circ p_{j}=p_{i}$ is just the equation $p_{i}^{j}(g_{j})=g_{i}$.
Let $(H,q_{i})$ be a second family such that $p_{i}^{j}\circ q_{j}=q_{i}$. The
image of the homomorphism
\[
h\mapsto(q_{i}(h))\colon H\rightarrow\prod G_{i}%
\]
is contained in $G$, and this is the unique homomorphism $H\rightarrow G$
carrying $q_{i}$ to $p_{i}$. Hence $(G,p_{i})=\varprojlim(G_{i},p_{i}^{j})$.
\end{example}

\begin{example}
\label{ig15}Let $(G_{i},p_{i}^{j}\colon G_{j}\rightarrow G_{i})$ be an inverse
system of topological groups and continuous homomorphisms. When endowed with
the product topology, $\prod G_{i}$ becomes a topological group
\[
G=\{(g_{i})\in\prod G_{i}\mid p_{i}^{j}(g_{j})=g_{i}\text{ all }i\leq j\},
\]
and $G$ becomes a topological subgroup with the subspace topology. The
projection maps $p_{i}$ are continuous. Let $H$ be $(H,q_{i})$ be a second
family such that $p_{i}^{j}\circ q_{j}=q_{i}$. The homomorphism
\[
h\mapsto(q_{i}(h))\colon H\rightarrow\prod G_{i}%
\]
is continuous because its composites with projection maps are continuous
(universal property of the product). Therefore $H\rightarrow G$ is continuous,
and this shows that $(G,p_{i})=\varprojlim(G_{i},p_{i}^{j})$.
\end{example}

\begin{example}
\label{ig16}Let $(G_{i},p_{i}^{j}\colon G_{j}\rightarrow G_{i})$ be an inverse
system of finite groups, and regard it as an inverse system of topological
groups by giving each $G_{i}$ the discrete topology. A topological group $G$
arising as an inverse limit of such a system is said to be%
\index{group!profinite}
\emph{profinite}\footnote{An inverse limit is also called a projective limit.
Thus a profinite group is a projective limit of finite groups.}$.$

If $(x_{i})\notin G$, say $p_{i_{0}}^{j_{0}}(x_{j_{0}})$ $\neq x_{i_{0}}$,
then%
\[
G\cap\{(g_{j})\mid g_{j_{0}}=x_{j_{0}},\quad g_{i_{0}}=x_{i_{0}}%
\}=\emptyset\text{.}%
\]
As the second set is an open neighbourhood of $(x_{i})$, this shows that $G$
is closed in $%
%TCIMACRO{\tprod }%
%BeginExpansion
{\textstyle\prod}
%EndExpansion
G_{i}$. By Tychonoff's theorem, $%
%TCIMACRO{\tprod }%
%BeginExpansion
{\textstyle\prod}
%EndExpansion
G_{i}$ is compact, and so $G$ is also compact. The map $p_{i}\colon
G\rightarrow G_{i}$ is continuous, and its kernel $U_{i}$ is an open subgroup
of finite index in $G$ (hence also closed). As $\bigcap U_{i}=\{e\}$, the
connected component of $G$ containing $e$ is just $\{e\}$. By homogeneity, the
same is true for every point of $G$: the connected components of $G$ are the
one-point sets --- $G$ is totally disconnected.

We have shown that a profinite group is compact and totally disconnected, and
it is an exercise to prove the converse.\footnote{More precisely, it is
Exercise 3 of \S 7 of Chapter 3 of Bourbaki's General Topology.}
\end{example}

\begin{example}
\label{ig16a}Let $\Omega$ be a Galois extension of $F$. The composite of two
finite Galois extensions of in $\Omega$ is again a finite Galois extension,
and so the finite Galois subextensions of $\Omega$ form a directed set $I$.
For each $E$ in $I$ we have a finite group $\Gal(E/F)$, and for each $E\subset
E^{\prime}$ we have a restriction homomorphism $p_{E}^{E^{\prime}}%
\colon\Gal(E^{\prime}/F)\rightarrow\Gal(E/F)$. In this way, we get an inverse
system of finite groups $(\Gal(E/F),p_{E}^{E^{\prime}})$ indexed by $I$.

For each $E$, there is a restriction homomorphism $p_{E}\colon\Gal(\Omega
/F)\rightarrow\Gal(E/F)$ and, because of the universal property of inverse
limits, these maps define a homomorphism%
\[
\Gal(\Omega/F)\rightarrow\varprojlim\Gal(E/F)\text{.}%
\]
This map is an isomorphism of topological groups. This is a restatement of
what we showed in the proof of (\ref{ig3d}).
\end{example}

\section{Nonopen subgroups of finite index}

We apply Zorn's lemma\footnote{This is really needed --- see mo106216.} to
construct a nonopen subgroup of finite index in $\Gal(\mathbb{Q}^{\mathrm{al}%
}/\mathbb{Q})$.\footnote{Contrast: \textquotedblleft\ldots\ it is not known,
even when $G=\Gal(\mathbb{\bar{Q}}/\mathbb{Q}{})$, whether every subgroup of
finite index in $G{}$ is open; this is one of a number of related unsolved
problems, all of which appear to be very difficult.\textquotedblright%
\ \noindent Swinnerton-Dyer, H. P. F., A brief guide to algebraic number
theory. Cambridge, 2001, p.\ 133.}

\begin{lemma}
\label{ig18}Let $V$ be an infinite-dimensional vector space. For all $n\geq1$,
there exists a subspace $V_{n}$ of $V$ such that $V/V_{n}$ has dimension $n$.
\end{lemma}

\begin{proof}
Zorn's lemma shows that $V$ contains maximal linearly independent subsets, and
then the usual argument shows that such a subset spans $V$, i.e., is a basis.
Choose a basis, and take $V_{n}$ to be the subspace spanned by the set
obtained by omitting $n$ elements from the basis.
\end{proof}

\begin{proposition}
\label{ig17}The group $\Gal(\mathbb{Q}^{\mathrm{al}}/\mathbb{Q}{})$ has
nonopen normal subgroups of index $2^{n}$ for all $n>1$.
\end{proposition}

\begin{proof}
Let $E$ be the subfield $\mathbb{Q}{}[\sqrt{-1},\sqrt{2},\ldots,\sqrt
{p},\ldots]$, $p$ prime, of $\mathbb{C}$. For each $p$,
\[
\Gal(\mathbb{Q}[\sqrt{-1},\sqrt{2},\ldots,\sqrt{p}]/\mathbb{Q})
\]
is a product of copies of $\mathbb{Z}/2\mathbb{Z}$ indexed by the set
$\{\mathrm{primes}\leq p\}\cup\{\infty\}$ (apply \ref{ag22}; see also
\ref{ag20b}b). As
\[
\Gal(E/\mathbb{Q}{})=\varprojlim\Gal(\mathbb{Q}{}[\sqrt{-1},\sqrt{2}%
,\ldots,\sqrt{p}]/\mathbb{Q}{}),
\]
it is a direct product of copies of $\mathbb{Z}/2\mathbb{Z}$ indexed by the
primes $l$ of $\mathbb{Q}{}$ (including $l=\infty$) endowed with the product
topology. Let $G=\Gal(E/\mathbb{Q}{})$, and let
\[
H=\{(a_{l})\in G\mid a_{l}=0\text{ for all but finitely many }l\}.
\]
This is a subgroup of $G$ (in fact, it is a direct \textit{sum} of copies of
$\mathbb{Z}/2\mathbb{Z}$ indexed by the primes of $\mathbb{Q}{}$), and it is
dense in $G$ because\footnote{Alternatively, let $(a_{l})\in G$; then the
sequence
\[
(a_{\infty},0,0,0,\ldots)\text{, }(a_{\infty},a_{2},0,0,\ldots)\text{,
}(a_{\infty},a_{2},a_{3},0,\ldots),\ldots
\]
$\quad$\ in $H$ converges to $(a_{l})$.} clearly every open subset of $G$
contains an element of $H$. We can regard $G/H$ as vector space over
$\mathbb{F}{}_{2}$ and apply the lemma to obtain subgroups $G_{n}$ of index
$2^{n}$ in $G$ containing $H$. If $G_{n}$ is open in $G$, then it is closed,
which contradicts the fact that $H$ is dense. Therefore, $G_{n}$ is not open,
and its inverse image in $\Gal(\mathbb{Q}{}^{\mathrm{al}}/\mathbb{Q}{})$ is
the desired subgroup.\footnote{The inverse image is not open because every
continuous homomorphism from a compact group to a separated group is open.
Alternatively, if the inverse image were open, its fixed field would be a
nontrivial extension $E$ of $\mathbb{Q}$ contained in $\mathbb{Q}{}[\sqrt
{-1},\sqrt{2},\ldots,\sqrt{p},\ldots]$; but then $E$ would be fixed by $G_{n}%
$, which is dense.}
\end{proof}

\begin{aside}
\label{ig19}Let $G=\Gal(\mathbb{Q}^{\mathrm{al}}/\mathbb{Q}{})$. We showed in
the above proof that there is a closed normal subgroup $N=\Gal(\mathbb{Q}%
{}^{\mathrm{al}}/E{})$ of $G$ such that $G/N$ is an uncountable vector space
over $\mathbb{F}{}_{2}$. Let $(G/N)^{\vee}$ be the dual of this vector space
(also uncountable). Every nonzero $f\in(G/N)^{\vee}$ defines a surjective map
$G\rightarrow\mathbb{F}{}_{2}$ whose kernel is a subgroup of index $2$ in $G$.
These subgroups are distinct, and so $G$ has uncountably many subgroups of
index $2$. Only countably many of them are open because $\mathbb{Q}{}$ has
only countably many quadratic extensions in a fixed algebraic closure.
\end{aside}

\begin{aside}
\label{ig19a}Let $G$ be a profinite group that is finitely generated as a
topological group. It is a difficult theorem, only recently proved, that every
subgroup of finite index in $G$ is open (Nikolov, Nikolay; Segal, Dan. On
finitely generated profinite groups. I. Strong completeness and uniform
bounds. Ann. of Math. (2) 165 (2007), no. 1, 171--238.)
\end{aside}

\section{Exercises}

\begin{exercise}
\label{x83} Let $p$ be a prime number, and let $\Omega$ be the subfield of
$\mathbb{C}{}$ generated over $\mathbb{Q}{}$ by all $p^{m}$th roots of $1$ for
$m\in\mathbb{N}{}$. Show that $\Omega$ is Galois over $\mathbb{Q}{}$ with
Galois group $\mathbb{Z}{}_{p}^{\times}=\varprojlim
(\mathbb{Z}{}/p^{m}\mathbb{Z}{})^{\times}$. Hint: Use that $\Omega$ is the
union of a tower of subfields
\[
\mathbb{Q}{}\subset\mathbb{Q}{}[\zeta_{p}]\subset\cdots\subset\mathbb{Q}%
{}[\zeta_{p^{m}}]\subset\mathbb{Q}{}[\zeta_{p^{m+1}}]\subset\cdots.
\]
For $p$ odd, show that $\Gal(\mathbb{Q}(\zeta_{p^{\infty}})/\mathbb{Q}%
(\zeta_{p}))\simeq\mathbb{Z}_{p}$. Hint: Let $a\in\mathbb{Z}_{p}$ correspond
to \[\zeta_{p^{k}}\mapsto\zeta_{p^{k}}^{(1+p)^{a\bmod p^{k-1}}}.\]
\end{exercise}

\begin{exercise}
\label{x84} Let $\mathbb{F}{}$ be an algebraic closure of $\mathbb{F}_{p}$,
and let $\mathbb{F}{}_{p^{m}}$ be the subfield of $\mathbb{F}{}$ with $p^{m}$
elements. Show that
\[
\varprojlim_{m\geq1}\Gal(\mathbb{F}{}_{p^{m}}/\mathbb{F}{}_{p})\simeq
\varprojlim_{m\geq1}\mathbb{Z}{}/m\mathbb{Z}%
\]
and deduce that $\Gal(\mathbb{F}{}/\mathbb{F}{}_{p})\simeq{}\mathbb{\hat{Z}}%
{}.$
\end{exercise}

\begin{exercise}
For a profinite group $G$, define $G^{\mathrm{ab}}$ to be the quotient of $G$
by the closure of its commutator subgroup. Is $G^{\mathrm{ab}}=\varprojlim
G_{i}^{\mathrm{ab}}$ where the $G_{i}$ range over the finite quotients of
$G$.
%Barnea, Ilan; Shelah, Saharon The abelianization of inverse limits of groups. Israel J. Math. 227 (2018), no. 1, 455--483.

\end{exercise}

\clearpage


\chapter{The Galois theory of \'{e}tale algebras}

For Grothendieck, the classification of field extensions by Galois groups, and
the classification of covering spaces by fundamental groups, are two aspects
of the same theory. In this chapter, we re-interprete classical Galois theory
from Grothendieck's point of view. We assume the reader is familiar with the
language of category theory (Wikipedia: category theory; equivalence of categories).

Throughout, $F$ is a field, all rings and $F$-algebras are commutative, and
unadorned tensor products are over $F$. An $F$-algebra $A$ is finite if it is
finitely generated as an $F$-module.

\section{Review of commutative algebra}

We'll need the following standard results from commutative algebra.

Two ideals $I$ and $J$ in a ring $A$ are said to be%
\index{relatively prime}
\emph{relatively prime} if $I+J=A$. For example, any two distinct maximal
ideals in $A$ are relatively prime.{}

\begin{theorem}
[Chinese Remainder Theorem]\label{ca0}
\index{theorem!Chinese remainder}%
Let $I_{1},\ldots,I_{n}$ be ideals in a ring $A$. If $I_{i}$ is relatively
prime to $I_{j}$ whenever $i\neq j$, then the map
\begin{equation}
a\mapsto(\ldots,a+I_{i},\ldots)\colon A\rightarrow A/I_{1}\times\cdots\times
A/I_{n} \label{caq1}%
\end{equation}
is surjective with kernel $\prod I_{i}$ (so $\prod I_{i}=\bigcap I_{i}$).
\end{theorem}

\begin{proof}
CA 2.13.
\end{proof}

\begin{theorem}
[Strong Nullstellensatz]\label{ca1}%
\index{theorem!strong Nullstellensatz}%
Let $I$ be an ideal in the polynomial ring $F[X_{1},\ldots,X_{n}]$ and let
$Z(I)$ denote the set of zeros of $I$ in $(F^{\mathrm{al}})^{n}$. If a
polynomial $h\in F[X_{1},\ldots,X_{n}]$ vanishes on $Z(I\mathfrak{)}$, then
some power of it lies in $I$.
\end{theorem}

\begin{proof}
CA 13.10.
\end{proof}

The \emph{radical} of an ideal $I$ in a ring $A$ is the set of $f\in A$ such
that $f^{n}\in I$ for some $n\in\mathbb{N}{}$. It is again an ideal, and it is
equal to its own radical.

The \emph{nilradical} $N$ of $A$ is the radical of the ideal $(0)$. It
consists of the nilpotents in $A$. If $N=0$, then $A$ is said to be
\emph{reduced}.

\begin{proposition}
\label{ca2}Let $A$ be a finitely generated $F$-algebra, and let $I$ be an
ideal in $A$. The radical of $I$ is equal to the intersection of the maximal
ideals containing it,
\[
\mathrm{rad}(I)=\bigcap\{M\mid M\supset I\text{, }M\text{ maximal}\}.
\]
In particular, $A$ is reduced if and only if $\bigcap\{M\mid M$ maximal$\}=0$.
\end{proposition}

\begin{proof}
Because of the correspondence between ideals in a ring and in a quotient of
the ring, it suffices to prove this for $A=F[X_{1},\ldots,X_{n}]$.

The inclusion $\mathrm{rad}(I)\subset\bigcap\{M\mid M\supset I$, $M$
maximal$\}$ holds in any ring (because maximal ideals are radical and
$\mathrm{rad}(I)$ is the smallest radical ideal containing $I$).

For the reverse inclusion, let $h$ lie in all maximal ideals containing $I$,
and let $(a_{1},\ldots,a_{n})\in Z(I\mathfrak{)}{}$. The image of the
evaluation map
\[
f\mapsto f(a_{1},\ldots,a_{n})\colon F[X_{1},\ldots,X_{n}]\rightarrow
F^{\mathrm{al}}%
\]
is a subring of $F^{\mathrm{al}}$ which is algebraic over $F$, and hence is a
field (see \ref{ef20}a). Therefore, the kernel of the map is a maximal ideal,
which contains $I$, and hence also $h$. This shows that $h(a_{1},\ldots
,a_{n})=0$, and we conclude from the strong Nullstellensatz that
$h\in\mathrm{\mathrm{rad}}(I)$.
\end{proof}

\section{\'{E}tale algebras over a field}

Let $F^{n}=F\times\cdots\times F$ ($n$-copies) regarded as an $F$-algebra by
the diagonal map.

\begin{definition}
\label{B65}An $F$-algebra $A$ is%
\index{algebra!diagonalizable}
\emph{diagonalizable} if it is isomorphic to $F^{n}$ for some $n$, and it is%
\index{algebra!etale@\'{e}tale}
\emph{\'{e}tale} if $L\otimes A$ is diagonalizable for some field $L$
containing $F$.\footnote{This is Bourbaki's terminology} The \emph{degree}%
\index{degree!of an algebra}
$[A\colon F]$ of a finite $F$-algebra $A$ is its dimension as an $F$-vector space.
\end{definition}

Let $A$ be a finite $F$-algebra. For any finite set $S$ of maximal ideals in
$A$, the Chinese remainder theorem (\ref{ca0}) shows that the map
$A\rightarrow\prod\nolimits_{M{}\in S}A/M{}$ is surjective with kernel
$\bigcap\nolimits_{M{}\in S}M{}$. In particular, $\left\vert S\right\vert
\leq\lbrack A\colon F]$, and so $A$ has only finitely many maximal ideals. If
$S$ is the set of all maximal ideals in $A$, then $\bigcap\nolimits_{M{}\in
S}M$ is the nilradical $N$ of $A$ (\ref{ca2}), and so $A/N{}$ is a finite
product of fields.

\begin{proposition}
\label{B66}The following conditions on a finite $F$-algebra $A$ are equivalent:

\begin{enumerate}
\item $A$ is \'{e}tale;

\item $L\otimes A$ is reduced for all fields $L$ containing $F;$

\item $A$ is a product of separable field extensions of $F$.
\end{enumerate}
\end{proposition}

\begin{proof}
(a)$\Rightarrow$(b). Let $L$ be a field containing $F$. By hypothesis, there
exists a field $L^{\prime}$ containing $F$ such that $L^{\prime}\otimes A$ is
diagonalizable. Let $L^{\prime\prime}$ be a field containing (copies of) both
$L$ and $L^{\prime}$ (e.g., take $L^{\prime\prime}$ to be a quotient of
$L\otimes L^{\prime}$ by a maximal ideal). Then $L^{\prime\prime}\otimes
A=L^{\prime\prime}\otimes_{L^{\prime}}L^{\prime}\otimes A$ is diagonalizable,
and the map $L\otimes A\rightarrow L^{\prime\prime}\otimes A$ defined by the
inclusion $L\rightarrow L^{\prime\prime}$ is injective, and so $L\otimes A$ is reduced.

(b)$\Rightarrow$(c). In particular, $A=A\otimes F$ is reduced, and so it is a
finite product of fields (see the above discussion). Suppose that one of the
factor fields $F^{\prime}$ of $A$ is not separable over $F$. Then $F$ has
characteristic $p\neq0$ and there exists an element $u$ of $F^{\prime}$ whose
minimal polynomial is of the form $g(X^{p})$ with $g\in F[X]$ (see \ref{ft10m}
\textit{et seq.}). Let $L$ be a field containing $F$. Then%
\[
L\otimes F[u]\simeq L\otimes(F[X]/(g(X^{p}))\simeq L[X]/(g(X^{p}))\text{.}%
\]
If $L$ is chosen so that the coefficients of $g(X)$ become $p$th powers in it,
then $g(X^{p})$ is a $p$th power in $L[X]$ (see the proof of \ref{ft5}), and
so $L\otimes F[u]$ is not reduced. But $L\otimes F[u]\subset L\otimes A$, and
so this contradicts the hypothesis.

(c)$\Rightarrow$(a). We may suppose that $A$ itself is a separable field
extension of $F$. From the primitive element theorem (\ref{ag1}), we know that
$A=F[u]$ for some $u$. Because $F[u]$ is separable over $F$, the minimal
polynomial $f(X)$ of $u$ is separable, which means that
\[
f(X)=\prod(X-u_{i}),\quad u_{i}\neq u_{j}\text{ for }i\neq j,
\]
in a splitting field $L$ for $f$. Now%
\[
L\otimes A\simeq L\otimes F[X]/(f)\simeq L[X]/(f)\text{,}%
\]
and, according to the Chinese remainder theorem (\ref{ca0}),%
\[
L[X]/(f)\simeq\prod\nolimits_{i}L[X]/(X-u_{i})\simeq L\times\cdots\times
L\text{.}%
\]

\end{proof}

\begin{corollary}
\label{B67}An $F$-algebra $A$ is \'{e}tale if and only if $F^{\mathrm{sep}%
}\otimes A$ is diagonalizable.
\end{corollary}

\begin{proof}
The proof that (c) implies (a) in (\ref{B66}) shows that $L\otimes A$ is
diagonalizable if certain separable polynomials split in $L$. By definition,
all separable polynomials split in $F^{\mathrm{sep}}$.
\end{proof}

\begin{example}
\label{ag41}Let $f\in F[X]$, and let $A=F[X]/(f)$. Let $f=\prod f_{i}^{m_{i}}$
with the $f_{i}$ irreducible and distinct. According to the Chinese remainder
theorem (CA 2.13)%
\[
A\simeq\prod\nolimits_{i}F[X]/(f_{i}^{m_{i}}).
\]
The $F$-algebra $F[X]/(f_{i}^{m_{i}})$ is a field if and only if $m_{i}=1$, in
which case it is a separable extension of $F$ if and only if $f_{i}$ is
separable. Therefore $A$ is an \'{e}tale $F$-algebra if and only if $f$ is a
separable polynomial.

Thus, $F[X]/(f)$ is \'{e}tale if $f\in F[X]$ is separable, but not all
\'{e}tale $F$-algebras are of this form; for example, $F[X]/(f)\times
F[X]/(f)$ is not.
\end{example}

\begin{proposition}
\label{B68}Finite products, tensor products, and quotients of diagonalizable
(resp.\ \'{e}tale) $F$-algebras are diagonalizable (resp.\ \'{e}tale).
\end{proposition}

\begin{proof}
This is obvious for diagonalizable algebras, and it follows for \'{e}tale algebras.
\end{proof}

\begin{corollary}
\label{B69}The composite of any finite set of \'{e}tale $F$-subalgebras of an
$F$-algebra is \'{e}tale.
\end{corollary}

\begin{proof}
Let $A$ be an $F$-algebra, and, for $i=1,\ldots,n$, let $A_{i}$ be an
\'{e}tale subalgebra of $A$. The composite $A_{1}\cdots A_{n}$ of the $A_{i}$
(i.e., the smallest $F$-subalgebra containing the $A_{i}$) is the image of the
map
\[
a_{1}\otimes\cdots\otimes a_{n}\mapsto a_{1}\cdots a_{n}\colon A_{1}%
\otimes\cdots\otimes A_{n}\rightarrow A,
\]
which is a quotient of $A_{1}\otimes\cdots\otimes A_{n}$.
\end{proof}

\begin{proposition}
\label{B70}If $A$ is an \'{e}tale $F$-algebra, then $F^{\prime}\otimes A$ is
an \'{e}tale $F^{\prime}$-algebra for any extension $F^{\prime}$ of $F$.
\end{proposition}

\begin{proof}
Let $L$ be an extension of $F$ such that $L\otimes A\approx L^{m}$, and let
$L^{\prime}$ be a field containing (copies of) both $L$ and $F^{\prime}$.
Then
\[
L^{\prime}\otimes_{F^{\prime}}\left(  F^{\prime}\otimes A\right)  \simeq
L^{\prime}\otimes A\simeq L^{\prime}\otimes_{L}(L\otimes A)\approx L^{\prime
}\otimes_{L}L^{m}\simeq\left(  L^{\prime}\right)  ^{m}\text{.}%
\]

\end{proof}

\begin{remark}
\label{b70}Let $A$ be an \'{e}tale algebra over $F$, and write $A$ as a
product of fields, $A=\prod\nolimits_{i}A_{i}$. A generator $\alpha$ for $A$
as an $F$-algebra is a tuple $(\alpha_{i})$ with each $\alpha_{i}$ a generator
for $A_{i}$ as an $F$-algebra. Because each $A_{i}$ is separable over $F$,
such an $\alpha$ exists (primitive element theorem \ref{ag1}). Choose an
$\alpha$, and let $f=\prod\nolimits_{i}f_{i}$ be the product of the minimal
polynomials of the $\alpha_{i}$. Then $f$ is a monic polynomial whose
irreducible factors are separable.

Conversely, let $f$ be a monic polynomial whose irreducible factors
$(f_{i})_{i}$ are separable. Then $A\overset{\df}{=}%
\prod\nolimits_{i}F[X]/(f_{i})$ is an \'{e}tale algebra over $F$ with a
canonical generator.

In this way, we get a one-to-one correspondence between the set of isomorphism
classes of pairs $(A,\alpha)$ consisting of an \'{e}tale $F$-algebra and a
generator and the set of monic polynomials whose irreducible factors are separable.
\end{remark}

\begin{plain}
\label{b71}In preparation for the next section, we review a little linear
algebra. Let $\Omega$ be a Galois extension of $F$ (possibly infinite) with
Galois group $G$. Let $V$ be a vector space over $F$, and let $V_{\Omega
}=\Omega\otimes_{F}V$. Then $G$ acts on $V_{\Omega}$ through its action on
$\Omega$, and the map
\[
v\mapsto1\otimes v\colon V\rightarrow(V_{\Omega})^{G}%
\overset{\df}{=}\{v\in V_{\Omega}\mid\sigma v=v\text{ for all
}\sigma\in G\}
\]
is an isomorphism. To see this, choose an $F$-basis $\{e_{1},\ldots,e_{n}\}$
for $V$. Then $\{e_{1},\ldots,e_{n}\}$ is also an $\Omega$-basis
for$V_{\Omega}$, and%
\[
\sigma(a_{1}e_{1}+\cdots+a_{n}e_{n})=(\sigma a_{1})e_{1}+\cdots+(\sigma
a_{n})e_{n},\quad a_{i}\in\Omega.
\]
Therefore $a_{1}e_{1}+\cdots+a_{n}e_{n}$ is fixed by all $\sigma\in G$ if and
only if $a_{1},\ldots,a_{n}\in F$.

Similarly, if $W$ is a second vector space over $F$, then $G$ acts on
$\Hom_{\Omega\text{-linear}}(V_{\Omega},W_{\Omega})$ by $\sigma\alpha
=\sigma\circ\alpha\circ\sigma^{-1}$, and
\[
\Hom_{F\text{-linear}}(V,W)\simeq\Hom_{\Omega\text{-linear}}(V_{\Omega
},W_{\Omega})^{G}.
\]
Again, this can be proved by choosing bases.
\end{plain}

\section{Classification of \'{e}tale algebras over a field}

We fix a separable closure $\Omega$ of $F$, and let $G=\Gal(\Omega/F)$. Recall
(Chapter 7) that for every subfield $E$ of $\Omega$ finite and Galois over
$F$, the homomorphism%
\[
\sigma\mapsto\sigma|E\colon G\rightarrow\Gal(E/F)
\]
is surjective, and its kernel is an open normal subgroup of $G$. Every open
normal subgroup of $G$ is of this form, and $G=\varprojlim\Gal(E/F)$.

By a
\index{Gset@$G$-set}%
$G$\emph{-set} we mean a set $S$ equipped with an action of $G$ such that the
map%
\[
G\times S\rightarrow S
\]
is continuous with respect to the Krull topology on $G$ and the discrete
topology on $S$. This is equivalent to saying that the stabilizer of every
point of $S$ is an \textit{open} subgroup of $G$. When $S$ is finite, it is
equivalent to saying that the action factors through $G\rightarrow\Gal(E/F)$
for some subfield $E$ of $\Omega$ finite and Galois over $F$.

\subsection{The functor $\mathcal{F}$}

For an \'{e}tale $F$-algebra $A$, let $\mathcal{F}{}(A)$ denote the set of
$F$-algebra homomorphisms $f\colon A\rightarrow\Omega$. We let $G$ act on
$\mathcal{F}{}(A)$ through its action on $\Omega$,%
\[
(\sigma f)(a)=\sigma(f(a)),\quad\sigma\in G\text{, }f\in\mathcal{F}%
{}(A)\text{, }a\in A,
\]
For some finite Galois extension $E$ of $F$ in $\Omega$, the images of all
homomorphism $A\rightarrow\Omega$ are contained in $E$,\footnote{Write
$A=F_{1}\times\cdots\times F_{n}$ with each $F_{i}$ a field; embed each
$F_{i}$ in $\Omega$, take its Galois closure, and then take the composite of
the fields obtained.} and so the action of $G$ on $\mathcal{F}(A)$ factors
through $\Gal(E/F)$. Therefore $\mathcal{F}{}(A)$ is a $G$-set.

\begin{plain}
\label{b74}Let $A=F[X]/(f)$ where $f$ is a separable polynomial in $F[X]$, and
let $F[X]/(f)=F[x]$. For every homomorphism $\varphi\colon A\rightarrow\Omega$
of $F$-algebras, $\varphi(x)$ is a root of $f(X)$ in $\Omega$, and the map
$\varphi\mapsto\varphi(x)$ defines a one-to-onc correspondence
\[
\mathcal{F}{}(A)\leftrightarrow\{\text{roots of }f(X)\text{ in }\Omega\}
\]
commuting with the actions of $G$. This is obvious (cf.~\ref{sf1}).
\end{plain}

\begin{plain}
\label{b75}Let $A=A_{1}\times\cdots\times A_{n}$ with each $A_{i}$ an
\'{e}tale $F$-algebra. Because $\Omega$ is an integral domain, every
homomorphism $f\colon A\rightarrow\Omega$ is zero on all but one $A_{i}$, and
so, to give a homomorphism $A\rightarrow\Omega$ amounts to giving a
homomorphism $A_{i}\rightarrow\Omega$ for some $i$. In other words,%
\[
\mathcal{F}{}(\tstyle\prod\nolimits_{i}A_{i})\simeq\bigsqcup\nolimits_{i}%
\mathcal{F}{}(A_{i})\quad\text{(disjoint sum).}%
\]
In particular, for an \'{e}tale $F$-algebra $A=\prod\nolimits_{i}F_{i},$
$F_{i}$ a field,%
\[
\mathcal{F}{}(A)\simeq\bigsqcup\nolimits_{i}\Hom_{F}(F_{i},\Omega)\text{.}%
\]
From Proposition \ref{sf7}, we deduce that $\mathcal{F}{}(A)$ is finite of
order $[A\colon F]$.
\end{plain}

Thus, $\mathcal{F}{}$ is a functor from \'{e}tale $F$-algebras to finite $G$-sets.

\subsection{The functor $\mathcal{A}{}$}

For a $G$-set $S$, we let $G$ act on the $F$-algebra $\Omega^{S}$ of maps
$S\rightarrow\Omega$ through its actions on $S$ and $\Omega,$%
\[
(\sigma f)(s)=\sigma(f(\sigma^{-1}s)),\quad\sigma\in G\text{, }f\in\Omega
^{S}\text{, }s\in S,
\]
We define $\mathcal{A}{}(S)$ to be the set of elements of $\Omega^{S}$ fixed
by $G$. Thus $\mathcal{A}{}(S)$ is the $F$-subalgebra of $\Omega^{S}$
consisting of the maps $f\colon S\rightarrow\Omega$ such that $f(\sigma
s)=\sigma f(s)$ for all $\sigma\in G$, $s\in S$.

\begin{plain}
\label{b76}Suppose that $G$ acts transitively on $S$. Choose an $s\in S$, and
let $H\subset G$ be its stabilizer. Then $H$ is an open subgroup of $G$, and
so $E=\Omega^{H}$ is a finite extension of $F$ (\ref{ig6}). An element $f$ of
$\mathcal{A}{}(S)$ is determined by its value on $s$, which can be any element
of $\Omega$ fixed by $H$. It follows that the map
\[
f\mapsto f(s)\colon\mathcal{A}{}(S)\rightarrow E
\]
is an isomorphism of $F$-algebras.

Every element of $S$ is of the form $\sigma s$ with $\sigma\in G$, and $\sigma
s=\sigma^{\prime}s$ if and only if $\sigma H=\sigma^{\prime}H$. Similarly,
every element of $\mathcal{F}{}(E)$ is of the form $\sigma|E$ with $\sigma\in
G$, and $\sigma|E=\sigma|E^{\prime}$ if and only if $\sigma H=\sigma^{\prime
}H$. It follows that the map
\[
\sigma s\mapsto\sigma|E\colon S\rightarrow\mathcal{F}(E)
\]
is an isomorphism of $G$-sets.

Let $E$ be a finite separable extension $E$ of $F$. Let $S=\mathcal{F}{}(E)$
and choose an $s\in S$, i.e., an embedding $s\colon E\hookrightarrow\Omega$.
The above calculation shows that $\mathcal{A}{}(S)=sE$. In particular, $s$
defines an isomorphism $E\rightarrow\mathcal{A}{}(\mathcal{F}{}(E))$.
\end{plain}

\begin{proposition}
\label{b77}Let $S$ be a finite $G$-set, and let $S=S_{1}\sqcup\ldots\sqcup
S_{n}$ be the decomposition of $S$ into its $G$-orbits. For each $i$, choose
an $s_{i}\in S_{i}$, and let $F_{i}$ be the subfield of $\Omega$ fixed by the
stabilizer of $s_{i}$.

\begin{enumerate}
\item Each $F_{i}$ is a finite separable extension of $F$.

\item The map%
\[
f\mapsto(f(s_{1}),\ldots,f(s_{n}))\colon\mathcal{A}{}(S)\rightarrow
F_{1}\times\cdots\times F_{n}%
\]
is an isomorphism of $F$-algebras.

\item The map sending $\sigma s_{i}\in S_{i}\subset S$ to $\sigma|F_{i}%
\in\mathcal{F}{}(F_{i})\subset\mathcal{F}{}(F_{1}\times\cdots\times F_{n})$ is
an isomorphism of $G$-sets%
\[
S\rightarrow\mathcal{F}{}\left(  F_{1}\times\cdots\times F_{n}\right)
\text{.}%
\]

\end{enumerate}
\end{proposition}

\begin{proof}
This follows easily from the special case considered in (\ref{b76})
\end{proof}

\begin{proposition}
\label{b78}For every finite $G$-set $S$, the $F$-algebra $\mathcal{A}{}(S)$ is
\'{e}tale with degree equal to $|S|$. Moreover, every \'{e}tale $F$-algebra
$A$ is of the form $\mathcal{A}{}(S)$ for some $G$-set $S$. More precisely,%
\[
A\simeq\mathcal{A}{}(\mathcal{F}{}(A)).
\]

\end{proposition}

\begin{proof}
The first statement follows from (\ref{B66}) and (\ref{b77}). We prove the
third statement. There is a canonical isomorphism of $\Omega$-algebras%
\[
a\otimes c\mapsto(\sigma a\cdot c)_{\sigma\in\mathcal{F}{}(A)}\colon
\Omega\otimes A\rightarrow\Omega^{\mathcal{F}{}(A)}.
\]
When we let $G$ act on $\Omega\otimes A$ through $\Omega$, and pass to the
fixed elements, we obtain an isomorphism%
\[
A\overset{\text{\ref{b71}}}{=}(\Omega\otimes A)^{G}\simeq\mathcal{A}%
{}(\mathcal{F}{}(A)).
\]
This implies the second statement of the proposition (which can also be
deduced from \ref{b77}).
\end{proof}

\begin{proposition}
\label{b79}Let $S$ be a finite $G$-set. An element $s\in S$ defines a
homomorphism of $F$-algebras $f\mapsto f(s)\colon\mathcal{A}{}(S)\rightarrow
\Omega$, and every homomorphism of $F$-algebras $\mathcal{A}{}(S)\rightarrow
\Omega$ is of this form for a unique $s$. Thus $S\simeq\mathcal{F}%
{}(\mathcal{A}{}(S))$.
\end{proposition}

\begin{proof}
We leave this as an exercise.
\end{proof}

\begin{proposition}
\label{b81}For all \'{e}tale $F$-algebras $A$ and $B$, the map%
\[
\Hom_{F\text{-algebras}}(A,B)\rightarrow\Hom_{G\text{-sets}}(\mathcal{F}%
{}(B),\mathcal{F}{}(A))
\]
defined by $\mathcal{F}{}$ is bijective.
\end{proposition}

\begin{proof}
Let $A$ and $B$ be \'{e}tale $F$-algebras. Under the isomorphism%
\[
\Hom_{F\text{-linear}}(A,B)\overset{\text{\ref{b71}}}{\simeq}\Hom_{\Omega
\text{-linear}}(A_{\Omega},B_{\Omega})^{G},
\]
$F$-algebra homomorphisms correspond to $\Omega$-algebra homomorphisms, and so%
\[
\Hom_{F\text{-algebra}}(A,B)\simeq\Hom_{\Omega\text{-algebra}}(A_{\Omega
},B_{\Omega})^{G}\text{.}%
\]
From (\ref{B67}), we know that $A_{\Omega}$ (resp.\ $B_{\Omega}$) is a product
of copies of $\Omega$ indexed by the elements of $\mathcal{F}{}(A)$ (resp.
$\mathcal{F}{}(B)$). Let $t$ be a map of sets $\mathcal{F}{}(B)\rightarrow
\mathcal{F}{}(A)$. Then
\[
(a_{i})_{i\in\mathcal{F}{}(A)}\mapsto(b_{j})_{j\in\mathcal{F}{}(B)},\quad
b_{j}=a_{t(j)},
\]
is a homomorphism of $\Omega$-algebras $A_{\Omega}\rightarrow B_{\Omega}$, and
every homomorphism of $\Omega$-algebras $A_{\Omega}\rightarrow B_{\Omega}$ is
of this form for a unique $t$. Thus%
\[
\Hom_{\Omega\text{-algebra}}(A_{\Omega},B_{\Omega})\simeq\Hom_{\text{Sets}%
}(\mathcal{F}(B),\mathcal{F}{}(A)).
\]
This isomorphism is compatible with the actions of $G$, and so%
\[
\Hom_{\Omega\text{-algebra}}(A_{\Omega},B_{\Omega})^{G}\simeq\Hom_{\text{Sets}%
}(\mathcal{F}(B),\mathcal{F}{}(A))^{G}.
\]
In other words,%
\[
\Hom_{F\text{-algebra}}(A,B)\simeq\Hom_{G\text{-sets}}(\mathcal{F}%
{}(B),\mathcal{F}{}(A)).
\]

\end{proof}

\begin{theorem}
\label{B71}The functor $A\rightsquigarrow\mathcal{F}{}(A)$ is a contravariant
equivalence from the category of \'{e}tale $F$-algebras to the category of
finite $G$-sets with quasi-inverse $\mathcal{A}{}$.
\end{theorem}

\begin{proof}
This summarizes the results in the last three propositions.
\end{proof}

\subsection{Variant of Theorem \ref{B71}}

Let $\Omega$ be a Galois extension of $F$ (finite or infinite), and let
$G=\Gal(\Omega/F)$. An \'{e}tale $F$-algebra $A$ is \emph{split} by $\Omega$
if $\Omega\otimes A$ is isomorphic to a product of copies of $\Omega$. For
such an $F$-algebra, let $\mathcal{F}{}(A)=\Hom_{k\text{-algebra}}(A,\Omega)$.

\begin{theorem}
\label{B71a}The functor $A\rightsquigarrow\mathcal{F}{}(A)$ is a contravariant
equivalence from the category of \'{e}tale $F$-algebras split by $\Omega$ to
the category of finite $G$-sets.
\end{theorem}

The proof is the same as that of Theorem \ref{B71}. When $\Omega$ is a finite
extension of $F$, the continuity condition for $G$-sets can be omitted.

\subsection{Geometric re-statement of Theorem \ref{B71}}

In this subsection, we assume that the reader is familiar with the notion of
an algebraic variety over a field $F$ (geometrically reduced separated scheme
of finite type over $F$). The functor $A\rightsquigarrow\Spec(A)$ is a
contravariant equivalence from the category of \'{e}tale algebras over $F$ to
the category of zero-dimensional algebraic varieties over $F$. In particular,
all zero-dimensional algebraic varieties are affine. If $V=\Spec(A)$, then
\[
\Hom_{F\text{-algebra}}(A,\Omega)\simeq\Hom_{\Spec(F)}(\Spec(\Omega
),V)\overset{\df}{=}V(\Omega)
\]
(set of points of $V$ with coordinates in $\Omega$).

\begin{theorem}
\label{B72}The functor $V\rightsquigarrow V(\Omega)$ is an equivalence from
the category of zero-dimensional algebraic varieties over $F$ to the category
of finite continuous $G$-sets. Under this equivalence, connected varieties
correspond to sets with a transitive action.
\end{theorem}

\begin{proof}
Combine Theorem \ref{B71} with the equivalence $A\rightsquigarrow\Spec(A)$.
\end{proof}

\section{Comparison with the theory of covering spaces.}

The reader should note the similarity of (\ref{B71}) and (\ref{B72}) with the
following statement:\bquote Let $F$ be a connected and locally simply
connected topological space, and let $\pi\colon\Omega\rightarrow F$ be a
universal covering space of $F$. Let $G$ denote the group of covering
transformations of $\Omega/F$ (the choice of a point $e\in\Omega$ determines
an isomorphism of $G$ with the fundamental group $\pi_{1}(F,\pi e)$). For a
covering space $E$ of $F$, let $\mathcal{F}{}(E)$ denote the set of covering
maps $\Omega\rightarrow E$. Then $E\rightsquigarrow\mathcal{F}{}(E)$ is an
equivalence from the category of covering spaces of $F$ to the category of
(right) $G$-sets.\equote For more on this, see the section on the \'{e}tale
fundamental group in my notes \textit{Lectures on \'{E}tale Cohomology} or
Szamuely, Tam\'{a}s, Galois groups and fundamental groups. CUP, 2009.

\begin{aside}
[for the experts]\label{B73}It is possible to define the \textquotedblleft
absolute Galois group\textquotedblright\ of a field $F$ canonically and
without assuming the axiom of choice. Consider the category of Artin motives
over $F$ (Milne and Deligne 1982, \S 6). This is a Tannakian category
equivalent to the category of sheaves $S$ of $\mathbb{Q}$-vector spaces on
$\Spec(F)_{\mathrm{et}}$ such that $S(A)$ is a finite-dimensional vector space
for all $A$ and the dimension of $S(K)$, $K$ a field, is bounded. Define the
absolute Galois group $\pi$ of $F$ to be the fundamental group of this
category --- this is an affine group scheme in the category (Deligne 1989, Le
groupe fondamental \ldots, \S 6). For any choice of a separable closure
$F^{\mathrm{sep}}$ of $F$, we get a fibre functor $\omega$ on the category and
$\omega(\pi)=\Gal(F^{\mathrm{sep}}/F)$. See Julian Rosen, A choice-free
absolute Galois group and Artin motives, arXiv:1706.06573.
\end{aside}

\chapter{Transcendental Extensions}

In this chapter we consider fields $\Omega\supset F$ with $\Omega$ much bigger
than $F$. For example, we could have $\mathbb{C}\supset\mathbb{Q}.$

\section{Algebraic independence}

Elements $\alpha_{1},...,\alpha_{n}$ of $\Omega$ give rise to an
$F$-homomorphism%
\[
f\mapsto f(\alpha_{1},...,\alpha_{n})\colon F[X_{1},\ldots,X_{n}%
]\rightarrow\Omega\text{.}%
\]
If the kernel of this homomorphism is zero, then the $\alpha_{i}$ are said to
be \emph{algebraically independent\/}\label{ai}%
\index{algebraically independent}
over $F$, and otherwise, they are \emph{algebraically dependent}%
\index{algebraically dependent}%
\emph{ }over $F$. Thus, the $\alpha_{i}$ are algebraically dependent over $F$
if there exists a nonzero polynomial $f(X_{1},...,X_{n})\in F[X_{1}%
,...,X_{n}]$ such that $f(\alpha_{1},...,\alpha_{n})=0$, and they are
algebraically independent if
\[
a_{i_{1},...,i_{n}}\in F,\quad\sum a_{i_{1},...,i_{n}}\alpha_{1}^{i_{1}%
}...\alpha_{n}^{i_{n}}=0\implies a_{i_{1},...,i_{n}}=0\text{\ all }%
i_{1},...,i_{n}.
\]
Note the similarity with linear independence. In fact, if $f$ is required to
be homogeneous of degree 1, then the definition becomes that of linear independence.

\begin{example}
\label{te1}(a) A single element $\alpha$ is algebraically independent over $F$
if and only if it is transcendental over $F.$

(b) The complex numbers $\pi$ and $e$ are almost certainly algebraically
independent over $\mathbb{Q}$, but this has not been proved.
\end{example}

An infinite set $A$ is \emph{algebraically independent\/} over $F$ if every
finite subset of $A$ is algebraically independent; otherwise, it is
\emph{algebraically dependent} over $F$.

\begin{remark}
\label{te2}If $\alpha_{1},...,\alpha_{n}$ are algebraically independent over
$F$, then the map
\[
f(X_{1},...,X_{n})\mapsto f(\alpha_{1},...,\alpha_{n})\colon F[X_{1}%
,...,X_{n}]\rightarrow F[\alpha_{1},...,\alpha_{n}]
\]
is an injection, and hence an isomorphism. This isomorphism then extends to
the fields of fractions,
\[
X_{i}\mapsto\alpha_{i}\colon F(X_{1},...,X_{n})\rightarrow F(\alpha
_{1},...,\alpha_{n})
\]
In this case, $F(\alpha_{1},...,\alpha_{n})$ is called a \emph{pure
transcendental } \emph{extension\/} of $F$. The polynomial
\[
f(X)=X^{n}-\alpha_{1}X^{n-1}+\cdots+(-1)^{n}\alpha_{n}%
\]
has Galois group $S_{n}$ over $F(\alpha_{1},...,\alpha_{n})$ (see \ref{ag28}).
\end{remark}

\begin{lemma}
\label{te2m}Let $\gamma\in\Omega$ and let $A\subset\Omega$. The following
conditions are equivalent:

\begin{enumerate}
\item $\gamma$ is algebraic over $F(A)$;

\item there exist $\beta_{1},\ldots,\beta_{n}\in F(A)$ such that $\gamma
^{n}+\beta_{1}\gamma^{n-1}+\cdots+\beta_{n}=0$;

\item there exist $\beta_{0},\beta_{1},\ldots,\beta_{n}\in F[A]$, not all $0$,
such that $\beta_{0}\gamma^{n}+\beta_{1}\gamma^{n-1}+\cdots+\beta_{n}=0$;

\item there exists an $f(X_{1},\ldots,X_{m},Y)\in F[X_{1}\ldots,X_{m},Y]$ and
$\alpha_{1},\ldots,\alpha_{m}\in A$ such that $f(\alpha_{1},\ldots,\alpha
_{m},Y)\neq0$ but $f(\alpha_{1},\ldots,\alpha_{m},\gamma)=0$.
\end{enumerate}
\end{lemma}

\begin{proof}
(a)$\implies$(b)$\implies$(c)$\implies$(a) are obvious.

(d)$\implies$(c). Write $f(X_{1},\ldots,X_{m},Y)$ as a polynomial in $Y$ with
coefficients in the ring $F[X_{1},\ldots,X_{m}]$,%
\[
f(X_{1},\ldots,X_{m},Y)=%
%TCIMACRO{\tsum }%
%BeginExpansion
{\textstyle\sum}
%EndExpansion
f_{i}(X_{1},\ldots,X_{m})Y^{n-i}\text{.}%
\]
Then (c) holds with $\beta_{i}=f_{i}(\alpha_{1},\ldots,\alpha_{m})$.

(c)$\implies$(d). The $\beta_{i}$ in (c) can be expressed as polynomials in a
finite number of elements $\alpha_{1},\ldots,\alpha_{m}$ of $A$, say,
$\beta_{i}=f_{i}(\alpha_{1},\ldots,\alpha_{m})$ with $f_{i}\in F[X_{1}%
,\ldots,X_{m}]$. Then (d) holds with $f=%
%TCIMACRO{\tsum }%
%BeginExpansion
{\textstyle\sum}
%EndExpansion
f_{i}(X_{1},\ldots,X_{m})Y^{n-i}$.
\end{proof}

\begin{definition}
When $\gamma$ satisfies the equivalent conditions of Lemma \ref{te2m}, it is
said to be \emph{algebraically dependent\/} on $A$ (over $F)$. A set $B$ is
\emph{algebraically dependent\/} on $A$ if each element of $B$ is
algebraically dependent on $A$.
\end{definition}

The theory in the remainder of this chapter is logically very similar to a
part of linear algebra. It is useful to keep the following correspondences in
mind:\hfill

\begin{center}%
\begin{tabular}
[c]{|c|c|}\hline
Linear algebra & Transcendence\\\hline
linearly independent & algebraically independent\\\hline
$A\subset\text{span}(B)$ & $A$ algebraically dependent on $B$\\\hline
basis & transcendence basis\\\hline
dimension & transcendence degree\\\hline
\end{tabular}



\end{center}

\section{Transcendence bases}

\begin{theorem}
[Fundamental result]\label{te3} Let $A=\{\alpha_{1},...,\alpha_{m}\}$ and
$B=\{\beta_{1},...,\beta_{n}\}$ be two subsets of $\Omega$. Assume

\begin{enumerate}
\item $A$ is algebraically independent (over $F$);

\item $A$ is algebraically dependent on $B$ (over $F$).
\end{enumerate}

\noindent Then $m\leq n$.
\end{theorem}

We first prove two lemmas.

\begin{lemma}
[The exchange property]\label{te4}Let $\{\alpha_{1},...,\alpha_{m}\}$ be a
subset of $\Omega$; if $\beta$ is algebraically dependent on $\{\alpha
_{1},...,\alpha_{m}\}$ but not on $\{\alpha_{1},...,\alpha_{m-1}\}$, then
$\alpha_{m}$ is algebraically dependent on $\{\alpha_{1},...,\alpha
_{m-1},\beta\}.$
\end{lemma}

\begin{proof}
\ Because $\beta$ is algebraically dependent on $\{\alpha_{1},\ldots
,\alpha_{m}\}$, there exists a polynomial $f(X_{1},...,X_{m},Y)$ with
coefficients in $F$ such that
\[
f(\alpha_{1},...,\alpha_{m},Y)\neq0,\quad f(\alpha_{1},...,\alpha_{m}%
,\beta)=0.
\]
Write $f$ as a polynomial in $X_{m}$,
\[
f(X_{1},...,X_{m},Y)=\sum_{i}a_{i}(X_{1},...,X_{m-1},Y)X_{m}^{n-i},
\]
and observe that, because $f(\alpha_{1},\ldots,\alpha_{m},Y)\neq0$, at least
one of the polynomials
\[
a_{i}(\alpha_{1},...,\alpha_{m-1},Y),
\]
say $a_{i_{0}}$, is not the zero polynomial. Because $\beta$ is not
algebraically dependent on
\[
\{\alpha_{1},...,\alpha_{m-1}\},
\]
$a_{i_{0}}(\alpha_{1},...,\alpha_{m-1},\beta)\neq0$. Therefore, $f(\alpha
_{1},...,\alpha_{m-1},X_{m},\beta)\neq0$. Since $f(\alpha_{1},...,\alpha
_{m},\beta)=0$, this shows that $\alpha_{m}$ is algebraically dependent on
$\{\alpha_{1},...,\alpha_{m-1},\beta\}$.
\end{proof}

\begin{lemma}
[Transitivity of algebraic dependence]\label{te5} If $C$ is algebraically
dependent on $B$, and $B$ is algebraically dependent on $A$, then $C$ is
algebraically dependent on $A$.
\end{lemma}

\begin{proof}
The argument in the proof of Proposition \ref{sf10} shows that if $\gamma$ is
algebraic over a field $E$ which is algebraic over a field $F$, then $\gamma$
is algebraic over $F$ (if $a_{1},\ldots,a_{n}$ are the coefficients of the
minimal polynomial of $\gamma$ over $E$, then the field $F[a_{1},\ldots
,a_{n},\gamma]$ has finite degree over $F$). Apply this with $E=F(A\cup B)$
and $F=F(A)$.
\end{proof}

\begin{pf}
[of Theorem \ref{te3}]Let $k$ be the number of elements that $A$ and $B$ have
in common. If $k=m$, then $A\subset B$, and certainly $m\leq n$. Suppose that
$k<{}m$, and write $B=\{\alpha_{1},...,\alpha_{k},\beta_{k+1},...,\beta_{n}%
\}$. Since $\alpha_{k+1}$ is algebraically dependent on $\{\alpha
_{1},...,\alpha_{k},\beta_{k+1},...,\beta_{n}\}$ but not on $\{\alpha
_{1},...,\alpha_{k}\}$, there will be a $\beta_{j}$, $k+1\leq j\leq n$, such
that $\alpha_{k+1}$ is algebraically dependent on $\{\alpha_{1},...,\alpha
_{k},\beta_{k+1},...,\beta_{j}\}$ but not
\[
\{\alpha_{1},...,\alpha_{k},\beta_{k+1},...,\beta_{j-1}\}.
\]
The exchange lemma then shows that $\beta_{j}$ is algebraically dependent on
\[
B_{1}\overset{\df}{=}B\cup\{\alpha_{k+1}\}\smallsetminus
\{\beta_{j}\}.
\]
Therefore $B$ is algebraically dependent on $B_{1}$, and so $A$ is
algebraically dependent on $B_{1}$ (by \ref{te5}). If $k+1<m$, repeat the
argument with $A$ and $B_{1}$. Eventually we'll achieve $k=m$, and $m\leq n.$
\end{pf}

\begin{definition}
\label{te6}A \emph{transcendence basis}%
\index{basis!transcendence}%
\emph{\/} for $\Omega$ over $F$ is an algebraically independent set $A$ such
that $\Omega$ is algebraic over $F(A).$
\end{definition}

\begin{lemma}
\label{te7}If $\Omega$ is algebraic over $F(A)$, and $A$ is minimal among
subsets of $\Omega$ with this property, then it is a transcendence basis for
$\Omega$ over $F$.
\end{lemma}

\begin{proof}
If $A$ is not algebraically independent, then there is an $\alpha\in A$ that
is algebraically dependent on $A\smallsetminus\{\alpha\}$. It follows from
Lemma \ref{te5} that $\Omega$ is algebraic over $F(A\smallsetminus
\{\alpha\}).$
\end{proof}

\begin{theorem}
\label{te8}If there is a finite subset $A\subset\Omega$ such that $\Omega$ is
algebraic over $F(A)$, then $\Omega$ has a finite transcendence basis over
$F$. Moreover, every transcendence basis is finite, and they all have the same
number of elements.
\end{theorem}

\begin{proof}
In fact, every minimal subset $A^{\prime}$ of $A$ such that $\Omega$ is
algebraic over $F(A^{\prime})$ will be a transcendence basis. The second
statement follows from Theorem \ref{te3}.
\end{proof}

\begin{lemma}
\label{te11}Suppose that $A$ is algebraically independent, but that
$A\cup\{\beta\}$ is algebraically dependent. Then $\beta$ is algebraic over
$F(A).$
\end{lemma}

\begin{proof}
The hypothesis is that there exists a nonzero polynomial
\[
f(X_{1},...,X_{n},Y)\in F[X_{1},...,X_{n},Y]
\]
such that $f(\alpha_{1},...,\alpha_{n},\beta)=0$, some distinct $\alpha
_{1},...,\alpha_{n}\in A$. Because $A$ is algebraically independent, $Y$ does
occur in $f$. Therefore
\[
f=g_{0}Y^{m}+g_{1}Y^{m-1}+\cdots+g_{m},\quad g_{i}\in F[X_{1},...,X_{n}],\quad
g_{0}\neq0,\quad m\geq1.
\]
As $g_{0}\neq0$ and the $\alpha_{i}$ are algebraically independent,
$g_{0}(\alpha_{1},...,\alpha_{n})\neq0$. Because $\beta$ is a root of
\[
f=g_{0}(\alpha_{1},...,\alpha_{n})X^{m}+g_{1}(\alpha_{1},...,\alpha
_{n})X^{m-1}+\cdots+g_{m}(\alpha_{1},...,\alpha_{n}),
\]
it is algebraic over $F(\alpha_{1},...,\alpha_{n})\subset F(A).$
\end{proof}

\begin{proposition}
\label{te12}Every maximal algebraically independent subset of $\Omega$ is a
transcendence basis for $\Omega$ over $F$.
\end{proposition}

\begin{proof}
We have to prove that $\Omega$ is algebraic over $F(A)$ if $A$ is maximal
among algebraically independent subsets. But the maximality implies that, for
every $\beta\in\Omega\smallsetminus A$, $A\cup\{\beta\}$ is algebraically
dependent, and so the lemma shows that $\beta$ is algebraic over $F(A)$.
\end{proof}

Recall that (except in \S 7), we use an asterisk to signal a result depending
on Zorn's lemma.

\begin{theorem}
[*]\label{te13}Every algebraically independent subset of $\Omega$ is contained
in a transcendence basis for $\Omega$ over $F$; in particular, transcendence
bases exist.
\end{theorem}

\begin{proof}
Let $S$ be the set of algebraically independent subsets of $\Omega$ containing
the given set. We can partially order it by inclusion. Let $T$ be a totally
ordered subset of $S$, and let $B=\bigcup\{A\mid A\in T\}$. I claim that $B\in
S$, i.e., that $B$ is algebraically independent. If not, there exists a finite
subset $B^{\prime}$ of $B$ that is not algebraically independent. But such a
subset will be contained in one of the sets in $T$, which is a contradiction.
Now Zorn's lemma shows that there exists a maximal algebraically independent
containing $S$, which Proposition \ref{te12} shows to be a transcendence basis
for $\Omega$ over $F$.
\end{proof}

It is possible to show that any two (possibly infinite) transcendence bases
for $\Omega$ over $F$ have the same cardinality. The cardinality of a
transcendence basis for $\Omega$ over $F$ is called the%
\index{transcendence degree}
\emph{transcendence degree\/} of $\Omega$ over $F$. For example, the pure
transcendental extension $F(X_{1},\ldots,X_{n})$ has transcendence degree $n$
over $F$.

\begin{example}
\label{te9}Let $p_{1},\ldots,p_{n}$ be the elementary symmetric polynomials in
$X_{1},\ldots,X_{n}$. The field $F(X_{1},\ldots,X_{n})$ is algebraic over
$F(p_{1},\ldots,p_{n})$, and so $\{p_{1},p_{2},\ldots,p_{n}\}$ contains a
transcendence basis for $F(X_{1},\ldots,X_{n})$. Because $F(X_{1},\ldots
,X_{n})$ has transcendence degree $n$, the $p_{i}$'s must themselves be a
transcendence basis.
\end{example}

\begin{example}
\label{te10}Let $\Omega$ be the field of meromorphic functions on a compact
complex manifold $M$.

(a) The only meromorphic functions on the Riemann sphere are the rational
functions in $z$. Hence, in this case, $\Omega$ is a pure transcendental
extension of $\mathbb{C}$ of transcendence degree $1$.

(b) If $M$ is a Riemann surface, then the transcendence degree of $\Omega$
over $\mathbb{C}$ is $1$, and $\Omega$ is a pure transcendental extension of
$\mathbb{C}$ $\iff$ $M$ is isomorphic to the Riemann sphere

(c) If $M$ has complex dimension $n$, then the transcendence degree is $\leq
n$, with equality holding if $M$ is embeddable in some projective space.
\end{example}

\begin{proposition}
\ \label{te14}Any two algebraically closed fields with the same transcendence
degree over $F$ are $F$-isomorphic.
\end{proposition}

\begin{proof}
\ Choose transcendence bases $A$ and $A^{\prime}$ for the two fields. By
assumption, there exists a bijection $A\rightarrow A^{\prime}$, which extends
uniquely to an $F$-isomorphism $F[A]\rightarrow F[A^{\prime}]$, and hence to
an $F$-isomorphism of the fields of fractions $F(A)\rightarrow F(A^{\prime})$.
Use this isomorphism to identify $F(A)$ with $F(A^{\prime})$. Then the two
fields in question are algebraic closures of the same field, and hence are
isomorphic (Theorem \ref{sf16}).
\end{proof}

\begin{remark}
\label{te15}Any two algebraically closed fields with the same uncountable
cardinality and the same characteristic are isomorphic. The idea of the proof
is as follows. Let $F$ and $F^{\prime}$ be the prime subfields of $\Omega$ and
$\Omega^{\prime}$; we can identify $F$ with $F^{\prime}$. Then show that when
$\Omega$ is uncountable, the cardinality of $\Omega$ is the same as the
cardinality of a transcendence basis over $F$. Finally, apply the proposition.
\end{remark}

\begin{remark}
\label{te16}What are the automorphisms of $\mathbb{C}$? There are only two
continuous automorphisms (cf.\ Exercise \ref{x31} and solution). If we assume
Zorn's lemma, then it is easy to construct many: choose any transcendence
basis $A$ for $\mathbb{C}$ over $\mathbb{Q}$, and choose any permutation
$\alpha$ of $A$; then $\alpha$ defines an isomorphism $\mathbb{Q}%
(A)\rightarrow\mathbb{Q}(A)$ that can be extended to an automorphism of
$\mathbb{C}$. Without Zorn's lemma, there are only two, because the
noncontinuous automorphisms are nonmeasurable,\footnote{A fairly elementary
theorem of G. Mackey says that measurable homomorphisms of Lie groups are
continuous (see Theorem B.3, p.\,198 of Zimmer, Robert J., Ergodic theory and
semisimple groups. Birkh\"auser, 1984.)} and it is known that the Zorn's lemma
is required to construct nonmeasurable functions.\footnote{\textquotedblleft
We show that the existence of a non-Lebesgue measurable set cannot be proved
in Zermelo-Frankel set theory (ZF) if use of the axiom of choice is
disallowed...\textquotedblright\ R. Solovay, Ann. of Math., 92 (1970), 1--56.}
\end{remark}

\section{L\"{u}roth's theorem}

\begin{theorem}
[L\"{u}roth]\label{te17}Let $L=F(X)$ with $X$ transcendental over $F$. Every
subfield $E$ of $L$ properly containing $F$ is of the form $E=F(u)$ for some
$u\in L$ transcendental over $F$.
\end{theorem}

We first sketch a geometric proof of L\"{u}roth's theorem. The inclusion of
$E$ into $L$ corresponds to a map from the projective line $\mathbb{P}{}^{1}$
onto a complete regular curve $C$. Now the Riemann-Hurwitz formula shows that
$C$ has genus $0$. Since it has an $F$-rational point (the image of any
$F$-rational point of $\mathbb{P}{}^{1}$), it is isomorphic to $\mathbb{P}%
{}^{1}$. Therefore $E=F(u)$ for some $u\in L$ transcendental over $F$.

Before giving the elementary proof, we review Gauss's lemma and its consequences.

\subsection{Gauss's lemma}

Let $R$ be a unique factorization domain, and let $Q$ be its field of
fractions, for example, $R=F[X]$ and $Q=F(X)$. A polynomial $f(T)=\sum
a_{i}T^{i}$ in $R[T]$ is said to be \emph{primitive}%
\index{polynomial!primitive}
if its coefficients $a_{i}$ have no common factor other than units. Every
polynomial $f$ in $Q[X]$ can be written $f=c(f)\cdot f_{1}$ with $c(f)\in Q$
and $f_{1}$ primitive (write $f=af/a$ with $a$ a common denominator for the
coefficients of $f$, and then write $f=(b/a)f_{1}$ with $b$ the greatest
common divisor of the coefficients of $af$). The element $c(f)$ is uniquely
determined up to a unit, and $f\in R[X]$ if and only if $c(f)\in R$.

\begin{E}
\label{te16a}If $f,g\in R[T]$ are primitive, so also is $fg$.
\end{E}

\noindent Let $f=\sum a_{i}T^{i}$ and $g=\sum b_{i}T^{i}$, and let $p$ be a
prime element of $R$. Because $f$ is primitive, there exists a coefficient
$a_{i}$ not divisible by $p$ --- let $a_{i_{1}}$ be the first such
coefficient. Similarly, let $b_{i_{2}}$ be the first coefficient of $g$ not
divisible by $p$. Then the coefficient of $T^{i_{1}+i_{2}}$ in $fg$ is not
divisible by $p$. This shows that $fg$ is primitive.

\begin{E}
\label{te16b}For any $f,g\in R[T]$, $c(fg)=c(f)c(g)$ and $(fg)_{1}=f_{1}g_{1}$.
\end{E}

\noindent Let $f=c(f)f_{1}$ and $g=c(g)g_{1}$ with $f_{1}$ and $g_{1}$
primitive. Then $fg=c(f)c(g)f_{1}g_{1}$ with $f_{1}g_{1}$ primitive, and so
$c(fg)=c(f)c(g)$ and $(fg)_{1}=f_{1}g_{1}$.

\begin{E}
\label{te16c}Let $f$ be a polynomial in $R[T]$. If $f$ factors into the
product of two nonconstant polynomials in $Q[T]$, then it factors into the
product of two nonconstant polynomials in $R[T]$.
\end{E}

\noindent Suppose that $f=gh$ in $Q[T]$. Then $f_{1}=g_{1}h_{1}$ in $R[T]$,
and so $f=c(f)\cdot f_{1}=$ $(c(f)\cdot g_{1})h_{1}$ with $c(f)\cdot g_{1}$
and $h_{1}$ in $R[T]$.

\begin{E}
\label{te16d}Let $f,g\in R[T]$. If $f$ divides $g$ in $Q[T]$ and $f$ is
primitive, then it divides $g$ in $R[T]$.
\end{E}

\noindent Let $fq=g$ with $q\in Q[T]$. Then $c(q)=c(g)\in R$, and so $q\in
R[T]$.

\subsection{Proof of L\"{u}roth's theorem}

We define the degree $\deg(u)$ of an element $u$ of $F(X)$ to be the larger of
the degrees of the numerator and denominator of $u$ when it is expressed in
its simplest form.

\begin{lemma}
\label{te17a}Let $u\in F(X)\smallsetminus F$. Then $u$ is transcendental over
$F$, $X$ is algebraic over $F(u)$, and $[F(X)\colon F(u)]=\deg(u).$
\end{lemma}

\begin{proof}
Let $u(X)=a(X)/b(X)$ with $a(X)$ and $b(X)$ relatively prime polynomials. Now
$a(T)-b(T)u\in F(u)[T]$, and it has $X$ as a root, and so $X$ is algebraic
over $F(u)$. It follows that $u$ is transcendental over $F$ (otherwise $X$
would be algebraic over $F$; \ref{ef20}b).

The polynomial $a(T)-b(T)Z\in F[Z,T]$ is clearly irreducible. As $u$ is
transcendental over $F$,
\[
F[Z,T]\simeq F[u,T],\quad Z\leftrightarrow u,\quad T\leftrightarrow T,
\]
and so $a(T)-b(T)u$ is irreducible in $F[u,T]$, and hence also in $F(u)[T]$ by
Gauss's lemma (\ref{te16c}). It has $X$ as a root, and so, up to a constant,
it is the minimal polynomial of $X$ over $F(u)$, and its degree is $\deg(u)$,
which proves the lemma.
\end{proof}

\begin{example}
\label{te17b}We have $F(X)=F(u)$ if and if
\[
u=\frac{aX+b}{cX+d}%
\]
with $ac\neq0$ and neither $aX+b$ nor $cX+d$ a constant multiple of the other.
These conditions are equivalent to $ad-bc\neq0.$
\end{example}

We now prove Theorem \ref{te17}. Let $u$ be an element of $E$ not in $F$. Then%
\[
\lbrack F(X)\colon E]\leq\lbrack F(X)\colon F(u)]=\deg(u),
\]
and so $X$ is algebraic over $E$. Let%
\[
f(T)=T^{n}+a_{1}T^{n-1}+\cdots+a_{n},\quad a_{i}\in E,
\]
be its minimal polynomial. As $X$ is transcendental over $F$, some
$a_{j}\notin F$, and we'll show that $E=F(a_{j})$.

Let $d(X)\in F[X]$ be a polynomial of least degree such that $d(X)a_{i}(X)\in
F[X]$ for all $i$, and let%
\[
f_{1}(X,T)=df(T)=dT^{n}+da_{1}T^{n-1}+\cdots+da_{n}\in F[X,T].
\]
Then $f_{1}$ is primitive as a polynomial in $T$, i.e., $\gcd(d,da_{1}%
,\ldots,da_{n})=1$ in $F[X]$. The degree $m$ of $f_{1}$ in $X$ is the largest
degree of one of the polynomials $\,da_{1},\,da_{2},\ldots$, say
$m=\deg(da_{i})$. Write $a_{i}=b/c$ with $b,c$ relatively prime polynomials in
$F[X]$. Now $b(T)-c(T)a_{i}(X)$ is a polynomial in $E[T]$ having $X$ as a
root, and so it is divisible by $f$, say%
\[
f(T)\cdot q(T)=b(T)-c(T)\cdot a_{i}(X),\quad q(T)\in E[T]\text{.}%
\]
On multiplying through by $c(X)$, we find that%
\[
c(X)\cdot f(T)\cdot q(T)=c(X)\cdot b(T)-c(T)\cdot b(X).
\]
As $f_{1}$ differs from $f$ by a nonzero element of $F(X)$, the equation shows
that $f_{1}$ divides $c(X)\cdot b(T)-c(T)\cdot b(X)$ in $F(X)[T]$. But $f_{1}$
is primitive in $F[X][T]$, and so it divides $c(X)\cdot b(T)-c(T)\cdot b(X)$
in $F[X][T]=F[X,T]$ (by \ref{te16d}), i.e., there exists a polynomial $h\in
F[X,T]$ such that%
\begin{equation}
f_{1}(X,T)\cdot h(X,T)=c(X)\cdot b(T)-c(T)\cdot b(X)\text{.} \label{eq3}%
\end{equation}


In (\ref{eq3}), the polynomial $c(X)\cdot b(T)-c(T)\cdot b(X)$ has degree at
most $m$ in $X$, and $m$ is the degree of $f_{1}(X,T)$ in $X$. Therefore,
$c(X)\cdot b(T)-c(T)\cdot b(X)$ has degree exactly $m$ in $X$, and $h(X,T)$
has degree $0$ in $X$, i.e., $h\in F[T]$. It now follows from (\ref{eq3}) that
$c(X)\cdot b(T)-c(T)\cdot b(X)$ is not divisible by a nonconstant polynomial
in $F[X]$.

The polynomial $c(X)\cdot b(T)-c(T)\cdot b(X)$ is symmetric in $X$ and $T$,
i.e., it is unchanged when they are swapped. Therefore, it has degree $m$ in
$T$ and it is not divisible by a nonconstant polynomial in $F[T]$. It now
follows from (\ref{eq3}) that $h$ is not divisible by a nonconstant polynomial
in $F[T]$, and so it lies in $F^{\times}$. We conclude that $f_{1}(X,T)$ is a
constant multiple of $c(X)\cdot b(T)-c(T)\cdot b(X)$.

On comparing degrees in $T$ in (\ref{eq3}), we see that $n=m$. Thus%
\[
\lbrack F(X)\colon F(a_{i})]\overset{\text{\ref{te17a}}}{=}\deg(a_{i})\leq
\deg(da_{i})=m=n=[F(X)\colon E]\leq\lbrack F(X)\colon F(a_{i})].
\]
Hence, equality holds throughout, and so $E=F[a_{i}]$.

Finally, if $a_{j}\notin F$, then%
\[
\lbrack F(X)\colon E]\leq\lbrack F(X)\colon F(a_{j}%
)]\overset{\text{\ref{te17a}}}{=}\deg(a_{j})\leq\deg(da_{j})\leq\deg
(da_{i})=m=[F(X)\colon E],
\]
and so $E=F(a_{j})$ as claimed.

\begin{remark}
\label{te18}L\"{u}roth's theorem fails when there is more than one variable
--- see Zariski's example (footnote to Remark \ref{ag4m}) and Swan's example
(Remark \ref{ag30}). However, the following is true: if $[F(X,Y)\colon
E]<\infty$ and $F$ is algebraically closed of characteristic zero, then $E$ is
a pure transcendental extension of $F$ (Theorem of Zariski, 1958).
\end{remark}

\begin{nt}
L\"{u}roth proved his theorem over $\mathbb{C}{}$ in 1876. For general fields,
it was proved by Steinitz in 1910, by the above argument.
\end{nt}

\section{Separating transcendence bases}

Let $E\supset F$ be fields with $E$ finitely generated over $F$. A subset
$\{x_{1},\ldots,x_{d}\}$ of $E$ is a%
\index{basis!separating transcendence}
\emph{separating transcendence basis }for $E/F$ if it is algebraically
independent over $F$ and $E$ is a finite \textit{separable} extension of
$F(x_{1},\ldots,x_{d})$.

\begin{theorem}
\label{te19}If $F$ is perfect, then every finitely generated extension $E$ of
$F$ admits a separating transcendence basis over $F$.
\end{theorem}

\begin{proof}
If $F$ has characteristic zero, then every transcendence basis is separating,
and so the statement becomes that of (\ref{te8}). Thus, we may assume $F$ has
characteristic $p\neq0$. Because $F$ is perfect, every polynomial in
$X_{1}^{p},\ldots,X_{n}^{p}$ with coefficients in $F$ is a $p$th power in
$F[X_{1},\ldots,X_{n}]$:%
\[
\sum a_{i_{1}\cdots i_{n}}X_{1}^{i_{1}p}\ldots X_{n}^{i_{n}p}=\left(  \sum
a_{i_{1}\cdots i_{n}}^{\frac{1}{p}}X_{1}^{i_{1}}\ldots X_{n}^{i_{n}}\right)
^{p}.
\]


Let $E=F(x_{1},\ldots,x_{n})$, and assume $n>d+1$ where $d$ is the
transcendence degree of $E$ over $F$. After renumbering, we may suppose that
$x_{1},\ldots,x_{d}$ are algebraically independent (\ref{te7}). Then
$f(x_{1},\ldots,x_{d+1})=0$ for some nonzero irreducible polynomial
$f(X_{1},\ldots,X_{d+1})$ with coefficients in $F$. Not all $\partial
f/\partial X_{i}$ are zero, for otherwise $f$ would be a polynomial in
$X_{1}^{p},\ldots,X_{d+1}^{p}$, which implies that it is a $p$th power. After
renumbering $x_{1},\ldots,x_{d+1}$, we may suppose that $\partial f/\partial
X_{d+1}\neq0$. Then $x_{d+1}$ is separably algebraic over $F(x_{1}%
,\ldots,x_{d})$ and $F(x_{1},\ldots,x_{d+1},x_{d+2})$ is algebraic over
$F(x_{1},\ldots,x_{d+1})$, hence over $F(x_{1},\ldots,x_{d})$ (\ref{ef20}),
and so, by the primitive element theorem (\ref{ag1}), there is an element $y$
such that $F(x_{1},\ldots,x_{d+2})=F(x_{1},\ldots,x_{d},y)$. Thus $E$ is
generated by $n-1$ elements (as a field containing $F)$. After repeating the
process, possibly several times, we will have $E=F(z_{1},\ldots,z_{d+1})$ with
$z_{d+1}$ separable over $F(z_{1},\ldots,z_{d})$.
\end{proof}

\begin{aside}
\label{te20}In fact, we showed that $E$ admits a separating transcendence
basis with $d+1$ elements where $d$ is the transcendence degree. This has the
following geometric interpretation: every irreducible algebraic variety of
dimension $d$ over a perfect field $F$ is birationally equivalent with a
hypersurface $H$ in $\mathbb{A}{}^{d+1}$ for which the projection
$(a_{1},\ldots,a_{d+1})\mapsto(a_{1},\ldots,a_{d})$ realizes $F(H)$ as a
finite separable extension of $F(\mathbb{A}{}^{d})$ (see my notes on Algebraic Geometry).
\end{aside}

\section{Transcendental Galois theory}

\begin{theorem}
\label{te21} Let $\Omega$ be an algebraically closed field and let $F$ be a
perfect subfield of $\Omega$. If $\alpha\in\Omega$ is fixed by all
$F$-automorphisms of $\Omega$, then $\alpha\in F$, i.e., $\Omega
^{\Aut(\Omega/F)}=F$.
\end{theorem}

\begin{proof}
Let $\alpha\in\Omega\smallsetminus F$. If $\alpha$ is algebraic over $F$, then
there is an $F$-homomorphism $F[\alpha]\rightarrow\Omega$ sending $\alpha$ to
a conjugate of $\alpha$ in $\Omega$ different from $\alpha$. This homomorphism
extends to a homomorphism from the algebraic closure $F^{\mathrm{al}}$ of $F$
in $\Omega$ to $\Omega$ (by \ref{sf16}). Now choose a transcendence basis $A$
for $\Omega$ over $F^{\mathrm{al}}$. We can extend our homomorphism to a
homomorphism $F(A)\rightarrow\Omega$ by mapping each element of $A$ to itself.
Finally, we can extend this homomorphism to a homomorphism from the algebraic
closure $\Omega$ of $F(A)$ to $\Omega$. The $F$-homomorphism $\Omega
\rightarrow\Omega$ we obtain is automatically an isomorphism (cf.\ \ref{sf16}).

If $\alpha$ is transcendental over $F$, then it is part of a transcendence
basis $A$ for $\Omega$ over $F$ (see \ref{te13}). If $A$ has at least two
elements, then there exists an automorphism $\sigma$ of $A$ such that
$\sigma(\alpha)\neq\alpha$. Now $\sigma$ defines an $F$-homomorphism
$F(A)\rightarrow\Omega$, which extends to an isomorphism $\Omega
\rightarrow\Omega$ as before. If $A=\{\alpha\}$, then we let $F(\alpha
)\rightarrow\Omega$ be the $F$-homomorphism sending $\alpha$ to $\alpha+1$.
Again, this extends to an isomorphism $\Omega\rightarrow\Omega$.
\end{proof}

%\begin{remark}
%\label{te21a} Theorem \ref{te21} holds with $\Omega$ only separably closed. To
%see this, let $\Omega^{\mathrm{al}}$ be an algebraic closure of $\Omega$. Then
%every automorphism $\sigma$ of $\Omega/F$ extends uniquely to an automorphism
%$\tilde{\sigma}$ of $\Omega^{\mathrm{al}}/F$: let $\alpha\in\Omega
%^{\mathrm{al}}$ and let $\alpha^{p^{n}}\in\Omega$; then $\tilde{\sigma}%
%(\alpha)$ is the unique root of $X^{p^{n}}-\sigma(\alpha^{p^{n}})$ in
%$\Omega^{\mathrm{al}}$. Thus, if $\alpha\in\Omega$ is fixed by all
%$F$-automorphisms of $\Omega$, then it is fixed by all $F$-automorphisms of
%$\Omega^{\mathrm{al}}$, and so it lies in $F$.
%\end{remark}


\addtocounter{X}{1}

Let $\Omega\supset F$ be fields and let $G{}=\Aut(\Omega/F)$. For any finite
subset $S$ of $\Omega$, let%
\[
G{}(S)=\{\sigma\in G{}\mid\sigma s=s\text{ for all }s\in S\}\text{.}%
\]
Then, as in \S 7, the subgroups $G{}(S)$ of $G{}$ form a neighbourhood base
for a unique topology on $G{}$, which we again call the%
\index{topology!Krull}
\emph{Krull topology}. The same argument as in \S 7 shows that this topology
is Hausdorff (but it is not necessarily compact).

\begin{theorem}
\label{te22}Let $\Omega\supset F$ be fields such that $\Omega^{G{}}=F$,
$G=\Aut(\Omega/F)$.

(a) For every finite extension $E$ of $F$ in $\Omega$, $\Omega^{\Aut(\Omega
/E)}=E$.

(b) The maps%
\begin{equation}
H\mapsto\Omega^{H},\quad M\mapsto\Aut(\Omega/M) \label{e32}%
\end{equation}
are inverse bijections between the set of compact subgroups of $G$ and the set
of intermediate fields over which $\Omega$ is Galois (possibly infinite):%
\[
\{\text{compact subgroups of }G\}\leftrightarrow\{\text{fields }M\text{ such
that }F\subset M\overset{\text{Galois}}{\subset}\Omega\}.
\]


(c) If there exists an $M$ finitely generated over $F$ such that $\Omega$ is
Galois over $M$, then $G$ is locally compact, and under (\ref{e32}):%
\[
\{\text{open compact subgroups of }G\}\overset{1\colon1}{\leftrightarrow
}\{\text{fields }M\text{ such that }F\overset{\text{finitely generated}%
}{\subset}M\overset{\text{Galois}}{\subset}\Omega\}.
\]


(d) Let $H$ be a subgroup of $G$, and let $M=\Omega^{H}$. Then the algebraic
closure $M_{1}$ of $M$ is Galois over $M$. If moreover $H=\Aut(\Omega/M)$,
then $\Aut(\Omega/M_{1})$ is a normal subgroup of $H$, and $\sigma
\mapsto\sigma|M_{1}$ maps $H/\Aut(\Omega/M_{1})$ isomorphically onto a dense
subgroup of $\Aut(M_{1}/M)$.
\end{theorem}

\begin{proof}
See 6.3 of Shimura, Goro., Introduction to the arithmetic theory of
automorphic functions. Princeton, 1971.
\end{proof}

\section{Exercises}

\begin{exercise}
\label{x91} Find the centralizer of complex conjugation in $\Aut(\mathbb{C}%
{}/\mathbb{Q}{})$.
\end{exercise}

\appendix\clearpage


\chapter{Review Exercises}

\renewcommand{\theY}{A-\arabic{Y}}

\begin{exercise}
\label{x24}Let $p$ be a prime number, and let $m$ and $n$ be positive integers.

\begin{enumerate}
\item Give necessary and sufficient conditions on $m$ and $n$ for
$\mathbb{F}_{p^{n}}$ to have a subfield isomorphic with $\mathbb{F}_{p^{m}}$.
Prove your answer.

\item If there is such a subfield, how many subfields isomorphic with
$\mathbb{F}_{p^{m}}$ are there, and why?
\end{enumerate}
\end{exercise}

\begin{exercise}
\label{x25} Show that the Galois group of the splitting field $F$ of $X^{3}-7$
over ${\mathbb{Q}}$ is isomorphic to $S_{3}$, and exhibit the fields between
${\mathbb{Q}}$ and $F$. Which of the fields between ${\mathbb{Q}}$ and $F$ are
normal over ${\mathbb{Q}}$?
\end{exercise}

\begin{exercise}
\label{x26} Prove that the two fields ${\mathbb{Q}}[\sqrt7]$ and ${\mathbb{Q}%
}[\sqrt{11}]$ are not isomorphic.
\end{exercise}

\begin{exercise}
\label{x27}

\begin{enumerate}
\item Prove that the multiplicative group of all nonzero elements in a finite
field is cyclic.

\item Construct explicitly a field of order $9$, and exhibit a generator for
its multiplicative group.
\end{enumerate}
\end{exercise}

\begin{exercise}
\label{x28} Let $X$ be transcendental over a field $F$, and let $E$ be a
subfield of $F(X)$ properly containing $F$. Prove that $X$ is algebraic over
$E$.
\end{exercise}

\begin{exercise}
\label{x29} Prove as directly as you can that if $\zeta$ is a primitive $p$th
root of $1$, $p$ prime, then the Galois group of ${\mathbb{Q}}[\zeta]$ over
${\mathbb{Q}}$ is cyclic of order $p-1$.
\end{exercise}

\begin{exercise}
\label{x30} Let $G$ be the Galois group of the polynomial $X^{5}-2$ over
${\mathbb{Q}}$.

\begin{enumerate}
\item Determine the order of $G$.

\item Determine whether $G$ is abelian.

\item Determine whether $G$ is solvable.
\end{enumerate}
\end{exercise}

\begin{exercise}
\label{x31}

\begin{enumerate}
\item Show that every field homomorphism from $\mathbb{R}$ to $\mathbb{R}$ is bijective.

\item Prove that $\mathbb{C}$ is isomorphic to infinitely many different
subfields of itself.
\end{enumerate}
\end{exercise}

\begin{exercise}
\label{x32} Let $F$ be a field with $16$ elements. How many roots in $F$ does
each of the following polynomials have? $X^{3}-1$; $X^{4}-1$; $X^{15}-1$;
$X^{17}-1$.
\end{exercise}

\begin{exercise}
\label{x33} Find the degree of a splitting field of the polynomial
$(X^{3}-5)(X^{3}-7)$ over ${\mathbb{Q}}$.
\end{exercise}

\begin{exercise}
\label{x34} Find the Galois group of the polynomial $X^{6}-5$ over each of the
fields ${\mathbb{Q}}$ and $\mathbb{R}$.
\end{exercise}

\begin{exercise}
\label{x35} The coefficients of a polynomial $f(X)$ are algebraic over a field
$F$. Show that $f(X)$ divides some nonzero polynomial $g(X)$ with coefficients
in $F$.
\end{exercise}

\begin{exercise}
\label{x36} Let $f(X)$ be a polynomial in $F[X]$ of degree $n$, and let $E$ be
a splitting field of $f$. Show that $[E\colon F]$ divides $n!$.
\end{exercise}

\begin{exercise}
\label{x37} Find a primitive element for the field ${\mathbb{Q}}[\sqrt
3,\sqrt7]$ over ${\mathbb{Q}}$, i.e., an element such that ${\mathbb{Q}}%
[\sqrt3,\sqrt7]={\mathbb{Q}}[\alpha]$.
\end{exercise}

\begin{exercise}
\label{x38} Let $G$ be the Galois group of $(X^{4}-2)(X^{3}-5)$ over
${\mathbb{Q}}$.

\begin{enumerate}
\item Give a set of generators for $G$, as well as a set of defining relations.

\item What is the structure of $G$ as an abstract group (is it cyclic,
dihedral, alternating, symmetric, etc.)?
\end{enumerate}
\end{exercise}

\begin{exercise}
\label{x39} Let $F$ be a finite field of characteristic $\neq2$. Prove that
$X^{2}=-1$ has a solution in $F$ if and only if $\left\vert F\right\vert
\equiv1\mod4$.
\end{exercise}

\begin{exercise}
\label{x40} Let $E$ be the splitting field over ${\mathbb{Q}}$ of
$(X^{2}-2)(X^{2}-5)(X^{2}-7)$. Find an element $\alpha$ in $E$ such that
$E={\mathbb{Q}}[\alpha]$. (You must prove that $E={\mathbb{Q}}[\alpha]$.)
\end{exercise}

\begin{exercise}
\label{x41} Let $E$ be a Galois extension of $F$ with Galois group $S_{n}$,
$n>1$ not prime. Let $H_{1}$ be the subgroup of $S_{n}$ of elements fixing
$1$, and let $H_{2}$ be the subgroup generated by the cycle $(123\ldots n)$.
Let $E_{i}=E^{H_{i}}$, $i=1, 2$. Find the degrees of $E_{1}$, $E_{2}$,
$E_{1}\cap E_{2}$, and $E_{1}E_{2}$ over $F$. Show that there exists a field
$M$ such that $F\subset M\subset E_{2}$, $M\neq F$, $M\neq E_{2}$, but that no
such field exists for $E_{1}$.
\end{exercise}

\begin{exercise}
\label{x42} Let $\zeta$ be a primitive $12$th root of $1$ over ${\mathbb{Q}}$.
How many fields are there strictly between ${\mathbb{Q}}[\zeta^{3}]$ and
${\mathbb{Q}}[\zeta]$.
\end{exercise}

\begin{exercise}
\label{x43} For the polynomial $X^{3}-3$, find explicitly its splitting field
over ${\mathbb{Q}}$ and elements that generate its Galois group.
\end{exercise}

\begin{exercise}
\label{x44} Let $E={\mathbb{Q}}[\zeta]$, $\zeta^{5}=1$, $\zeta\neq1$. Show
that $i\notin E$, and that if $L=E[i]$, then $-1$ is a norm from $L$ to $E$.
Here $i=\sqrt{-1}$.
\end{exercise}

\begin{exercise}
\label{x45} Let $E$ be an extension of $F$, and let $\Omega$ be an algebraic
closure of $E$. Let $\sigma_{1},\ldots,\sigma_{n}$ be distinct $F$%
-isomorphisms $E\to\Omega$.

\begin{enumerate}
\item Show that $\sigma_{1},\ldots,\sigma_{n}$ are linearly dependent over
$\Omega$.

\item Show that $[E\colon F]\geq m$.

\item Let $F$ have characteristic $p>0$, and let $L$ be a subfield of $\Omega$
containing $E$ and such that $a^{p}\in E$ for all $a\in L$. Show that each
$\sigma_{i}$ has a unique extension to a homomorphism $\sigma_{i}^{\prime
}\colon L\rightarrow\Omega$.
\end{enumerate}
\end{exercise}

\begin{exercise}
\label{x46} Identify the Galois group of the splitting field $F$ of $X^{4}-3$
over ${\mathbb{Q}}$. Determine the number of quadratic subfields.
\end{exercise}

\begin{exercise}
\label{x47} Let $F$ be a subfield of a finite field $E$. Prove that the trace
map $T=\Tr_{E/F}$ and the norm map $N=\Nm_{E/F}$ of $E$ over $F$ both map $E$
\textit{onto }$F$. (You may quote basic properties of finite fields and the
trace and norm.)
\end{exercise}

\begin{exercise}
\label{x48} Prove or disprove by counterexample.

\begin{enumerate}
\item If $L/F$ is an extension of fields of degree $2$, then there is an
automorphism $\sigma$ of $L$ such that $F$ is the fixed field of $\sigma$.

\item The same as (a) except that $L$ is also given to be finite.
\end{enumerate}
\end{exercise}

\begin{exercise}
\label{x49} A finite Galois extension $L$ of a field $K$ has degree $8100$.
Show that there is a field $F$ with $K\subset F\subset L$ such that $[F\colon
K]=100$.
\end{exercise}

\begin{exercise}
\label{x50} An algebraic extension $L$ of a field $K$ of characteristic $0$ is
generated by an element $\theta$ that is a root of both of the polynomials
$X^{3}-1$ and $X^{4}+X^{2}+1$. Given that $L\neq K$, find the minimal
polynomial of $\theta$.
\end{exercise}

\begin{exercise}
\label{x51} Let $F/{\mathbb{Q}}$ be a Galois extension of degree $3^{n}$,
$n\geq1$. Prove that there is a chain of fields
\[
{\mathbb{Q}}=F_{0}\subset F_{1}\subset\cdots F_{n}=F
\]
such that for every $i$, $0\leq i\leq n-1$, $[F_{i+1}\colon F_{i}]=3$.
\end{exercise}

\begin{exercise}
\label{x52} Let $L$ be the splitting field over ${\mathbb{Q}}$ of an equation
of degree $5$ with distinct roots. Suppose that $L$ has an automorphism that
fixes three of these roots while interchanging the other two and also an
automorphism $\alpha\neq1$ of order $5$.

\begin{enumerate}
\item Prove that the group of automorphisms of $L$ is the symmetric group on
$5$ elements.

\item How many proper subfields of $L$ are normal extensions of ${\mathbb{Q}}%
$? For each such field $F$, what is $[F\colon{\mathbb{Q}}]$?
\end{enumerate}
\end{exercise}

\begin{exercise}
\label{x53} If $L/K$ is a separable algebraic field extension of finite degree
$d$, show that the number of fields between $K$ and $L$ is at most $2^{d!}$.
[This is far from best possible. See math.stackexchange.com, question 522976.]
\end{exercise}

\begin{exercise}
\label{x54} Let $K$ be the splitting field over ${\mathbb{Q}}$ of $X^{5}-1$.
Describe the Galois group $\Gal(K/{\mathbb{Q}})$ of $K$ over ${\mathbb{Q}}$,
and show that $K$ has exactly one subfield of degree $2$ over ${\mathbb{Q}}$,
namely, ${\mathbb{Q}}[\zeta+\zeta^{4}]$, $\zeta\neq1$ a root of $X^{5}-1$.
Find the minimal polynomial of $\zeta+\zeta^{4}$ over ${\mathbb{Q}}$. Find
$\Gal(L/{\mathbb{Q}})$ when $L$ is the splitting field over ${\mathbb{Q}}$ of

\begin{enumerate}
\item $(X^{2}-5)(X^{5}-1)$;

\item $(X^{2}+3)(X^{5}-1)$.
\end{enumerate}
\end{exercise}

\begin{exercise}
\label{x55} Let $\Omega_{1}$ and $\Omega_{2}$ be algebraically closed fields
of transcendence degree $5$ over ${\mathbb{Q}}$, and let $\alpha\colon
\Omega_{1}\rightarrow\Omega_{2}$ be a homomorphism (in particular,
$\alpha(1)=1$). Show that $\alpha$ is a bijection. (State carefully all
theorems you use.)
\end{exercise}

\begin{exercise}
\label{x56} Find the group of ${\mathbb{Q}}$-automorphisms of the field
$k={\mathbb{Q}}[\sqrt{-3},\sqrt{-2}]$.
\end{exercise}

\begin{exercise}
\label{x57} Prove that the polynomial $f(X)=X^{3}-5$ is irreducible over the
field ${\mathbb{Q}}[\sqrt7]$. If $L$ is the splitting field of $f(X)$ over
${\mathbb{Q}}[\sqrt7]$, prove that the Galois group of $L/{\mathbb{Q}}%
[\sqrt7]$ is isomorphic to $S_{3}$. Prove that there must exist a subfield $K$
of $L$ such that the Galois group of $L/K$ is cyclic of order $3$.
\end{exercise}

\begin{exercise}
\label{x58} Identify the Galois group $G$ of the polynomial $f(X)=X^{5}%
-6X^{4}+3$ over $F$, when (a) $F={\mathbb{Q}}$ and when (b) $F=\mathbb{F}_{2}%
$. In each case, if $E$ is the splitting field of $f(X)$ over $F$, determine
how many fields $K$ there are such that $E\supset K\supset F$ with $[K\colon
F]=2$.
\end{exercise}

\begin{exercise}
\label{x59} Let $K$ be a field of characteristic $p$, say with $p^{n}$
elements, and let $\theta$ be the automorphism of $K$ that maps every element
to its $p$th power. Show that there exists an automorphism $\alpha$ of $K$
such that $\theta\alpha^{2}=1$ if and only if $n$ is odd.
\end{exercise}

\begin{exercise}
\label{x60} Describe the splitting field and Galois group, over ${\mathbb{Q}}%
$, of the polynomial $X^{5}-9$.
\end{exercise}

\begin{exercise}
\label{x61} Suppose that $E$ is a Galois field extension of a field $F$ such
that $[E\colon F]=5^{3}\cdot(43)^{2}$. Prove that there exist fields $K_{1}$
and $K_{2}$ lying strictly between $F$ and $E$ with the following properties:
(i) each $K_{i}$ is a Galois extension of $F$; (ii) $K_{1}\cap K_{2}=F$; and
(iii) $K_{1}K_{2}=E$.
\end{exercise}

\begin{exercise}
\label{x62} Let $F=\mathbb{F}_{p}$ for some prime $p$. Let $m$ be a positive
integer not divisible by $p$, and let $K$ be the splitting field of $X^{m}-1$.
Find $[K\colon F]$ and prove that your answer is correct.
\end{exercise}

\begin{exercise}
\label{x63} Let $F$ be a field of 81 elements. For each of the following
polynomials $g(X)$, determine the number of roots of $g(X)$ that lie in $F$:
$X^{80}-1$, $X^{81}-1$, $X^{88}- 1$.
\end{exercise}

\begin{exercise}
\label{x64} Describe the Galois group of the polynomial $X^{6}-7$ over
${\mathbb{Q}}$.
\end{exercise}

\begin{exercise}
\label{x65} Let $K$ be a field of characteristic $p>0$ and let $F=K(u,v)$ be a
field extension of degree $p^{2}$ such that $u^{p}\in K$ and $v^{p}\in K$.
Prove that $K$ is not finite, that $F$ is not a simple extension of $K$, and
that there exist infinitely many intermediate fields $F\supset L\supset K$.
\end{exercise}

\begin{exercise}
\label{x66} Find the splitting field and Galois group of the polynomial
$X^{3}-5$ over the field ${\mathbb{Q}}[\sqrt2]$.
\end{exercise}

\begin{exercise}
\label{x67} For every prime $p$, find the Galois group over ${\mathbb{Q}}$ of
the polynomial $X^{5}-5p^{4}X+p$.
\end{exercise}

\begin{exercise}
\label{x68} Factorize $X^{4}+1$ over each of the finite fields (a)
$\mathbb{F}_{5}$; (b) $\mathbb{F}_{25}$; and (c) $\mathbb{F}_{125} $. Find its
splitting field in each case.
\end{exercise}

\begin{exercise}
\label{x69} Let ${\mathbb{Q}}[\alpha]$ be a field of finite degree over
${\mathbb{Q}}$. Assume that there is a $q\in{\mathbb{Q}}$, $q\neq0$, such that
$|\rho(\alpha)|=q$ for all homomorphisms $\rho\colon{\mathbb{Q}}%
[\alpha]\rightarrow\mathbb{C}$. Show that the set of roots of the minimal
polynomial of $\alpha$ is the same as that of $q^{2}/\alpha$. Deduce that
there exists an automorphism $\sigma$ of ${\mathbb{Q}}[\alpha]$ such that

\begin{enumerate}
\item $\sigma^{2}=1$ and

\item $\rho(\sigma\gamma)=\overline{\rho(\gamma)}$ for all $\gamma
\in{\mathbb{Q}}[\alpha]$ and $\rho\colon{\mathbb{Q}}[\alpha]\rightarrow
\mathbb{C}$.
\end{enumerate}
\end{exercise}

\begin{exercise}
\label{x70} Let $F$ be a field of characteristic zero, and let $p$ be a prime
number. Suppose that $F$ has the property that all irreducible polynomials
$f(X)\in F[X]$ have degree a power of $p$ $(1=p^{0}$ is allowed). Show that
every equation $g(X)=0$, $g\in F[X]$, is solvable by extracting radicals.
\end{exercise}

\begin{exercise}
\label{x71} Let $K={\mathbb{Q}}[\sqrt5,\sqrt{-7}]$ and let $L$ be the
splitting field over ${\mathbb{Q}}$ of $f(X)=X^{3}-10$.

\begin{enumerate}
\item Determine the Galois groups of $K$ and $L$ over ${\mathbb{Q}}$.

\item Decide whether $K$ contains a root of $f$.

\item Determine the degree of the field $K\cap L$ over ${\mathbb{Q}}$.
\end{enumerate}

[Assume all fields are subfields of $\mathbb{C}{}$.]
\end{exercise}

\begin{exercise}
\label{x72} Find the splitting field (over $\mathbb{F}_{p}$) of $X^{p^{r}%
}-X\in\mathbb{F}_{p}[X]$, and deduce that $X^{p^{r}}-X$ has an irreducible
factor $f\in\mathbb{F}_{p}[X]$ of degree $r$. Let $g(X)\in\mathbb{Z}[X]$ be a
monic polynomial that becomes equal to $f(X)$ when its coefficients are read
modulo $p$. Show that $g(X)$ is irreducible in ${\mathbb{Q}}[X]$.
\end{exercise}

\begin{exercise}
\label{x73} Let $E$ be the splitting field of $X^{3}-51$ over ${\mathbb{Q}}$.
List all the subfields of $E$, and find an element $\gamma$ of $E$ such that
$E={\mathbb{Q}}[\gamma]$.
\end{exercise}

\begin{exercise}
\label{x74} Let $k=\mathbb{F}_{1024}$ be the field with $1024$ elements, and
let $K$ be an extension of $k$ of degree $2$. Prove that there is a unique
automorphism $\sigma$ of $K$ of order $2$ which leaves $k$ elementwise fixed
and determine the number of elements of $K^{\times}$ such that $\sigma
(x)=x^{-1}$.
\end{exercise}

\begin{exercise}
\label{x75} Let $F$ and $E$ be finite fields of the same characteristic. Prove
or disprove these statements:

\begin{enumerate}
\item There is a ring homomorphism of $F$ into $E$ if and only if $\left\vert
E\right\vert $ is a power of $\left\vert F\right\vert $.

\item There is an injective group homomorphism of the multiplicative group of
$F$ into the multiplicative group of $E$ if and only if $\left\vert
E\right\vert $ is a power of $\left\vert F\right\vert $.
\end{enumerate}
\end{exercise}

\begin{exercise}
\label{x76} Let $L/K$ be an algebraic extension of fields. Prove that $L$ is
algebraically closed if every polynomial over $K$ factors completely over $L$.
\end{exercise}

\begin{exercise}
\label{x77} Let $K$ be a field, and let $M=K(X)$, $X$ an indeterminate. Let
$L$ be an intermediate field different from $K$. Prove that $M$ is
finite-dimensional over $L$.
\end{exercise}

\begin{exercise}
\label{x78} Let $\theta_{1},\theta_{2},\theta_{3}$ be the roots of the
polynomial $f(X)=X^{3}+X^{2}-9X+1$.

\begin{enumerate}
\item Show that the $\theta_{i}$ are real, nonrational, and distinct.

\item Explain why the Galois group of $f(X)$ over ${\mathbb{Q}}$ must be
either $A_{3}$ or $S_{3}$. Without carrying it out, give a brief description
of a method for deciding which it is.

\item Show that the rows of the matrix
\[
\left(
\begin{matrix}
3 & 9 & 9 & 9\\
3 & \theta_{1} & \theta_{2} & \theta_{3}\\
3 & \theta_{2} & \theta_{3} & \theta_{1}\\
3 & \theta_{3} & \theta_{1} & \theta_{2}%
\end{matrix}
\right)
\]
are pairwise orthogonal; compute their lengths, and compute the determinant of
the matrix.
\end{enumerate}
\end{exercise}

\begin{exercise}
\label{x79} Let $E/K$ be a Galois extension of degree $p^{2}q $ where $p$ and
$q$ are primes, $q<p$ and $q$ not dividing $p^{2}-1$. Prove that:

\begin{enumerate}
\item there exist intermediate fields $L$ and $M$ such that $[L\colon
K]=p^{2}$ and $[M\colon K]=q$;

\item such fields $L$ and $M$ must be Galois over $K$; and

\item the Galois group of $E/K$ must be abelian.
\end{enumerate}
\end{exercise}

\begin{exercise}
\label{x80} Let $\zeta$ be a primitive $7$th root of $1$ (in $\mathbb{C}$).

\begin{enumerate}
\item Prove that $1+X+X^{2}+X^{3}+X^{4}+X^{5}+X^{6}$ is the minimal polynomial
of $\zeta$ over ${\mathbb{Q}}$.

\item Find the minimal polynomial of $\zeta+\frac1{\zeta}$ over ${\mathbb{Q}}$.
\end{enumerate}
\end{exercise}

\begin{exercise}
\label{x81} Find the degree over ${\mathbb{Q}}$ of the Galois closure $K$ of
${\mathbb{Q}}[2^{\frac14}]$ and determine the isomorphism class of
$\Gal(K/{\mathbb{Q}})$.
\end{exercise}

\begin{exercise}
\label{x82} Let $p,q$ be distinct positive prime numbers, and consider the
extension $K={\mathbb{Q}}[\sqrt p,\sqrt q]\supset{\mathbb{Q}}$.

\begin{enumerate}
\item Prove that the Galois group is isomorphic to $C_{2}\times C_{2}$.

\item Prove that every subfield of $K$ of degree $2$ over ${\mathbb{Q}}$ is of
the form ${\mathbb{Q}}[\sqrt m]$ where $m\in\{p,q, pq\}$.

\item Show that there is an element $\gamma\in K$ such that $K={\mathbb{Q}%
}[\gamma]$.
\end{enumerate}
\end{exercise}

\clearpage


\chapter{Two-hour Examination}

\textbf{1.} (a) Let $\sigma$ be an automorphism of a field $E$. If $\sigma
^{4}=1$ and
\[
\sigma(\alpha)+\sigma^{3}(\alpha)=\alpha+\sigma^{2}(\alpha)\qquad
\text{\textrm{all }} \alpha\in E,
\]
show that $\sigma^{2}=1$.

\noindent(b) Let $p$ be a prime number and let $a,b$ be rational numbers such
that $a^{2}+pb^{2}=1$. Show that there exist rational numbers $c,d$ such that
$a=\frac{c^{2}-pd^{2}}{c^{2}+pd^{2}}$ and $b=\frac{2cd}{c^{2}+pd^{2}}$.
\bigskip!!Check!!

\medskip\noindent\textbf{2.} Let $f(X)$ be an irreducible polynomial of degree
$4$ in ${\mathbb{Q}}[X]$, and let $g(X)$ be the resolvent cubic of $f$. What
is the relation between the Galois group of $f$ and that of $g$? Find the
Galois group of $f$ if

\begin{enumerate}
\item $g(X)=X^{3}-3X+1$;

\item $g(X)=X^{3}+3X+1$.
\end{enumerate}

\medskip\noindent\textbf{3.} (a) How many monic irreducible factors does
$X^{255}-1\in\mathbb{F}_{2}[X]$ have, and what are their degrees.

\noindent(b) How many monic irreducible factors does $X^{255}-1\in{\mathbb{Q}%
}[X]$ have, and what are their degrees?

\medskip\noindent\textbf{4.} Let $E$ be the splitting field of $(X^{5}%
-3)(X^{5}- 7)\in{\mathbb{Q}}[X]$. What is the degree of $E$ over ${\mathbb{Q}%
}$? How many proper subfields of $E$ are there that are not contained in the
splitting fields of both $X^{5}-3$ and $X^{5}-7$?

\noindent[You may assume that $7$ is not a $5$th power in the splitting field
of $X^{5}-3$.]

\medskip\noindent\textbf{5.} Consider an extension $\Omega\supset F$ of
fields. Define $a\in\Omega$ to be $F$\textit{-constructible\/} if it is
contained in a field of the form
\[
F[\sqrt{a_{1}},\ldots,\sqrt{a_{n}}],\qquad a_{i}\in F[\sqrt{a_{1}}%
,\ldots,\sqrt{a_{i-1}}].
\]
Assume $\Omega$ is a finite Galois extension of $F$ and construct a field $E
$, $F\subset E\subset\Omega$, such that every $a\in\Omega$ is $E$%
-constructible and $E$ is minimal with this property.

\medskip\noindent\textbf{6.} Let $\Omega$ be an extension field of a field $F
$. Show that every $F$-homomorphism $\Omega\to\Omega$ is an isomorphism provided:

\begin{enumerate}
\item $\Omega$ is algebraically closed, and

\item $\Omega$ has finite transcendence degree over $F$.
\end{enumerate}

Can either of the conditions (i) or (ii) be dropped? (Either prove, or give a counterexample.)

\medskip\noindent\textit{You should prove all answers. You may use results
proved in class or in the notes, but you should indicate clearly what you are
using. }

\medskip\noindent\textit{Possibly useful facts:\/} The discriminant of
$X^{3}+aX+b$ is $-4a^{3}-27b^{2}$ and $2^{8}-1=255=3\times5\times17$.

\clearpage


\chapter{Solutions to the Exercises}

\textit{These solutions fall somewhere between hints and complete solutions.
Students were expected to write out complete solutions}.

\noindent\textbf{\ref{x1}.} Similar to Example \ref{ef17}.

\medskip\noindent\textbf{\ref{x2}.} Verify that $3$ is not a square in
${\mathbb{Q}}[\sqrt{2}]$, and so $[{\mathbb{Q}}[\sqrt{2},\sqrt{3}%
]\colon{\mathbb{Q}}]=4$.

\medskip\noindent\textbf{\ref{x3}.} (a) Apply the division algorithm, to get
$f(X)=q(X)(X-a)+r(X)$ with $r(X)$ constant, and put $X=a$ to find $r=f(a)$.

\noindent(c) Use that factorization in $F[X]$ is unique (or use induction on
the degree of $f$).

\noindent(d) If $G$ had two cyclic factors $C$ and $C^{\prime}$ whose orders
were divisible by a prime $p$, then $G$ would have (at least) $p^{2}$ elements
of order dividing $p$. This doesn't happen, and it follows that $G$ is cyclic.

\noindent(e) The elements of order $m$ in $F^{\times}$ are the roots of the
polynomial $X^{m}-1$, and so there are at most $m$ of them. Hence every finite
subgroup $G$ of $F^{\times}$ satisfies the condition in (d).

\medskip\noindent\textbf{\ref{x4}.} Note that it suffices to construct
$\alpha=\cos\frac{2\pi}{7}$, and that $[{\mathbb{Q}}[\alpha]\colon{\mathbb{Q}%
}]=\frac{7-1}{2}=3$, and so its minimal polynomial has degree $3$ (see Example
\ref{ft19}). There is a standard method (once taught in high schools) for
solving cubics using the equation
\[
\cos3\theta=4\cos^{3}\theta-3\cos\theta.
\]
By \textquotedblleft completing the cube\textquotedblright, reduce the cubic
to the form $X^{3}-pX-q$. Then construct a square root $a$ of $\frac{4p}{3}$,
so that $a^{2}=\frac{4p}{3}$. Let $3\theta$ be the angle such that
$\cos3\theta=\frac{4q}{a^{3}}$, and use the angle trisector to construct
$\cos\theta$. From the displayed equation, we find that $\alpha=a\cos\theta$
is a root of $X^{3}-pX-q$. For a geometric construction, see sx93476.

\medskip\noindent\textbf{\ref{x4a}.} Let $f_{1}$ be an irreducible factor of
$f$ in $E[X]$, and let $(L,\alpha)$ be a stem field for $f_{1}$ over $E$. Then
$m|[L\colon F]$ because $L\supset E$ (\ref{ef10}). But $f(\alpha)=0$, and so
$(F[\alpha],\alpha)$ is a stem field for $f$ over $F$, which implies that
$[F[\alpha]\colon F]=n$. Now $n|[L\colon F]$ because $L\supset F[\alpha]$. We
deduce that $[L\colon F]=mn$ and $[L\colon E]=n$. But $[L\colon E]=\deg
(f_{1})$, and so $f_{1}=f$.

\medskip\noindent\textbf{\ref{x4b}.} The polynomials $f(X)-1$ and $f(X)+1$
have only finitely many roots, and so there exists an $n\in\mathbb{Z}{}$ such
that $f(n)\neq\pm1$. Let $p$ be a prime dividing $f(n)$. Then $f(n)=0$ modulo
$p$, and so $f$ has a root in $\mathbb{F}{}_{p}$. Thus it is not irreducible
in $\mathbb{F}{}_{p}[X]$.

\medskip\noindent\textbf{\ref{x4c}.} It is easy to see that $R$ is ring, and
so it remains to show that every nonzero element $a+b\alpha+c\alpha^{2}$ has
an inverse in $R$. Let $f(X)=X^{3}-2$ and $g(X)=cX^{2}+bX+a$. As $f$ is
irreducible and $\deg(g)<\deg(f)$, $f$ and $g$ are relatively prime. Therefore
Euclid's algorithm gives polynomials $u(X)$ and $v(X)$ with $\deg v<3$ such
that $u(X)f(X)+v(X)g(X)=1$. On putting $X=\alpha$ in this equation, we find
that $v(\alpha)g(\alpha)=1$, i.e., $v(\alpha)$ is inverse to $g(\alpha
)=a+b\alpha+c\alpha^{2}$. Alternatively, $R$ is an integral domain (being a
subring of $\mathbb{C}{}$), and so (\ref{ef14}) shows that $R$ is a field.

\medskip\noindent\textbf{\ref{x5}.} (a) is obvious, as is the ``only if'' in
(b). For the ``if'' note that for any $a\in S(E)$, $a\notin F^{2}$, $E\approx
F[X]/(X^{2} -a)$.

(c) Take $E_{i}={\mathbb{Q}}[\sqrt{p_{i}}]$ with $p_{i}$ the $i$th prime.
Check that $p_{i}$ is the only prime that becomes a square in $E_{i}$. For
this use that $(a+b\sqrt p)^{2}\in{\mathbb{Q}}\implies2ab=0$.

(d) Every field of characteristic $p$ contains (an isomorphic copy of)
$\mathbb{F}_{p}$, and so we are looking at the quadratic extensions of
$\mathbb{F}_{p}$. The homomorphism $a\mapsto a^{2}\colon\mathbb{F}_{p}%
^{\times}\rightarrow\mathbb{F}_{p}^{\times}$ has kernel $\{\pm1\}$, and so its
image has index $2$ in $\mathbb{F}_{p}^{\times}$. Thus the only possibility
for $S(E)$ is $\mathbb{F}_{p}^{\times}$, and so there is at most one $E$ (up
to $\mathbb{F}_{p}$-isomorphism). To get one, take $E=F[X]/(X^{2}-a)$,
$a\notin\mathbb{F}_{p}^{2}$.

\medskip\noindent\textbf{\ref{x6}.} (a) If $\alpha$ is a root of
$f(X)=X^{p}-X-a$ (in some splitting field), then the remaining roots are
$\alpha+1,\ldots,\alpha+p-1$, which obviously lie in whichever field contains
$\alpha$. Moreover, they are distinct. Suppose that, in $F[X]$,
\[
f(X)=(X^{r}+a_{1}X^{r-1}+\cdots+a_{r})(X^{p-r}+\cdots),\quad0<r<p.
\]
Then $-a_{1}$ is a sum of $r$ of the roots of $f$, $-a_{1}=r\alpha+d$ some
$d\in\mathbb{Z}\cdot1_{F}$, and it follows that $\alpha\in F$.

(b) As $0$ and $1$ are not roots of $X^{p}-X-1$ in $\mathbb{F}_{p}$ it can't
have $p$ distinct roots in $\mathbb{F}{}_{p}$, and so (a) implies that
$X^{p}-X-1$ is irreducible in $\mathbb{F}_{p}[X]$ and hence also in
$\mathbb{Z}[X]$ and $\mathbb{Q}{}[X]$ (see \ref{ef8m}, \ref{ef6}).

\medskip\noindent\textbf{\ref{x7}.} Let $\alpha$ be the real $5$th root of
$2$. Eisenstein's criterion shows that $X^{5}-2$ is irreducible in
${\mathbb{Q}}[X]$, and so ${\mathbb{Q}}[\sqrt[5]{2}]$ has degree $5$ over
${\mathbb{Q}}$. The remaining roots of $X^{5}-2$ are $\zeta\alpha,\zeta
^{2}\alpha,\zeta^{3}\alpha,\zeta^{4}\alpha$, where $\zeta$ is a primitive
$5$th root of $1$. It follows that the subfield of $\mathbb{C}$ generated by
the roots of $X^{5}-2$ is ${\mathbb{Q}}[\zeta,\alpha]$. The degree of
${\mathbb{Q}}[\zeta,\alpha]$ is $20$, since it must be divisible by
$[{\mathbb{Q}}[\zeta]\colon{\mathbb{Q}}]=4$ and $[{\mathbb{Q}}[\alpha
]\colon{\mathbb{Q}}]=5$.

\medskip\noindent\textbf{\ref{x8}.} It's $\mathbb{F}_{p}$ because $X^{p^{m}%
}-1=(X-1)^{p^{m}}$\textbf{.} (Perhaps I meant $X^{p^{m}}-X$ --- that would
have been more interesting.)

\medskip\noindent\textbf{\ref{x9}.} If $f(X)=\prod(X-\alpha_{i})^{m_{i}}$,
$\alpha_{i}\neq\alpha_{j}$, then
\[
f^{\prime}(X)=\sum m_{i}\frac{f(X)}{X-\alpha_{i}}%
\]
and so $d(X)=\prod_{m_{i}>1}(X-\alpha_{i})^{m_{i}-1}$. Therefore $g(X)
=\prod(X-\alpha_{i})$.

\medskip\noindent\textbf{\ref{x10}.} From (\ref{ft3}) we know that either $f$
is separable or $f(X)=f_{1}(X^{p})$ for some polynomial $f_{1}$. Clearly
$f_{1}$ is also irreducible. If $f_{1}$ is not separable, it can be written
$f_{1}(X)=f_{2}(X^{p})$. Continue in the way until you arrive at a separable
polynomial. For the final statement, note that $g(X)=\prod(X-a_{i})$,
$a_{i}\neq a_{j}$, and so $f(X)=g(X^{p^{e}})=\prod(X-\alpha_{i})^{p^{e}}$ with
$\alpha_{i}^{p^{e}}=a_{i}$.

\medskip\noindent\textbf{\ref{x11}.} Let $\sigma$ and $\tau$ be automorphisms
of $F(X)$ given by $\sigma(X)=-X$ and $\tau(X)=1-X$. Then $\sigma$ and $\tau$
fix $X^{2}$ and $X^{2}-X$ respectively, and so $\sigma\tau$ fixes
$E\overset{\df}{=}F(X)\cap F(X^{2}-X)$. But $\alpha\tau
X=1+X$, and so $(\sigma\tau)^{m}(X)=m+X$. Thus $\Aut(F(X)/E)$ is infinite,
which implies that $[F(X)\colon E]$ is infinite (otherwise $F(X)=E[\alpha
_{1},\ldots,\alpha_{n}]$; an $E$-automorphism of $F(X)$ is determined by its
values on the $\alpha_{i}$, and its value on $\alpha_{i}$ is a root of the
minimal polynomial of $\alpha_{i}$). If $E$ contains a polynomial $f(X)$ of
degree $m>0$, then $[F(X)\colon E]\leq\lbrack F(X)\colon F(f(X))]=m$ --- contradiction.

\medskip\noindent\textbf{\ref{x12}.} Since $1+\zeta+\cdots+\zeta^{p-1}=0$, we
have $\alpha+\beta=-1$. If $i\in H$, then $iH=H$ and $i(G\smallsetminus
H)=G\smallsetminus H$, and so $\alpha$ and $\beta$ are fixed by $H$. If $j\in
G\smallsetminus H$, then $jH=G\smallsetminus H$ and $j(G\smallsetminus H)=H$,
and so $j\alpha=\beta$ and $j\beta=\alpha$. Hence $\alpha\beta\in{\mathbb{Q}}%
$, and $\alpha$ and $\beta$ are the roots of $X^{2}+X+\alpha\beta$. Note that
\[
\alpha\beta=\sum_{i,j}\zeta^{i+j},\quad i\in H,\quad j\in G\smallsetminus H.
\]
How many times do we have $i+j=0$? If $i+j=0$, then $-1=i^{-1}j$, which is a
nonsquare; conversely, if $-1$ is a nonsquare, take $i=1$ and $j=-1$ to get
$i+j=0$. Hence
\[
i+j=0\text{\textrm{\ some }}i\in H,\quad j\in G\smallsetminus H\iff
-1\text{\textrm{\ is a square mod }}p\iff p\equiv-1\mod4.
\]
If we do have a solution to $i+j=0$, we get all solutions by multiplying it
through by the $\frac{p-1}{2}$ squares. So in the sum for $\alpha\beta$ we see
1 a total of $\frac{p-1}{2}$ times when $p\equiv3\mod4$ and not at all if
$p\equiv1\mod4$. In either case, the remaining terms add to a rational number,
which implies that each power of $\zeta$ occurs the same number of times. Thus
for $p\equiv1\mod4$, $\alpha\beta=-(\frac{p-1}{2})^{2}/(p-1)=-\frac{p-1}{4}$;
the polynomial satisfied by $\alpha$ and $\beta$ is $X^{2}+X-\frac{p-1}{4}$,
whose roots are $(-1\pm\sqrt{1+p-1})/2$; the fixed field of $H$ is
${\mathbb{Q}}[\sqrt{p}]$. For $p\equiv-1\mod4$, $\alpha\beta=\frac{p-1}%
{2}+(-1)\left(  (\frac{p-1}{2})^{2}-\frac{p-1}{2}\right)  /(p-1)=\frac{p-1}%
{2}-\frac{p-3}{4}=\frac{p+1}{4}$; the polynomial is $X^{2}+X+\frac{p-1}{4}$,
with roots $(-1\pm\sqrt{1-p-1})/2$; the fixed field of $H$ is ${\mathbb{Q}%
}[\sqrt{-p}]$.

%See also sx 984457


\medskip\noindent\textbf{\ref{x13}.} (a) It is easy to see that $M$ is Galois
over ${\mathbb{Q}}$ with Galois group $\langle\sigma,\tau\rangle$:%
\[
\left\{
\begin{array}
[c]{c}%
\sigma\sqrt{2}=-\sqrt{2}\newline\\
\sigma\sqrt{3}=\sqrt{3}%
\end{array}
\right.  \quad\quad\left\{
\begin{array}
[c]{c}%
\tau\sqrt{2}=\sqrt{2}\newline\\
\tau\sqrt{3}=-\sqrt{3}%
\end{array}
\right.  .
\]
\noindent(b) We have
\[
\frac{\sigma\alpha^{2}}{\alpha^{2}}=\frac{2-\sqrt{2}}{2+\sqrt{2}}%
=\frac{(2-\sqrt{2})^{2}}{4-2}=\left(  \frac{2-\sqrt{2}}{\sqrt{2}}\right)
^{2}=(\sqrt{2}-1)^{2},
\]
i.e., $\sigma\alpha^{2}=((\sqrt{2}-1)\alpha)^{2}$. Thus, if $\alpha\in M$,
then $\sigma\alpha=\pm(\sqrt{2}-1)\alpha$, and
\[
\sigma^{2}\alpha=(-\sqrt{2}-1)(\sqrt{2}-1)\alpha=-\alpha;
\]
as $\sigma^{2}\alpha=\alpha\neq0$, this is impossible. Hence $\alpha\notin M$,
and so $[E\colon{\mathbb{Q}}]=8$.

\noindent Extend $\sigma$ to an automorphism (also denoted $\sigma$) of $E$.
Again $\sigma\alpha=\pm(\sqrt2-1)\alpha$ and $\sigma^{2}\alpha=-\alpha$, and
so $\sigma^{2}\neq1$. Now $\sigma^{4}\alpha=\alpha$, $\sigma^{4}|M=1$, and so
we can conclude that $\sigma$ has order $4$. After possibly replacing $\sigma$
with its inverse, we may suppose that $\sigma\alpha=(\sqrt2-1)\alpha$.

\noindent Repeat the above argument with $\tau$: $\frac{\tau\alpha^{2}}%
{\alpha^{2}}=\frac{3-\sqrt3}{3+\sqrt3}=\left(  \frac{3-\sqrt3}{\sqrt6}\right)
^{2}$, and so we can extend $\tau$ to an automorphism of $L$ (also denoted
$\tau$) with $\tau\alpha=\frac{3-\sqrt3}{\sqrt6}\alpha$. The order of $\tau$
is $4$.

\noindent Finally compute that
\[
\sigma\tau\alpha=\frac{3-\sqrt{3}}{-\sqrt{6}}(\sqrt{2}-1)\alpha;\quad
\tau\sigma\alpha=(\sqrt{2}-1)\frac{3-\sqrt{3}}{\sqrt{6}}\alpha.
\]
Hence $\sigma\tau\neq\tau\sigma$, and $\Gal(E/{\mathbb{Q}})$ has two
noncommuting elements of order $4$. Since it has order $8$, it must be the
quaternion group.

%See also sx983458


\medskip\noindent\textbf{\ref{x13b}.} Let $G=\Aut(E/F)$. Then $E$ is Galois
over $E^{G}$ with Galois group $G$, and so $|G|=[E\colon E^{G}]$. Now
$[E\colon F]=[E\colon E^{G}][E^{G}\colon F]=|G|[E^{G}\colon F]$.

\medskip\noindent\textbf{\ref{x14}.} The splitting field is the smallest field
containing all $m$th roots of $1$. Hence it is $\mathbb{F}_{p^{n}}$ where $n$
is the smallest positive integer such that $m_{0}|p^{n}-1$, $m=m_{0}p^{r}$,
where $p$ is prime and does not divide $m_{0}$.

\medskip\noindent\textbf{\ref{x15}.} We have $X^{4}-2X^{3}-8X-3=(X^{3}%
+X^{2}+3X+1)(X-3)$, and $g(X)=X^{3}+X^{2}+3X+1$ is irreducible over
${\mathbb{Q}}$ (use \ref{ef4}), and so its Galois group is either $A_{3}$ or
$S_{3}$. Either check that its discriminant is not a square or, more simply,
show by examining its graph that $g(X)$ has only one real root, and hence its
Galois group contains a transposition (cf. the proof of \ref{cg13}).

\medskip\noindent\textbf{\ref{x16}.} Eisenstein's criterion shows that
$X^{8}-2$ is irreducible over ${\mathbb{Q}}$, and so $[{\mathbb{Q}}%
[\alpha]\colon{\mathbb{Q}}]=8$ where $\alpha$ is a positive $8$th root of $2$.
As usual for polynomials of this type, the splitting field is ${\mathbb{Q}%
}[\alpha,\zeta]$ where $\zeta$ is any primitive $8$th root of $1$. For
example, $\zeta$ can be taken to be $\frac{1+i}{\sqrt{2}}$, which lies in
${\mathbb{Q}}[\alpha,i]$. It follows that the splitting field is ${\mathbb{Q}%
}[\alpha,i]$. Clearly ${\mathbb{Q}}[\alpha,i]\neq{\mathbb{Q}}[\alpha]$,
because ${\mathbb{Q}}[\alpha]$, unlike $i$, is contained in $\mathbb{R}$, and
so $[{\mathbb{Q}}[\alpha,i]\colon{\mathbb{Q}}[\alpha]]=2$. Therefore the
degree is $2\times8=16$.

\medskip\noindent\textbf{\ref{x17}.} Find an extension $L/F$ with Galois group
$S_{4}$, and let $E$ be the fixed field of $S_{3}\subset S_{4}$. There is no
subgroup strictly between $S_{n}$ and $S_{n-1}$, because such a subgroup would
be transitive and contain an $(n-1)$-cycle and a transposition, and so would
equal $S_{n}$. We can take $E=L^{S_{3}}$. More specifically, we can take $L$
to be the splitting field of $X^{4}-X+2$ over $\mathbb{Q}{}$ and $E$ to be the
subfield generated by a root of the polynomial (see \ref{ft23r}).

\medskip\noindent\textbf{\ref{x18}.} Type: ``Factor$(X^{343}-X)$ mod 7;'' and
discard the $7$ factors of degree $1$.

\medskip\noindent\textbf{\ref{x19}.} Type \textquotedblleft galois$(X^{6}%
+2X^{5}+3X^{4}+4X^{3}+5X^{2}+6X+7)$;\textquotedblright. It is the group
$\PGL_{2}(\mathbb{F}_{5})$ (group of invertible $2\times2$ matrices over
$\mathbb{F}_{5}$ modulo scalar matrices) which has order $120$. Alternatively,
note that there are the following factorizations: mod $3$, irreducible; mod
$5$ (deg $3$)(deg $3$); mod $13$ (deg $1$)(deg $5$); mod $19 $, (deg $1)^{2}%
$(deg $4$); mod $61$ (deg $1)^{2}$(deg $2)^{2}$; mod $79$, (deg $2)^{3}$. Thus
the Galois group has elements of type:
\[
6,\quad3+3,\quad1+5,\quad1+1+4,\quad1+1+2+2,\quad2+2+2.
\]
No element of type $2$, $3$, $3+2$, or $4+2$ turns up by factoring modulo any
of the first $400$ primes (or, so I have been told). This suggests it is the
group $T14$ in the tables in Butler and McKay, which is indeed $\PGL_{2}%
(\mathbb{F}_{5})$.

\medskip\noindent\textbf{\ref{x20}.} $\impliedby$: Condition (a) implies that
$G_{f}$ contains a $5$-cycle, condition (b) implies that $G_{f}\subset A_{5}$,
and condition (c) excludes $A_{5}$. That leaves $D_{5}$ and $C_{5}$ as the
only possibilities (see, for example, Jacobson, Basic Algebra I, p305, Ex 6).
The derivative of $f$ is $5X^{4}+a$, which has at most $2$ real zeros, and so
(from its graph) we see that $f$ can have at most $3$ real zeros. Thus complex
conjugation acts as an element of order $2$ on the splitting field of $f$, and
this shows that we must have $G_{f}=D_{5}$.

\noindent$\implies$: Regard $D_{5}$ as a subgroup of $S_{5}$ by letting it act
on the vertices of a regular pentagon---all subgroups of $S_{5}$ isomorphic to
$D_{5}$ look like this one. If $G_{f}=D_{5}$, then (a) holds because $D_{5}$
is transitive, (b) holds because $D_{5}\subset A_{5}$, and (c) holds because
$D_{5}$ is solvable.

\medskip\noindent\textbf{\ref{x20a}.} Suppose that $f$ is irreducible of
degree $n$. Then $f$ has no root in a field $\mathbb{F}{}_{p^{m}}$ with $m<n$,
which implies (a). However, every root $\alpha$ of $f$ lies in $\mathbb{F}%
{}_{p^{n}}$, and so $\alpha^{p^{n}}-\alpha=0$. Hence $(X-\alpha)|(X^{p^{n}%
}-X)$, which implies (b) because $f$ has no multiple roots.

Conversely, suppose that (a) and (b) hold. It follows from (b) that all roots
of $f$ lie in $\mathbb{F}{}_{p^{n}}$. Suppose that $f$ had an irreducible
factor $g$ of degree $m<n$. Then every root of $g$ generates $\mathbb{F}%
{}_{p^{m}}$, and so $\mathbb{F}{}_{p^{m}}\subset\mathbb{F}{}_{p^{n}}$.
Consequently, $m$ divides $n$, and so $m$ divides $n/p_{i}$ for some $i$. But
then $g$ divides both $f$ and $X^{p^{n/p_{i}}}-X$, contradicting (a). Thus $f$
is irreducible.

\medskip\noindent\textbf{\ref{x20b}.} Let $a_{1},a_{2}$ be conjugate nonreal
roots, and let $a_{3}$ be a real root. Complex conjugation defines an element
$\sigma$ of the Galois group of $f$ switching $a_{1}$ and $a_{2}$ and fixing
$a_{3}$. On the other hand, because $f$ is irreducible, its Galois group acts
transitively on its roots, and so there is a $\tau$ such that $\tau
(a_{3})=a_{1}$. Now%
\begin{align*}
&  a_{3}\overset{\tau}{\mapsto}a_{1}\overset{\sigma}{\mapsto}a_{2}\\
&  a_{3}\overset{\sigma}{\mapsto}a_{3}\overset{\tau}{\mapsto}a_{1\text{.}}%
\end{align*}
This statement is false for reducible polynomials --- consider for example
$f(X)=(X^{2}+1)(X-1)$.

p.131. The following corrections to the solution to Question 5-1 are based on
emails from .

In the case $a=-4$, we can't apply Dedekind's theorem modulo 2 because the
discriminant is even. Instead, show that the discriminant is negative, so G
cannot be $A_{3}$, or else use Dedekind's theorem for p=13 to get the
factorization $f(x)=(x-1)(x-2)(x^{2}+4x-2)$ modulo 13 and conclude the
existence of a 2-cycle.

I claim that 4 is the maximum number of distinct groups, however, when $a=12$,
$f(x)=(3 - 2 x + x^{2}) (4 + 3 x + x^{2})$ is reducible and in fact the Galois
group is $V_{4}$ in this case (the one not sitting inside $A_{4}$). So the
number of distinct groups seems to be 5.

To prove that 4 is the maximum number, I suggest checking modulo 2 to see that
there must be a 2-cycle or a 4-cycle. However, when $a$ is even, the
discriminant is even, and so we can't apply Dedekind's theorem.

Instead, we can show that the discriminant is not a square. Ip writes: this
will require more work (checking $|a|<10000$ shows that none gives a square,
so I think the conclusion is still correct.)

Later Ip writes: showing that the discriminant is not a square is equivalent
to solving for integral points of the elliptic curve $y^{2}=256x^{3}%
-203x^{2}+88x-16$. By substituting $x\mapsto x/2^{16}$, $y\mapsto y/2^{20}$,
the resulting equation becomes

$y^{2}= x^{3}-51968x^{2} + 1476395008 x-17592186044416$

and one can put this in e.g. MAGMA to compute the Mordell-Weil group (i.e. the
group of rational points of this elliptic curve) which turns out to be
trivial. This means that the equation has no rational solutions, or in other
words, D cannot be a square.

\medskip\noindent\textbf{\ref{x21}.} For $a=1$, this is the polynomial
$\Phi_{5}(X)$, whose Galois group is cyclic of order $4$.

\noindent For $a=0$, $f(X)=X(X^{3}+X^{2}+X+1)=X(X+1)(X^{2}+1)$, whose Galois
group is cyclic of order $2$.

\noindent For $a=12$, $f(X)=(X^{2}-2X+3)(X^{2}+3X+4)$, whose Galois group is
$V_{4}$ (the one not sitting inside $A_{4}$).

\noindent For $a=-4$, $f(X)=(X-1)(X^{3}+2X^{2}+3X+4)$. The cubic does not have
$\pm1,\pm2,$ or $\pm4$ as roots, and so it is irreducible in ${\mathbb{Q}}%
[X]$. Hence its Galois group is $S_{3}$ or $A_{3}$. Modulo $13$,
$f(X)=(X-1)(X-2)(X^{2}+4X-2)$, and so the Galois group contains a $2$-cycle by
Dedekind's theorem. Therefore, it is $S_{3}$. Alternatively, use that the
discriminant of the cubic is $-200$, which is not a square. Note that, because
$2$ divides the discriminant, we can't use Dedekind's theorem with $p=2$.

\noindent For a general $a$, the resolvent cubic is
\[
g(X)=X^{3}-X^{2}+(1-4a)X+3a-1.
\]
For $a=-1$, $f=X^{4}+X^{3}+X^{2}+X-1$ is irreducible modulo $2$, and so it is
irreducible over $\mathbb{Q}{}$. The resolvant cubic is $g=X^{3}-X^{2}+5X-4$,
which is irreducible. Moreover
\[
g^{\prime}(x)=3x^{2}-2x+5=3(x-\tfrac{1}{3})^{2}+4\tfrac{2}{3}>0\text{, all
}x,
\]
and so $g$ has exactly one real root. Hence the Galois group of $g$ is $S_{3}%
$, and it follows that the Galois group of $f$ is $S_{4}$.

Thus we have found the following Galois groups (in $S_{4}$): $C_{2}$, $C_{4}$,
$V_{4}$ ($\nsubseteq A_{4})$, $S_{3}$, $S_{4}$. This seems to be all. The
discriminant of $f$ is $256a^{3}-203a^{2}+88a-16$. If $a$ is odd, this is odd,
and we can apply Dedekind's theorem with $p=2$ to show that the Galois group
contains a $2$-cycle or a $4$-cycle, and so $1,A_{3},A_{4},V_{4}$ are not
possible. In the general case, the discriminant is not a square, and so the
Galois group is not contained in $A_{4}$.

Showing that the discriminant is not a square is equivalent to solving for
integral points on the elliptic curve $Y^{2}=256X^{3}-203X^{2}+88X-16$. The
substitution $X\mapsto X/2^{16}$, $Y\mapsto Y/2^{20}$ turns this into the equation%

\[
Y^{2}=X^{3}-51968X^{2}+1476395008X-17592186044416\text{. }%
\]
According to PARI this has no nonzero rational points, and so the discriminant
can't be a square. (I thank Ivan Ip for his help with this solution.)

\medskip\noindent\textbf{\ref{x22}.} We have $\Nm(a+ib)=a^{2}+b^{2}$. Hence
$a^{2}+b^{2}=1$ if and only $a+ib=\frac{s+it}{s-it}$ for some $s,t\in
{\mathbb{Q}}$ (Hilbert's Theorem 90). The rest is easy.

\medskip\noindent\textbf{\ref{x23}.} The degree $[{\mathbb{Q}}[\zeta
_{n}]\colon{\mathbb{Q}}]=\varphi(n)$, $\zeta_{n}$ a primitive $n$th root of
$1$, and $\varphi(n)\rightarrow\infty$ as $n\rightarrow\infty$.

\medskip\noindent\textbf{\ref{x91}.} If some element centralizes the complex
conjugation, then it must preserve the real numbers as a set. Now, since any
automorphism of the real numbers preserves the set of squares, it must
preserve the order; and hence be continuous. Since $\mathbb{Q}$ is fixed, this
implies that the real numbers are fixed pointwise. It follows that any element
which centralized the complex conjugation must be the identity or the complex
conjugation itself. See mo121083, Andreas Thom.

\medskip\noindent\textbf{\ref{x24}.} (a) Need that $m|n$, because
\[
n=[\mathbb{F}{}_{p^{n}}\colon\mathbb{F}{}_{p}]=[\mathbb{F}{}_{p^{n}}%
\colon\mathbb{F}{}_{p^{m}}]\cdot\lbrack\mathbb{F}{}_{p^{m}}\colon\mathbb{F}%
{}_{p}]=[\mathbb{F}{}_{p^{n}}\colon\mathbb{F}{}_{p^{m}}]\cdot m.
\]
Use Galois theory to show there exists one, for example. (b) Only one; it
consists of all the solutions of $X^{p^{m}}-X=0$.

\medskip\noindent\textbf{\ref{x25}.} The polynomial is irreducible by
Eisenstein's criterion. The polynomial has only one real root, and therefore
complex conjugation is a transposition in $G_{f}$. This proves that
$G_{f}\approx S_{3}$. The discriminant is $-1323=-3^{3}7^{2}$. Only the
subfield ${\mathbb{Q}}[\sqrt{-3}]$ is normal over ${\mathbb{Q}}$. The
subfields ${\mathbb{Q}}[\sqrt[3]{7}]$, ${\mathbb{Q}}[\zeta\sqrt[3]{7}]$
${\mathbb{Q}}[\zeta^{2}\sqrt[3]{7}]$ are not normal over ${\mathbb{Q}}$. [The
discriminant of $X^{3}-a$ is $-27a^{2}=-3(3a)^{2}$.]

\medskip\noindent\textbf{\ref{x26}.} The prime $7$ becomes a square in the
first field, but $11$ does not: $(a+b\sqrt{7})^{2}=a^{2}+7b^{2}+2ab\sqrt{7}$,
which lies in ${\mathbb{Q}}$ only if $ab=0$. Hence the rational numbers that
become squares in ${\mathbb{Q}}[\sqrt{7}]$ are those that are already squares
or lie in $7{\mathbb{Q}}^{\times2}$.

\medskip\noindent\textbf{\ref{x27}.}(a) See Exercise 3.

(b) Let $F=\mathbb{F}_{3}[X]/(X^{2}+1)$. Modulo $3$
\[
X^{8}-1=(X-1)(X+1)(X^{2}+1)(X^{2}+X+2)(X^{2}+2X+2).
\]
Take $\alpha$ to be a root of $X^{2}+X+2$.

\medskip\noindent\textbf{\ref{x28}.} Since $E\neq F$, $E$ contains an element
$\frac{f}{g}$ with the degree of $f$ or $g>0$. Now
\[
f(T)-\frac{f(X)}{g(X)}g(T)
\]
is a nonzero polynomial having $X$ as a root.

\medskip\noindent\textbf{\ref{x29}.} Use Eisenstein to show that
$X^{p-1}+\cdots+1$ is irreducible, etc. Done in class.

\medskip\noindent\textbf{\ref{x30}.} The splitting field is ${\mathbb{Q}%
}[\zeta,\alpha]$ where $\zeta^{5}=1$ and $\alpha^{5}=2$. It is generated by
$\sigma=(12345)$ and $\tau=(2354)$, where $\sigma\alpha=\zeta\alpha$ and
$\tau\zeta=\zeta^{2}$. The group has order $20$. It is not abelian (because
${\mathbb{Q}}[\alpha]$ is not Galois over ${\mathbb{Q}}$), but it is solvable
(its order is $<60$).

\medskip\noindent\textbf{\ref{x31}.} (a) A homomorphism $\alpha\colon
\mathbb{R}{}\rightarrow\mathbb{R}{}$ acts as the identity map on $\mathbb{Z}%
{}$, hence on $\mathbb{Q}{}$, and it maps positive real numbers to positive
real numbers, and therefore preserves the order. Hence, for each real number
$a$,
\[
\{r\in{\mathbb{Q}}\mid a<r\}=\{r\in{\mathbb{Q}}\mid\alpha(a)<r\},
\]
which implies that $\alpha(a)=a$.

(b) Choose a transcendence basis $A$ for $\mathbb{C}$ over ${\mathbb{Q}}$.
Because it is infinite, there is a bijection $\alpha\colon A\rightarrow
A^{\prime}$ from $A$ onto a proper subset. Extend $\alpha$ to an isomorphism
${\mathbb{Q}}(A)\rightarrow{\mathbb{Q}}(A^{\prime})$, and then extend it to an
isomorphism $\mathbb{C}\rightarrow\mathbb{C}^{\prime}$ where $\mathbb{C}%
^{\prime}$ is the algebraic closure of ${\mathbb{Q}}(A^{\prime})$ in
$\mathbb{C}$.

\medskip\noindent\textbf{\ref{x32}.} The group $F^{\times}$ is cyclic of order
$15$. It has $3$ elements of order dividing $3$, $1$ element of order dividing
$4$, $15$ elements of order dividing $15$, and $1$ element of order dividing
$17$.

\medskip\noindent\textbf{\ref{x33}.} If $E_{1}$ and $E_{2}$ are Galois
extensions of $F$, then $E_{1}E_{2}$ and $E_{1}\cap E_{2}$ are Galois over
$F$, and there is an exact sequence
\[
1\to\Gal(E_{1}E_{2}/F)\to\Gal(E_{1}/F)\times\Gal(E_{2}/F)\to\Gal(E_{1}\cap
E_{2}/F)\to1.
\]
In this case, $E_{1}\cap E_{2}={\mathbb{Q}}[\zeta]$ where $\zeta$ is a
primitive cube root of $1$. The degree is $18$.

\medskip\noindent\textbf{\ref{x34}.} Over ${\mathbb{Q}}$, the splitting field
is ${\mathbb{Q}}[\alpha,\zeta]$ where $\alpha^{6}=5$ and $\zeta^{3}=1$
(because $-\zeta$ is then a primitive $6$th root of $1$). The degree is $12$,
and the Galois group is $D_{6}$ (generators $(26)(35)$ and $(123456)$).

Over $\mathbb{R}$, the Galois group is $C_{2}$.

\medskip\noindent\textbf{\ref{x35}.} Let the coefficients of $f$ be
$a_{1},\ldots,a_{n}$ --- they lie in the algebraic closure $\Omega$ of $F$.
Let $g(X)$ be the product of the minimal polynomials over $F$ of the roots of
$f$ in $\Omega$.

Alternatively, the coefficients will lie in some finite extension $E$ of $F$,
and we can take the norm of $f(X)$ from $E[X]$ to $F[X]$.

\medskip\noindent\textbf{\ref{x36}.} If $f$ is separable, $[E\colon
F]=(G_{f}\colon1)$, which is a subgroup of $S_{n}$. Etc..

\medskip\noindent\textbf{\ref{x37}.} $\sqrt{3}+\sqrt{7}$ will do.

\medskip\noindent\textbf{\ref{x38}.} The splitting field of $X^{4}-2$ is
$E_{1}={\mathbb{Q}}[i,\alpha]$ where $\alpha^{4}=2$; it has degree $8$, and
Galois group $D_{4}$. The splitting field of $X^{3}-5$ is $E_{2}={\mathbb{Q}%
}[\zeta,\beta]$; it has degree $6$, and Galois group $D_{3}$. The Galois group
is the product (they could only intersect in ${\mathbb{Q}}[\sqrt{3}]$, but
$\sqrt{3}$ does not become a square in $E_{1}$).

\medskip\noindent\textbf{\ref{x39}.} The multiplicative group of $F$ is cyclic
of order $q-1$. Hence it contains an element of order $4$ if and only if
$4|q-1$.

\medskip\noindent\textbf{\ref{x40}.} Take $\alpha=\sqrt{2}+\sqrt{5}+\sqrt{7}$.

\medskip\noindent\textbf{\ref{x41}.} We have $E_{1}=E^{H_{1}}$, which has
degree $n$ over $F$, and $E_{2}=E^{<1\cdots n>}$, which has degree $(n-1)!$
over $F$, etc.. This is really a problem in group theory posing as a problem
in field theory.

\medskip\noindent\textbf{\ref{x42}.} We have ${\mathbb{Q}}[\zeta]={\mathbb{Q}%
}[i,\zeta^{\prime}]$ where $\zeta^{\prime}$ is a primitive cube root of $1 $
and $\pm i=\zeta^{3}$ etc..

\medskip\noindent\textbf{\ref{x43}.} The splitting field is ${\mathbb{Q}%
}[\zeta,\sqrt[3]{3}]$, and the Galois group is $S_{3}$.

\medskip\noindent\textbf{\ref{x44}.} Use that%
\[
(\zeta+\zeta^{4})(1+\zeta^{2})=\zeta+\zeta^{4}+\zeta^{3}+\zeta
\]


\medskip\noindent\textbf{\ref{x45}.} (a) is Dedekind's theorem. (b) is Artin's
theorem \ref{ft10}. (c) is O.K. because $X^{p}-a^{p}$ has a unique root in
$\Omega$.

\medskip\noindent\textbf{\ref{x46}.} The splitting field is ${\mathbb{Q}%
}[i,\alpha]$ where $\alpha^{4}=3$, and the Galois group is $D_{4}$ with
generators $(1234)$ and $(13)$ etc..

\medskip\noindent\textbf{\ref{x47}.} From Hilbert's theorem 90, we know that
the kernel of the map $N\colon E^{\times}\rightarrow F^{\times}$ consists of
elements of the form $\frac{\sigma\alpha}{\alpha}$. The map $E^{\times
}\rightarrow E^{\times}$, $\alpha\mapsto\frac{\sigma\alpha}{\alpha}$, has
kernel $F^{\times}$. Therefore the kernel of $N$ has order $\frac{q^{m}%
-1}{q-1}$, and hence its image has order $q-1$. There is a similar proof for
the trace --- I don't know how the examiners expected you to prove it.

\medskip\noindent\textbf{\ref{x48}.} (a) is false---could be inseparable. (b)
is true---couldn't be inseparable.

\medskip\noindent\textbf{\ref{x49}.} Apply the Sylow theorem to see that the
Galois group has a subgroup of order $81$. Now the Fundamental Theorem of
Galois theory shows that $F$ exists.

\medskip\noindent\textbf{\ref{x50}.} The greatest common divisor of the two
polynomials over ${\mathbb{Q}}$ is $X^{2}+X+1$, which must therefore be the
minimal polynomial for $\theta$.

\medskip\noindent\textbf{\ref{x51}.} Theorem on $p$-groups plus the
Fundamental Theorem of Galois Theory.

\medskip\noindent\textbf{\ref{x52}.} It was proved in class that $S_{p}$ is
generated by an element of order $p$ and a transposition (\ref{cg12}). There
is only one $F$, and it is quadratic over ${\mathbb{Q}}$.

\medskip\noindent\textbf{\ref{x53}.} Let $L=K[\alpha]$. The splitting field of
the minimal polynomial of $\alpha$ has degree at most $d!$, and a set with
$d!$ elements has at most $2^{d!}$ subsets. [Of course, this bound is much too
high: the subgroups are very special subsets. For example, they all contain
$1$ and they are invariant under $a\mapsto a^{-1}$.]

\medskip\noindent\textbf{\ref{x54}.} The Galois group is $(\mathbb{Z}%
/5\mathbb{Z})^{\times}$, which cyclic of order $4$, generated by $2$.
\[
(\zeta+\zeta^{4})+(\zeta^{2}+\zeta^{3})=-1,\quad(\zeta+\zeta^{4})(\zeta
^{2}+\zeta^{3})=-1.
\]


(a) Omit.

(b) Certainly, the Galois group is a product $C_{2}\times C_{4}$.

\medskip\noindent\textbf{\ref{x55}.} Let $a_{1},\ldots,a_{5}$ be a
transcendence basis for $\Omega_{1}/{\mathbb{Q}}$. Their images are
algebraically independent, therefore they are a maximal algebraically
independent subset of $\Omega_{2}$, and therefore they form a transcendence
basis, etc..

\medskip\noindent\textbf{\ref{x56}.} $C_{2}\times C_{2}$.

\medskip\noindent\textbf{\ref{x57}.} If $f(X)$ were reducible over
${\mathbb{Q}}[\sqrt{7}]$, it would have a root in it, but it is irreducible
over ${\mathbb{Q}}$ by Eisenstein's criterion. The discriminant is $-675$,
which is not a square in $\mathbb{R}$, much less ${\mathbb{Q}}[\sqrt{7}]$.

\medskip\noindent\textbf{\ref{x58}.} (a) Should be $X^{5}-6X^{4}+3$. The
Galois group is $S_{5}$, with generators $(12)$ and $(12345)$ --- it is
irreducible (Eisenstein) and (presumably) has exactly $2$ nonreal roots. (b)
It factors as $(X+1)(X^{4}+X^{3}+X^{2}+X+1)$. Hence the splitting field has
degree $4$ over $\mathbb{F}_{2}$, and the Galois group is cyclic.

\medskip\noindent\textbf{\ref{x59}.} This is really a theorem in group theory,
since the Galois group is a cyclic group of order $n$ generated by $\theta$.
If $n$ is odd, say $n=2m+1$, then $\alpha=\theta^{m}$ does.

\medskip\noindent\textbf{\ref{x60}.} It has order $20$, generators $(12345)$
and $(2354)$.

\medskip\noindent\textbf{\ref{x61}.} Take $K_{1}$ and $K_{2}$ to be the fields
corresponding to the Sylow $5$ and Sylow $43$ subgroups. Note that of the
possible numbers $1,6,11,16,21,...$ of Sylow $5$-subgroups, only $1$ divides
$43$. There are $1$, $44$, $87$, ... subgroups of ....

\medskip\noindent\textbf{\ref{x62}.} See Exercise 14.

\medskip\noindent\textbf{\ref{x63}.} The group $F^{\times}$ is cyclic of order
$80$; hence $80$, $1$, $8$.

\medskip\noindent\textbf{\ref{x64}.} It's $D_{6}$, with generators $(26)(35)$
and $(123456)$. The polynomial is irreducible by Eisenstein's criterion, and
its splitting field is ${\mathbb{Q}}[\alpha,\zeta]$ where $\zeta\neq1$ is a
cube root of $1$.

\medskip\noindent\textbf{\ref{x65}.} Example \ref{ag4m}.

\medskip\noindent\textbf{\ref{x66}.} Omit.

\medskip\noindent\textbf{\ref{x67}.} It's irreducible by Eisenstein. Its
derivative is $5X^{4}-5p^{4}$, which has the roots $X=\pm p$. These are the
max and mins, $X=p$ gives negative; $X=-p$ gives positive. Hence the graph
crosses the $x$-axis $3$ times and so there are $2$ imaginary roots. Hence the
Galois group is $S_{5}$.

\medskip\noindent\textbf{\ref{x68}.} Its roots are primitive $8$th roots of
$1$. It splits completely in $\mathbb{F}_{25}$. (a) $(X^{2}+2)(X^{2}+3)$.

\medskip\noindent\textbf{\ref{x69}.} $\rho(\alpha)\overline{\rho(\alpha
)}=q^{2}$, and $\rho(\alpha)\rho(\frac{q^{2}}{\alpha})=q^{2}$. Hence
$\rho(\frac{q^{2}}{\alpha})$ is the complex conjugate of $\rho(\alpha)$. Hence
the automorphism induced by complex conjugation is independent of the
embedding of ${\mathbb{Q}}[\alpha]$ into $\mathbb{C}$.

\medskip\noindent\textbf{\ref{x70}.} The argument that proves the Fundamental
Theorem of Algebra, shows that its Galois group is a $p$-group. Let $E$ be the
splitting field of $g(X)$, and let $H$ be the Sylow $p$-subgroup. Then
$E^{H}=F$, and so the Galois group is a $p$-group.

\medskip\noindent\textbf{\ref{x71}.} (a) $C_{2}\times C_{2}$ and $S_{3}$. (b)
No. (c). 1

\medskip\noindent\textbf{\ref{x72}.} Omit.

\medskip\noindent\textbf{\ref{x73}.} Omit.

\medskip\noindent\textbf{\ref{x74}.} $1024=2^{10}$. Want $\sigma x\cdot x=1$,
i.e., $Nx=1$. They are the elements of the form $\frac{\sigma x}{x}$; have
\[
1\xrightarrow{\phantom{x\mapsto\frac {\sigma x}x}}k^{\times}%
\xrightarrow{\phantom{x\mapsto\frac {\sigma x}x}}K^{\times}%
\xrightarrow{x\mapsto\frac {\sigma x}x}K^{\times} .
\]
Hence the number is $2^{11}/2^{10}=2$.

\medskip\noindent\textbf{\ref{x75}.} Pretty standard. False; true.

\medskip\noindent\textbf{\ref{x76}.} Omit.

\medskip\noindent\textbf{\ref{x77}.} Similar to a previous problem.

\medskip\noindent\textbf{\ref{x78}.} Omit.

\medskip\noindent\textbf{\ref{x79}.} This is really a group theory problem
disguised as a field theory problem.

\medskip\noindent\textbf{\ref{x80}.} (a) Prove it's irreducible by apply
Eisenstein to $f(X+1)$. (b) See example worked out in class.

\medskip\noindent\textbf{\ref{x81}.} It's $D_{4}$, with generators $(1234)$
and $(12)$.

\medskip\noindent\textbf{\ref{x82}.} Omit.

\subsection{Solutions for the exam.}

\medskip\noindent\textbf{1.} (a) Let $\sigma$ be an automorphism of a field
$E$. If $\sigma^{4}=1$ and
\[
\sigma(\alpha)+\sigma^{3}(\alpha)=\alpha+\sigma^{2}(\alpha)\qquad
\text{\textrm{all }} \alpha\in E,
\]
show that $\sigma^{2}=1$.

If $\sigma^{2}\neq1$, then $1,\sigma,\sigma^{2},\sigma^{3}$ are distinct
automorphisms of $E$, and hence are linearly independent (Dedekind \ref{ag13})
--- contradiction. [If $\sigma^{2}=1$, then the condition becomes $2\sigma=2$,
so either $\sigma=1$ or the characteristic is $2$ (or both).]

\noindent(b) Let $p$ be a prime number and let $a,b$ be rational numbers such
that $a^{2}+pb^{2}=1$. Show that there exist rational numbers $c,d$ such that
$a=\frac{c^{2}+pd^{2}}{c^{2}-pd^{2}}$ and $b=\frac{2cd}{ c^{2}-pd^{2}}$.

Apply Hilbert's Theorem 90 to ${\mathbb{Q}}[\sqrt p]$ (or ${\mathbb{Q}}[\sqrt{
-p}]$, depending how you wish to correct the sign). \bigskip

\medskip\noindent\textbf{2.} Let $f(X)$ be an irreducible polynomial of degree
$4$ in ${\mathbb{Q}}[X]$, and let $g(X)$ be the resolvent cubic of $f$. What
is the relation between the Galois group of $f$ and that of $g$? Find the
Galois group of $f$ if

\begin{enumerate}
\item $g(X)=X^{3}-3X+1$;

\item $g(X)=X^{3}+3X+1$.
\end{enumerate}

We have $G_{g}=G_{f}/G_{f}\cap V$, where $V=\{1,(12)(34),\ldots\}$. The two
cubic polynomials are irreducible, because their only possible roots are
$\pm1$. From their discriminants, one finds that the first has Galois group
$A_{3}$ and the second $S_{3}$. Because $f(X)$ is irreducible, $4|(G_{f}%
\colon1)$ and it follows that $G_{f}=A_{4}$ and $S_{4}$ in the two cases.

\medskip\noindent\textbf{3.} (a) How many monic irreducible factors does
$X^{255}-1\in\mathbb{F}_{2}[X]$ have, and what are their degrees?

Its roots are the nonzero elements of $\mathbb{F}_{2^{8}}$, which has
subfields $\mathbb{F}_{2^{4}}\mathbb{\supset F}_{2^{2}}\mathbb{\supset F}_{2}
$. There are $256-16$ elements not in $\mathbb{F}_{16}$, and their minimal
polynomials all have degree $8$. Hence there are $30$ factors of degree $8$,
$3$ of degree $4$, and $1$ each of degrees $2$ and $1$.

\noindent(b) How many monic irreducible factors does $X^{255}-1\in{\mathbb{Q}%
}[X]$ have, and what are their degrees?

Obviously, $X^{255}-1=\prod_{d|255}\Phi_{d}=\Phi_{1}\Phi_{3}\Phi_{5}\Phi
_{15}\cdots\Phi_{255}$, and we showed in class that the $\Phi_{d}$ are
irreducible. They have degrees $1,2,4,8,16,32,64,128$.

\medskip\noindent\textbf{4.} Let $E$ be the splitting field of $(X^{5}%
-3)(X^{5}- 7)\in{\mathbb{Q}}[X]$. What is the degree of $E$ over ${\mathbb{Q}%
}$? How many proper subfields of $E$ are there that are not contained in the
splitting fields of both $X^{5}-3$ and $X^{5}-7$?

The splitting field of $X^{5}-3$ is ${\mathbb{Q}}[\zeta,\alpha]$, which has
degree $5$ over ${\mathbb{Q}}[\zeta]$ and $20$ over ${\mathbb{Q}}$. The Galois
group of $X^{5}-7$ over ${\mathbb{Q}}[\zeta,\alpha]$ is (by ...) a subgroup of
a cyclic group of order $5$, and hence has order $1$ or $5$. Since $7$ is not
a $5$th power in ${\mathbb{Q}}[\zeta,\alpha]$, it must be $5$. Thus
$[E\colon{\mathbb{Q}}]=100$, and
\[
G=\Gal(E/{\mathbb{Q}})=(C_{5}\times C_{5})\rtimes C_{4}.
\]
We want the nontrivial subgroups of $G$ not containing $C_{5}\times C_{5}$.
The subgroups of order $5$ of $C_{5}\times C_{5}$ are lines in $(\mathbb{F}%
_{5})^{2}$, and hence $C_{5}\times C_{5}$ has $6+1=7$ proper subgroups. All
are normal in $G$. Each subgroup of $C_{5}\times C_{5}$ is of the form
$H\cap(C_{5}\times C_{5})$ for exactly $3$ subgroups $H$ of $G$ corresponding
to the three possible images in $G/(C_{5}\times C_{5})=C_{4}$. Hence we have
$21$ subgroups of $G$ not containing $C_{5}\times C_{5}$, and $20$ nontrivial
ones. Typical fields: ${\mathbb{Q}}[\alpha]$, ${\mathbb{Q}}[\alpha,\cos
\frac{2\pi}{5}]$, ${\mathbb{Q}}[\alpha,\zeta]$.

\noindent[You may assume that $7$ is not a $5$th power in the splitting field
of $X^{5}-3$.]

\medskip\noindent\textbf{5.} Consider an extension $\Omega\supset F$ of
fields. Define $\alpha\in\Omega$ to be $F$\textit{-constructible\/} if it is
contained in a field of the form
\[
F[\sqrt{a_{1}},\ldots,\sqrt{a_{n}}],\qquad a_{i}\in F[\sqrt{a_{1}}%
,\ldots,\sqrt{a_{i-1}}].
\]
Assume $\Omega$ is a finite Galois extension of $F$ and construct a field $E
$, $F\subset E\subset\Omega$, such that every $a\in\Omega$ is $E$%
-constructible and $E$ is minimal with this property.

Suppose $E$ has the required property. From the primitive element theorem, we
know $\Omega=E[a]$ for some $a$. Now $a$ $E$-constructible $\implies$
$[\Omega\colon E]$ is a power of $2$. Take $E=\Omega^{H}$, where $H$ is the
Sylow $2$-subgroup of $\Gal(\Omega/F)$.

\medskip\noindent\textbf{6.} Let $\Omega$ be an extension field of a field $F
$. Show that every $F$-homomorphism $\Omega\to\Omega$ is an isomorphism provided:

\begin{enumerate}
\item $\Omega$ is algebraically closed, and

\item $\Omega$ has finite transcendence degree over $F$.
\end{enumerate}

Can either of the conditions (i) or (ii) be dropped? (Either prove, or give a counterexample.)

Let $A$ be a transcendence basis for $\Omega/F$. Because $\sigma\colon
\Omega\rightarrow\Omega$ is injective, $\sigma(A)$ is algebraically
independent over $F$, and hence (because it has the right number of elements)
is a transcendence basis for $\Omega/F$. Now $F[\sigma A]\subset\sigma
\Omega\subset\Omega$. Because $\Omega$ is algebraic over $F[\sigma A]$ and
$\sigma\Omega$ is algebraically closed, the two are equal. Neither condition
can be dropped. E.g., $\mathbb{C}(X)\mathbb{\rightarrow C}(X)$, $X\mapsto
X^{2}$. E.g., $\Omega=$ the algebraic closure of $\mathbb{C}(X_{1},X_{2}%
,X_{3},\ldots)$, and consider an extension of the map $X_{1}\mapsto X_{2}$,
$X_{2}\mapsto X_{3}$, $\ldots$.\clearpage%

%TCIMACRO{\TeXButton{footnotesize}{\footnotesize}}%
%BeginExpansion
\footnotesize
%EndExpansion


\printindex

\end{document}
